% beamer shouldn't load amssymb given mathdesign
\expandafter\let\csname ver@amssymb.sty\endcsname\empty
\documentclass{beamer}
% \documentclass[handout,t]{beamer}
\expandafter\let\csname ver@amssymb.sty\endcsname\relax
\let\Tiny=\tiny


\usefonttheme[onlymath]{serif}

\usepackage[charter,cal=cmcal]{mathdesign}


% \batchmode

\usepackage{pgfpages}
% \pgfpagesuselayout{4 on 1}[letterpaper,landscape,border shrink=5mm]
% \pgfpagesuselayout{2 on 1}[letterpaper, portrait, border shrink=5mm]


\usepackage{amsmath}
\usepackage{amsthm}
\usepackage{latexsym}
\usepackage{enumerate}
\usepackage{epsfig}
\usepackage{calc}
\usepackage{color}
\usepackage{ifthen}
\usepackage{capt-of}
\usepackage{array,colortbl,booktabs}
\usepackage{tikz,pgfplots}
\usepackage{exscale,relsize}
\usepackage{mycommands}

\pgfplotsset{compat=1.3}

% \usetikzlibrary{plotmarks,pgfplots.groupplots,shapes,arrows,positioning}
\usetikzlibrary{plotmarks,shapes,arrows,positioning}


\usetheme{Berlin}
\usecolortheme{mit}

% Colors defined in mit-beamer.sty:
% mitred  
% mitgray 
% darkgray

%% Customizations:
% ------------------------------------------------------------------------------------------------------------ %
% Add Logo:
\pgfdeclareimage[height=0.5cm]{mit-logo}{mit-logo.pdf}
\logo{\vspace{-0.25cm}\pgfuseimage{mit-logo}\hspace*{0.025cm}}

% Show outline at beginning of each section
\AtBeginSection[]
{
  \begin{frame}<beamer>
    \frametitle{Outline}
    \tableofcontents[currentsection]
  \end{frame}
}


% \setbeamertemplate{blocks}[rounded]
\setbeamertemplate{blocks}[rounded][shadow=true]

% \setbeamertemplate{headline}[default]

\setbeamertemplate{navigation symbols}{}


% \beamerdefaultoverlayspecification{<+->}

% Set color for 'alert' text
\setbeamercolor{alerted text}{fg=blue}

% Modify some default font sizes
\setbeamerfont{itemize/enumerate body}{size=\normalfont}
\setbeamerfont{itemize/enumerate subbody}{size=\smaller, shape=\upshape}
\setbeamerfont{frametitle}{size=\large, series=\bfseries}


% \setbeamertemplate{bibliography entry title}{}
% \setbeamertemplate{bibliography entry location}{}
% \setbeamertemplate{bibliography entry note}{}
\setbeamertemplate{bibliography item}[text] 

% \setbeamertemplate{items}[ball]
% \setbeamertemplate{itemize subitem}[circle-symbol]
% \setbeamertemplate{background canvas}[vertical shading][bottom=mitgray!25,top=white]

% Use the shrink option to squeeze lots of text on a slide
% \frame[shrink]{
% …
% }

\colorlet{dark green}{green!50!black}

\graphicspath{{./Figures/FlowCase/}{./Figures/DiffusionCase}}

\newcommand{\packin}{\setlength\abovedisplayskip{2pt}\setlength\belowdisplayskip{2pt}}

\tikzstyle{refbox} = [shape = rectangle, fill = mitred, inner sep = 2pt, text=white, font=\footnotesize]

\newcommand{\numberInBox}[2][0.9]%
	{\scalebox{#1}{{\tikz \draw (0,0) node[refbox] {\makebox[\totalheight]{#2}};}}}

\newcommand{\enumref}[2][0.9] {\numberInBox[#1]{\ref{#2}}}

% Small arrow pointing down and hooking right
\newcommand{\drarrow}{\scalebox{1.5}{\reflectbox{\rotatebox[c]{180}{$\boldsymbol{\smash[b]{\Rsh}}$}}}}


\newenvironment{prettydescript}[1]
	{\begin{list}{}%
		{\renewcommand\makelabel[1]{\itshape\bfseries\color{mitred} ##1:\hfill}%
		\settowidth\labelwidth{\makelabel{#1}}%
		\setlength\leftmargin{\labelwidth}%
		\addtolength\leftmargin{\labelsep}}}%
	{\end{list}}


\newenvironment{customdescript}[1]
	{\begin{list}{}%
		{\renewcommand\makelabel[1]{\bfseries\color{mitred} ##1\hfill}%
		\settowidth\labelwidth{\makelabel{#1}}%
		\setlength\leftmargin{\labelwidth}%
		\addtolength\leftmargin{\labelsep}}}%
	{\end{list}}


\makeatletter
\newenvironment{customlist}[2]{
  \ifnum\@itemdepth >2\relax\@toodeep\else
      \advance\@itemdepth\@ne%
      \beamer@computepref\@itemdepth%
      \usebeamerfont{itemize/enumerate \beameritemnestingprefix body}%
      \usebeamercolor[fg]{itemize/enumerate \beameritemnestingprefix body}%
      \usebeamertemplate{itemize/enumerate \beameritemnestingprefix body begin}%
      \begin{list}
        {
            \usebeamertemplate{itemize \beameritemnestingprefix item}
        }
        { \leftmargin=#1 \itemindent=#2
            \def\makelabel##1{%
              {%  
                  \hss\llap{{%
                    \usebeamerfont*{itemize \beameritemnestingprefix item}%
                        \usebeamercolor[fg]{itemize \beameritemnestingprefix item}##1}}%
              }%  
            }%  
        }
  \fi
}
{
  \end{list}
  \usebeamertemplate{itemize/enumerate \beameritemnestingprefix body end}%
}
\makeatother


\newenvironment<>{varblock}[2][\textwidth]{%
  \setlength{\textwidth}{#1}
  \begin{actionenv}#3%
    \def\insertblocktitle{#2}%
    \par%
    \usebeamertemplate{block begin}}
  {\par%
    \usebeamertemplate{block end}%
  \end{actionenv}}


%% Notational commands:
\newcommand{\params}{\ensuremath{\xi}}
\newcommand{\vparms}{\ensuremath{\gvect{\params}}}




\renewcommand{\thefootnote}{\ensuremath{\fnsymbol{footnote}}}
\setcounter{footnote}{2}

\renewcommand{\thempfootnote}{\ensuremath{\fnsymbol{mpfootnote}}}

\newcommand{\newsubsection}[1]{\subsection{#1}\setcounter{subsection}{0}}



%% Title Page
\title[Stochastic Projection Method for UQ in CFD]{The Stochastic Projection Method for Uncertainty Quantification in Computational Fluid Dynamics}
\author[]{Dustin Langewisch \\[8pt] \and Prof. Jacopo Buongiorno (Advisor)}
\institute[\insertpagenumber]{}
\date{\today} 


% -----------------------------------------------------------------------------
\begin{document}
% -----------------------------------------------------------------------------

\frame{\titlepage}

\section[Outline]{}
% ---------------------------------------------------------------------------------------------------------- %
\begin{frame}{Outline}
  \tableofcontents
\end{frame}
% ---------------------------------------------------------------------------------------------------------- %

\section{Introduction}

% \newsubsection{Motivation}

\begin{frame}{Motivation}

	\setbeamerfont{itemize/enumerate body}{size=\smaller}
% 	\setbeamerfont{itemize/enumerate subbody}{size=\relsize{-0}}
	\setbeamerfont{itemize/enumerate subbody}{size=\smaller}

	\begin{customlist}{0pt}{0pt}
		\item The popularity of advanced simulation tools such as CFD (multiphase CFD, especially) is rapidly 
			  rising in the nuclear community.

		\begin{itemize}
			\item[$\blacktriangleright$] Advantages are numerous (higher resolution, fewer empirical dependencies, etc.),
						but \emph{very} expensive
		\end{itemize}

		\item Inevitably, the issue of {\color{red} uncertainty quantification} (UQ) for CFD must be addressed.

		\begin{itemize}
			\item[$\blacktriangleright$] this is extremely {\color{red} challenging due to the high cost} of CFD
		\end{itemize}

		\item Traditional UQ tools (\eg Monte Carlo) entail repeating expensive CFD simulations hundreds to thousands
			  of times.

		\begin{itemize}
			\item[$\blacktriangleright$] these methods are clearly prohibitive, and \alert{alternatives are needed}.
		\end{itemize}

		\item The stochastic projection method (SPM) has been shown to be very efficient for UQ with single-phase CFD

		\begin{itemize}
			\item[$\blacktriangleright$] \alert{this work seeks to extend the SPM to multiphase CFD}
		\end{itemize}

	\end{customlist}


\end{frame}



\newsubsection{Formulating the UQ Problem}

\begin{frame}

	\frametitle{The Stochastic Problem}

	\setbeamerfont{itemize/enumerate body}{size=\smaller}
	\setbeamerfont{block body}{size=\smaller}


	\begin{customlist}{0pt}{0pt}
		\item Given a set $\vparms = \{\params_1,\dots,\params_N\}$ of {\color{blue} uncertain inputs} to our 
			  model\footnote{\relsize{-1} At this level of abstraction, the model is arbitrary, but we will
			  be mostly\\\hspace{5ex}\! interested in the Navier-Stokes equations.}, the 
			  {\color{red} output} $u(\vect{x},t;\vparms)$ will be {\color{red} uncertain}.
	\end{customlist}

	\begin{prettydescript}{\smaller Goal}\smaller
		\item[Goal] Quantify this output uncertainty.
	\end{prettydescript}

	% ~~~~~~~~~~~~~~~~~~~~~~~~~~~~~~~~~~~~~~~~~~~~~~~~~~~~~~~~~~~~~~~~~~~~~~~~~~~~~~~~~~~~~~~~~~~~~~~~~~~~~~~~~~~~~~~~~~~~~~~~~~~~~ %
	%% Draw diagram:
	% ~~~~~~~~~~~~~~~~~~~~~~~~~~~~~~~~~~~~~~~~~~~~~~~~~~~~~~~~~~~~~~~~~~~~~~~~~~~~~~~~~~~~~~~~~~~~~~~~~~~~~~~~~~~~~~~~~~~~~~~~~~~~~ %
	\begin{center}
	\scalebox{0.65}{%
	\begin{tikzpicture}

		\definecolor{pdfout}{named}{red};
		\definecolor{pdfin}{named}{blue};

		\tikzstyle{labels}    = [minimum height = 1.0cm, text centered, anchor = south];


		\tikzstyle{connector} = [->, >=stealth, thick, shorten >=2pt];

		\tikzstyle{input  pdf} = [scale=0.75, color=pdfin , thick,  fill = pdfin!50];
		\tikzstyle{output pdf} = [scale=0.85, color=pdfout,  thick, fill = pdfout!50];

		\colorlet{bgcolor}{mitgray!20};

		\def\blockwidth{4cm};
		\def\blockheight{4cm};
		\def\elemheight{1cm};
		\def\elemwidth{0.75*\blockwidth};

		\def\elemvsep{0.5cm};
		\def\elemhsep{1.25cm};

		\def\labelheight{1.5*\elemheight + 1.5*\elemvsep};

 		\tikzstyle{ghost} = [rectangle, fill=none, draw=none, minimum width = \blockwidth, minimum height = \blockheight];		

		\tikzstyle{frame} = [rounded corners, fill=bgcolor, draw=black, double, very thick];


		\draw[frame] (+0.25cm,-0.75*\elemvsep) rectangle +(13.cm,5.85);
		\draw[draw=none,clip] (+0.25cm,-0.75*\elemvsep) rectangle +(13.cm,5.85);

		\begin{scope}

		\tikzstyle{element} = [rectangle, rounded corners, draw = black, very thick, minimum height=\elemheight, minimum width=2.25cm, text width=2.25cm];

		\node[element,anchor=south] (input 3) at (0.5*\blockwidth,0.0+0.0) 			{$\params_N$};

		\node[element,anchor=south, above = \elemvsep of input 3.north] (input 2) 	{$\params_i$};
		\node[element,anchor=south, above = \elemvsep of input 2.north] (input 1) 	{$\params_1$};

		\end{scope}

		\node[anchor=south] at (0.5*\blockwidth, \elemheight-0.2mm) {$\vdots$};
		\node[anchor=south] at (0.5*\blockwidth, 2.0*\elemheight + \elemvsep+0.1mm) {$\vdots$};


		\begin{scope}

		\tikzstyle{element} = [rectangle, rounded corners, draw = black, very thick, minimum height=\elemheight, minimum width=0.85*\blockwidth];

		\node[element,anchor=west, right = \elemhsep of input 2.east] (model) {$\vparms\mapsto u(\vect{x},t;\vparms)$};

		\end{scope}


		\begin{scope}

		\tikzstyle{element} = [rectangle, rounded corners, draw = black, very thick, minimum height=\elemheight, minimum width=0.90*\blockwidth, text width=3.25cm];

		\node[element,anchor=west, right = \elemhsep of model.east] (output) {$u(\vect{x},t;\vparms)$};

		\end{scope}

 		\node[labels, above = \labelheight of input 2.center] 	{{\bfseries\underline{Inputs}}};
 		\node[labels, above = \labelheight of model.center] 	{{\bfseries\underline{Model}}};
		\node[labels, above = \labelheight of output.center] 	{{\bfseries\underline{Output}}};


		%% Draw Connectors:
		% ---------------------------------------------------------------- %
		\draw[connector] (input 1) to [out=0,in=170] (model);
		\draw[connector] (input 2) to [out=0,in=180] (model);
		\draw[connector] (input 3) to [out=0,in=190] (model);
		\draw[connector] (model)   to [out=0,in=180] (output);

		\tikzstyle{every plot} = [samples=100, domain=-1:1];

		\draw[input pdf,  yshift=0.2cm, xshift=3cm]  plot (\x, {1.0*exp(-0.5*\x*\x*10)}) -- cycle;		
		\draw[input pdf,  yshift=2.25cm, xshift=3cm] plot (\x, {1.0*exp(-0.5*\x*\x*10)}) -- cycle;		
		\draw[input pdf,  yshift=4.3cm, xshift=3cm]  plot (\x, {1.0*exp(-0.5*\x*\x*10)}) -- cycle;		

		\draw[output pdf, yshift=1.95cm, xshift=13.25cm] plot ({0.75*({1+\x})},{8*({1+\x})*(exp(-3*({1+\x})))}) -- (1.5,0) -- cycle;	

	\end{tikzpicture}
	}
	\end{center}
	% ~~~~~~~~~~~~~~~~~~~~~~~~~~~~~~~~~~~~~~~~~~~~~~~~~~~~~~~~~~~~~~~~~~~~~~~~~~~~~~~~~~~~~~~~~~~~~~~~~~~~~~~~~~~~~~~~~~~~~~~~~~~~~ %
\end{frame}



\begin{frame}[t]{A Few Restrictions}
	\setbeamerfont{block body}{size=\smaller, shape=\upshape}
	\setbeamerfont{itemize/enumerate subbody}{size=\relsize{-0}, shape=\upshape}

	\begin{block}{Assumptions:}
	\begin{enumerate}

		\item  	The model is conditionally deterministic.
 		\item 	Uncertainties arising from numerical errors or model inadequacy are not considered.
		\item 	The inputs are reducible to a finite set of mutually independent RVs.

	\end{enumerate}
	\end{block} 

	\vspace{0.5em}

	\raisebox{-0.15ex}{\numberInBox{1}} and \raisebox{-0.15ex}{\numberInBox{2}}	have important (and useful) consequences:
	\begin{prettydescript}{Theoretical}
		\item[Practical]   the problem reduces to uncertainty propagation
		\item[Theoretical] $u$ and $\params$ reside in the same probability space 
	\end{prettydescript}
	\vspace{-0.5em}
	\begin{customlist}{7em}{0em}
		\item[$\blacktriangleright$] \alert{We know where to look for the solution!}
	\end{customlist}

	\raisebox{-0.15ex}{\numberInBox{3}} {\color{mitred} $\implies$} 
	\vspace{-1.5em}
	\begin{customlist}{7.25ex}{0pt}
		\item[] some care is needed to handle random fields (\eg uncertain initial/boundary conditions).			
	\end{customlist}

\end{frame}

\newsubsection{Methods}

\begin{frame}{Why SPM?}{a comparison with MCS}
	\relsize{-1}
% 	\setbeamerfont{block body}{size=\smaller, shape=\upshape}
	\setbeamerfont{itemize/enumerate body}{size=\relsize{-1}, shape=\upshape}

	\begin{block}{Monte Carlo Simulation}

		\begin{customlist}{4ex}{0pt}
		 \item  \begin{prettydescript}{Disadvantage}
				\item [Features] 
						{\color{mitred} $\bullet$} a non-intrusive (``black-box'') method \\
						{\color{mitred} $\bullet$} relies on random sampling and the law of large numbers
		        \end{prettydescript}
		 \item \begin{prettydescript}{Disadvantage}
					\item [Advantage]	{\color{mitred} $\bullet$} requires no modifications to existing deterministic codes
				\end{prettydescript}
		 \item	\begin{prettydescript}{Disadvantage}
					\item [Disadvantage] {\color{mitred} $\bullet$} slow $O\big(N^{-\frac{1}{2}}\big)$ convergence 
										 $\fimply$ computationally expensive
				\end{prettydescript}

		\end{customlist}

	\end{block}

% 	Recently, intrusive methods such as SPM have found favor for certain classes of long-running simulations

	\begin{block}{Stochastic Projection Method}

		\begin{customlist}{4ex}{0pt}
		 \item  \begin{prettydescript}{Disadvantage}
				\item [Features] 
						{\color{mitred} $\bullet$} an intrusive method \\
						{\color{mitred} $\bullet$} deterministic -- no random sampling
		        \end{prettydescript}
		 \item 	\begin{prettydescript}{Disadvantage}
				\item [Advantage] {\color{mitred} $\bullet$} potentially large cost reductions
				\end{prettydescript}
		 \item 	\begin{prettydescript}{Disadvantage}
				\item [Disadvantage]	{\color{mitred} $\bullet$} must ``open the box'' $\fimply$ new codes are needed
				\end{prettydescript}
		\end{customlist}

	\end{block}



% 	Traditionally, non-intrusive (``black-box'') sampling-based methods such as MCS\footnote[2]{Monte Carlo Simulation} 
% 	have been popular
% 
% 		\begin{customlist}{4ex}{0pt}
% 		 \item  \begin{prettydescript}{Disadvantage}
% 				\item [Features] relies on random sampling and the law of large numbers
% 		        \end{prettydescript}
% 		 \item \begin{prettydescript}{Disadvantage}
% 					\item [Advantage]	 requires no modifications to existing deterministic codes
% 				\end{prettydescript}
% 		 \item	\begin{prettydescript}{Disadvantage}
% 					\item [Disadvantage] slow $O\big(N^{-\frac{1}{2}}\big)$ convergence $\fimply$ computationally expensive
% 				\end{prettydescript}
% 
% 		\end{customlist}
% 
% 
% 	Recently, intrusive methods such as SPM have found favor for certain classes of long-running simulations
% 
% 		\begin{customlist}{4ex}{0pt}
% 		 \item  \begin{prettydescript}{Disadvantage}
% 				\item [Features] deterministic -- no random sampling 
% 		        \end{prettydescript}
% 		 \item 	\begin{prettydescript}{Disadvantage}
% 				\item [Advantage] potentially large efficiency gains $\fimply$ reduced cost
% 				\end{prettydescript}
% 		 \item 	\begin{prettydescript}{Disadvantage}
% 				\item [Disadvantage]	must ``open the box'' -- new codes are needed
% 				\end{prettydescript}
% 		\end{customlist}


\end{frame}



\section[Stochastic Projection Method]{The Stochastic Projection Method (SPM)}

\newsubsection{Overview}

% \begin{frame}[t]{The Procedure}
\begin{frame}[t]{The Stochastic Projection Method}

	\relsize{-1}

	\begin{block}{The Procedure:}
	\begin{enumerate}
		\item 	\alert{Identify relevant uncertain inputs} $\gvect{\xi} = \{\xi_i\}_{i=0}^{N}$\\[1em]

		\item 	\alert{Construct a stochastic basis} -- a set of RVs $\{\Psi_k(\gvect{\xi})\}_{k=0}^{\infty}$ 
				spanning the probability space, $\mathcal{S}$, for the problem

		\item 	\alert{Truncate} the basis: $\{\Psi_k(\gvect{\xi})\}_{k=0}^{\infty} \to \{\Psi_k(\gvect{\xi})\}_{k=0}^{P}$\\[1em]

		\item 	\alert{Discretize}: express all RVs in terms of this basis, \ie 
				$u = \sum_{k=0}^{P} u_k \Psi_k$\\[1em]

		\item 	\alert{Project} (Galerkin) the model onto the subspace $\mathcal{S}_P \subset \mathcal{S}$ spanned by 
					$\{\Psi_k\}_{k=0}^{P}$ 

		\item 	\alert{Solve} for the stochastic modes $u_k$
	\end{enumerate}
	\end{block} 
	\vspace{2pt}
	\raisebox{-.2ex}{\numberInBox{1}} is straightforward if all inputs are parameters, but in general we must 
		\emph{parameterize} a random field.

\end{frame}


\newsubsection{Parameterizing Random Fields}

\begin{frame}[t]{The Karhunen-Lo\`{e}ve Expansion}
	\relsize{-1}
	\begin{definition}\packin
		Let $\kappa(\vect{x};\omega)$ be a \second-order\footnote[2]{Finite variance} random field with mean $\bar{\kappa}(\vect{x})$ and covariance 
		function $C(\vect{x},\vect{y})$.	The KL-expansion of $\kappa$ is 
		\[
			\kappa(\vect{x};\omega) = \bar{\kappa}(\vect{x}) + \sum_{i=1}^{\infty} \sqrt{\lambda_i} \, \hat{\kappa}_i(\vect{x}) \, \xi_{i}, \vspace{-1.2em}
		\]
		where 
		\begin{itemize}\smaller
			\item $\lambda_{i}$ is the $i^{\textrm{th}}$ ordered (decreasing) eigenvalue of $C(\vect{x},\vect{y})$,
			\item $\hat{\kappa}_i(\vect{x})$ is the corresponding eigenfunction, and
			\item $\{\xi_i\}_{i=1}^{\infty}$ are a set of mutually uncorrelated, zero-mean RVs.
		\end{itemize}
	\end{definition}

	\begin{customlist}{0ex}{0pt}
		\item[$\therefore$\!] a random field is reducible to a \emph{countable} set of \emph{mutually uncorrelated} RVs.  
	\end{customlist}
	\vspace{-1em}
	\begin{customlist}{5ex}{0pt}
		\item[$\blacktriangleright$] In practice the series is truncated after $N_{KL}$ terms.
	\end{customlist}
\end{frame}



\newsubsection{Constructing a Stochastic Basis}

\begin{frame}{Generalized Polynomial Chaos (gPC) Expansion}

	\relsize{-2}
	\setbeamerfont{itemize/enumerate subbody}{size=\relsize{-0}, shape=\upshape}
		
	\begin{definition}\packin
		The gPC expansion of $u(\vect{x},t;\gvect{\xi})$ is given by
		\[
			u(\vect{x},t;\gvect{\xi}) =\displaystyle \sum_{i=0}^{\infty} \hat{u}_{i}(\vect{x},t) \Psi_{i}(\gvect{\xi}),
		\]
		where the $\Psi_i (\gvect{\xi})$ are multivariate polynomials orthogonal under product expectation:
		\[
			\langle \Psi_i, \Psi_j \rangle \equiv E(\Psi_i \Psi_j) = E(\Psi_i^2) \,\delta_{ij}.
		\]
	\end{definition}

	\begin{columns}[t]

	\column{0.5\textwidth}
	\vspace{-3em}	 
	\begin{table}
	\begin{center}
	\renewcommand{\arraystretch}{1.15}
	\begin{tabular}{@{\,}|c|c|@{\,}} 

	\multicolumn{2}{l}{{\color{mitred}\textbf{\!\!\!\!Univariate gPC Basis Functions*:}}} \\[1pt]

	\toprule

	\makebox[0.9in]{\textbf{Distribution -- $\xi_j$}} & 
	\makebox[0.9in]{\textbf{Basis -- $\psi_i(\xi_j)$}} \\

	\midrule \addlinespace[-1pt] \midrule %\addlinespace[-0.1pt]

	Gaussian 	& Hermite 		\\ 
	Uniform		& Legendre 		\\ 
	Beta		& Jacobi		\\ 
	Gamma		& Laguerre		\\ 
	Poisson		& Charlier		\\ 
	Binomial	& Krawtchouk	\\ 

	\bottomrule

	\end{tabular}
	\end{center}
	\end{table}


	\column{0.5\textwidth}

	\begin{customlist}{4pt}{0pt}\packin
		\item[*] Give optimal representation of RVs with given distribution, \eg:

		\item[--] If $u\sim\mathcal{N}(\mu,\sigma^2)$ and $\psi_i$ are Hermite polynomials,
		then \[ u(\xi) = \mu \psi_0 + \sigma^2 \psi_1 = \mu + \sigma^2 \xi \vspace{2pt}\] 

		\item[--] \alert{The PC expansion terminates!}



	\end{customlist}

	\end{columns}	

\end{frame}


\newsubsection{Demonstration}

\begin{frame}[t]{Steady Stochastic Diffusion Equation}
	\relsize{-2}
	\setbeamerfont{itemize/enumerate subbody}{size=\relsize{+0}}

	\only<1>{
		\begin{columns}[t]
		\column{0.5\textwidth}
		\vspace{-1em}
		\begin{block}{Governing Equations}
		\setlength\abovedisplayskip{1pt}\setlength\belowdisplayskip{2pt}
		\begin{align*}
			-\nabla\cdot\big(\kappa(\vect{x};\omega)\, \nabla u(\vect{x};\omega)\big) &= f(\vect{x}), 
			&\vect{x} &\in \mathcal{D} \\[1pt]
			u(\vect{x};\omega) &= 0, 
			&\vect{x} &\in \partial\mathcal{D}
		\end{align*}
		\end{block}
		\column{0.5\textwidth}

		Impose the following:
		\begin{enumerate}
			\item $f(\vect{x})$ is deterministic, as are the BCs
			\item $\kappa(\vect{x};\omega) = \bar{\kappa}(\vect{x}) + \sum_{i=1}^{N_{KL}} \sqrt{\lambda_i} \hat{\kappa}_i(\vect{x}) \xi_{i}$
		\end{enumerate}
		\vspace{-4pt}
		\begin{customlist}{3.0em}{0pt}
			\item[$\implies$] all uncertainty given by $\gvect{\xi}=\{\xi_i\}_{i=1}^{N_{KL}}$
		\end{customlist}
		\end{columns}

		\medskip
		\begin{customlist}{0pt}{0pt}

			\item From $\gvect{\xi}$, {\color{mitred} construct the gPC basis} $\{ \Psi_j(\gvect{\xi}) \}_{j=0}^{P}$

			\begin{itemize}
				\item[--] ordered such that $\Psi_0 = 1$, and $\Psi_i = \xi_i$ for $1 \le i \le N_{KL}$
				\[ 
% 					{\smash{\numberInBox[0.8]{2}}} 
					{\raisebox{-0.175ex}{\numberInBox[0.8]{2}}} 
					{\color{mitred}\quad\boldsymbol{\longrightarrow}\quad}
					\kappa(\vect{x};\omega) = \sum_{i=0}^{N_{KL}} \tilde{\kappa}_i(\vect{x}) \Psi_{i} 
						\quad\text{where}\quad
						\tilde{\kappa}_i(\vect{x}) \equiv
						\begin{cases}
							\bar{\kappa}(\vect{x}), & i = 0 \\
							\lambda_i^{1/2} \hat{\kappa}_i(\vect{x}), & 1 \le i \le N_{KL}
						\end{cases}				
				\]				
			\end{itemize}

			\item {\color{mitred} Expand the unknown}: 
				$\displaystyle u(\vect{x};\omega) = \sum_{j=0}^{P} \hat{u}_{j} (\vect{x}) \Psi_j(\gvect{\xi})$

			\item {\color{mitred} Discretize} (plug-in expansions):
				\(\displaystyle
					\sum_{j=0}^{P}\sum_{i=0}^{N_{KL}} \Psi_j \Psi_i 
					\left[
						-\nabla\cdot\big(\tilde{\kappa}_i(\vect{x})\,\nabla \hat{u}_j(\vect{x})\big)
					\right]
					= f(\vect{x})
				\)

		\end{customlist}
	}

	\only<2->{

		\begin{customlist}{0pt}{0pt}
			\item {\color{mitred} Project}:
			\(\displaystyle
				\sum_{j=0}^{P}
				\sum_{i=0}^{N_{KL}} 
				\only<2>{{\color{red} 
				\overbrace{
				\frac{\langle \Psi_j \Psi_i, \Psi_k \rangle}{\langle \Psi_k, \Psi_k \rangle}
				}^{\equiv C_{ijk}}
				}}
				\only<3>{
				\frac{\langle \Psi_j \Psi_i, \Psi_k \rangle}{\langle \Psi_k, \Psi_k \rangle}
				}
				\only<2>{{\color{dark green} 
				\overbrace{
				\Big(
					-\nabla\cdot\big(\tilde{\kappa}_i(\vect{x})\, \nabla \big) 
				\Big)
				}^{\equiv \mathcal{L}_i}
				}}
				\only<3>{
				\Big(
					-\nabla\cdot\big(\tilde{\kappa}_i(\vect{x})\, \nabla \big) 
				\Big)
				}
				\hat{u}_j(\vect{x})
				= \frac{\langle f(\vect{x}), \Psi_k \rangle}{\langle \Psi_k, \Psi_k \rangle} 
				= f(\vect{x}) 
				\only<2>{{\color{blue} 
				\overbrace{
				\frac{\langle \Psi_0, \Psi_k \rangle}{\langle \Psi_k, \Psi_k \rangle}
				}^{=\delta_{k0}}
				}}
				\only<3>{
				\frac{\langle \Psi_0, \Psi_k \rangle}{\langle \Psi_k, \Psi_k \rangle}
				}
			\)
		\end{customlist}
		\[
			{\color{mitred}\implies}\quad
			\sum_{j=0}^{P} 
			  \only<2>{
			  \Bigg[ 
				\sum_{i=0}^{N_{KL}} 
				{\color<2>{red} C_{ijk}}
				{\color<2>{dark green} \mathcal{L}_i } 
			  \Bigg] 			
			  }
			  \only<3>{{\color{red}
				\underbrace{
				\Bigg[ 
				  \sum_{i=0}^{N_{KL}} 
				  {\color<2>{red} C_{ijk}}
				  {\color<2>{dark green} \mathcal{L}_i } 
				\Bigg] 			
				}_{\equiv \small\mathcal{A}_{kj}}
			  }}
			  {\color<3>{blue} \hat{u}_j(\vect{x})} = 
			  {\color<3>{dark green} f(\vect{x})\, {\color<2>{blue} \delta_{k0}}}
			  \visible<3>{
			  \quad{\color{mitred}\implies}\quad
			  \tiny
				\left[
				{\color{red}
				\begin{matrix}
					\mathcal{A}_{00} & \mathcal{A}_{01} & \dots & \mathcal{A}_{0P} \\
					\mathcal{A}_{10} & \mathcal{A}_{11} & & \vdots \\
					\vdots 		& & \ddots	& \vdots \\
					\mathcal{A}_{P0} & \dots & \dots & \mathcal{A}_{PP}
				\end{matrix}
				}
				\right]
				\left(
				{\color{blue}
				\begin{matrix}
					\hat{u}_{0}(\vect{x}) \\
					\vdots \\
					\vdots \\
					\hat{u}_{P}(\vect{x}) 
				\end{matrix}
				}
				\right) = 
				\left(
				{\color{dark green}
				\begin{matrix}
					f(\vect{x})\mathstrut \\
					\phantom{\vdots}\!\raisebox{1.25ex}{0} \\
					\vdots \\
					\raisebox{0ex}{0}
				\end{matrix}
				}
				\right)	
			}
		\]		

	}

	\visible<3>{
		\begin{customlist}{4ex}{0ex}
			\item[$\blacktriangleright$] Problem size increases by a factor of $P+1$.
			\item[$\blacktriangleright$] When discretized (in space), each $\mathcal{A}_{kj}$ becomes a block matrix.
	
			\begin{customlist}{8ex}{0pt}
				\item[$\bullet$] Sparse block structure: most $\mathcal{A}_{kj} = 0$ by orthogonality
				\item[$\bullet$] Symmetry: $\mathcal{A}_{kj} = \mathcal{A}_{jk}$
			\end{customlist}

			\item[$\blacktriangleright$] Only the $N_{KL}+1$ deterministic system matrices, $\mathcal{L}_i$, 
					must be explicitly constructed.

		
		\end{customlist}
	}

\end{frame}



\begin{frame}[t]{Extension to the Transient Case: $\D{u}/Dt -\nabla\cdot\big(\kappa\, \nabla u \big) = f(\vect{x})$}

\setbeamerfont{itemize/enumerate subbody}{size=\relsize{+0}}

\vspace{-1em}
{\relsize{-2}
\[	\hspace{-1em}
	\Big\langle {\color{red} \D{{\color{black} u}}/Dt}, \Psi_k \Big\rangle = 
	{\color{red} \D{}/Dt} \big\langle u, \Psi_k \big\rangle = 
	{\color{red} \D{ {\color{black}\hat{u}_{k}} }/Dt}
	\quad\implies\quad\tiny
	\left[
	\begin{matrix}
		{\color{red} \D{}/Dt} + \mathcal{A}_{00} & \mathcal{A}_{01} & \dots & \mathcal{A}_{0P} \\
		\mathcal{A}_{10} & {\color{red} \D{}/Dt} + \mathcal{A}_{11} & & \vdots \\
		\vdots 		& & \ddots	& \vdots \\
		\mathcal{A}_{P0} & \dots & \dots & {\color{red} \D{}/Dt} + \mathcal{A}_{PP}
	\end{matrix}
	\right]
	\left(
	\begin{matrix}
		\hat{u}_{0}(\vect{x}) \\
		\vdots \\
		\vdots \\
		\hat{u}_{P}(\vect{x}) 
	\end{matrix}
	\right) = 
	\left(
	\begin{matrix}
		f(\vect{x})\mathstrut \\
		\phantom{\vdots}\!\raisebox{1.25ex}{0} \\
		\vdots \\
		\raisebox{0ex}{0}
	\end{matrix}
	\right)	
\]
}

% \vspace{-1em}
\smaller
\begin{customlist}{0pt}{0pt}
	\item The extension is trivial because the \alert{$\Psi_k$ are time-independent}
	\item This has an important practical implication:

	\begin{itemize}
		\item[$\blacktriangleright$] \alert{A sufficient basis for $t\approx 0$ may poorly
				represent $u$ for $t \gg 0$}.  
		\item[$\blacktriangleright$] The stochasticity can evolve out of
			$\mathcal{S}_P = \mathrm{span}\big( \{ \Psi_k \}_{k=0}^{P} \big)$
	\end{itemize}

	\item This limitation is well-known, and generalizations have been proposed, \ie 
		  Time-Dependent gPC (TDgPC) \cite{Gerritsma2010}. 

\end{customlist}



% \begin{itemize}
% 
% 	\item[$\therefore$]\relsize{-1}
% 		\(\displaystyle
% 			\left[
% 			\begin{matrix}
% 				{\color{red} \D{}/Dt} + \mathcal{A}_{00} & \mathcal{A}_{01} & \dots & \mathcal{A}_{0P} \\
% 				\mathcal{A}_{10} & {\color{red} \D{}/Dt} + \mathcal{A}_{11} & & \vdots \\
% 				\vdots 		& & \ddots	& \vdots \\
% 				\mathcal{A}_{P0} & \dots & \dots & {\color{red} \D{}/Dt} + \mathcal{A}_{PP}
% 			\end{matrix}
% 			\right]
% 			\left(
% 			\begin{matrix}
% 				\hat{u}_{0}(\vect{x}) \\
% 				\vdots \\
% 				\vdots \\
% 				\hat{u}_{P}(\vect{x}) 
% 			\end{matrix}
% 			\right) = 
% 			\left(
% 			\begin{matrix}
% 				f(\vect{x})\mathstrut \\
% 				\phantom{\vdots}\!\raisebox{1.25ex}{0} \\
% 				\vdots \\
% 				\raisebox{0ex}{0}
% 			\end{matrix}
% 			\right)	
% 		\)
% 		
% \end{itemize}

% This is true because the \alert{$\Psi_k$ are time-independent}, but this has an important practical implication:
% \begin{itemize}
% 	\item As
% \end{itemize}



\end{frame}


\begin{frame}[fragile]{Computational Tools}

	\relsize{-1}

	\begin{customlist}{0pt}{0pt}
		\item The SPM is currently being implemented in \texttt{C++} using the Trilinos library \cite{Heroux2005}%\cite{Trilinos-Overview}
		\item Trilinos consists of numerous tools for numerical computing, but its most important feature 
			is the \alert{Stokhos} package.
			\begin{customlist}{4ex}{0pt}\relsize{+1}
				\item[$\blacktriangleright$] Contains tools to facilitate development of intrusive UQ methods
				\item[$\blacktriangleright$] Some of the more useful offerings are listed below:
			\end{customlist}
	\end{customlist}

	\relsize{-1}

	\begin{center}\begin{tabular}{c l} 

		\toprule

		\multicolumn{2}{l}{\bfseries Some Useful Classes} \\

		\midrule
			$\{\psi_k(\xi_i)\}_{k=0}^{r}$		& 	\texttt{Stokhos::OneDOrthogPolyBasis}	\\[4pt]
			$\{\Psi_k(\gvect{\xi})\}_{k=0}^{P}$	&	\texttt{Stokhos::OrthogPolyBasis}		\\[4pt]
			$C_{ijk}$							&	\texttt{Stokhos::Sparse3Tensor}			\\
		\midrule

		\multicolumn{2}{l}{\bfseries Other Useful Tools} \\

		\midrule
			
			\multicolumn{2}{l}{$C_{ijk} = \big(\Psi_k\big)$\!\texttt{.computeTripleProductTensor()}} 			\\[4pt]
			\multicolumn{2}{l}{$\mathcal{A} = $ \texttt{KLMatrixFreeEpetraOp(}$C_{ijk},\, \Psi_k,\, \mathcal{L}_{i}$\texttt{)}} 	\\

		\bottomrule

	\end{tabular}\end{center}

\end{frame}


\begin{frame}[t]{Results -- Steady Problem}
	\relsize{-2}

	\only<1>{
 		\begin{block}{Solution Mean and Variance ($N_{KL} = 2$, $P = 20$)}
		\begin{center}
		\input{Figures/DiffusionCase/mean_var_surf32.tikz}
		\end{center}	
		\end{block}

		\begin{customlist}{0pt}{0pt}
		 \item[] $f(\vect{x}) = 2.5$, $\mathcal{\overline{D}} = [0,1]^2$, and $\kappa$ has mean $\bar{\kappa}= 0.5$ with 
				expontial correlation kernel ($\sigma^2 = 0.01$).
		\end{customlist}
	}

	\only<2>{
		\begin{columns}
		\column{0.8\textwidth}
		\begin{block}{Mean $\pm$ Standard Deviation ($y = 0.5$)}
		\begin{center}
		\input{Figures/DiffusionCase/mean_errbar.tikz}
		\end{center}
		\end{block}
		\end{columns}		
	}

	\only<3>{
		\begin{block}{$P$-Convergence}
		\begin{center}
		\input{Figures/DiffusionCase/mean_var_Pconvergence.tikz}
		\end{center}
		\end{block}
	}

\end{frame}



\begin{frame}{Results -- Transient Problem}

	\begin{align*}
		&{\color{mitred}\text{Initial Condition:}}&
		u(\vect{x},0) &= 0, \qquad \vect{x} \in \mathcal{D} = \big[-\tfrac{1}{2},+\tfrac{1}{2}\big]^2
		\\%[1pt]
		&{\color{mitred}\text{Boundary Condition:}}&
		u(\vect{x},t) &= 1, \qquad \vect{x} \in \partial\mathcal{D}	
		\\[1pt]
		&{\color{mitred}\text{No Source:}}&
		f(\vect{x},t) &= 0	
	\end{align*}
	\[ 
		\kappa(\vect{x};\omega) \text{ is unchanged from the steady problem} 
	\]

	\medskip

	\begin{center}
		{\color{red}$\big<$\texttt{animation}$\big>$}
	\end{center}

\end{frame}


\section[SPM for Incompressible Flow]{Application of SPM to Incompressible Flow Problems}


		\newlength{\arrowsize}  
		\pgfarrowsdeclare{smallertip}{smallertip}{  
		\setlength{\arrowsize}{0.5pt}  
		\addtolength{\arrowsize}{.5\pgflinewidth}  
		\pgfarrowsrightextend{0}  
		\pgfarrowsleftextend{-5\arrowsize}  
		}{  
		\setlength{\arrowsize}{0.5pt}  
		\addtolength{\arrowsize}{.5\pgflinewidth}  
		\pgfpathmoveto{\pgfpoint{-3\arrowsize}{1\arrowsize}}  
		\pgfpathlineto{\pgfpointorigin}  
		\pgfpathlineto{\pgfpoint{-3\arrowsize}{-1\arrowsize}}  
		\pgfusepathqstroke  
		}  

\subsection{The Stochastic Navier-Stokes Equations}
\setcounter{subsection}{0}

\begin{frame}{The Deterministic Flow Equations}

	\relsize{-1}

	{\color{mitred} {\bfseries Incompressible Navier-Stokes Equations} (\textit{constant fluid properties})\textbf{:}}
	\[
		\D{\vect{u}}/Dt + (\vect{u}\cdot\nabla)\vect{u} = -\nabla p + \nu \nabla^2 \vect{u} \qquad\text{with}\qquad
		\nabla\cdot\vect{u} = 0
	\]

	\medskip
	{\color{mitred} {\bfseries Fully-Explicit Pressure-Projection Method:}}
	\vspace{-0.5em}
	\begin{center}
		\scalebox{1.0}{\input{Figures/Deterministic_PressureProject_Schematic.tikz}}
	\end{center}

\end{frame}


\begin{frame}{The Stochastic Flow Equations}
	\relsize{-1}

	\begin{customlist}{0pt}{0pt}
	\item Expanding all flow quantities as $\alpha = \sum_{i=0}^{P} \hat{\alpha}_i \Psi_i$, where $\alpha = (\vect{u},p,\nu)$,
		and projecting yields:
	\end{customlist}
	\[
		\D{\hat{\vect{u}}_k}/Dt = -\nabla\hat{p}_k + 
			\sum_{i=0}^{P}\sum_{j=0}^{P} C_{ijk} 
			\Big[ \hat{\nu}_j \nabla^2 \hat{\vect{u}}_i - (\hat{\vect{u}}_j \cdot \nabla) \hat{\vect{u}}_i \Big]
	\]

	\medskip
	\visible<1>{{\color{mitred} {\bfseries Updated Algorithm:}}}
	\vspace{-2.5em}

	\begin{center}
		\scalebox{1.0}{\input{Figures/Stochastic_PressureProject_Schematic.tikz}}
	\end{center}

\end{frame}


\begin{frame}{Comments}

\begin{itemize}

	\item Solving the stochastic Navier-Stokes equations is a work-in-progress.

	\item Parallelizing the pressure solves is critical to the cost-effectiveness of SPM.

	\item Implementing Stokhos in parallel is a challenge due to limited documentation.

		\begin{itemize}
			\item[$\blacktriangleright$] the previous examples were all sufficiently small to be run in serial
		\end{itemize}

	\item Furthermore, a good deterministic solver is clearly a prerequisite.  

\end{itemize}
 

\end{frame}


\subsection{Deterministic Case Studies}\setcounter{subsection}{0}


\begin{frame}{Lid-Driven Cavity Flow}
	\framesubtitle<1>{Streamline plot for \textrm{Re} = 100}
	\framesubtitle<2>{Validation against data of Ghia et al.}
	\smaller
	\begin{center}
	\only<1>{
		\includegraphics[width=0.65\textwidth,keepaspectratio=true,clip=true,trim=0.5in 0.15in 0.25in 0.75in]{cavflow_streamlines}
	}
	\onslide<2>{
		\scalebox{0.975}{\input{Figures/FlowCase/CavityFlowValidation.tikz}}
	}
	\end{center}
\end{frame}

% \subsubsection{Channel Flow}

\begin{frame}{Channel Flow}
	\framesubtitle<1>{Streamline plot for uniform ($\boldsymbol{u_{in} = 10}$) inflow (\textrm{Re} = 100)}
	\framesubtitle<2>{Validation against Truchas code (LANL)}
	\begin{center}
	\only<1>{
		\includegraphics[width=0.99\textwidth,keepaspectratio=true,clip=true,trim=1in 0in 1.25in 0in]{chanflow_streamlines}		
	}
	\only<2>{
		\vspace{1cm}
		\scalebox{0.825}{\input{Figures/FlowCase/ChannelFlowValidation.tikz}}
	}
	\end{center}

\end{frame}




\section[Conclusions]{Conclusions \& Future Work}

\begin{frame}{Conclusions}

	\begin{itemize}

	\item UQ in conjunction with CFD is issue that inevitably will need to be addressed.

	\item The cost of simulation tools such as CFD precludes the use of sampling-based UQ methods.

	\item SPM has clear advantages over alternatives, particularly for incompressible flow problems due 
		  to decoupling of the stochastic pressure modes.

	\item Stokhos/Trilinos offer many utilities, but these tools come with a \emph{very} steep learning curve. 

	\item Preliminary results for the deterministic single-phase flow solver are promising

	\end{itemize}

\end{frame}

\begin{frame}{Future Work}

	The road ahead is long, and many tasks remain:

	\begin{enumerate}
		\item Finish the deterministic, single-phase flow solver:

		\begin{itemize}
		 \item[$\blacktriangleright$] mostly optimization work
		\end{itemize}
	
		\item Apply SPM to select single-phase flow examples

		\begin{itemize}
		 \item[$\blacktriangleright$] in progress
		\end{itemize}

		\item Develop a multiphase flow solver using the front-tracking method

		\begin{itemize}
		 \item[$\blacktriangleright$] this, itself, is a huge task and will likely take considerable time
		\end{itemize}

		\item Implement the SPM within the multiphase flow solver

		\begin{itemize}
		 \item[$\blacktriangleright$] with the lessons learned from applying SPM to single-phase flow problems,
				this \emph{should} be relatively straightforward -- in comparison to \raisebox{-0.15ex}{\numberInBox{3}}.
		\end{itemize}

	\end{enumerate}



\end{frame}


\section[]{}

\begin{frame}{That's All, Folks!}
	\begin{center}
		\Huge\bfseries Questions?
	\end{center}
\end{frame}


  
% \begin{frame}{References}
% \relsize{-2}
% \bibliographystyle{siam}
% \bibliography{/home/dustin/Documents/LaTeX/bibliography/mylib}
% \end{frame}





% ~~~~~~~~~~~~~~~~~~~~~~~~~~~~~~~~~~~~~~~~~~~~~~~~~~~~~~~~~~~~~~~~~~~~~~~~~~~~~~~~~~~~~~~~~~~~~~~~~~~~~~~~~~ %
%	APPENDIX
% ~~~~~~~~~~~~~~~~~~~~~~~~~~~~~~~~~~~~~~~~~~~~~~~~~~~~~~~~~~~~~~~~~~~~~~~~~~~~~~~~~~~~~~~~~~~~~~~~~~~~~~~~~~ %
\appendix

\begin{frame}{\;}
	\begin{center}
		\Huge\bfseries Extra Slides
	\end{center}	
\end{frame}



\begin{frame}{Additional Comments on the Karhunen-Lo\`{e}ve Expansion}
	\setbeamerfont{itemize/enumerate body}{size=\relsize{-1}}

	\begin{itemize}[<1->]

		\item 	The $\{\xi_i\}_{i=1}^{\infty}$ are uncorrelated, {\color{red} not necessarily independent}.
		\begin{itemize}
			\item[$\blacktriangleright$] They are \emph{approximately} independent (to $1^{\mathrm{st}}$-order).
		\end{itemize}


		\item 	The series must be truncated after a finite number $N_{KL}$ of terms.
		\begin{itemize}
			\item[$\blacktriangleright$] The truncated KL expansion is \alert{mean-square optimal} ($L^2$-optimal). 
		\end{itemize}

		\item 	The optimal $N_{KL}$ depends on spectrum (eigenvalues) of $C(\cdot\,,\cdot)$
	
 		\begin{enumerate}
			\item The further the correlation extends in space, the ``faster'' the eigenvalues decay $\implies$ smaller $N_{KL}$. 
			\vspace{1ex}
			\item Conversely, $\lambda_i \to 1$ $\forall i$ as $C(\vect{x},\vect{y}) \to \delta(\vect{x}-\vect{y})$ 
				  (vanishing correlation)

			\begin{itemize}\relsize{-1}
				\item[$\bullet$] Knowing $\kappa(\vect{x};\omega)$ says nothing of $\kappa(\vect{y};\omega)$ for $\vect{x}\ne\vect{y}$. 
			\end{itemize}

		\end{enumerate}
		\vspace{-1.1ex}
		\begin{customlist}{2.75em}{0pt}
			\item[\drarrow] analogous to the Fourier (and Heisenberg) uncertainty principle		 
		\end{customlist}
		
	\end{itemize}

\end{frame}


\begin{frame}{The gPC Expansion in $N$-dimensions}

	\setbeamerfont{itemize/enumerate body}{size=\smaller}
 
	\begin{itemize}

	\item Let $\xi$ and $\eta$ be mutually independent RVs with known distributions (need not be ID).

	\item Let $\psi_k(\xi)$ and $\varphi_k(\eta)$ denote the respective $k^{\mathrm{th}}$ gPC basis function.

	\item The product basis $\{\Psi_k(\xi,\eta)\}_{k=0}^{\infty}$ is given by:
	\[\begin{matrix}
		\Psi_0 = \psi_0 \varphi_0		&
		\Psi_2 = \psi_0 \varphi_1		& 
		\Psi_5 = \psi_0 \varphi_2		&\dots\\

		\Psi_1 = \psi_1 \varphi_0		&
		\Psi_4 = \psi_1 \varphi_1		&
		\\

		\Psi_3 = \psi_2 \varphi_0 &
		\\
		\vdots
	\end{matrix}\]
	\vspace{-1.5em}
	\item Repeating \emph{ad infinitum}, a basis for any dimension can be constructed.

	\item By construction, \alert{orthogonality is preserved}:
 	\[
 		E(\Psi_i \Psi_j) \equiv \langle \Psi_i, \Psi_j \rangle = E(\Psi_i^2) \delta_{ij}
 	\]

	\end{itemize}

\end{frame}

\begin{frame}{Staggered Grid Arrangement}
\begin{center}
	\scalebox{0.65}{\input{Figures/FlowCase/Staggered_Stencil.tikz}}
\end{center}
\end{frame}


% ========================================================================================================== %
\end{document}
