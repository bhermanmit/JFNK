%%%%%%%%%%%%   LaTeX Preamble %%%%%%%%%%%%%%

\documentclass{beamer}

\usefonttheme[onlymath]{serif}
\usepackage{hyperref}
\usepackage{pgfpages}
\usepackage{amsmath}
\usepackage{latexsym}
\usepackage{enumerate}
\usepackage{color}
\usepackage{ifthen}
\usepackage{animate}
\usepackage{tikz,pgfplots}
\usepackage{mycommands}
\pgfplotsset{compat=1.3}
\usetikzlibrary{plotmarks,shapes,arrows,positioning}
\usetheme{Berlin}
\usecolortheme{mit}

% Set Logo
\pgfdeclareimage[height=0.5cm]{mit-logo}{mit-logo.pdf}
\logo{\vspace{-0.25cm}\pgfuseimage{mit-logo}\hspace*{0.025cm}}

% Show outline at beginning of each section
\AtBeginSection[]
{
  \begin{frame}<beamer>
    \frametitle{Outline}
    \tableofcontents[currentsection]
  \end{frame}
}

\setbeamertemplate{blocks}[rounded][shadow=true]
\setbeamertemplate{navigation symbols}{} % the pdf navigation stuff
\newcommand{\packin}{\setlength\abovedisplayskip{2pt}\setlength\belowdisplayskip{2pt}}
\tikzstyle{refbox} = [shape = rectangle, fill = mitred, inner sep = 2pt, text=white, font=\footnotesize]
\newcommand{\numberInBox}[2][0.9]%
	{\scalebox{#1}{{\tikz \draw (0,0) node[refbox] {\makebox[\totalheight]{#2}};}}}
\newcommand{\enumref}[2][0.9] {\numberInBox[#1]{\ref{#2}}}

% Small arrow pointing down and hooking right
\newcommand{\drarrow}{\scalebox{1.5}{\reflectbox{\rotatebox[c]{180}{$\boldsymbol{\smash[b]{\Rsh}}$}}}}


\newenvironment{prettydescript}[1]
	{\begin{list}{}%
		{\renewcommand\makelabel[1]{\itshape\bfseries\color{mitred} ##1:\hfill}%
		\settowidth\labelwidth{\makelabel{#1}}%
		\setlength\leftmargin{\labelwidth}%
		\addtolength\leftmargin{\labelsep}}}%
	{\end{list}}


\newenvironment{customdescript}[1]
	{\begin{list}{}%
		{\renewcommand\makelabel[1]{\bfseries\color{mitred} ##1\hfill}%
		\settowidth\labelwidth{\makelabel{#1}}%
		\setlength\leftmargin{\labelwidth}%
		\addtolength\leftmargin{\labelsep}}}%
	{\end{list}}


\makeatletter
\newenvironment{customlist}[2]{
  \ifnum\@itemdepth >2\relax\@toodeep\else
      \advance\@itemdepth\@ne%
      \beamer@computepref\@itemdepth%
      \usebeamerfont{itemize/enumerate \beameritemnestingprefix body}%
      \usebeamercolor[fg]{itemize/enumerate \beameritemnestingprefix body}%
      \usebeamertemplate{itemize/enumerate \beameritemnestingprefix body begin}%
      \begin{list}
        {
            \usebeamertemplate{itemize \beameritemnestingprefix item}
        }
        { \leftmargin=#1 \itemindent=#2
            \def\makelabel##1{%
              {%  
                  \hss\llap{{%
                    \usebeamerfont*{itemize \beameritemnestingprefix item}%
                        \usebeamercolor[fg]{itemize \beameritemnestingprefix item}##1}}%
              }%  
            }%  
        }
  \fi
}
{
  \end{list}
  \usebeamertemplate{itemize/enumerate \beameritemnestingprefix body end}%
}
\makeatother
\newenvironment<>{varblock}[2][\textwidth]{%
  \setlength{\textwidth}{#1}
  \begin{actionenv}#3%
    \def\insertblocktitle{#2}%
    \par%
    \usebeamertemplate{block begin}}
  {\par%
    \usebeamertemplate{block end}%
  \end{actionenv}}
%% Notational commands:
\newcommand{\params}{\ensuremath{\xi}}
\newcommand{\vparms}{\ensuremath{\gvect{\params}}}
\renewcommand{\thefootnote}{\ensuremath{\fnsymbol{footnote}}}
\setcounter{footnote}{2}
\renewcommand{\thempfootnote}{\ensuremath{\fnsymbol{mpfootnote}}}
\newcommand{\newsubsection}[1]{\subsection{#1}\setcounter{subsection}{0}}

\graphicspath{{./tikz/}}

%% Title Page
\title[JFNK Methods for Coupled Nonlinear Systems]{Jacobian-Free Newton-Krylov (JFNK) Methods for Nonlinear Neutronics/Thermal-Hydrualic Equations}
\author[]{Bryan Herman}
\institute[\insertpagenumber]{}
\date{\today} 

\newcounter{angle}
\setcounter{angle}{0}

% -----------------------------------------------------------------------------
\begin{document}
% -----------------------------------------------------------------------------
\frame{\titlepage}

\section[Outline]{}
% -------------------------------------------------------------------------------------------------------------
\begin{frame}{Outline}
  \tableofcontents
\end{frame}
% -------------------------------------------------------------------------------------------------------------

%=======================================================
\begin{section}{Introduction}

%-------------------------------------------------------------------------------------------------------------
\begin{frame}{Motivation}

	\begin{customlist}{0pt}{0pt}

		\item Research is key

	\end{customlist}

\end{frame}
%-----------------------------------------------------------------------------------------------------------

\begin{subsection}{Formulating the Nonlinear Problem}

%-----------------------------------------------------------------------------------------------------------
\begin{frame}{Common Approach to Coupling - Operator Splitting}

	\begin{center}
	\scalebox{0.65}{%

		\begin{tikzpicture}

	% Define background color
	\colorlet{bgcolor}{mitgray!20};

	% Define Dimensions
	\def\blockwidth{2.75cm};
	\def\blockheight{1.5cm};
	\def\elemheight{1cm};
	\def\elemwidth{2cm};
	\def\elemvsep{0.5cm};
	\def\elemhsep{0.75cm};
	
	% Define Outer Frame
	\tikzstyle{frame} = [rounded corners, fill=bgcolor, draw=black, double, very thick];
	\draw[frame] (0.0,0.0) rectangle (4*\blockwidth+0.1cm,2*\blockheight+0.5cm);

	% Define Elements in Frame
	\begin{scope}
		\tikzstyle{element} = [rectangle, rounded corners, draw = black, very thick, minimum height=\elemheight, minimum width=\elemwidth];	
		\node[element] (time 1t) at (0.5*\elemwidth+0.5*\elemhsep,0.5*\elemheight+0.5*\elemvsep) {T/H};
		\node[element,above = \elemvsep of time 1t.north] (time 1n) {Neutronics};
		\node[element,right = \elemhsep of time 1t.east] (time 2t) {T/H};
		\node[element,above = \elemvsep of time 2t.north] (time 2n) {Neutronics};
		\node[element,right = \elemhsep of time 2t.east] (time 3t) {T/H};
		\node[element,above = \elemvsep of time 3t.north] (time 3n) {Neutronics};
		\node[element,right = \elemhsep of time 3t.east] (time 4t) {T/H};
		\node[element,above = \elemvsep of time 4t.north] (time 4n) {Neutronics};
	\end{scope}
		
	% Draw connectors between elements
	\tikzstyle{connector} = [->, >=stealth, thick, shorten >=2pt];
	\draw[connector] (time 1n) to [out=270,in=90] (time 1t);
	\draw[connector] (time 1t) to [out=0,in=180]  (time 2n);
	\draw[connector] (time 2n) to [out=270,in=90] (time 2t);
	\draw[connector] (time 2t) to [out=0,in=180]  (time 3n);
	\draw[connector] (time 3n) to [out=270,in=90] (time 3t);
	\draw[connector,dashed] (time 3t) to [out=0,in=180]  (time 4n);
	\draw[connector] (time 4n) to [out=270,in=90] (time 4t);
	
	% Put labels in
	\tikzstyle{labels}    = [minimum height = 1.0cm, text centered, anchor = south];
	\node[labels,above = 0.4cm of time 1n.center] {$t=0$};
	\node[labels,above = 0.4cm of time 2n.center] {$t=\Delta t$};
	\node[labels,above = 0.4cm of time 3n.center] {$t=2\Delta t$};
	\node[labels,above = 0.4cm of time 4n.center] {$t=n\Delta t$};

\end{tikzpicture}

	}
	\end{center}

\end{frame}
%-----------------------------------------------------------------------------------------------------------

\end{subsection}

\end{section}

%======================================================
\begin{section}{Governing Equations}

\end{section}
%======================================================

%======================================================
\begin{section}{Solvers}

\end{section}
%======================================================
\begin{section}{Results}

%------------------------------------------------------------------------------------------------------------
\begin{frame}{Flux Results - Steady Solution}

	\begin{block}{Flux Results}
		\begin{center}
			\begin{tikzpicture}[scale=0.5]

\def\figwidth{0.375\textwidth};
\def\figheight{0.325\textwidth};

\begin{axis}[%
name=plot1,
scale only axis,
width=4.52083in,
height=3.56562in,
xmin=0, xmax=400,
ymin=0, ymax=0.09,
view={-37.5}{20},
xlabel={Slab Length [cm]},
ylabel={Flux},
axis on top]
\addplot [
color=red,
solid,
line width=1.0pt
]
coordinates{
 (1,0.00320041)(2,0.00465222)(3,0.00610246)(4,0.00755063)(5,0.00899626)(6,0.0104389)(7,0.0118779)(8,0.013313)(9,0.0147436)(10,0.0161692)(11,0.0175894)(12,0.0190038)(13,0.0204117)(14,0.0218129)(15,0.0232068)(16,0.0245929)(17,0.0259709)(18,0.0273403)(19,0.0287007)(20,0.0300516)(21,0.0313926)(22,0.0327233)(23,0.0340433)(24,0.0353523)(25,0.0366497)(26,0.0379353)(27,0.0392086)(28,0.0404693)(29,0.041717)(30,0.0429514)(31,0.0441721)(32,0.0453789)(33,0.0465713)(34,0.0477491)(35,0.0489119)(36,0.0500596)(37,0.0511916)(38,0.0523079)(39,0.0534082)(40,0.0544921)(41,0.0555595)(42,0.05661)(43,0.0576436)(44,0.0586599)(45,0.0596587)(46,0.0606399)(47,0.0616033)(48,0.0625487)(49,0.063476)(50,0.0643849)(51,0.0652754)(52,0.0661473)(53,0.0670005)(54,0.0678349)(55,0.0686504)(56,0.0694468)(57,0.0702242)(58,0.0709824)(59,0.0717214)(60,0.0724411)(61,0.0731415)(62,0.0738226)(63,0.0744843)(64,0.0751266)(65,0.0757495)(66,0.0763531)(67,0.0769373)(68,0.0775022)(69,0.0780477)(70,0.078574)(71,0.0790811)(72,0.0795691)(73,0.0800379)(74,0.0804877)(75,0.0809187)(76,0.0813307)(77,0.081724)(78,0.0820987)(79,0.0824549)(80,0.0827926)(81,0.0831121)(82,0.0834134)(83,0.0836967)(84,0.0839621)(85,0.0842099)(86,0.08444)(87,0.0846527)(88,0.0848482)(89,0.0850266)(90,0.0851882)(91,0.085333)(92,0.0854613)(93,0.0855732)(94,0.0856691)(95,0.0857489)(96,0.0858131)(97,0.0858616)(98,0.0858949)(99,0.085913)(100,0.0859163)(101,0.0859048)(102,0.0858789)(103,0.0858387)(104,0.0857845)(105,0.0857165)(106,0.0856349)(107,0.08554)(108,0.085432)(109,0.0853111)(110,0.0851775)(111,0.0850316)(112,0.0848734)(113,0.0847034)(114,0.0845216)(115,0.0843283)(116,0.0841239)(117,0.0839084)(118,0.0836822)(119,0.0834454)(120,0.0831984)(121,0.0829413)(122,0.0826745)(123,0.082398)(124,0.0821122)(125,0.0818173)(126,0.0815135)(127,0.0812011)(128,0.0808802)(129,0.0805512)(130,0.0802142)(131,0.0798694)(132,0.0795172)(133,0.0791576)(134,0.078791)(135,0.0784176)(136,0.0780375)(137,0.0776509)(138,0.0772582)(139,0.0768595)(140,0.0764549)(141,0.0760448)(142,0.0756293)(143,0.0752087)(144,0.074783)(145,0.0743525)(146,0.0739175)(147,0.0734781)(148,0.0730344)(149,0.0725867)(150,0.0721352)(151,0.07168)(152,0.0712213)(153,0.0707593)(154,0.0702942)(155,0.0698261)(156,0.0693552)(157,0.0688817)(158,0.0684057)(159,0.0679274)(160,0.0674469)(161,0.0669644)(162,0.06648)(163,0.065994)(164,0.0655063)(165,0.0650173)(166,0.064527)(167,0.0640355)(168,0.063543)(169,0.0630496)(170,0.0625554)(171,0.0620606)(172,0.0615654)(173,0.0610697)(174,0.0605737)(175,0.0600776)(176,0.0595814)(177,0.0590853)(178,0.0585894)(179,0.0580937)(180,0.0575984)(181,0.0571036)(182,0.0566093)(183,0.0561157)(184,0.0556228)(185,0.0551308)(186,0.0546397)(187,0.0541496)(188,0.0536606)(189,0.0531727)(190,0.0526861)(191,0.0522008)(192,0.0517169)(193,0.0512344)(194,0.0507535)(195,0.0502741)(196,0.0497964)(197,0.0493204)(198,0.0488462)(199,0.0483738)(200,0.0479033)(201,0.0474347)(202,0.0469681)(203,0.0465036)(204,0.0460411)(205,0.0455807)(206,0.0451225)(207,0.0446665)(208,0.0442128)(209,0.0437613)(210,0.0433122)(211,0.0428654)(212,0.042421)(213,0.041979)(214,0.0415395)(215,0.0411024)(216,0.0406679)(217,0.0402358)(218,0.0398064)(219,0.0393795)(220,0.0389552)(221,0.0385335)(222,0.0381144)(223,0.037698)(224,0.0372842)(225,0.0368732)(226,0.0364648)(227,0.0360591)(228,0.0356561)(229,0.0352558)(230,0.0348583)(231,0.0344635)(232,0.0340715)(233,0.0336822)(234,0.0332956)(235,0.0329118)(236,0.0325308)(237,0.0321525)(238,0.031777)(239,0.0314042)(240,0.0310342)(241,0.0306669)(242,0.0303024)(243,0.0299407)(244,0.0295817)(245,0.0292254)(246,0.0288718)(247,0.028521)(248,0.0281729)(249,0.0278275)(250,0.0274848)(251,0.0271448)(252,0.0268074)(253,0.0264728)(254,0.0261408)(255,0.0258114)(256,0.0254847)(257,0.0251606)(258,0.0248391)(259,0.0245202)(260,0.0242038)(261,0.0238901)(262,0.0235788)(263,0.0232702)(264,0.022964)(265,0.0226603)(266,0.0223591)(267,0.0220604)(268,0.0217642)(269,0.0214703)(270,0.0211789)(271,0.0208899)(272,0.0206032)(273,0.0203189)(274,0.0200369)(275,0.0197573)(276,0.01948)(277,0.0192049)(278,0.0189321)(279,0.0186615)(280,0.0183931)(281,0.0181269)(282,0.0178629)(283,0.0176011)(284,0.0173414)(285,0.0170838)(286,0.0168282)(287,0.0165748)(288,0.0163233)(289,0.0160739)(290,0.0158265)(291,0.0155811)(292,0.0153376)(293,0.015096)(294,0.0148564)(295,0.0146186)(296,0.0143827)(297,0.0141486)(298,0.0139163)(299,0.0136858)(300,0.0134571)(301,0.0132301)(302,0.0130048)(303,0.0127812)(304,0.0125593)(305,0.0123391)(306,0.0121204)(307,0.0119034)(308,0.0116879)(309,0.011474)(310,0.0112615)(311,0.0110506)(312,0.0108412)(313,0.0106332)(314,0.0104267)(315,0.0102215)(316,0.0100177)(317,0.00981531)(318,0.00961422)(319,0.00941444)(320,0.00921593)(321,0.00901869)(322,0.00882267)(323,0.00862787)(324,0.00843425)(325,0.00824178)(326,0.00805046)(327,0.00786024)(328,0.00767112)(329,0.00748305)(330,0.00729602)(331,0.00711001)(332,0.00692499)(333,0.00674093)(334,0.00655782)(335,0.00637562)(336,0.00619431)(337,0.00601388)(338,0.00583429)(339,0.00565552)(340,0.00547754)(341,0.00530034)(342,0.00512389)(343,0.00494816)(344,0.00477313)(345,0.00459878)(346,0.00442508)(347,0.00425201)(348,0.00407954)(349,0.00390766)(350,0.00373633)(351,0.00356553)(352,0.00339524)(353,0.00322543)(354,0.00305609)(355,0.00288718)(356,0.00271869)(357,0.00255058)(358,0.00238284)(359,0.00221544)(360,0.00204836)(361,0.00188157)(362,0.00171505)(363,0.00154878)(364,0.00138273)(365,0.00121688)(366,0.00105121)(367,0.000885682)(368,0.000720285)(369,0.000554991)(370,0.000389777) 
};

\end{axis}

\end{tikzpicture}

		\end{center}	
	\end{block}

\end{frame}
%------------------------------------------------------------------------------------------------------------
\begin{frame}{Animation}

	\begin{center}

		\begin{animateinline}[loop, poster = first, controls, palindrome]{2}
			\begin{tikzpicture}[scale=0.5]

\def\figwidth{0.375\textwidth};
\def\figheight{0.325\textwidth};

\begin{axis}[%
name=plot1,
scale only axis,
width=4.52083in,
height=3.56562in,
xmin=0, xmax=400,
ymin=0, ymax=0.09,
view={-37.5}{20},
xlabel={Slab Length [cm]},
ylabel={Flux},
axis on top]
\addplot [
color=red,
solid,
line width=1.0pt
]
coordinates{
 (1,0.00320041)(2,0.00465222)(3,0.00610246)(4,0.00755063)(5,0.00899626)(6,0.0104389)(7,0.0118779)(8,0.013313)(9,0.0147436)(10,0.0161692)(11,0.0175894)(12,0.0190038)(13,0.0204117)(14,0.0218129)(15,0.0232068)(16,0.0245929)(17,0.0259709)(18,0.0273403)(19,0.0287007)(20,0.0300516)(21,0.0313926)(22,0.0327233)(23,0.0340433)(24,0.0353523)(25,0.0366497)(26,0.0379353)(27,0.0392086)(28,0.0404693)(29,0.041717)(30,0.0429514)(31,0.0441721)(32,0.0453789)(33,0.0465713)(34,0.0477491)(35,0.0489119)(36,0.0500596)(37,0.0511916)(38,0.0523079)(39,0.0534082)(40,0.0544921)(41,0.0555595)(42,0.05661)(43,0.0576436)(44,0.0586599)(45,0.0596587)(46,0.0606399)(47,0.0616033)(48,0.0625487)(49,0.063476)(50,0.0643849)(51,0.0652754)(52,0.0661473)(53,0.0670005)(54,0.0678349)(55,0.0686504)(56,0.0694468)(57,0.0702242)(58,0.0709824)(59,0.0717214)(60,0.0724411)(61,0.0731415)(62,0.0738226)(63,0.0744843)(64,0.0751266)(65,0.0757495)(66,0.0763531)(67,0.0769373)(68,0.0775022)(69,0.0780477)(70,0.078574)(71,0.0790811)(72,0.0795691)(73,0.0800379)(74,0.0804877)(75,0.0809187)(76,0.0813307)(77,0.081724)(78,0.0820987)(79,0.0824549)(80,0.0827926)(81,0.0831121)(82,0.0834134)(83,0.0836967)(84,0.0839621)(85,0.0842099)(86,0.08444)(87,0.0846527)(88,0.0848482)(89,0.0850266)(90,0.0851882)(91,0.085333)(92,0.0854613)(93,0.0855732)(94,0.0856691)(95,0.0857489)(96,0.0858131)(97,0.0858616)(98,0.0858949)(99,0.085913)(100,0.0859163)(101,0.0859048)(102,0.0858789)(103,0.0858387)(104,0.0857845)(105,0.0857165)(106,0.0856349)(107,0.08554)(108,0.085432)(109,0.0853111)(110,0.0851775)(111,0.0850316)(112,0.0848734)(113,0.0847034)(114,0.0845216)(115,0.0843283)(116,0.0841239)(117,0.0839084)(118,0.0836822)(119,0.0834454)(120,0.0831984)(121,0.0829413)(122,0.0826745)(123,0.082398)(124,0.0821122)(125,0.0818173)(126,0.0815135)(127,0.0812011)(128,0.0808802)(129,0.0805512)(130,0.0802142)(131,0.0798694)(132,0.0795172)(133,0.0791576)(134,0.078791)(135,0.0784176)(136,0.0780375)(137,0.0776509)(138,0.0772582)(139,0.0768595)(140,0.0764549)(141,0.0760448)(142,0.0756293)(143,0.0752087)(144,0.074783)(145,0.0743525)(146,0.0739175)(147,0.0734781)(148,0.0730344)(149,0.0725867)(150,0.0721352)(151,0.07168)(152,0.0712213)(153,0.0707593)(154,0.0702942)(155,0.0698261)(156,0.0693552)(157,0.0688817)(158,0.0684057)(159,0.0679274)(160,0.0674469)(161,0.0669644)(162,0.06648)(163,0.065994)(164,0.0655063)(165,0.0650173)(166,0.064527)(167,0.0640355)(168,0.063543)(169,0.0630496)(170,0.0625554)(171,0.0620606)(172,0.0615654)(173,0.0610697)(174,0.0605737)(175,0.0600776)(176,0.0595814)(177,0.0590853)(178,0.0585894)(179,0.0580937)(180,0.0575984)(181,0.0571036)(182,0.0566093)(183,0.0561157)(184,0.0556228)(185,0.0551308)(186,0.0546397)(187,0.0541496)(188,0.0536606)(189,0.0531727)(190,0.0526861)(191,0.0522008)(192,0.0517169)(193,0.0512344)(194,0.0507535)(195,0.0502741)(196,0.0497964)(197,0.0493204)(198,0.0488462)(199,0.0483738)(200,0.0479033)(201,0.0474347)(202,0.0469681)(203,0.0465036)(204,0.0460411)(205,0.0455807)(206,0.0451225)(207,0.0446665)(208,0.0442128)(209,0.0437613)(210,0.0433122)(211,0.0428654)(212,0.042421)(213,0.041979)(214,0.0415395)(215,0.0411024)(216,0.0406679)(217,0.0402358)(218,0.0398064)(219,0.0393795)(220,0.0389552)(221,0.0385335)(222,0.0381144)(223,0.037698)(224,0.0372842)(225,0.0368732)(226,0.0364648)(227,0.0360591)(228,0.0356561)(229,0.0352558)(230,0.0348583)(231,0.0344635)(232,0.0340715)(233,0.0336822)(234,0.0332956)(235,0.0329118)(236,0.0325308)(237,0.0321525)(238,0.031777)(239,0.0314042)(240,0.0310342)(241,0.0306669)(242,0.0303024)(243,0.0299407)(244,0.0295817)(245,0.0292254)(246,0.0288718)(247,0.028521)(248,0.0281729)(249,0.0278275)(250,0.0274848)(251,0.0271448)(252,0.0268074)(253,0.0264728)(254,0.0261408)(255,0.0258114)(256,0.0254847)(257,0.0251606)(258,0.0248391)(259,0.0245202)(260,0.0242038)(261,0.0238901)(262,0.0235788)(263,0.0232702)(264,0.022964)(265,0.0226603)(266,0.0223591)(267,0.0220604)(268,0.0217642)(269,0.0214703)(270,0.0211789)(271,0.0208899)(272,0.0206032)(273,0.0203189)(274,0.0200369)(275,0.0197573)(276,0.01948)(277,0.0192049)(278,0.0189321)(279,0.0186615)(280,0.0183931)(281,0.0181269)(282,0.0178629)(283,0.0176011)(284,0.0173414)(285,0.0170838)(286,0.0168282)(287,0.0165748)(288,0.0163233)(289,0.0160739)(290,0.0158265)(291,0.0155811)(292,0.0153376)(293,0.015096)(294,0.0148564)(295,0.0146186)(296,0.0143827)(297,0.0141486)(298,0.0139163)(299,0.0136858)(300,0.0134571)(301,0.0132301)(302,0.0130048)(303,0.0127812)(304,0.0125593)(305,0.0123391)(306,0.0121204)(307,0.0119034)(308,0.0116879)(309,0.011474)(310,0.0112615)(311,0.0110506)(312,0.0108412)(313,0.0106332)(314,0.0104267)(315,0.0102215)(316,0.0100177)(317,0.00981531)(318,0.00961422)(319,0.00941444)(320,0.00921593)(321,0.00901869)(322,0.00882267)(323,0.00862787)(324,0.00843425)(325,0.00824178)(326,0.00805046)(327,0.00786024)(328,0.00767112)(329,0.00748305)(330,0.00729602)(331,0.00711001)(332,0.00692499)(333,0.00674093)(334,0.00655782)(335,0.00637562)(336,0.00619431)(337,0.00601388)(338,0.00583429)(339,0.00565552)(340,0.00547754)(341,0.00530034)(342,0.00512389)(343,0.00494816)(344,0.00477313)(345,0.00459878)(346,0.00442508)(347,0.00425201)(348,0.00407954)(349,0.00390766)(350,0.00373633)(351,0.00356553)(352,0.00339524)(353,0.00322543)(354,0.00305609)(355,0.00288718)(356,0.00271869)(357,0.00255058)(358,0.00238284)(359,0.00221544)(360,0.00204836)(361,0.00188157)(362,0.00171505)(363,0.00154878)(364,0.00138273)(365,0.00121688)(366,0.00105121)(367,0.000885682)(368,0.000720285)(369,0.000554991)(370,0.000389777) 
};

\end{axis}

\end{tikzpicture}

			\newframe
			\begin{tikzpicture}[scale=0.5]

\begin{axis}[%
scale only axis,
width=4.52083in,
height=3.56562in,
xmin=0, xmax=400,
ymin=0.65, ymax=0.75,
xlabel={Slab Length [cm]},
ylabel={$\text{Coolant Density [g}/\text{cc]}$},
xmajorgrids,
ymajorgrids]
\addplot [
color=blue,
solid,
line width=1.0pt
]
coordinates{
 (1,0.74028)(2,0.740263)(3,0.740239)(4,0.74021)(5,0.740174)(6,0.740131)(7,0.740082)(8,0.740027)(9,0.739966)(10,0.739898)(11,0.739824)(12,0.739744)(13,0.739658)(14,0.739566)(15,0.739467)(16,0.739362)(17,0.739252)(18,0.739135)(19,0.739012)(20,0.738883)(21,0.738748)(22,0.738607)(23,0.73846)(24,0.738308)(25,0.73815)(26,0.737985)(27,0.737815)(28,0.73764)(29,0.737459)(30,0.737272)(31,0.737079)(32,0.736882)(33,0.736678)(34,0.73647)(35,0.736256)(36,0.736036)(37,0.735812)(38,0.735582)(39,0.735347)(40,0.735107)(41,0.734862)(42,0.734613)(43,0.734358)(44,0.734098)(45,0.733834)(46,0.733565)(47,0.733292)(48,0.733014)(49,0.732731)(50,0.732444)(51,0.732153)(52,0.731857)(53,0.731558)(54,0.731254)(55,0.730946)(56,0.730634)(57,0.730318)(58,0.729998)(59,0.729675)(60,0.729347)(61,0.729017)(62,0.728682)(63,0.728345)(64,0.728003)(65,0.727659)(66,0.727311)(67,0.72696)(68,0.726606)(69,0.726249)(70,0.725889)(71,0.725526)(72,0.725161)(73,0.724792)(74,0.724421)(75,0.724048)(76,0.723672)(77,0.723294)(78,0.722913)(79,0.72253)(80,0.722145)(81,0.721758)(82,0.721368)(83,0.720977)(84,0.720584)(85,0.720189)(86,0.719793)(87,0.719394)(88,0.718995)(89,0.718593)(90,0.718191)(91,0.717787)(92,0.717381)(93,0.716975)(94,0.716567)(95,0.716158)(96,0.715749)(97,0.715338)(98,0.714927)(99,0.714514)(100,0.714101)(101,0.713688)(102,0.713274)(103,0.712859)(104,0.712444)(105,0.712028)(106,0.711612)(107,0.711196)(108,0.71078)(109,0.710363)(110,0.709947)(111,0.70953)(112,0.709114)(113,0.708697)(114,0.708281)(115,0.707865)(116,0.707449)(117,0.707034)(118,0.706619)(119,0.706205)(120,0.705791)(121,0.705377)(122,0.704964)(123,0.704552)(124,0.704141)(125,0.70373)(126,0.70332)(127,0.702911)(128,0.702503)(129,0.702096)(130,0.70169)(131,0.701284)(132,0.70088)(133,0.700477)(134,0.700076)(135,0.699675)(136,0.699276)(137,0.698878)(138,0.698481)(139,0.698086)(140,0.697692)(141,0.697299)(142,0.696908)(143,0.696519)(144,0.696131)(145,0.695745)(146,0.69536)(147,0.694977)(148,0.694595)(149,0.694215)(150,0.693837)(151,0.693461)(152,0.693087)(153,0.692714)(154,0.692343)(155,0.691974)(156,0.691607)(157,0.691242)(158,0.690879)(159,0.690517)(160,0.690158)(161,0.689801)(162,0.689445)(163,0.689092)(164,0.688741)(165,0.688392)(166,0.688045)(167,0.6877)(168,0.687357)(169,0.687016)(170,0.686677)(171,0.686341)(172,0.686007)(173,0.685675)(174,0.685345)(175,0.685017)(176,0.684692)(177,0.684368)(178,0.684047)(179,0.683729)(180,0.683412)(181,0.683098)(182,0.682786)(183,0.682476)(184,0.682169)(185,0.681864)(186,0.681561)(187,0.68126)(188,0.680962)(189,0.680666)(190,0.680372)(191,0.680081)(192,0.679792)(193,0.679505)(194,0.679221)(195,0.678938)(196,0.678659)(197,0.678381)(198,0.678106)(199,0.677833)(200,0.677562)(201,0.677294)(202,0.677028)(203,0.676764)(204,0.676503)(205,0.676244)(206,0.675987)(207,0.675732)(208,0.67548)(209,0.67523)(210,0.674982)(211,0.674737)(212,0.674494)(213,0.674253)(214,0.674014)(215,0.673777)(216,0.673543)(217,0.673311)(218,0.673081)(219,0.672854)(220,0.672628)(221,0.672405)(222,0.672184)(223,0.671965)(224,0.671748)(225,0.671534)(226,0.671322)(227,0.671111)(228,0.670903)(229,0.670697)(230,0.670493)(231,0.670292)(232,0.670092)(233,0.669894)(234,0.669699)(235,0.669506)(236,0.669314)(237,0.669125)(238,0.668938)(239,0.668752)(240,0.668569)(241,0.668388)(242,0.668209)(243,0.668031)(244,0.667856)(245,0.667683)(246,0.667511)(247,0.667342)(248,0.667174)(249,0.667009)(250,0.666845)(251,0.666683)(252,0.666523)(253,0.666365)(254,0.666209)(255,0.666054)(256,0.665902)(257,0.665751)(258,0.665602)(259,0.665455)(260,0.66531)(261,0.665166)(262,0.665024)(263,0.664884)(264,0.664746)(265,0.664609)(266,0.664475)(267,0.664341)(268,0.66421)(269,0.66408)(270,0.663952)(271,0.663826)(272,0.663701)(273,0.663578)(274,0.663456)(275,0.663336)(276,0.663218)(277,0.663101)(278,0.662986)(279,0.662873)(280,0.662761)(281,0.662651)(282,0.662542)(283,0.662434)(284,0.662329)(285,0.662224)(286,0.662121)(287,0.66202)(288,0.66192)(289,0.661822)(290,0.661725)(291,0.66163)(292,0.661536)(293,0.661443)(294,0.661352)(295,0.661262)(296,0.661174)(297,0.661087)(298,0.661001)(299,0.660917)(300,0.660834)(301,0.660753)(302,0.660673)(303,0.660594)(304,0.660516)(305,0.66044)(306,0.660366)(307,0.660292)(308,0.66022)(309,0.660149)(310,0.660079)(311,0.660011)(312,0.659944)(313,0.659878)(314,0.659813)(315,0.65975)(316,0.659688)(317,0.659627)(318,0.659567)(319,0.659509)(320,0.659452)(321,0.659396)(322,0.659341)(323,0.659287)(324,0.659235)(325,0.659183)(326,0.659133)(327,0.659084)(328,0.659036)(329,0.65899)(330,0.658944)(331,0.6589)(332,0.658856)(333,0.658814)(334,0.658773)(335,0.658733)(336,0.658694)(337,0.658657)(338,0.65862)(339,0.658585)(340,0.65855)(341,0.658517)(342,0.658485)(343,0.658454)(344,0.658424)(345,0.658395)(346,0.658367)(347,0.65834)(348,0.658314)(349,0.65829)(350,0.658266)(351,0.658243)(352,0.658222)(353,0.658201)(354,0.658182)(355,0.658164)(356,0.658146)(357,0.65813)(358,0.658115)(359,0.6581)(360,0.658087)(361,0.658075)(362,0.658064)(363,0.658054)(364,0.658045)(365,0.658037)(366,0.65803)(367,0.658024)(368,0.658019)(369,0.658015)(370,0.658012) 
};

\end{axis}
\end{tikzpicture}

		\end{animateinline}	

	\end{center}

\end{frame}
%-----------------------------------------------------------------------------------------------------------
\end{section}
%======================================================
\begin{section}{Conclusions}
\end{section}
%======================================================
%======================================================
\end{document}


