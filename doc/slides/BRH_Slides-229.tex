%%%%%%%%%%%%   LaTeX Preamble %%%%%%%%%%%%%%

\documentclass{beamer}

% list all packages
\usefonttheme[onlymath]{serif}
\usepackage{comment}
\usepackage{hyperref}
\usepackage{pgfpages}
\usepackage{amsmath}
\usepackage{latexsym}
\usepackage{enumerate}
\usepackage{color}
\usepackage{ifthen}
\usepackage{animate}
\usepackage{tikz,pgfplots}
\usepackage{mycommands}
\pgfplotsset{compat=1.3}
\usetikzlibrary{plotmarks,shapes,arrows,positioning,snakes}
\usepackage[latin1]{inputenc}
\usepackage{xcolor}
\usepackage{tikz}
\usetikzlibrary{decorations.pathmorphing}
\usepackage{graphicx}
\usepackage{adjustbox}
\usepackage{scalefnt}

\def\TTiny{\fontsize{3pt}{3pt}\selectfont}
% slide theme
\usetheme{Berlin}
\usecolortheme{mit}

% Set Logo
\pgfdeclareimage[height=0.5cm]{mit-logo}{mit-logo.pdf}
\logo{\vspace{-0.25cm}\pgfuseimage{mit-logo}\hspace*{0.025cm}}

% Show outline at beginning of each section
\AtBeginSection[]
{
  \begin{frame}<beamer>
    \frametitle{Outline}
    \tableofcontents[currentsection]
  \end{frame}
}

% Include Custom environments
% \setbeamertemplate{blocks}[rounded]
\setbeamertemplate{blocks}[rounded][shadow=true]

% \setbeamertemplate{headline}[default]
\setbeamertemplate{navigation symbols}{}

% \beamerdefaultoverlayspecification{<+->}

% Set color for 'alert' text

\setbeamercolor{alerted text}{fg=blue}

% Modify some default font sizes
\setbeamerfont{itemize/enumerate body}{size=\normalfont}
\setbeamerfont{itemize/enumerate subbody}{size=\smaller, shape=\upshape}
\setbeamerfont{frametitle}{size=\large, series=\bfseries}


% \setbeamertemplate{bibliography entry title}{}
% \setbeamertemplate{bibliography entry location}{}
% \setbeamertemplate{bibliography entry note}{}
\setbeamertemplate{bibliography item}[text] 

% \setbeamertemplate{items}[ball]
% \setbeamertemplate{itemize subitem}[circle-symbol]
% \setbeamertemplate{background canvas}[vertical shading][bottom=mitgray!25,top=white]

% Use the shrink option to squeeze lots of text on a slide

% \frame[shrink]{

% …

% }

\colorlet{dark green}{green!50!black}

\newcommand{\packin}{\setlength\abovedisplayskip{2pt}\setlength\belowdisplayskip{2pt}}

\tikzstyle{refbox} = [shape = rectangle, fill = mitred, inner sep = 2pt, text=white, font=\footnotesize]

\newcommand{\numberInBox}[2][0.9]%
	{\scalebox{#1}{{\tikz \draw (0,0) node[refbox] {\makebox[\totalheight]{#2}};}}}

\newcommand{\enumref}[2][0.9] {\numberInBox[#1]{\ref{#2}}}


% Small arrow pointing down and hooking right
\newcommand{\drarrow}{\scalebox{1.5}{\reflectbox{\rotatebox[c]{180}{$\boldsymbol{\smash[b]{\Rsh}}$}}}}

\newenvironment{prettydescript}[1]
	{\begin{list}{}%
		{\renewcommand\makelabel[1]{\itshape\bfseries\color{mitred} ##1:\hfill}%
		\settowidth\labelwidth{\makelabel{#1}}%
		\setlength\leftmargin{\labelwidth}%
		\addtolength\leftmargin{\labelsep}}}%
	{\end{list}}

\newenvironment{customdescript}[1]
	{\begin{list}{}%
		{\renewcommand\makelabel[1]{\bfseries\color{mitred} ##1\hfill}%
		\settowidth\labelwidth{\makelabel{#1}}%
		\setlength\leftmargin{\labelwidth}%
		\addtolength\leftmargin{\labelsep}}}%
	{\end{list}}

\makeatletter

\newenvironment{customlist}[2]{
  \ifnum\@itemdepth >2\relax\@toodeep\else
      \advance\@itemdepth\@ne%
      \beamer@computepref\@itemdepth%
      \usebeamerfont{itemize/enumerate \beameritemnestingprefix body}%
      \usebeamercolor[fg]{itemize/enumerate \beameritemnestingprefix body}%
      \usebeamertemplate{itemize/enumerate \beameritemnestingprefix body begin}%
      \begin{list}
        {
            \usebeamertemplate{itemize \beameritemnestingprefix item}
        }
        { \leftmargin=#1 \itemindent=#2
            \def\makelabel##1{%
              {%  
                  \hss\llap{{%
                    \usebeamerfont*{itemize \beameritemnestingprefix item}%
                        \usebeamercolor[fg]{itemize \beameritemnestingprefix item}##1}}%
              }%  
            }%  
        }
  \fi
}
{
  \end{list}
  \usebeamertemplate{itemize/enumerate \beameritemnestingprefix body end}%
}
\makeatother

\newenvironment<>{varblock}[2][\textwidth]{%
  \setlength{\textwidth}{#1}
  \begin{actionenv}#3%
    \def\insertblocktitle{#2}%
    \par%
    \usebeamertemplate{block begin}}
  {\par%
    \usebeamertemplate{block end}%
  \end{actionenv}}

%% Notational commands:
\newcommand{\params}{\ensuremath{\xi}}
\newcommand{\vparms}{\ensuremath{\gvect{\params}}}
\renewcommand{\thefootnote}{\ensuremath{\fnsymbol{footnote}}}
\setcounter{footnote}{2}
\renewcommand{\thempfootnote}{\ensuremath{\fnsymbol{mpfootnote}}}
\newcommand{\newsubsection}[1]{\subsection{#1}\setcounter{subsection}{0}}

\newcommand{\proton}[1]{%
    \shade[ball color=red] (#1) circle (.25);\draw (#1) node{$+$};
}

%\neutron{xposition,yposition}
\newcommand{\neutron}[1]{%
    \shade[ball color=green] (#1) circle (.25);
}

%\electron{xwidth,ywidth,rotation angle}
\newcommand{\electron}[3]{%
    \draw[rotate = #3](0,0) ellipse (#1 and #2)[color=blue];
    \shade[ball color=yellow] (0,#2)[rotate=#3] circle (.1);
}

\newcommand{\nucleus}{%
    \neutron{0.1,0.3}
    \proton{0,0}
    \neutron{0.3,0.2}
    \proton{-0.2,0.1}
    \neutron{-0.1,0.3}
    \proton{0.2,-0.15}
    \neutron{-0.05,-0.12}
    \proton{0.17,0.21}
}


% Title Page
\title[JFNK Methods for Coupled Nonlinear Systems]{Jacobian-Free Newton-
Krylov (JFNK) Methods for Nonlinear Neutronics/Thermal-Hydrualic Equations}
\author[]{Bryan Herman}
\institute[\insertpagenumber]{}
\date{\today} 

% -----------------------------------------------------------------------------
\begin{document}
% -----------------------------------------------------------------------------

% Inset title page
\frame{\titlepage}

% Outline slide
\begin{frame}{Outline}
  \tableofcontents
\end{frame}

%==============================================================================
\begin{section}{Introduction}

%------------------------------------------------------------------------------
\begin{frame}{Motivation}

\begin{itemize}

	\item Eventually will be part of thesis work
	\vfill\item JFNK method not currently used in nuclear reactor 
		analysis
	\vfill\item Incorporates a lot of ideas from 2.29 class
	\vfill\item Coupled physics is fun!

\end{itemize}

\end{frame}
%------------------------------------------------------------------------------
\begin{frame}{Nuclear Reactor Systems}
	\scalebox{0.5}{% Pressurized Water Reactor
% Author: Gloria Faccanoni <http://www.science.unitn.it/~gloria/home.htm>
%
\begin{tikzpicture}[
        scale=0.7,
        annotline/.style = {stealth-},
        arrows1loop/.style={->,red},
        arrows2loop/.style={->,white},
        arrows3loop/.style={->,draw=gray},
    ]
\draw[draw=gray,double=gray!10,double distance=4pt]
    (12,12) to[out=135,in=45](0,12)--(0,0)--(22,0)--(22,12)--(12,12)--(12,0);
\node[text width=4cm, text centered,font=\small] at (6,13)
    {Containment\\structure};
% legend
\begin{scope}[yshift=-2cm]
    \filldraw[draw=red,fill=red!10] (1,0) rectangle ++(2,1);
    \node[text width=4cm, font=\small,right] at (3,0.5)
        {Pressurized water\\(primary loop)};
    \filldraw[draw=blue,bottom color=blue!40,top color=gray!30]
        (11,0) rectangle ++(2,1);
    \node[text width=4cm, font=\small,right] at (13,0.5)
        {Water and steam\\(secondary loop)};
    \filldraw[draw=blue,fill=blue!10] (21,0) rectangle ++(2,1);
    \node[text width=4cm, font=\small,right] at (23,0.5)
        {Water\\(cooling loop)};
\end{scope}
% 2nd loop --------------------------------------------------------------------
\begin{scope}[xshift=7.25cm,yshift=3cm]
    % vessel left
    \filldraw[draw=blue,bottom color=blue!40,top color=gray!30]
        (0,0) to[out=-20,in=200] (3.5,0) --
        (3.5,4.5) to[out=120,in=60] (0,4.5) -- (0,0);
    % vessel right
    \filldraw[draw=blue,bottom color=blue!40,top color=gray!30,xshift=7cm]
        (0,0) to[out=-20,in=200] (3.5,0) --
        (3.5,5) to[out=120,in=60] (0,5) -- (0,0);
    % circuits
    \draw[draw=blue,double=blue!40,double distance=4pt]
      (1.75,-0.3) -- ++(0,-1) -- ++(7,0) -- ++(0,1);
    \draw[draw=blue,double=gray!30,double distance=4pt]
        (1.75,5.38) -- ++(0,1) -- ++(4,0) -- ++(0,1) -- ++(3,0) -- ++(0,-1.5);
    % arrows
    \draw[arrows2loop] (3.5,-1.3) -- (3,-1.3);
    \draw[arrows2loop] (1.75,-0.9) -- (1.75,-0.4);
    \draw[arrows2loop] (4.5,6.38) -- (5,6.38);
    \draw[arrows2loop] (7,7.38) -- (7.5,7.38);
    \draw[arrows2loop] (8.75,6.4) -- (8.75,5.9);
    \draw[arrows2loop] (8.75,-0.4) -- (8.75,-0.9);
    %
    \foreach \x in {0.5,1,...,3}
        \draw[arrows2loop,xshift=7cm] (\x,3) -- (\x,2.5);
    % labels
    \draw[annotline] (2.5,-1.3) -- ++(3.5,1.3)
        node[text width=1cm,font=\small,above] {Liquid};
    \draw[annotline] (2.5,6.38) -- ++(3.5,-1.3)
        node[text width=1cm,font=\small,below] {Vapor};
    % pump
    \begin{scope}[xshift=160,yshift=-40]
        \filldraw[fill=blue!20,draw=blue] (0,0) circle (0.5cm);
        \node[below,font=\small] at (0,-0.5) {Pump};
        \filldraw[fill=blue!40,draw=blue,yshift=-0.5cm]
            (0,0) arc (240:180:0.4cm)  arc (200:280:0.4cm) ;
        \filldraw[fill=blue!40,draw=blue,yshift=+0.5cm,rotate=180]
            (0,0) arc (240:180:0.4cm)  arc (200:280:0.4cm) ;
        \filldraw[fill=blue!40,draw=blue,xshift=+0.5cm,rotate=90]
            (0,0) arc (240:180:0.4cm)  arc (200:280:0.4cm) ;
        \filldraw[fill=blue!40,draw=blue,xshift=-0.5cm,rotate=-90]
            (0,0) arc (240:180:0.4cm)  arc (200:280:0.4cm) ;
    \end{scope}
    % generator ...
    \draw[xshift=6.5cm,draw=gray,double=gray!10,double distance=4pt] 
        (3,4) -- ++(2,0);
    \filldraw[xshift=6.5cm,fill=orange!10,draw=orange] 
        (1.8,4) -- (3.0,3.3) -- (3.0,4.7) -- cycle;
    \filldraw[xshift=6.5cm,fill=orange!10,draw=orange] 
        (1.5,4) -- (2.5,3.4) -- (2.5,4.6) -- cycle;
    \filldraw[xshift=6.5cm,fill=orange!10,draw=orange] 
        (1.2,4) -- (2  ,3.5) -- (2  ,4.5) -- cycle;
    \filldraw[xshift=6.5cm,fill=orange!10,draw=orange] 
        (4.5,3.3) rectangle (7.3,4.7);
    %labels
    \node[text width=3cm, text centered,font=\small] at (1.75,4) 
        {Steam generator\\ (heat change)};
    \node[text width=2cm, text centered,font=\small] at (8.8,5) {Turbine};
    \node[text width=2cm, text centered,font=\small] at (12.4,4) {Generator};
    % transmission lines
    \node (aa) at (11.1,4.6) {};
    \node (bb) at (11.6,4.6) {};
    \node (cc) at (12.1,4.6) {};
    \node (dd) at (12.6,4.6) {};
    \node (ee) at (13.1,4.6) {};
    \node (ff) at (13.6,4.6) {};

\end{scope}
% 3 loop --------------------------------------------------------------------
\begin{scope}[xshift=23cm,yshift=1cm]
    % circuit
    \draw[draw=blue,double=blue!10,double distance=4pt]
      (1,2.5) -- ++(-8.5,0) -- ++(0,+1.5) -- ++(8.5,0);
    % arrows
    \draw[arrows3loop] (-5.5,2.5) -- (-6,2.5);
    \draw[arrows3loop] (-1.5,2.5) -- (-2,2.5);
    \draw[arrows3loop] (-6,4) -- (-5.5,4);
    \draw[arrows3loop] (-2,4) -- (-1.5,4);
    % tower
    \filldraw[draw=gray,fill=gray!20] (1,7) to[out=270,in=80]
                  (0,0) to[out=-20,in=200]
                  (6,0) to[out=100,in=270]
                  (5,7);
    \filldraw[draw=gray,fill=gray!40] (1,7) to[out=30,in=150]
                  (5,7) to[out=200,in=-20]
                  (1,7);
    % labels
    \node[text width=3cm, text centered,font=\small] at (3,3.5)
        {Cooling\\tower};
    \node[text width=2cm, text centered,font=\small] at (-3.5,1.5)
        {Cooling\\water};
    \node[text width=2cm, text centered,font=\small] at (-6,3.25)
        {Condenser};
    % pump
    \begin{scope}[xshift=-10,yshift=115]
        \filldraw[fill=purple!20,draw=purple] (0,0) circle (0.5cm);
        \node[below,font=\small] at (0,-0.5) {Pump};
        \filldraw[fill=purple!40,draw=purple,yshift=-0.5cm]
            (0,0) arc (240:180:0.4cm)  arc (200:280:0.4cm) ;
        \filldraw[fill=purple!40,draw=purple,yshift=+0.5cm,rotate=180]
            (0,0) arc (240:180:0.4cm)  arc (200:280:0.4cm) ;
        \filldraw[fill=purple!40,draw=purple,xshift=+0.5cm,rotate=90]
            (0,0) arc (240:180:0.4cm)  arc (200:280:0.4cm) ;
        \filldraw[fill=purple!40,draw=purple,xshift=-0.5cm,rotate=-90]
            (0,0) arc (240:180:0.4cm)  arc (200:280:0.4cm) ;
    \end{scope}
\end{scope}
%1 loop --------------------------------------------------------------------
\begin{scope}[xshift=2cm,yshift=4cm]
% Reactor vessel
\filldraw[draw=red,fill=red!10] (0,-0.5) to[out=-20,in=200]
              (3.5,-0.5) --
              (3.5,4.5) to[out=160,in=20]
              (0,4.5) --
              (0,-0.5);
% circuit
\draw[draw=red,double=red!10,double distance=4pt]
  (0.1,1) --  ++(-1,0) -- ++(0,-3) -- ++(5,0) -- ++(0,1.5) --
  ++(3,0) -- ++(0,2) -- ++(-3.7,0);
% Pressurizer
\draw[draw=red,double=red!10,double distance=4pt] (4.2,1.6) -- ++(0,0.8);
\filldraw[draw=green,bottom color=red!40,top color=green!20]
              (4,2.4) to[out=-20,in=200]
              (4.5,2.4) --
              (4.5,3.6) to[out=160,in=20]
              (3.9,3.6) --
              (3.9,2.4);
% arrows
\draw[arrows1loop] (-0.7,1) -- (-0.2,1);
\draw[arrows1loop] (-0.9,-0.5) -- (-0.9,0);
\draw[arrows1loop] (0.7,-2) -- (0.2,-2);
\draw[arrows1loop] (4.5,1.5) -- (5,1.5);
\draw[arrows1loop] (7.1,0.5) -- (7.1,0);
\draw[arrows1loop] (5.5,-0.5) -- (5,-0.5);

% pump
\begin{scope}[xshift=75,yshift=-55,fill=red!20,draw=red]
    \filldraw (0,0) circle (0.5cm);
    \node[below,font=\small] at (0,-0.5) {Pump};
    \filldraw[yshift=-0.5cm] (0,0) arc (240:180:0.4cm)  arc (200:280:0.4cm) ;
    \filldraw[yshift=+0.5cm,rotate=180]
        (0,0) arc (240:180:0.4cm)  arc (200:280:0.4cm) ;
    \filldraw[xshift=+0.5cm,rotate=90]
        (0,0) arc (240:180:0.4cm)  arc (200:280:0.4cm) ;
    \filldraw[xshift=-0.5cm,rotate=-90]
        (0,0) arc (240:180:0.4cm)  arc (200:280:0.4cm) ;
\end{scope}
% reactor core
\filldraw[fill=red!30,draw=red] (0.7,0) rectangle (2.8,2);

% control rods
\foreach \x in {1.0,1.5,2.0,2.5}
  \draw[draw=gray,double=gray!50,double distance=0.5pt] (\x,0.3) -- (\x,3.7);

%labels
\draw[annotline] (0.6,0.5) -- ++(-3.3,-1.5)
    node[text width=1cm,font=\small,left] {Reactor core};
\node[text width=2cm, text centered,font=\small] at (1.75,5.4) {Reactor vessel};
\draw[annotline] (0.9,2.8) -- ++(-3.3,1.5)
    node[text width=2cm, text centered,font=\small,left=-8pt] {Control\\rods};
\draw[annotline] (4.2,3.7) -- ++(0.5,1.5)
    node[text width=2cm, text centered,font=\small,above] {Pressurizer};
\draw[annotline] (3.9,1.5) -- ++(1.3,-0.6)
    node[text width=2.4cm, text centered,below=-2pt,font=\small]
        {Water coolant (\unit{330}{\degreecelsius})};
\draw[annotline] (-0.1,-2) -- ++(-0.3,-0.6)
    node[text width=2.4cm, text centered,below=-2pt,font=\small]
        {Water coolant (\unit{280}{\degreecelsius})};
\end{scope}
% clouds ----------------------------------
\begin{scope}[xshift=26cm,yshift=10cm, fill=blue!10, draw=blue,
    decoration={bumps,segment length=0.5cm}]
    \filldraw[yshift=-1.5cm,rotate=-25,decorate]
        (0,0) -- ++(-0.4,1.25)-- ++(-0.1,0.75)-- ++(0.2,0.5)-- ++(0.3,0.5)--
        ++(0.3,-0.5)-- ++(0.2,-0.5)-- ++(-0.1,-0.75)-- ++(-0.4,-1.25);
    \filldraw[xshift=0.5cm,yshift=-2cm,rotate=-30,decorate]
        (0,0) -- ++(-0.4,1.25)-- ++(-0.1,0.75)-- ++(0.2,0.5)-- ++(0.3,0.5)--
        ++(0.3,-0.5)-- ++(0.2,-0.5)-- ++(-0.1,-0.75)-- ++(-0.4,-1.25);
    \filldraw[xshift=-1.05cm,yshift=-2.15cm,rotate=-20,decorate]
        (0,0) -- ++(-0.4,1.25)-- ++(-0.1,0.75)-- ++(0.2,0.5)-- ++(0.3,0.5)--
        ++(0.3,-0.5)-- ++(0.2,-0.5)-- ++(-0.1,-0.75)-- ++(-0.4,-1.25);
    %labels
    \node[text width=1cm, text centered,font=\small] at (0.2,1.5) {Water vapor};
\end{scope}

% palo della luce
\begin{scope}[xscale=0.2,xshift=113cm,yshift=19cm,line width=1pt,brown]
    \draw (0,0) -- (-6,-6)
          (0,0) -- ( 6,-6)
          (-1,-1) -- ( 1,-1)
          (-1,-1) -- ( 2,-2)
          ( 1,-1) -- (-2,-2)
          (-2,-2) -- ( 2,-2)
          (-2,-2) -- ( 3,-3)
          ( 2,-2) -- (-3,-3)
          ( 3,-3) -- (-3,-3)
          (-3,-3) -- ( 4,-4)
          ( 3,-3) -- (-4,-4)
          ( 4,-4) -- (-4,-4)
          (-4,-4) -- ( 5,-5)
          ( 4,-4) -- (-5,-5)
          ( 5,-5) -- (-5,-5)
          (-6,-6) -- ( 0,-5.2)
          ( 6,-6) -- ( 0,-5.2);
    \draw (-1.5,-1.5) -- (-4,-1.5) -- (-1,-1)
          ( 1.5,-1.5) -- ( 4,-1.5) -- ( 1,-1);
    \path (-4,-1.4) node (a) {}
          ( 4,-1.4) node (b) {};
    \draw[line width=1pt,brown] (-3.5,-3.5) -- (-7.5,-3.5) -- (-3,-3)
                                ( 3.5,-3.5) -- ( 7.5,-3.5) -- ( 3,-3);
    \path (-7.5,-3.4) node (c) {}
          ( 7.5,-3.4) node (d) {}
          (-5.5,-3.4) node (e) {}
          ( 5.5,-3.4) node (f) {};
\end{scope}
% transmission lines
\draw[dashed,gray] (c) -- (aa)
                   (a) -- (bb)
                   (e) -- (cc)
                   (b) -- (dd)
                   (f) -- (ee)
                   (d) -- (ff);
\end{tikzpicture}}
	\vfill
\end{frame}
%------------------------------------------------------------------------------
\begin{frame}{Nuclear Feedback}
\begin{center}
	\scalebox{0.5}{\begin{tikzpicture}[scale=0.65]
	
	% Draw nucleus
	\tikzstyle{nucleon}=[shape = circle, shading=ball, minimum size=0.25cm];
	\node[nucleon, ball color = green] (neut) at (0.1,0.3) {};
	\node[nucleon, ball color = red] (prot) at (0.0,0.0) {};
	\node[nucleon, ball color = green] (neut2) at (0.3,0.2) {};
        \node[nucleon, ball color = red] (prot2) at (-0.2,0.1) {};
	\node[nucleon, ball color = green] (neut3) at (-0.1,0.3) {};
        \node[nucleon, ball color = red] (prot3) at (0.2,-0.15) {};
	\node[nucleon, ball color = green] (neut4) at (-0.05,-0.12) {};
        \node[nucleon, ball color = red] (prot4) at (0.17,0.21) {};
	\draw (0.17,0.21) node{$+$};

	% Draw electrons
	\draw[rotate = 80] (0,0) ellipse (1.5 and 0.75)[color=blue];
	\shade[ball color=yellow] (0,0.75)[rotate=80] circle (.1);
        \draw[rotate = 260] (0,0) ellipse (1.2 and 1.4)[color=blue];
        \shade[ball color=yellow] (0,1.4)[rotate=260] circle (.1);
        \draw[rotate = 30] (0,0) ellipse (4 and 2)[color=blue];
        \shade[ball color=yellow] (0,2)[rotate=30] circle (.1);
        \draw[rotate = 60] (0,0) ellipse (5 and 1)[color=blue];
        \shade[ball color=yellow] (0,1)[rotate=60] circle (.1);
        \draw[rotate = 150] (0,0) ellipse (5.5 and 1.5)[color=blue];
        \shade[ball color=yellow] (0,1.5)[rotate=150] circle (.1);
        \draw[rotate = 80] (0,0) ellipse (2.8 and 2.25)[color=blue];
        \shade[ball color=yellow] (0,2.25)[rotate=80] circle (.1);

	% Draw incident neutron
	\node[nucleon, ball color = green] (neut) at (-5.0,-5.0) {};
	\draw[snake=coil,segment aspect=0,very thick,segment length=20pt,line after snake=1mm,->,draw=orange] (-4.5cm,-4.5cm) -- (-0.4,-0.4);

        % Draw text
	\node[shape = rectangle,color=black,minimum width=8,minimum height=5,draw=black,above=5cm of prot.north] (N)   {Neutronics};
	\node[shape = rectangle,color=black,minimum width=8,minimum height=5,draw=black,below=5cm of prot.south] (T)   {Thermal Hydraulics};
	\node[shape = rectangle,color=black,minimum width=8,minimum height=5,draw=black,right=5cm of prot.east]  (NT)  {Neut/TH};
	\node[shape = rectangle,color=black,minimum width=8,minimum height=5,draw=black,left=5cm of prot.west]   (TN)  {TH/Neut};

	% Draw arrows
	\tikzstyle{connector} = [->,>=stealth, thick];
	\draw[connector] (N)  to [out=0,in=90]    (NT);
	\draw[connector] (NT) to [out=270,in=0]   (T);
	\draw[connector] (T)  to [out=180,in=270] (TN);
	\draw[connector] (TN) to [out=90,in=180]  (N);

\end{tikzpicture}
}
	\begin{itemize}
		\item Doppler Feedback
		\item Coolant Density Feedback
	\end{itemize}
\end{center}
\end{frame}
%------------------------------------------------------------------------------
\begin{frame}{1-D Slab Reactor Geometry}

\end{frame}
%------------------------------------------------------------------------------
\end{section}
%==============================================================================
\begin{section}{Governing Equations}
%------------------------------------------------------------------------------
\begin{frame}{Neutronics}

\end{frame}
%------------------------------------------------------------------------------
\begin{frame}{Neutronics to Thermal Hydraulics}

\end{frame}
%------------------------------------------------------------------------------
\begin{frame}{Thermal Hydraulics}

\end{frame}
%------------------------------------------------------------------------------
\begin{frame}{Thermal Hydraulics to Neutronics}

\end{frame}
%------------------------------------------------------------------------------
\end{section}
%==============================================================================
\begin{section}{Solvers}
%------------------------------------------------------------------------------
\begin{frame}{Newton's Method}

\end{frame}
%------------------------------------------------------------------------------
\begin{frame}{Krylov Subspace Methods}

\end{frame}
%------------------------------------------------------------------------------
\begin{frame}{Generalized Minimal RESidual Method}

\end{frame}
%------------------------------------------------------------------------------
\begin{frame}{Inexact Newton's Method}

\end{frame}
%------------------------------------------------------------------------------
\begin{frame}{Jacobian-Free Approximation}

\end{frame}
%------------------------------------------------------------------------------
\end{section}
%==============================================================================
\begin{section}{Results}
\end{section}
%==============================================================================
\begin{section}{Conclusions}
\end{section}
%==============================================================================
\end{document}
