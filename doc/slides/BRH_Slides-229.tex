%%%%%%%%%%%%   LaTeX Preamble %%%%%%%%%%%%%%

\documentclass{beamer}

\usefonttheme[onlymath]{serif}
\usepackage{comment}
\usepackage{hyperref}
\usepackage{pgfpages}
\usepackage{amsmath}
\usepackage{latexsym}
\usepackage{enumerate}
\usepackage{color}
\usepackage{ifthen}
\usepackage{animate}
\usepackage{tikz,pgfplots}
\usepackage{mycommands}
\pgfplotsset{compat=1.3}
\usetikzlibrary{plotmarks,shapes,arrows,positioning}
\usetheme{Berlin}
\usecolortheme{mit}

% Set Logo
\pgfdeclareimage[height=0.5cm]{mit-logo}{mit-logo.pdf}
\logo{\vspace{-0.25cm}\pgfuseimage{mit-logo}\hspace*{0.025cm}}

% Show outline at beginning of each section
\AtBeginSection[]
{
  \begin{frame}<beamer>
    \frametitle{Outline}
    \tableofcontents[currentsection]
  \end{frame}
}

\setbeamertemplate{blocks}[rounded][shadow=true]
\setbeamertemplate{navigation symbols}{} % the pdf navigation stuff
\newcommand{\packin}{\setlength\abovedisplayskip{2pt}\setlength\belowdisplayskip{2pt}}
\tikzstyle{refbox} = [shape = rectangle, fill = mitred, inner sep = 2pt, text=white, font=\footnotesize]
\newcommand{\numberInBox}[2][0.9]%
	{\scalebox{#1}{{\tikz \draw (0,0) node[refbox] {\makebox[\totalheight]{#2}};}}}
\newcommand{\enumref}[2][0.9] {\numberInBox[#1]{\ref{#2}}}

% Small arrow pointing down and hooking right
\newcommand{\drarrow}{\scalebox{1.5}{\reflectbox{\rotatebox[c]{180}{$\boldsymbol{\smash[b]{\Rsh}}$}}}}


\newenvironment{prettydescript}[1]
	{\begin{list}{}%
		{\renewcommand\makelabel[1]{\itshape\bfseries\color{mitred} ##1:\hfill}%
		\settowidth\labelwidth{\makelabel{#1}}%
		\setlength\leftmargin{\labelwidth}%
		\addtolength\leftmargin{\labelsep}}}%
	{\end{list}}


\newenvironment{customdescript}[1]
	{\begin{list}{}%
		{\renewcommand\makelabel[1]{\bfseries\color{mitred} ##1\hfill}%
		\settowidth\labelwidth{\makelabel{#1}}%
		\setlength\leftmargin{\labelwidth}%
		\addtolength\leftmargin{\labelsep}}}%
	{\end{list}}


\makeatletter
\newenvironment{customlist}[2]{
  \ifnum\@itemdepth >2\relax\@toodeep\else
      \advance\@itemdepth\@ne%
      \beamer@computepref\@itemdepth%
      \usebeamerfont{itemize/enumerate \beameritemnestingprefix body}%
      \usebeamercolor[fg]{itemize/enumerate \beameritemnestingprefix body}%
      \usebeamertemplate{itemize/enumerate \beameritemnestingprefix body begin}%
      \begin{list}
        {
            \usebeamertemplate{itemize \beameritemnestingprefix item}
        }
        { \leftmargin=#1 \itemindent=#2
            \def\makelabel##1{%
              {%  
                  \hss\llap{{%
                    \usebeamerfont*{itemize \beameritemnestingprefix item}%
                        \usebeamercolor[fg]{itemize \beameritemnestingprefix item}##1}}%
              }%  
            }%  
        }
  \fi
}
{
  \end{list}
  \usebeamertemplate{itemize/enumerate \beameritemnestingprefix body end}%
}
\makeatother
\newenvironment<>{varblock}[2][\textwidth]{%
  \setlength{\textwidth}{#1}
  \begin{actionenv}#3%
    \def\insertblocktitle{#2}%
    \par%
    \usebeamertemplate{block begin}}
  {\par%
    \usebeamertemplate{block end}%
  \end{actionenv}}
%% Notational commands:
\newcommand{\params}{\ensuremath{\xi}}
\newcommand{\vparms}{\ensuremath{\gvect{\params}}}
\renewcommand{\thefootnote}{\ensuremath{\fnsymbol{footnote}}}
\setcounter{footnote}{2}
\renewcommand{\thempfootnote}{\ensuremath{\fnsymbol{mpfootnote}}}
\newcommand{\newsubsection}[1]{\subsection{#1}\setcounter{subsection}{0}}

\graphicspath{{./tikz/}}

%% Title Page
\title[JFNK Methods for Coupled Nonlinear Systems]{Jacobian-Free Newton-Krylov (JFNK) Methods for Nonlinear Neutronics/Thermal-Hydrualic Equations}
\author[]{Bryan Herman}
\institute[\insertpagenumber]{}
\date{\today} 

\newcounter{angle}
\setcounter{angle}{0}

% -----------------------------------------------------------------------------
\begin{document}
% -----------------------------------------------------------------------------
\frame{\titlepage}

\section[Outline]{}
% -------------------------------------------------------------------------------------------------------------
\begin{frame}{Outline}
  \tableofcontents
\end{frame}
% -------------------------------------------------------------------------------------------------------------

%=======================================================
\begin{section}{Introduction}

%-------------------------------------------------------------------------------------------------------------
\begin{frame}{Motivation}

	\begin{customlist}{0pt}{0pt}

		\item Research is key

	\end{customlist}

\end{frame}
%-----------------------------------------------------------------------------------------------------------

\begin{subsection}{Formulating the Nonlinear Problem}

%-----------------------------------------------------------------------------------------------------------
\begin{frame}{Common Approach to Coupling - Operator Splitting}

	\begin{center}
	\scalebox{0.65}{%

		\begin{tikzpicture}

	% Define background color
	\colorlet{bgcolor}{mitgray!20};

	% Define Dimensions
	\def\blockwidth{2.75cm};
	\def\blockheight{1.5cm};
	\def\elemheight{1cm};
	\def\elemwidth{2cm};
	\def\elemvsep{0.5cm};
	\def\elemhsep{0.75cm};
	
	% Define Outer Frame
	\tikzstyle{frame} = [rounded corners, fill=bgcolor, draw=black, double, very thick];
	\draw[frame] (0.0,0.0) rectangle (4*\blockwidth+0.1cm,2*\blockheight+0.5cm);

	% Define Elements in Frame
	\begin{scope}
		\tikzstyle{element} = [rectangle, rounded corners, draw = black, very thick, minimum height=\elemheight, minimum width=\elemwidth];	
		\node[element] (time 1t) at (0.5*\elemwidth+0.5*\elemhsep,0.5*\elemheight+0.5*\elemvsep) {T/H};
		\node[element,above = \elemvsep of time 1t.north] (time 1n) {Neutronics};
		\node[element,right = \elemhsep of time 1t.east] (time 2t) {T/H};
		\node[element,above = \elemvsep of time 2t.north] (time 2n) {Neutronics};
		\node[element,right = \elemhsep of time 2t.east] (time 3t) {T/H};
		\node[element,above = \elemvsep of time 3t.north] (time 3n) {Neutronics};
		\node[element,right = \elemhsep of time 3t.east] (time 4t) {T/H};
		\node[element,above = \elemvsep of time 4t.north] (time 4n) {Neutronics};
	\end{scope}
		
	% Draw connectors between elements
	\tikzstyle{connector} = [->, >=stealth, thick, shorten >=2pt];
	\draw[connector] (time 1n) to [out=270,in=90] (time 1t);
	\draw[connector] (time 1t) to [out=0,in=180]  (time 2n);
	\draw[connector] (time 2n) to [out=270,in=90] (time 2t);
	\draw[connector] (time 2t) to [out=0,in=180]  (time 3n);
	\draw[connector] (time 3n) to [out=270,in=90] (time 3t);
	\draw[connector,dashed] (time 3t) to [out=0,in=180]  (time 4n);
	\draw[connector] (time 4n) to [out=270,in=90] (time 4t);
	
	% Put labels in
	\tikzstyle{labels}    = [minimum height = 1.0cm, text centered, anchor = south];
	\node[labels,above = 0.4cm of time 1n.center] {$t=0$};
	\node[labels,above = 0.4cm of time 2n.center] {$t=\Delta t$};
	\node[labels,above = 0.4cm of time 3n.center] {$t=2\Delta t$};
	\node[labels,above = 0.4cm of time 4n.center] {$t=n\Delta t$};

\end{tikzpicture}

	}
	\end{center}

\end{frame}
%-----------------------------------------------------------------------------------------------------------
\begin{frame}{PARCS Coupling}

	\begin{center}
	\scalebox{0.65}{%

		\begin{tikzpicture}

	% Define background color
	\colorlet{bgcolor}{mitred!50};
	\colorlet{boxcolor}{mitgray!20};

	% Define Dimensions
	\def\framewidth{14cm};
	\def\frameheight{10cm};
	
	% Define Outer Frame
	\tikzstyle{frame} = [rounded corners, fill=bgcolor, draw=black, double, very thick];
	\draw[frame] (0.0,0.0) rectangle (\framewidth,\frameheight);

	% Design Rectangles
	\begin{scope}
	
		\tikzstyle{calcs} = [rectangle, rounded corners, fill=boxcolor, draw = black, very thick, minimum height=5cm, minimum width=3cm, text width = 2.4 cm, text centered];	
		\node[calcs] (thcalc) at (2.0cm,3.5cm) {Thermal Hydraulics Solver};
		\node[calcs] (neutcalc) at (12.0cm,3.5cm) {Neutronics Solver};
		
		\tikzstyle{input} = [rectangle, fill=boxcolor, draw = black, very thick, minimum height=2cm, minimum width=3cm, text centered];	
		\node[input,above = 1cm of thcalc.north] (thinput) {T/H input};
		\node[input,above = 1cm of neutcalc.north] (neutinput) {Neutronics Input};

		\tikzstyle{interface} = [rectangle, rounded corners, fill=boxcolor, draw = black, very thick, minimum height=5cm, minimum width=1cm, text width=1ex, text centered];
		\node[interface] (thneutint) at (7cm,3.5cm) {I N T E R F A C E};
		
		\tikzstyle{map} = [rectangle, rounded corners, fill=boxcolor, draw = black, very thick, minimum height=2cm, minimum width=2cm, text width=1.0cm, text centered];
		\node[map,right = 0.4cm of thcalc.east](thmap) {T/H to Neut Map};
		\node[map,left  = 0.4cm of neutcalc.west](neutmap) {Neut to T/H Map};
		
	\end{scope}
	
	% Draw connectors
	\tikzstyle{connector} = [->, >=stealth, thick, shorten >=2pt];
	\draw[connector] (thinput) to [out=270,in=90] (thcalc);
	\draw[connector] (neutinput) to [out=270,in=90] (neutcalc);
	
	\tikzstyle{connectora} = [<->, >=stealth, thick, shorten >=2pt];
	\draw[connectora] (thcalc) to [out=0,in=180] (thmap);
	\draw[connectora] (neutcalc) to [out=180,in=0] (neutmap);
	\draw[connectora] (thmap) to [out=0,in=180] (thneutint);
	\draw[connectora] (neutmap) to [out=180,in=0] (thneutint);

\end{tikzpicture}

	}
	\end{center}

\end{frame}
%-----------------------------------------------------------------------------------------------------------

\end{subsection}

\end{section}

%======================================================
\begin{section}{Governing Equations}

\end{section}
%======================================================

%======================================================
\begin{section}{Solvers}

\end{section}
%======================================================
\begin{section}{Results}

%------------------------------------------------------------------------------------------------------------
\begin{frame}{Flux Results - Steady Solution}

	\begin{block}{Flux Results}
		\begin{center}
			\begin{tikzpicture}[scale=0.5]

\def\figwidth{0.375\textwidth};
\def\figheight{0.325\textwidth};

\begin{axis}[%
name=plot1,
scale only axis,
width=4.52083in,
height=3.56562in,
xmin=0, xmax=400,
ymin=0, ymax=0.09,
view={-37.5}{20},
xlabel={Slab Length [cm]},
ylabel={Flux},
axis on top]
\addplot [
color=red,
solid,
line width=1.0pt
]
coordinates{
 (1,0.00320041)(2,0.00465222)(3,0.00610246)(4,0.00755063)(5,0.00899626)(6,0.0104389)(7,0.0118779)(8,0.013313)(9,0.0147436)(10,0.0161692)(11,0.0175894)(12,0.0190038)(13,0.0204117)(14,0.0218129)(15,0.0232068)(16,0.0245929)(17,0.0259709)(18,0.0273403)(19,0.0287007)(20,0.0300516)(21,0.0313926)(22,0.0327233)(23,0.0340433)(24,0.0353523)(25,0.0366497)(26,0.0379353)(27,0.0392086)(28,0.0404693)(29,0.041717)(30,0.0429514)(31,0.0441721)(32,0.0453789)(33,0.0465713)(34,0.0477491)(35,0.0489119)(36,0.0500596)(37,0.0511916)(38,0.0523079)(39,0.0534082)(40,0.0544921)(41,0.0555595)(42,0.05661)(43,0.0576436)(44,0.0586599)(45,0.0596587)(46,0.0606399)(47,0.0616033)(48,0.0625487)(49,0.063476)(50,0.0643849)(51,0.0652754)(52,0.0661473)(53,0.0670005)(54,0.0678349)(55,0.0686504)(56,0.0694468)(57,0.0702242)(58,0.0709824)(59,0.0717214)(60,0.0724411)(61,0.0731415)(62,0.0738226)(63,0.0744843)(64,0.0751266)(65,0.0757495)(66,0.0763531)(67,0.0769373)(68,0.0775022)(69,0.0780477)(70,0.078574)(71,0.0790811)(72,0.0795691)(73,0.0800379)(74,0.0804877)(75,0.0809187)(76,0.0813307)(77,0.081724)(78,0.0820987)(79,0.0824549)(80,0.0827926)(81,0.0831121)(82,0.0834134)(83,0.0836967)(84,0.0839621)(85,0.0842099)(86,0.08444)(87,0.0846527)(88,0.0848482)(89,0.0850266)(90,0.0851882)(91,0.085333)(92,0.0854613)(93,0.0855732)(94,0.0856691)(95,0.0857489)(96,0.0858131)(97,0.0858616)(98,0.0858949)(99,0.085913)(100,0.0859163)(101,0.0859048)(102,0.0858789)(103,0.0858387)(104,0.0857845)(105,0.0857165)(106,0.0856349)(107,0.08554)(108,0.085432)(109,0.0853111)(110,0.0851775)(111,0.0850316)(112,0.0848734)(113,0.0847034)(114,0.0845216)(115,0.0843283)(116,0.0841239)(117,0.0839084)(118,0.0836822)(119,0.0834454)(120,0.0831984)(121,0.0829413)(122,0.0826745)(123,0.082398)(124,0.0821122)(125,0.0818173)(126,0.0815135)(127,0.0812011)(128,0.0808802)(129,0.0805512)(130,0.0802142)(131,0.0798694)(132,0.0795172)(133,0.0791576)(134,0.078791)(135,0.0784176)(136,0.0780375)(137,0.0776509)(138,0.0772582)(139,0.0768595)(140,0.0764549)(141,0.0760448)(142,0.0756293)(143,0.0752087)(144,0.074783)(145,0.0743525)(146,0.0739175)(147,0.0734781)(148,0.0730344)(149,0.0725867)(150,0.0721352)(151,0.07168)(152,0.0712213)(153,0.0707593)(154,0.0702942)(155,0.0698261)(156,0.0693552)(157,0.0688817)(158,0.0684057)(159,0.0679274)(160,0.0674469)(161,0.0669644)(162,0.06648)(163,0.065994)(164,0.0655063)(165,0.0650173)(166,0.064527)(167,0.0640355)(168,0.063543)(169,0.0630496)(170,0.0625554)(171,0.0620606)(172,0.0615654)(173,0.0610697)(174,0.0605737)(175,0.0600776)(176,0.0595814)(177,0.0590853)(178,0.0585894)(179,0.0580937)(180,0.0575984)(181,0.0571036)(182,0.0566093)(183,0.0561157)(184,0.0556228)(185,0.0551308)(186,0.0546397)(187,0.0541496)(188,0.0536606)(189,0.0531727)(190,0.0526861)(191,0.0522008)(192,0.0517169)(193,0.0512344)(194,0.0507535)(195,0.0502741)(196,0.0497964)(197,0.0493204)(198,0.0488462)(199,0.0483738)(200,0.0479033)(201,0.0474347)(202,0.0469681)(203,0.0465036)(204,0.0460411)(205,0.0455807)(206,0.0451225)(207,0.0446665)(208,0.0442128)(209,0.0437613)(210,0.0433122)(211,0.0428654)(212,0.042421)(213,0.041979)(214,0.0415395)(215,0.0411024)(216,0.0406679)(217,0.0402358)(218,0.0398064)(219,0.0393795)(220,0.0389552)(221,0.0385335)(222,0.0381144)(223,0.037698)(224,0.0372842)(225,0.0368732)(226,0.0364648)(227,0.0360591)(228,0.0356561)(229,0.0352558)(230,0.0348583)(231,0.0344635)(232,0.0340715)(233,0.0336822)(234,0.0332956)(235,0.0329118)(236,0.0325308)(237,0.0321525)(238,0.031777)(239,0.0314042)(240,0.0310342)(241,0.0306669)(242,0.0303024)(243,0.0299407)(244,0.0295817)(245,0.0292254)(246,0.0288718)(247,0.028521)(248,0.0281729)(249,0.0278275)(250,0.0274848)(251,0.0271448)(252,0.0268074)(253,0.0264728)(254,0.0261408)(255,0.0258114)(256,0.0254847)(257,0.0251606)(258,0.0248391)(259,0.0245202)(260,0.0242038)(261,0.0238901)(262,0.0235788)(263,0.0232702)(264,0.022964)(265,0.0226603)(266,0.0223591)(267,0.0220604)(268,0.0217642)(269,0.0214703)(270,0.0211789)(271,0.0208899)(272,0.0206032)(273,0.0203189)(274,0.0200369)(275,0.0197573)(276,0.01948)(277,0.0192049)(278,0.0189321)(279,0.0186615)(280,0.0183931)(281,0.0181269)(282,0.0178629)(283,0.0176011)(284,0.0173414)(285,0.0170838)(286,0.0168282)(287,0.0165748)(288,0.0163233)(289,0.0160739)(290,0.0158265)(291,0.0155811)(292,0.0153376)(293,0.015096)(294,0.0148564)(295,0.0146186)(296,0.0143827)(297,0.0141486)(298,0.0139163)(299,0.0136858)(300,0.0134571)(301,0.0132301)(302,0.0130048)(303,0.0127812)(304,0.0125593)(305,0.0123391)(306,0.0121204)(307,0.0119034)(308,0.0116879)(309,0.011474)(310,0.0112615)(311,0.0110506)(312,0.0108412)(313,0.0106332)(314,0.0104267)(315,0.0102215)(316,0.0100177)(317,0.00981531)(318,0.00961422)(319,0.00941444)(320,0.00921593)(321,0.00901869)(322,0.00882267)(323,0.00862787)(324,0.00843425)(325,0.00824178)(326,0.00805046)(327,0.00786024)(328,0.00767112)(329,0.00748305)(330,0.00729602)(331,0.00711001)(332,0.00692499)(333,0.00674093)(334,0.00655782)(335,0.00637562)(336,0.00619431)(337,0.00601388)(338,0.00583429)(339,0.00565552)(340,0.00547754)(341,0.00530034)(342,0.00512389)(343,0.00494816)(344,0.00477313)(345,0.00459878)(346,0.00442508)(347,0.00425201)(348,0.00407954)(349,0.00390766)(350,0.00373633)(351,0.00356553)(352,0.00339524)(353,0.00322543)(354,0.00305609)(355,0.00288718)(356,0.00271869)(357,0.00255058)(358,0.00238284)(359,0.00221544)(360,0.00204836)(361,0.00188157)(362,0.00171505)(363,0.00154878)(364,0.00138273)(365,0.00121688)(366,0.00105121)(367,0.000885682)(368,0.000720285)(369,0.000554991)(370,0.000389777) 
};

\end{axis}

\end{tikzpicture}

		\end{center}	
	\end{block}

\end{frame}
%------------------------------------------------------------------------------------------------------------
%\begin{comment}
\begin{frame}{Animation}

	\begin{center}

		\begin{animateinline}[loop, poster = first, controls]{2}
			\begin{tikzpicture}[scale=0.5]

\begin{axis}[%
scale only axis,
width=4.52083in,
height=3.56562in,
xmin=0, xmax=400,
ymin=0, ymax=400,
xlabel={Slab Length [cm]},
ylabel={Power [W]},
title={$\text{tstep }= 6$},
axis on top]
\addplot [
color=blue,
solid
]
coordinates{
 (1,13.9911)(2,20.3593)(3,26.7197)(4,33.0699)(5,39.4076)(6,45.7302)(7,52.0354)(8,58.3208)(9,64.584)(10,70.8228)(11,77.0348)(12,83.2176)(13,89.3691)(14,95.487)(15,101.569)(16,107.613)(17,113.617)(18,119.578)(19,125.496)(20,131.366)(21,137.189)(22,142.96)(23,148.68)(24,154.345)(25,159.955)(26,165.506)(27,170.998)(28,176.428)(29,181.795)(30,187.098)(31,192.335)(32,197.504)(33,202.604)(34,207.633)(35,212.591)(36,217.475)(37,222.285)(38,227.019)(39,231.677)(40,236.256)(41,240.757)(42,245.178)(43,249.519)(44,253.778)(45,257.954)(46,262.048)(47,266.057)(48,269.982)(49,273.822)(50,277.577)(51,281.245)(52,284.827)(53,288.323)(54,291.731)(55,295.052)(56,298.285)(57,301.43)(58,304.488)(59,307.457)(60,310.339)(61,313.133)(62,315.839)(63,318.457)(64,320.988)(65,323.431)(66,325.788)(67,328.057)(68,330.241)(69,332.338)(70,334.35)(71,336.276)(72,338.118)(73,339.876)(74,341.55)(75,343.141)(76,344.65)(77,346.078)(78,347.424)(79,348.69)(80,349.877)(81,350.985)(82,352.015)(83,352.967)(84,353.844)(85,354.645)(86,355.372)(87,356.025)(88,356.606)(89,357.114)(90,357.553)(91,357.921)(92,358.22)(93,358.452)(94,358.617)(95,358.716)(96,358.751)(97,358.722)(98,358.63)(99,358.477)(100,358.264)(101,357.991)(102,357.659)(103,357.271)(104,356.826)(105,356.327)(106,355.774)(107,355.167)(108,354.509)(109,353.801)(110,353.043)(111,352.236)(112,351.382)(113,350.482)(114,349.537)(115,348.547)(116,347.515)(117,346.441)(118,345.325)(119,344.17)(120,342.976)(121,341.744)(122,340.476)(123,339.172)(124,337.833)(125,336.461)(126,335.056)(127,333.619)(128,332.152)(129,330.654)(130,329.128)(131,327.575)(132,325.994)(133,324.387)(134,322.756)(135,321.1)(136,319.421)(137,317.719)(138,315.997)(139,314.253)(140,312.49)(141,310.707)(142,308.907)(143,307.089)(144,305.254)(145,303.404)(146,301.538)(147,299.659)(148,297.765)(149,295.859)(150,293.941)(151,292.012)(152,290.071)(153,288.121)(154,286.161)(155,284.192)(156,282.216)(157,280.232)(158,278.24)(159,276.243)(160,274.24)(161,272.232)(162,270.219)(163,268.202)(164,266.181)(165,264.158)(166,262.132)(167,260.104)(168,258.075)(169,256.044)(170,254.013)(171,251.982)(172,249.951)(173,247.92)(174,245.891)(175,243.864)(176,241.838)(177,239.815)(178,237.794)(179,235.776)(180,233.762)(181,231.752)(182,229.745)(183,227.743)(184,225.746)(185,223.753)(186,221.766)(187,219.785)(188,217.809)(189,215.839)(190,213.876)(191,211.919)(192,209.969)(193,208.027)(194,206.091)(195,204.163)(196,202.242)(197,200.33)(198,198.425)(199,196.529)(200,194.641)(201,192.762)(202,190.892)(203,189.03)(204,187.178)(205,185.334)(206,183.5)(207,181.676)(208,179.861)(209,178.055)(210,176.26)(211,174.474)(212,172.698)(213,170.933)(214,169.177)(215,167.432)(216,165.697)(217,163.972)(218,162.258)(219,160.554)(220,158.861)(221,157.178)(222,155.506)(223,153.845)(224,152.194)(225,150.554)(226,148.925)(227,147.307)(228,145.699)(229,144.102)(230,142.516)(231,140.941)(232,139.377)(233,137.823)(234,136.281)(235,134.749)(236,133.228)(237,131.718)(238,130.218)(239,128.73)(240,127.252)(241,125.785)(242,124.328)(243,122.883)(244,121.447)(245,120.023)(246,118.609)(247,117.206)(248,115.813)(249,114.43)(250,113.058)(251,111.697)(252,110.345)(253,109.004)(254,107.673)(255,106.352)(256,105.042)(257,103.741)(258,102.451)(259,101.17)(260,99.899)(261,98.6379)(262,97.3866)(263,96.1449)(264,94.9129)(265,93.6903)(266,92.4773)(267,91.2736)(268,90.0792)(269,88.8941)(270,87.7182)(271,86.5513)(272,85.3934)(273,84.2445)(274,83.1045)(275,81.9733)(276,80.8507)(277,79.7368)(278,78.6314)(279,77.5345)(280,76.4459)(281,75.3657)(282,74.2937)(283,73.2298)(284,72.1739)(285,71.126)(286,70.086)(287,69.0538)(288,68.0293)(289,67.0123)(290,66.003)(291,65.001)(292,64.0064)(293,63.0191)(294,62.039)(295,61.0659)(296,60.0999)(297,59.1407)(298,58.1884)(299,57.2428)(300,56.3038)(301,55.3714)(302,54.4454)(303,53.5259)(304,52.6125)(305,51.7054)(306,50.8044)(307,49.9094)(308,49.0203)(309,48.137)(310,47.2594)(311,46.3875)(312,45.5212)(313,44.6603)(314,43.8047)(315,42.9545)(316,42.1094)(317,41.2694)(318,40.4344)(319,39.6043)(320,38.779)(321,37.9585)(322,37.1426)(323,36.3312)(324,35.5243)(325,34.7217)(326,33.9234)(327,33.1293)(328,32.3393)(329,31.5532)(330,30.7711)(331,29.9928)(332,29.2182)(333,28.4472)(334,27.6798)(335,26.9158)(336,26.1551)(337,25.3978)(338,24.6436)(339,23.8925)(340,23.1443)(341,22.3991)(342,21.6567)(343,20.917)(344,20.18)(345,19.4455)(346,18.7134)(347,17.9837)(348,17.2563)(349,16.531)(350,15.8079)(351,15.0867)(352,14.3675)(353,13.65)(354,12.9343)(355,12.2203)(356,11.5078)(357,10.7967)(358,10.0871)(359,9.37866)(360,8.67146)(361,7.96537)(362,7.26029)(363,6.55613)(364,5.85281)(365,5.15023)(366,4.44831)(367,3.74696)(368,3.04609)(369,2.3456)(370,1.64541) 
};

\addplot [
color=red,
solid
]
coordinates{
 (1,13.9911)(2,20.3593)(3,26.7197)(4,33.0699)(5,39.4076)(6,45.7302)(7,52.0354)(8,58.3208)(9,64.584)(10,70.8228)(11,77.0348)(12,83.2176)(13,89.3691)(14,95.487)(15,101.569)(16,107.613)(17,113.617)(18,119.578)(19,125.496)(20,131.366)(21,137.189)(22,142.96)(23,148.68)(24,154.345)(25,159.955)(26,165.506)(27,170.998)(28,176.428)(29,181.795)(30,187.098)(31,192.335)(32,197.504)(33,202.604)(34,207.633)(35,212.591)(36,217.475)(37,222.285)(38,227.019)(39,231.677)(40,236.256)(41,240.757)(42,245.178)(43,249.519)(44,253.778)(45,257.954)(46,262.048)(47,266.057)(48,269.982)(49,273.822)(50,277.577)(51,281.245)(52,284.827)(53,288.323)(54,291.731)(55,295.052)(56,298.285)(57,301.43)(58,304.488)(59,307.457)(60,310.339)(61,313.133)(62,315.839)(63,318.457)(64,320.988)(65,323.431)(66,325.788)(67,328.057)(68,330.241)(69,332.338)(70,334.35)(71,336.276)(72,338.118)(73,339.876)(74,341.55)(75,343.141)(76,344.65)(77,346.078)(78,347.424)(79,348.69)(80,349.877)(81,350.985)(82,352.015)(83,352.967)(84,353.844)(85,354.645)(86,355.372)(87,356.025)(88,356.606)(89,357.114)(90,357.553)(91,357.921)(92,358.22)(93,358.452)(94,358.617)(95,358.716)(96,358.751)(97,358.722)(98,358.63)(99,358.477)(100,358.264)(101,357.991)(102,357.659)(103,357.271)(104,356.826)(105,356.327)(106,355.774)(107,355.167)(108,354.509)(109,353.801)(110,353.043)(111,352.236)(112,351.382)(113,350.482)(114,349.537)(115,348.547)(116,347.515)(117,346.441)(118,345.325)(119,344.17)(120,342.976)(121,341.744)(122,340.476)(123,339.172)(124,337.833)(125,336.461)(126,335.056)(127,333.619)(128,332.152)(129,330.654)(130,329.128)(131,327.575)(132,325.994)(133,324.387)(134,322.756)(135,321.1)(136,319.421)(137,317.719)(138,315.997)(139,314.253)(140,312.49)(141,310.707)(142,308.907)(143,307.089)(144,305.254)(145,303.404)(146,301.538)(147,299.659)(148,297.765)(149,295.859)(150,293.941)(151,292.012)(152,290.071)(153,288.121)(154,286.161)(155,284.192)(156,282.216)(157,280.232)(158,278.24)(159,276.243)(160,274.24)(161,272.232)(162,270.219)(163,268.202)(164,266.181)(165,264.158)(166,262.132)(167,260.104)(168,258.075)(169,256.044)(170,254.013)(171,251.982)(172,249.951)(173,247.92)(174,245.891)(175,243.864)(176,241.838)(177,239.815)(178,237.794)(179,235.776)(180,233.762)(181,231.752)(182,229.745)(183,227.743)(184,225.746)(185,223.753)(186,221.766)(187,219.785)(188,217.809)(189,215.839)(190,213.876)(191,211.919)(192,209.969)(193,208.027)(194,206.091)(195,204.163)(196,202.242)(197,200.33)(198,198.425)(199,196.529)(200,194.641)(201,192.762)(202,190.892)(203,189.03)(204,187.178)(205,185.334)(206,183.5)(207,181.676)(208,179.861)(209,178.055)(210,176.26)(211,174.474)(212,172.698)(213,170.933)(214,169.177)(215,167.432)(216,165.697)(217,163.972)(218,162.258)(219,160.554)(220,158.861)(221,157.178)(222,155.506)(223,153.845)(224,152.194)(225,150.554)(226,148.925)(227,147.307)(228,145.699)(229,144.102)(230,142.516)(231,140.941)(232,139.377)(233,137.823)(234,136.281)(235,134.749)(236,133.228)(237,131.718)(238,130.218)(239,128.73)(240,127.252)(241,125.785)(242,124.328)(243,122.883)(244,121.447)(245,120.023)(246,118.609)(247,117.206)(248,115.813)(249,114.43)(250,113.058)(251,111.697)(252,110.345)(253,109.004)(254,107.673)(255,106.352)(256,105.042)(257,103.741)(258,102.451)(259,101.17)(260,99.899)(261,98.6379)(262,97.3866)(263,96.1449)(264,94.9129)(265,93.6903)(266,92.4773)(267,91.2736)(268,90.0792)(269,88.8941)(270,87.7182)(271,86.5513)(272,85.3934)(273,84.2445)(274,83.1045)(275,81.9733)(276,80.8507)(277,79.7368)(278,78.6314)(279,77.5345)(280,76.4459)(281,75.3657)(282,74.2937)(283,73.2298)(284,72.1739)(285,71.126)(286,70.086)(287,69.0538)(288,68.0293)(289,67.0123)(290,66.003)(291,65.001)(292,64.0064)(293,63.0191)(294,62.039)(295,61.0659)(296,60.0999)(297,59.1407)(298,58.1884)(299,57.2428)(300,56.3038)(301,55.3714)(302,54.4454)(303,53.5259)(304,52.6125)(305,51.7054)(306,50.8044)(307,49.9094)(308,49.0203)(309,48.137)(310,47.2594)(311,46.3875)(312,45.5212)(313,44.6603)(314,43.8047)(315,42.9545)(316,42.1094)(317,41.2694)(318,40.4344)(319,39.6043)(320,38.779)(321,37.9585)(322,37.1426)(323,36.3312)(324,35.5243)(325,34.7217)(326,33.9234)(327,33.1293)(328,32.3393)(329,31.5532)(330,30.7711)(331,29.9928)(332,29.2182)(333,28.4472)(334,27.6798)(335,26.9158)(336,26.1551)(337,25.3978)(338,24.6436)(339,23.8925)(340,23.1443)(341,22.3991)(342,21.6567)(343,20.917)(344,20.18)(345,19.4455)(346,18.7134)(347,17.9837)(348,17.2563)(349,16.531)(350,15.8079)(351,15.0867)(352,14.3675)(353,13.65)(354,12.9343)(355,12.2203)(356,11.5078)(357,10.7967)(358,10.0871)(359,9.37866)(360,8.67146)(361,7.96537)(362,7.26029)(363,6.55613)(364,5.85281)(365,5.15023)(366,4.44831)(367,3.74696)(368,3.04609)(369,2.3456)(370,1.64541) 
};

\end{axis}
\end{tikzpicture}

			\newframe
			\begin{tikzpicture}[scale=0.5]

\begin{axis}[%
scale only axis,
width=4.52083in,
height=3.56562in,
xmin=0, xmax=400,
ymin=0, ymax=400,
xlabel={Slab Length [cm]},
ylabel={Power [W]},
title={$\text{tstep }= 12$},
axis on top]
\addplot [
color=blue,
solid
]
coordinates{
 (1,13.9911)(2,20.3593)(3,26.7197)(4,33.0699)(5,39.4076)(6,45.7302)(7,52.0354)(8,58.3208)(9,64.584)(10,70.8228)(11,77.0348)(12,83.2176)(13,89.3691)(14,95.487)(15,101.569)(16,107.613)(17,113.617)(18,119.578)(19,125.496)(20,131.366)(21,137.189)(22,142.96)(23,148.68)(24,154.345)(25,159.955)(26,165.506)(27,170.998)(28,176.428)(29,181.795)(30,187.098)(31,192.335)(32,197.504)(33,202.604)(34,207.633)(35,212.591)(36,217.475)(37,222.285)(38,227.019)(39,231.677)(40,236.256)(41,240.757)(42,245.178)(43,249.519)(44,253.778)(45,257.954)(46,262.048)(47,266.057)(48,269.982)(49,273.822)(50,277.577)(51,281.245)(52,284.827)(53,288.323)(54,291.731)(55,295.052)(56,298.285)(57,301.43)(58,304.488)(59,307.457)(60,310.339)(61,313.133)(62,315.839)(63,318.457)(64,320.988)(65,323.431)(66,325.788)(67,328.057)(68,330.241)(69,332.338)(70,334.35)(71,336.276)(72,338.118)(73,339.876)(74,341.55)(75,343.141)(76,344.65)(77,346.078)(78,347.424)(79,348.69)(80,349.877)(81,350.985)(82,352.015)(83,352.967)(84,353.844)(85,354.645)(86,355.372)(87,356.025)(88,356.606)(89,357.114)(90,357.553)(91,357.921)(92,358.22)(93,358.452)(94,358.617)(95,358.716)(96,358.751)(97,358.722)(98,358.63)(99,358.477)(100,358.264)(101,357.991)(102,357.659)(103,357.271)(104,356.826)(105,356.327)(106,355.774)(107,355.167)(108,354.509)(109,353.801)(110,353.043)(111,352.236)(112,351.382)(113,350.482)(114,349.537)(115,348.547)(116,347.515)(117,346.441)(118,345.325)(119,344.17)(120,342.976)(121,341.744)(122,340.476)(123,339.172)(124,337.833)(125,336.461)(126,335.056)(127,333.619)(128,332.152)(129,330.654)(130,329.128)(131,327.575)(132,325.994)(133,324.387)(134,322.756)(135,321.1)(136,319.421)(137,317.719)(138,315.997)(139,314.253)(140,312.49)(141,310.707)(142,308.907)(143,307.089)(144,305.254)(145,303.404)(146,301.538)(147,299.659)(148,297.765)(149,295.859)(150,293.941)(151,292.012)(152,290.071)(153,288.121)(154,286.161)(155,284.192)(156,282.216)(157,280.232)(158,278.24)(159,276.243)(160,274.24)(161,272.232)(162,270.219)(163,268.202)(164,266.181)(165,264.158)(166,262.132)(167,260.104)(168,258.075)(169,256.044)(170,254.013)(171,251.982)(172,249.951)(173,247.92)(174,245.891)(175,243.864)(176,241.838)(177,239.815)(178,237.794)(179,235.776)(180,233.762)(181,231.752)(182,229.745)(183,227.743)(184,225.746)(185,223.753)(186,221.766)(187,219.785)(188,217.809)(189,215.839)(190,213.876)(191,211.919)(192,209.969)(193,208.027)(194,206.091)(195,204.163)(196,202.242)(197,200.33)(198,198.425)(199,196.529)(200,194.641)(201,192.762)(202,190.892)(203,189.03)(204,187.178)(205,185.334)(206,183.5)(207,181.676)(208,179.861)(209,178.055)(210,176.26)(211,174.474)(212,172.698)(213,170.933)(214,169.177)(215,167.432)(216,165.697)(217,163.972)(218,162.258)(219,160.554)(220,158.861)(221,157.178)(222,155.506)(223,153.845)(224,152.194)(225,150.554)(226,148.925)(227,147.307)(228,145.699)(229,144.102)(230,142.516)(231,140.941)(232,139.377)(233,137.823)(234,136.281)(235,134.749)(236,133.228)(237,131.718)(238,130.218)(239,128.73)(240,127.252)(241,125.785)(242,124.328)(243,122.883)(244,121.447)(245,120.023)(246,118.609)(247,117.206)(248,115.813)(249,114.43)(250,113.058)(251,111.697)(252,110.345)(253,109.004)(254,107.673)(255,106.352)(256,105.042)(257,103.741)(258,102.451)(259,101.17)(260,99.899)(261,98.6379)(262,97.3866)(263,96.1449)(264,94.9129)(265,93.6903)(266,92.4773)(267,91.2736)(268,90.0792)(269,88.8941)(270,87.7182)(271,86.5513)(272,85.3934)(273,84.2445)(274,83.1045)(275,81.9733)(276,80.8507)(277,79.7368)(278,78.6314)(279,77.5345)(280,76.4459)(281,75.3657)(282,74.2937)(283,73.2298)(284,72.1739)(285,71.126)(286,70.086)(287,69.0538)(288,68.0293)(289,67.0123)(290,66.003)(291,65.001)(292,64.0064)(293,63.0191)(294,62.039)(295,61.0659)(296,60.0999)(297,59.1407)(298,58.1884)(299,57.2428)(300,56.3038)(301,55.3714)(302,54.4454)(303,53.5259)(304,52.6125)(305,51.7054)(306,50.8044)(307,49.9094)(308,49.0203)(309,48.137)(310,47.2594)(311,46.3875)(312,45.5212)(313,44.6603)(314,43.8047)(315,42.9545)(316,42.1094)(317,41.2694)(318,40.4344)(319,39.6043)(320,38.779)(321,37.9585)(322,37.1426)(323,36.3312)(324,35.5243)(325,34.7217)(326,33.9234)(327,33.1293)(328,32.3393)(329,31.5532)(330,30.7711)(331,29.9928)(332,29.2182)(333,28.4472)(334,27.6798)(335,26.9158)(336,26.1551)(337,25.3978)(338,24.6436)(339,23.8925)(340,23.1443)(341,22.3991)(342,21.6567)(343,20.917)(344,20.18)(345,19.4455)(346,18.7134)(347,17.9837)(348,17.2563)(349,16.531)(350,15.8079)(351,15.0867)(352,14.3675)(353,13.65)(354,12.9343)(355,12.2203)(356,11.5078)(357,10.7967)(358,10.0871)(359,9.37866)(360,8.67146)(361,7.96537)(362,7.26029)(363,6.55613)(364,5.85281)(365,5.15023)(366,4.44831)(367,3.74696)(368,3.04609)(369,2.3456)(370,1.64541) 
};

\addplot [
color=red,
solid
]
coordinates{
 (1,13.9911)(2,20.3593)(3,26.7197)(4,33.0699)(5,39.4076)(6,45.7302)(7,52.0354)(8,58.3208)(9,64.584)(10,70.8228)(11,77.0348)(12,83.2176)(13,89.3691)(14,95.487)(15,101.569)(16,107.613)(17,113.617)(18,119.578)(19,125.496)(20,131.366)(21,137.189)(22,142.96)(23,148.68)(24,154.345)(25,159.955)(26,165.506)(27,170.998)(28,176.428)(29,181.795)(30,187.098)(31,192.335)(32,197.504)(33,202.604)(34,207.633)(35,212.591)(36,217.475)(37,222.285)(38,227.019)(39,231.677)(40,236.256)(41,240.757)(42,245.178)(43,249.519)(44,253.778)(45,257.954)(46,262.048)(47,266.057)(48,269.982)(49,273.822)(50,277.577)(51,281.245)(52,284.827)(53,288.323)(54,291.731)(55,295.052)(56,298.285)(57,301.43)(58,304.488)(59,307.457)(60,310.339)(61,313.133)(62,315.839)(63,318.457)(64,320.988)(65,323.431)(66,325.788)(67,328.057)(68,330.241)(69,332.338)(70,334.35)(71,336.276)(72,338.118)(73,339.876)(74,341.55)(75,343.141)(76,344.65)(77,346.078)(78,347.424)(79,348.69)(80,349.877)(81,350.985)(82,352.015)(83,352.967)(84,353.844)(85,354.645)(86,355.372)(87,356.025)(88,356.606)(89,357.114)(90,357.553)(91,357.921)(92,358.22)(93,358.452)(94,358.617)(95,358.716)(96,358.751)(97,358.722)(98,358.63)(99,358.477)(100,358.264)(101,357.991)(102,357.659)(103,357.271)(104,356.826)(105,356.327)(106,355.774)(107,355.167)(108,354.509)(109,353.801)(110,353.043)(111,352.236)(112,351.382)(113,350.482)(114,349.537)(115,348.547)(116,347.515)(117,346.441)(118,345.325)(119,344.17)(120,342.976)(121,341.744)(122,340.476)(123,339.172)(124,337.833)(125,336.461)(126,335.056)(127,333.619)(128,332.152)(129,330.654)(130,329.128)(131,327.575)(132,325.994)(133,324.387)(134,322.756)(135,321.1)(136,319.421)(137,317.719)(138,315.997)(139,314.253)(140,312.49)(141,310.707)(142,308.907)(143,307.089)(144,305.254)(145,303.404)(146,301.538)(147,299.659)(148,297.765)(149,295.859)(150,293.941)(151,292.012)(152,290.071)(153,288.121)(154,286.161)(155,284.192)(156,282.216)(157,280.232)(158,278.24)(159,276.243)(160,274.24)(161,272.232)(162,270.219)(163,268.202)(164,266.181)(165,264.158)(166,262.132)(167,260.104)(168,258.075)(169,256.044)(170,254.013)(171,251.982)(172,249.951)(173,247.92)(174,245.891)(175,243.864)(176,241.838)(177,239.815)(178,237.794)(179,235.776)(180,233.762)(181,231.752)(182,229.745)(183,227.743)(184,225.746)(185,223.753)(186,221.766)(187,219.785)(188,217.809)(189,215.839)(190,213.876)(191,211.919)(192,209.969)(193,208.027)(194,206.091)(195,204.163)(196,202.242)(197,200.33)(198,198.425)(199,196.529)(200,194.641)(201,192.762)(202,190.892)(203,189.03)(204,187.178)(205,185.334)(206,183.5)(207,181.676)(208,179.861)(209,178.055)(210,176.26)(211,174.474)(212,172.698)(213,170.933)(214,169.177)(215,167.432)(216,165.697)(217,163.972)(218,162.258)(219,160.554)(220,158.861)(221,157.178)(222,155.506)(223,153.845)(224,152.194)(225,150.554)(226,148.925)(227,147.307)(228,145.699)(229,144.102)(230,142.516)(231,140.941)(232,139.377)(233,137.823)(234,136.281)(235,134.749)(236,133.228)(237,131.718)(238,130.218)(239,128.73)(240,127.252)(241,125.785)(242,124.328)(243,122.883)(244,121.447)(245,120.023)(246,118.609)(247,117.206)(248,115.813)(249,114.43)(250,113.058)(251,111.697)(252,110.345)(253,109.004)(254,107.673)(255,106.352)(256,105.042)(257,103.741)(258,102.451)(259,101.17)(260,99.899)(261,98.6379)(262,97.3866)(263,96.1449)(264,94.9129)(265,93.6903)(266,92.4773)(267,91.2736)(268,90.0792)(269,88.8941)(270,87.7182)(271,86.5513)(272,85.3934)(273,84.2445)(274,83.1045)(275,81.9733)(276,80.8507)(277,79.7368)(278,78.6314)(279,77.5345)(280,76.4459)(281,75.3657)(282,74.2937)(283,73.2298)(284,72.1739)(285,71.126)(286,70.086)(287,69.0538)(288,68.0293)(289,67.0123)(290,66.003)(291,65.001)(292,64.0064)(293,63.0191)(294,62.039)(295,61.0659)(296,60.0999)(297,59.1407)(298,58.1884)(299,57.2428)(300,56.3038)(301,55.3714)(302,54.4454)(303,53.5259)(304,52.6125)(305,51.7054)(306,50.8044)(307,49.9094)(308,49.0203)(309,48.137)(310,47.2594)(311,46.3875)(312,45.5212)(313,44.6603)(314,43.8047)(315,42.9545)(316,42.1094)(317,41.2694)(318,40.4344)(319,39.6043)(320,38.779)(321,37.9585)(322,37.1426)(323,36.3312)(324,35.5243)(325,34.7217)(326,33.9234)(327,33.1293)(328,32.3393)(329,31.5532)(330,30.7711)(331,29.9928)(332,29.2182)(333,28.4472)(334,27.6798)(335,26.9158)(336,26.1551)(337,25.3978)(338,24.6436)(339,23.8925)(340,23.1443)(341,22.3991)(342,21.6567)(343,20.917)(344,20.18)(345,19.4455)(346,18.7134)(347,17.9837)(348,17.2563)(349,16.531)(350,15.8079)(351,15.0867)(352,14.3675)(353,13.65)(354,12.9343)(355,12.2203)(356,11.5078)(357,10.7967)(358,10.0871)(359,9.37866)(360,8.67146)(361,7.96537)(362,7.26029)(363,6.55613)(364,5.85281)(365,5.15023)(366,4.44831)(367,3.74696)(368,3.04609)(369,2.3456)(370,1.64541) 
};

\end{axis}
\end{tikzpicture}

			\newframe
			\begin{tikzpicture}[scale=0.5]

\begin{axis}[%
scale only axis,
width=4.52083in,
height=3.56562in,
xmin=0, xmax=400,
ymin=0, ymax=400,
xlabel={Slab Length [cm]},
ylabel={Power [W]},
title={$\text{tstep }= 18$},
axis on top]
\addplot [
color=blue,
solid
]
coordinates{
 (1,13.9911)(2,20.3593)(3,26.7197)(4,33.0699)(5,39.4076)(6,45.7302)(7,52.0354)(8,58.3208)(9,64.584)(10,70.8228)(11,77.0348)(12,83.2176)(13,89.3691)(14,95.487)(15,101.569)(16,107.613)(17,113.617)(18,119.578)(19,125.496)(20,131.366)(21,137.189)(22,142.96)(23,148.68)(24,154.345)(25,159.955)(26,165.506)(27,170.998)(28,176.428)(29,181.795)(30,187.098)(31,192.335)(32,197.504)(33,202.604)(34,207.633)(35,212.591)(36,217.475)(37,222.285)(38,227.019)(39,231.677)(40,236.256)(41,240.757)(42,245.178)(43,249.519)(44,253.778)(45,257.954)(46,262.048)(47,266.057)(48,269.982)(49,273.822)(50,277.577)(51,281.245)(52,284.827)(53,288.323)(54,291.731)(55,295.052)(56,298.285)(57,301.43)(58,304.488)(59,307.457)(60,310.339)(61,313.133)(62,315.839)(63,318.457)(64,320.988)(65,323.431)(66,325.788)(67,328.057)(68,330.241)(69,332.338)(70,334.35)(71,336.276)(72,338.118)(73,339.876)(74,341.55)(75,343.141)(76,344.65)(77,346.078)(78,347.424)(79,348.69)(80,349.877)(81,350.985)(82,352.015)(83,352.967)(84,353.844)(85,354.645)(86,355.372)(87,356.025)(88,356.606)(89,357.114)(90,357.553)(91,357.921)(92,358.22)(93,358.452)(94,358.617)(95,358.716)(96,358.751)(97,358.722)(98,358.63)(99,358.477)(100,358.264)(101,357.991)(102,357.659)(103,357.271)(104,356.826)(105,356.327)(106,355.774)(107,355.167)(108,354.509)(109,353.801)(110,353.043)(111,352.236)(112,351.382)(113,350.482)(114,349.537)(115,348.547)(116,347.515)(117,346.441)(118,345.325)(119,344.17)(120,342.976)(121,341.744)(122,340.476)(123,339.172)(124,337.833)(125,336.461)(126,335.056)(127,333.619)(128,332.152)(129,330.654)(130,329.128)(131,327.575)(132,325.994)(133,324.387)(134,322.756)(135,321.1)(136,319.421)(137,317.719)(138,315.997)(139,314.253)(140,312.49)(141,310.707)(142,308.907)(143,307.089)(144,305.254)(145,303.404)(146,301.538)(147,299.659)(148,297.765)(149,295.859)(150,293.941)(151,292.012)(152,290.071)(153,288.121)(154,286.161)(155,284.192)(156,282.216)(157,280.232)(158,278.24)(159,276.243)(160,274.24)(161,272.232)(162,270.219)(163,268.202)(164,266.181)(165,264.158)(166,262.132)(167,260.104)(168,258.075)(169,256.044)(170,254.013)(171,251.982)(172,249.951)(173,247.92)(174,245.891)(175,243.864)(176,241.838)(177,239.815)(178,237.794)(179,235.776)(180,233.762)(181,231.752)(182,229.745)(183,227.743)(184,225.746)(185,223.753)(186,221.766)(187,219.785)(188,217.809)(189,215.839)(190,213.876)(191,211.919)(192,209.969)(193,208.027)(194,206.091)(195,204.163)(196,202.242)(197,200.33)(198,198.425)(199,196.529)(200,194.641)(201,192.762)(202,190.892)(203,189.03)(204,187.178)(205,185.334)(206,183.5)(207,181.676)(208,179.861)(209,178.055)(210,176.26)(211,174.474)(212,172.698)(213,170.933)(214,169.177)(215,167.432)(216,165.697)(217,163.972)(218,162.258)(219,160.554)(220,158.861)(221,157.178)(222,155.506)(223,153.845)(224,152.194)(225,150.554)(226,148.925)(227,147.307)(228,145.699)(229,144.102)(230,142.516)(231,140.941)(232,139.377)(233,137.823)(234,136.281)(235,134.749)(236,133.228)(237,131.718)(238,130.218)(239,128.73)(240,127.252)(241,125.785)(242,124.328)(243,122.883)(244,121.447)(245,120.023)(246,118.609)(247,117.206)(248,115.813)(249,114.43)(250,113.058)(251,111.697)(252,110.345)(253,109.004)(254,107.673)(255,106.352)(256,105.042)(257,103.741)(258,102.451)(259,101.17)(260,99.899)(261,98.6379)(262,97.3866)(263,96.1449)(264,94.9129)(265,93.6903)(266,92.4773)(267,91.2736)(268,90.0792)(269,88.8941)(270,87.7182)(271,86.5513)(272,85.3934)(273,84.2445)(274,83.1045)(275,81.9733)(276,80.8507)(277,79.7368)(278,78.6314)(279,77.5345)(280,76.4459)(281,75.3657)(282,74.2937)(283,73.2298)(284,72.1739)(285,71.126)(286,70.086)(287,69.0538)(288,68.0293)(289,67.0123)(290,66.003)(291,65.001)(292,64.0064)(293,63.0191)(294,62.039)(295,61.0659)(296,60.0999)(297,59.1407)(298,58.1884)(299,57.2428)(300,56.3038)(301,55.3714)(302,54.4454)(303,53.5259)(304,52.6125)(305,51.7054)(306,50.8044)(307,49.9094)(308,49.0203)(309,48.137)(310,47.2594)(311,46.3875)(312,45.5212)(313,44.6603)(314,43.8047)(315,42.9545)(316,42.1094)(317,41.2694)(318,40.4344)(319,39.6043)(320,38.779)(321,37.9585)(322,37.1426)(323,36.3312)(324,35.5243)(325,34.7217)(326,33.9234)(327,33.1293)(328,32.3393)(329,31.5532)(330,30.7711)(331,29.9928)(332,29.2182)(333,28.4472)(334,27.6798)(335,26.9158)(336,26.1551)(337,25.3978)(338,24.6436)(339,23.8925)(340,23.1443)(341,22.3991)(342,21.6567)(343,20.917)(344,20.18)(345,19.4455)(346,18.7134)(347,17.9837)(348,17.2563)(349,16.531)(350,15.8079)(351,15.0867)(352,14.3675)(353,13.65)(354,12.9343)(355,12.2203)(356,11.5078)(357,10.7967)(358,10.0871)(359,9.37866)(360,8.67146)(361,7.96537)(362,7.26029)(363,6.55613)(364,5.85281)(365,5.15023)(366,4.44831)(367,3.74696)(368,3.04609)(369,2.3456)(370,1.64541) 
};

\addplot [
color=red,
solid
]
coordinates{
 (1,13.9911)(2,20.3593)(3,26.7197)(4,33.0699)(5,39.4076)(6,45.7302)(7,52.0354)(8,58.3208)(9,64.584)(10,70.8228)(11,77.0348)(12,83.2176)(13,89.3691)(14,95.487)(15,101.569)(16,107.613)(17,113.617)(18,119.578)(19,125.496)(20,131.366)(21,137.189)(22,142.96)(23,148.68)(24,154.345)(25,159.955)(26,165.506)(27,170.998)(28,176.428)(29,181.795)(30,187.098)(31,192.335)(32,197.504)(33,202.604)(34,207.633)(35,212.591)(36,217.475)(37,222.285)(38,227.019)(39,231.677)(40,236.256)(41,240.757)(42,245.178)(43,249.519)(44,253.778)(45,257.954)(46,262.048)(47,266.057)(48,269.982)(49,273.822)(50,277.577)(51,281.245)(52,284.827)(53,288.323)(54,291.731)(55,295.052)(56,298.285)(57,301.43)(58,304.488)(59,307.457)(60,310.339)(61,313.133)(62,315.839)(63,318.457)(64,320.988)(65,323.431)(66,325.788)(67,328.057)(68,330.241)(69,332.338)(70,334.35)(71,336.276)(72,338.118)(73,339.876)(74,341.55)(75,343.141)(76,344.65)(77,346.078)(78,347.424)(79,348.69)(80,349.877)(81,350.985)(82,352.015)(83,352.967)(84,353.844)(85,354.645)(86,355.372)(87,356.025)(88,356.606)(89,357.114)(90,357.553)(91,357.921)(92,358.22)(93,358.452)(94,358.617)(95,358.716)(96,358.751)(97,358.722)(98,358.63)(99,358.477)(100,358.264)(101,357.991)(102,357.659)(103,357.271)(104,356.826)(105,356.327)(106,355.774)(107,355.167)(108,354.509)(109,353.801)(110,353.043)(111,352.236)(112,351.382)(113,350.482)(114,349.537)(115,348.547)(116,347.515)(117,346.441)(118,345.325)(119,344.17)(120,342.976)(121,341.744)(122,340.476)(123,339.172)(124,337.833)(125,336.461)(126,335.056)(127,333.619)(128,332.152)(129,330.654)(130,329.128)(131,327.575)(132,325.994)(133,324.387)(134,322.756)(135,321.1)(136,319.421)(137,317.719)(138,315.997)(139,314.253)(140,312.49)(141,310.707)(142,308.907)(143,307.089)(144,305.254)(145,303.404)(146,301.538)(147,299.659)(148,297.765)(149,295.859)(150,293.941)(151,292.012)(152,290.071)(153,288.121)(154,286.161)(155,284.192)(156,282.216)(157,280.232)(158,278.24)(159,276.243)(160,274.24)(161,272.232)(162,270.219)(163,268.202)(164,266.181)(165,264.158)(166,262.132)(167,260.104)(168,258.075)(169,256.044)(170,254.013)(171,251.982)(172,249.951)(173,247.92)(174,245.891)(175,243.864)(176,241.838)(177,239.815)(178,237.794)(179,235.776)(180,233.762)(181,231.752)(182,229.745)(183,227.743)(184,225.746)(185,223.753)(186,221.766)(187,219.785)(188,217.809)(189,215.839)(190,213.876)(191,211.919)(192,209.969)(193,208.027)(194,206.091)(195,204.163)(196,202.242)(197,200.33)(198,198.425)(199,196.529)(200,194.641)(201,192.762)(202,190.892)(203,189.03)(204,187.178)(205,185.334)(206,183.5)(207,181.676)(208,179.861)(209,178.055)(210,176.26)(211,174.474)(212,172.698)(213,170.933)(214,169.177)(215,167.432)(216,165.697)(217,163.972)(218,162.258)(219,160.554)(220,158.861)(221,157.178)(222,155.506)(223,153.845)(224,152.194)(225,150.554)(226,148.925)(227,147.307)(228,145.699)(229,144.102)(230,142.516)(231,140.941)(232,139.377)(233,137.823)(234,136.281)(235,134.749)(236,133.228)(237,131.718)(238,130.218)(239,128.73)(240,127.252)(241,125.785)(242,124.328)(243,122.883)(244,121.447)(245,120.023)(246,118.609)(247,117.206)(248,115.813)(249,114.43)(250,113.058)(251,111.697)(252,110.345)(253,109.004)(254,107.673)(255,106.352)(256,105.042)(257,103.741)(258,102.451)(259,101.17)(260,99.899)(261,98.6379)(262,97.3866)(263,96.1449)(264,94.9129)(265,93.6903)(266,92.4773)(267,91.2736)(268,90.0792)(269,88.8941)(270,87.7182)(271,86.5513)(272,85.3934)(273,84.2445)(274,83.1045)(275,81.9733)(276,80.8507)(277,79.7368)(278,78.6314)(279,77.5345)(280,76.4459)(281,75.3657)(282,74.2937)(283,73.2298)(284,72.1739)(285,71.126)(286,70.086)(287,69.0538)(288,68.0293)(289,67.0123)(290,66.003)(291,65.001)(292,64.0064)(293,63.0191)(294,62.039)(295,61.0659)(296,60.0999)(297,59.1407)(298,58.1884)(299,57.2428)(300,56.3038)(301,55.3714)(302,54.4454)(303,53.5259)(304,52.6125)(305,51.7054)(306,50.8044)(307,49.9094)(308,49.0203)(309,48.137)(310,47.2594)(311,46.3875)(312,45.5212)(313,44.6603)(314,43.8047)(315,42.9545)(316,42.1094)(317,41.2694)(318,40.4344)(319,39.6043)(320,38.779)(321,37.9585)(322,37.1426)(323,36.3312)(324,35.5243)(325,34.7217)(326,33.9234)(327,33.1293)(328,32.3393)(329,31.5532)(330,30.7711)(331,29.9928)(332,29.2182)(333,28.4472)(334,27.6798)(335,26.9158)(336,26.1551)(337,25.3978)(338,24.6436)(339,23.8925)(340,23.1443)(341,22.3991)(342,21.6567)(343,20.917)(344,20.18)(345,19.4455)(346,18.7134)(347,17.9837)(348,17.2563)(349,16.531)(350,15.8079)(351,15.0867)(352,14.3675)(353,13.65)(354,12.9343)(355,12.2203)(356,11.5078)(357,10.7967)(358,10.0871)(359,9.37866)(360,8.67146)(361,7.96537)(362,7.26029)(363,6.55613)(364,5.85281)(365,5.15023)(366,4.44831)(367,3.74696)(368,3.04609)(369,2.3456)(370,1.64541) 
};

\end{axis}
\end{tikzpicture}

			\newframe
			\begin{tikzpicture}[scale=0.5]

\begin{axis}[%
scale only axis,
width=4.52083in,
height=3.56562in,
xmin=0, xmax=400,
ymin=0, ymax=400,
xlabel={Slab Length [cm]},
ylabel={Power [W]},
title={$\text{tstep }= 24$},
axis on top]
\addplot [
color=blue,
solid
]
coordinates{
 (1,13.9911)(2,20.3593)(3,26.7197)(4,33.0699)(5,39.4076)(6,45.7302)(7,52.0354)(8,58.3208)(9,64.584)(10,70.8228)(11,77.0348)(12,83.2176)(13,89.3691)(14,95.487)(15,101.569)(16,107.613)(17,113.617)(18,119.578)(19,125.496)(20,131.366)(21,137.189)(22,142.96)(23,148.68)(24,154.345)(25,159.955)(26,165.506)(27,170.998)(28,176.428)(29,181.795)(30,187.098)(31,192.335)(32,197.504)(33,202.604)(34,207.633)(35,212.591)(36,217.475)(37,222.285)(38,227.019)(39,231.677)(40,236.256)(41,240.757)(42,245.178)(43,249.519)(44,253.778)(45,257.954)(46,262.048)(47,266.057)(48,269.982)(49,273.822)(50,277.577)(51,281.245)(52,284.827)(53,288.323)(54,291.731)(55,295.052)(56,298.285)(57,301.43)(58,304.488)(59,307.457)(60,310.339)(61,313.133)(62,315.839)(63,318.457)(64,320.988)(65,323.431)(66,325.788)(67,328.057)(68,330.241)(69,332.338)(70,334.35)(71,336.276)(72,338.118)(73,339.876)(74,341.55)(75,343.141)(76,344.65)(77,346.078)(78,347.424)(79,348.69)(80,349.877)(81,350.985)(82,352.015)(83,352.967)(84,353.844)(85,354.645)(86,355.372)(87,356.025)(88,356.606)(89,357.114)(90,357.553)(91,357.921)(92,358.22)(93,358.452)(94,358.617)(95,358.716)(96,358.751)(97,358.722)(98,358.63)(99,358.477)(100,358.264)(101,357.991)(102,357.659)(103,357.271)(104,356.826)(105,356.327)(106,355.774)(107,355.167)(108,354.509)(109,353.801)(110,353.043)(111,352.236)(112,351.382)(113,350.482)(114,349.537)(115,348.547)(116,347.515)(117,346.441)(118,345.325)(119,344.17)(120,342.976)(121,341.744)(122,340.476)(123,339.172)(124,337.833)(125,336.461)(126,335.056)(127,333.619)(128,332.152)(129,330.654)(130,329.128)(131,327.575)(132,325.994)(133,324.387)(134,322.756)(135,321.1)(136,319.421)(137,317.719)(138,315.997)(139,314.253)(140,312.49)(141,310.707)(142,308.907)(143,307.089)(144,305.254)(145,303.404)(146,301.538)(147,299.659)(148,297.765)(149,295.859)(150,293.941)(151,292.012)(152,290.071)(153,288.121)(154,286.161)(155,284.192)(156,282.216)(157,280.232)(158,278.24)(159,276.243)(160,274.24)(161,272.232)(162,270.219)(163,268.202)(164,266.181)(165,264.158)(166,262.132)(167,260.104)(168,258.075)(169,256.044)(170,254.013)(171,251.982)(172,249.951)(173,247.92)(174,245.891)(175,243.864)(176,241.838)(177,239.815)(178,237.794)(179,235.776)(180,233.762)(181,231.752)(182,229.745)(183,227.743)(184,225.746)(185,223.753)(186,221.766)(187,219.785)(188,217.809)(189,215.839)(190,213.876)(191,211.919)(192,209.969)(193,208.027)(194,206.091)(195,204.163)(196,202.242)(197,200.33)(198,198.425)(199,196.529)(200,194.641)(201,192.762)(202,190.892)(203,189.03)(204,187.178)(205,185.334)(206,183.5)(207,181.676)(208,179.861)(209,178.055)(210,176.26)(211,174.474)(212,172.698)(213,170.933)(214,169.177)(215,167.432)(216,165.697)(217,163.972)(218,162.258)(219,160.554)(220,158.861)(221,157.178)(222,155.506)(223,153.845)(224,152.194)(225,150.554)(226,148.925)(227,147.307)(228,145.699)(229,144.102)(230,142.516)(231,140.941)(232,139.377)(233,137.823)(234,136.281)(235,134.749)(236,133.228)(237,131.718)(238,130.218)(239,128.73)(240,127.252)(241,125.785)(242,124.328)(243,122.883)(244,121.447)(245,120.023)(246,118.609)(247,117.206)(248,115.813)(249,114.43)(250,113.058)(251,111.697)(252,110.345)(253,109.004)(254,107.673)(255,106.352)(256,105.042)(257,103.741)(258,102.451)(259,101.17)(260,99.899)(261,98.6379)(262,97.3866)(263,96.1449)(264,94.9129)(265,93.6903)(266,92.4773)(267,91.2736)(268,90.0792)(269,88.8941)(270,87.7182)(271,86.5513)(272,85.3934)(273,84.2445)(274,83.1045)(275,81.9733)(276,80.8507)(277,79.7368)(278,78.6314)(279,77.5345)(280,76.4459)(281,75.3657)(282,74.2937)(283,73.2298)(284,72.1739)(285,71.126)(286,70.086)(287,69.0538)(288,68.0293)(289,67.0123)(290,66.003)(291,65.001)(292,64.0064)(293,63.0191)(294,62.039)(295,61.0659)(296,60.0999)(297,59.1407)(298,58.1884)(299,57.2428)(300,56.3038)(301,55.3714)(302,54.4454)(303,53.5259)(304,52.6125)(305,51.7054)(306,50.8044)(307,49.9094)(308,49.0203)(309,48.137)(310,47.2594)(311,46.3875)(312,45.5212)(313,44.6603)(314,43.8047)(315,42.9545)(316,42.1094)(317,41.2694)(318,40.4344)(319,39.6043)(320,38.779)(321,37.9585)(322,37.1426)(323,36.3312)(324,35.5243)(325,34.7217)(326,33.9234)(327,33.1293)(328,32.3393)(329,31.5532)(330,30.7711)(331,29.9928)(332,29.2182)(333,28.4472)(334,27.6798)(335,26.9158)(336,26.1551)(337,25.3978)(338,24.6436)(339,23.8925)(340,23.1443)(341,22.3991)(342,21.6567)(343,20.917)(344,20.18)(345,19.4455)(346,18.7134)(347,17.9837)(348,17.2563)(349,16.531)(350,15.8079)(351,15.0867)(352,14.3675)(353,13.65)(354,12.9343)(355,12.2203)(356,11.5078)(357,10.7967)(358,10.0871)(359,9.37866)(360,8.67146)(361,7.96537)(362,7.26029)(363,6.55613)(364,5.85281)(365,5.15023)(366,4.44831)(367,3.74696)(368,3.04609)(369,2.3456)(370,1.64541) 
};

\addplot [
color=red,
solid
]
coordinates{
 (1,12.1668)(2,18.0311)(3,24.3721)(4,30.7034)(5,37.0227)(6,43.3275)(7,49.6154)(8,55.8841)(9,62.1312)(10,68.3543)(11,74.5512)(12,80.7196)(13,86.8571)(14,92.9616)(15,99.0308)(16,105.062)(17,111.055)(18,117.005)(19,122.912)(20,128.772)(21,134.585)(22,140.348)(23,146.059)(24,151.717)(25,157.319)(26,162.863)(27,168.349)(28,173.773)(29,179.136)(30,184.434)(31,189.667)(32,194.832)(33,199.929)(34,204.956)(35,209.912)(36,214.794)(37,219.603)(38,224.337)(39,228.995)(40,233.575)(41,238.077)(42,242.5)(43,246.843)(44,251.104)(45,255.284)(46,259.38)(47,263.394)(48,267.323)(49,271.168)(50,274.928)(51,278.602)(52,282.19)(53,285.692)(54,289.107)(55,292.435)(56,295.676)(57,298.829)(58,301.895)(59,304.874)(60,307.764)(61,310.568)(62,313.283)(63,315.912)(64,318.453)(65,320.907)(66,323.275)(67,325.556)(68,327.751)(69,329.86)(70,331.884)(71,333.823)(72,335.678)(73,337.448)(74,339.136)(75,340.741)(76,342.263)(77,343.705)(78,345.065)(79,346.346)(80,347.547)(81,348.669)(82,349.714)(83,350.682)(84,351.574)(85,352.391)(86,353.133)(87,353.802)(88,354.398)(89,354.923)(90,355.377)(91,355.762)(92,356.077)(93,356.326)(94,356.507)(95,356.623)(96,356.674)(97,356.662)(98,356.587)(99,356.451)(100,356.254)(101,355.998)(102,355.684)(103,355.313)(104,354.885)(105,354.403)(106,353.867)(107,353.278)(108,352.637)(109,351.946)(110,351.205)(111,350.415)(112,349.579)(113,348.696)(114,347.767)(115,346.795)(116,345.78)(117,344.723)(118,343.624)(119,342.486)(120,341.309)(121,340.094)(122,338.843)(123,337.556)(124,336.234)(125,334.878)(126,333.49)(127,332.07)(128,330.619)(129,329.138)(130,327.628)(131,326.091)(132,324.526)(133,322.936)(134,321.32)(135,319.681)(136,318.017)(137,316.332)(138,314.625)(139,312.897)(140,311.149)(141,309.382)(142,307.597)(143,305.794)(144,303.974)(145,302.139)(146,300.289)(147,298.424)(148,296.545)(149,294.654)(150,292.75)(151,290.835)(152,288.909)(153,286.973)(154,285.027)(155,283.072)(156,281.109)(157,279.139)(158,277.161)(159,275.177)(160,273.188)(161,271.193)(162,269.193)(163,267.189)(164,265.181)(165,263.171)(166,261.157)(167,259.142)(168,257.125)(169,255.107)(170,253.088)(171,251.068)(172,249.049)(173,247.031)(174,245.013)(175,242.998)(176,240.983)(177,238.971)(178,236.962)(179,234.955)(180,232.952)(181,230.953)(182,228.957)(183,226.965)(184,224.979)(185,222.997)(186,221.02)(187,219.048)(188,217.083)(189,215.123)(190,213.17)(191,211.223)(192,209.282)(193,207.349)(194,205.423)(195,203.504)(196,201.593)(197,199.69)(198,197.794)(199,195.907)(200,194.028)(201,192.157)(202,190.295)(203,188.442)(204,186.598)(205,184.763)(206,182.937)(207,181.121)(208,179.314)(209,177.516)(210,175.728)(211,173.95)(212,172.182)(213,170.424)(214,168.676)(215,166.938)(216,165.21)(217,163.492)(218,161.785)(219,160.088)(220,158.402)(221,156.726)(222,155.061)(223,153.406)(224,151.762)(225,150.128)(226,148.505)(227,146.893)(228,145.292)(229,143.701)(230,142.121)(231,140.552)(232,138.993)(233,137.446)(234,135.909)(235,134.383)(236,132.867)(237,131.363)(238,129.869)(239,128.385)(240,126.913)(241,125.451)(242,124)(243,122.559)(244,121.129)(245,119.709)(246,118.3)(247,116.902)(248,115.513)(249,114.136)(250,112.768)(251,111.411)(252,110.064)(253,108.727)(254,107.401)(255,106.084)(256,104.778)(257,103.482)(258,102.195)(259,100.919)(260,99.6517)(261,98.3947)(262,97.1473)(263,95.9095)(264,94.6812)(265,93.4625)(266,92.2531)(267,91.0531)(268,89.8623)(269,88.6808)(270,87.5084)(271,86.345)(272,85.1905)(273,84.045)(274,82.9083)(275,81.7803)(276,80.661)(277,79.5503)(278,78.4481)(279,77.3543)(280,76.2688)(281,75.1916)(282,74.1226)(283,73.0617)(284,72.0087)(285,70.9637)(286,69.9265)(287,68.8971)(288,67.8754)(289,66.8612)(290,65.8545)(291,64.8552)(292,63.8633)(293,62.8786)(294,61.901)(295,60.9304)(296,59.9669)(297,59.0102)(298,58.0603)(299,57.1171)(300,56.1805)(301,55.2505)(302,54.3268)(303,53.4095)(304,52.4985)(305,51.5936)(306,50.6948)(307,49.802)(308,48.915)(309,48.0339)(310,47.1584)(311,46.2886)(312,45.4243)(313,44.5654)(314,43.7119)(315,42.8636)(316,42.0205)(317,41.1825)(318,40.3494)(319,39.5213)(320,38.6979)(321,37.8792)(322,37.0651)(323,36.2556)(324,35.4505)(325,34.6497)(326,33.8532)(327,33.0609)(328,32.2726)(329,31.4883)(330,30.7079)(331,29.9312)(332,29.1583)(333,28.389)(334,27.6232)(335,26.8609)(336,26.1019)(337,25.3461)(338,24.5935)(339,23.844)(340,23.0974)(341,22.3538)(342,21.6129)(343,20.8748)(344,20.1393)(345,19.4063)(346,18.6757)(347,17.9476)(348,17.2216)(349,16.4979)(350,15.7762)(351,15.0565)(352,14.3387)(353,13.6227)(354,12.9085)(355,12.1959)(356,11.4848)(357,10.7752)(358,10.0669)(359,9.35997)(360,8.65419)(361,7.94951)(362,7.24584)(363,6.54309)(364,5.84117)(365,5.14)(366,4.43947)(367,3.73952)(368,3.04004)(369,2.34094)(370,1.64215) 
};

\end{axis}
\end{tikzpicture}

			\newframe
			\begin{tikzpicture}[scale=0.5]

\begin{axis}[%
scale only axis,
width=4.52083in,
height=3.56562in,
xmin=0, xmax=400,
ymin=0, ymax=400,
xlabel={Slab Length [cm]},
ylabel={Power [W]},
title={$\text{tstep }= 30$},
axis on top]
\addplot [
color=blue,
solid
]
coordinates{
 (1,13.9911)(2,20.3593)(3,26.7197)(4,33.0699)(5,39.4076)(6,45.7302)(7,52.0354)(8,58.3208)(9,64.584)(10,70.8228)(11,77.0348)(12,83.2176)(13,89.3691)(14,95.487)(15,101.569)(16,107.613)(17,113.617)(18,119.578)(19,125.496)(20,131.366)(21,137.189)(22,142.96)(23,148.68)(24,154.345)(25,159.955)(26,165.506)(27,170.998)(28,176.428)(29,181.795)(30,187.098)(31,192.335)(32,197.504)(33,202.604)(34,207.633)(35,212.591)(36,217.475)(37,222.285)(38,227.019)(39,231.677)(40,236.256)(41,240.757)(42,245.178)(43,249.519)(44,253.778)(45,257.954)(46,262.048)(47,266.057)(48,269.982)(49,273.822)(50,277.577)(51,281.245)(52,284.827)(53,288.323)(54,291.731)(55,295.052)(56,298.285)(57,301.43)(58,304.488)(59,307.457)(60,310.339)(61,313.133)(62,315.839)(63,318.457)(64,320.988)(65,323.431)(66,325.788)(67,328.057)(68,330.241)(69,332.338)(70,334.35)(71,336.276)(72,338.118)(73,339.876)(74,341.55)(75,343.141)(76,344.65)(77,346.078)(78,347.424)(79,348.69)(80,349.877)(81,350.985)(82,352.015)(83,352.967)(84,353.844)(85,354.645)(86,355.372)(87,356.025)(88,356.606)(89,357.114)(90,357.553)(91,357.921)(92,358.22)(93,358.452)(94,358.617)(95,358.716)(96,358.751)(97,358.722)(98,358.63)(99,358.477)(100,358.264)(101,357.991)(102,357.659)(103,357.271)(104,356.826)(105,356.327)(106,355.774)(107,355.167)(108,354.509)(109,353.801)(110,353.043)(111,352.236)(112,351.382)(113,350.482)(114,349.537)(115,348.547)(116,347.515)(117,346.441)(118,345.325)(119,344.17)(120,342.976)(121,341.744)(122,340.476)(123,339.172)(124,337.833)(125,336.461)(126,335.056)(127,333.619)(128,332.152)(129,330.654)(130,329.128)(131,327.575)(132,325.994)(133,324.387)(134,322.756)(135,321.1)(136,319.421)(137,317.719)(138,315.997)(139,314.253)(140,312.49)(141,310.707)(142,308.907)(143,307.089)(144,305.254)(145,303.404)(146,301.538)(147,299.659)(148,297.765)(149,295.859)(150,293.941)(151,292.012)(152,290.071)(153,288.121)(154,286.161)(155,284.192)(156,282.216)(157,280.232)(158,278.24)(159,276.243)(160,274.24)(161,272.232)(162,270.219)(163,268.202)(164,266.181)(165,264.158)(166,262.132)(167,260.104)(168,258.075)(169,256.044)(170,254.013)(171,251.982)(172,249.951)(173,247.92)(174,245.891)(175,243.864)(176,241.838)(177,239.815)(178,237.794)(179,235.776)(180,233.762)(181,231.752)(182,229.745)(183,227.743)(184,225.746)(185,223.753)(186,221.766)(187,219.785)(188,217.809)(189,215.839)(190,213.876)(191,211.919)(192,209.969)(193,208.027)(194,206.091)(195,204.163)(196,202.242)(197,200.33)(198,198.425)(199,196.529)(200,194.641)(201,192.762)(202,190.892)(203,189.03)(204,187.178)(205,185.334)(206,183.5)(207,181.676)(208,179.861)(209,178.055)(210,176.26)(211,174.474)(212,172.698)(213,170.933)(214,169.177)(215,167.432)(216,165.697)(217,163.972)(218,162.258)(219,160.554)(220,158.861)(221,157.178)(222,155.506)(223,153.845)(224,152.194)(225,150.554)(226,148.925)(227,147.307)(228,145.699)(229,144.102)(230,142.516)(231,140.941)(232,139.377)(233,137.823)(234,136.281)(235,134.749)(236,133.228)(237,131.718)(238,130.218)(239,128.73)(240,127.252)(241,125.785)(242,124.328)(243,122.883)(244,121.447)(245,120.023)(246,118.609)(247,117.206)(248,115.813)(249,114.43)(250,113.058)(251,111.697)(252,110.345)(253,109.004)(254,107.673)(255,106.352)(256,105.042)(257,103.741)(258,102.451)(259,101.17)(260,99.899)(261,98.6379)(262,97.3866)(263,96.1449)(264,94.9129)(265,93.6903)(266,92.4773)(267,91.2736)(268,90.0792)(269,88.8941)(270,87.7182)(271,86.5513)(272,85.3934)(273,84.2445)(274,83.1045)(275,81.9733)(276,80.8507)(277,79.7368)(278,78.6314)(279,77.5345)(280,76.4459)(281,75.3657)(282,74.2937)(283,73.2298)(284,72.1739)(285,71.126)(286,70.086)(287,69.0538)(288,68.0293)(289,67.0123)(290,66.003)(291,65.001)(292,64.0064)(293,63.0191)(294,62.039)(295,61.0659)(296,60.0999)(297,59.1407)(298,58.1884)(299,57.2428)(300,56.3038)(301,55.3714)(302,54.4454)(303,53.5259)(304,52.6125)(305,51.7054)(306,50.8044)(307,49.9094)(308,49.0203)(309,48.137)(310,47.2594)(311,46.3875)(312,45.5212)(313,44.6603)(314,43.8047)(315,42.9545)(316,42.1094)(317,41.2694)(318,40.4344)(319,39.6043)(320,38.779)(321,37.9585)(322,37.1426)(323,36.3312)(324,35.5243)(325,34.7217)(326,33.9234)(327,33.1293)(328,32.3393)(329,31.5532)(330,30.7711)(331,29.9928)(332,29.2182)(333,28.4472)(334,27.6798)(335,26.9158)(336,26.1551)(337,25.3978)(338,24.6436)(339,23.8925)(340,23.1443)(341,22.3991)(342,21.6567)(343,20.917)(344,20.18)(345,19.4455)(346,18.7134)(347,17.9837)(348,17.2563)(349,16.531)(350,15.8079)(351,15.0867)(352,14.3675)(353,13.65)(354,12.9343)(355,12.2203)(356,11.5078)(357,10.7967)(358,10.0871)(359,9.37866)(360,8.67146)(361,7.96537)(362,7.26029)(363,6.55613)(364,5.85281)(365,5.15023)(366,4.44831)(367,3.74696)(368,3.04609)(369,2.3456)(370,1.64541) 
};

\addplot [
color=red,
solid
]
coordinates{
 (1,7.4757)(2,11.0782)(3,14.9726)(4,19.2615)(5,24.058)(6,29.4886)(7,35.6967)(8,41.8885)(9,48.0616)(10,54.2136)(11,60.3424)(12,66.4455)(13,72.5208)(14,78.5659)(15,84.5787)(16,90.557)(17,96.4987)(18,102.402)(19,108.263)(20,114.082)(21,119.857)(22,125.584)(23,131.262)(24,136.889)(25,142.464)(26,147.985)(27,153.449)(28,158.855)(29,164.202)(30,169.487)(31,174.71)(32,179.868)(33,184.96)(34,189.985)(35,194.942)(36,199.828)(37,204.644)(38,209.386)(39,214.055)(40,218.65)(41,223.168)(42,227.61)(43,231.974)(44,236.259)(45,240.465)(46,244.59)(47,248.634)(48,252.597)(49,256.477)(50,260.275)(51,263.989)(52,267.619)(53,271.164)(54,274.625)(55,278.001)(56,281.292)(57,284.498)(58,287.617)(59,290.652)(60,293.6)(61,296.462)(62,299.239)(63,301.93)(64,304.535)(65,307.055)(66,309.49)(67,311.84)(68,314.105)(69,316.286)(70,318.383)(71,320.397)(72,322.327)(73,324.175)(74,325.94)(75,327.624)(76,329.227)(77,330.75)(78,332.193)(79,333.556)(80,334.842)(81,336.05)(82,337.181)(83,338.236)(84,339.215)(85,340.12)(86,340.952)(87,341.71)(88,342.397)(89,343.013)(90,343.558)(91,344.035)(92,344.443)(93,344.784)(94,345.059)(95,345.269)(96,345.414)(97,345.496)(98,345.516)(99,345.475)(100,345.373)(101,345.213)(102,344.994)(103,344.718)(104,344.386)(105,344)(106,343.559)(107,343.066)(108,342.521)(109,341.925)(110,341.28)(111,340.586)(112,339.845)(113,339.057)(114,338.224)(115,337.347)(116,336.426)(117,335.464)(118,334.46)(119,333.416)(120,332.332)(121,331.211)(122,330.053)(123,328.858)(124,327.628)(125,326.365)(126,325.068)(127,323.739)(128,322.379)(129,320.989)(130,319.569)(131,318.121)(132,316.645)(133,315.143)(134,313.616)(135,312.063)(136,310.487)(137,308.888)(138,307.267)(139,305.624)(140,303.961)(141,302.278)(142,300.577)(143,298.857)(144,297.12)(145,295.366)(146,293.597)(147,291.813)(148,290.015)(149,288.203)(150,286.378)(151,284.541)(152,282.692)(153,280.833)(154,278.963)(155,277.084)(156,275.196)(157,273.3)(158,271.396)(159,269.485)(160,267.568)(161,265.644)(162,263.715)(163,261.782)(164,259.844)(165,257.902)(166,255.957)(167,254.01)(168,252.06)(169,250.108)(170,248.155)(171,246.2)(172,244.246)(173,242.291)(174,240.337)(175,238.383)(176,236.431)(177,234.48)(178,232.531)(179,230.584)(180,228.64)(181,226.699)(182,224.761)(183,222.827)(184,220.897)(185,218.971)(186,217.05)(187,215.133)(188,213.222)(189,211.316)(190,209.415)(191,207.52)(192,205.632)(193,203.75)(194,201.874)(195,200.005)(196,198.143)(197,196.289)(198,194.441)(199,192.601)(200,190.77)(201,188.945)(202,187.129)(203,185.322)(204,183.522)(205,181.732)(206,179.949)(207,178.176)(208,176.412)(209,174.656)(210,172.91)(211,171.173)(212,169.445)(213,167.727)(214,166.018)(215,164.319)(216,162.63)(217,160.951)(218,159.281)(219,157.621)(220,155.971)(221,154.331)(222,152.702)(223,151.082)(224,149.473)(225,147.873)(226,146.284)(227,144.706)(228,143.137)(229,141.579)(230,140.031)(231,138.493)(232,136.966)(233,135.449)(234,133.943)(235,132.447)(236,130.961)(237,129.485)(238,128.02)(239,126.565)(240,125.121)(241,123.687)(242,122.263)(243,120.849)(244,119.445)(245,118.052)(246,116.669)(247,115.296)(248,113.933)(249,112.58)(250,111.237)(251,109.904)(252,108.581)(253,107.267)(254,105.964)(255,104.67)(256,103.387)(257,102.112)(258,100.848)(259,99.5929)(260,98.3474)(261,97.1114)(262,95.8848)(263,94.6675)(264,93.4595)(265,92.2606)(266,91.071)(267,89.8903)(268,88.7187)(269,87.5561)(270,86.4022)(271,85.2572)(272,84.1209)(273,82.9933)(274,81.8742)(275,80.7637)(276,79.6615)(277,78.5678)(278,77.4822)(279,76.4049)(280,75.3358)(281,74.2746)(282,73.2214)(283,72.1761)(284,71.1386)(285,70.1088)(286,69.0867)(287,68.0721)(288,67.065)(289,66.0652)(290,65.0728)(291,64.0876)(292,63.1095)(293,62.1385)(294,61.1745)(295,60.2173)(296,59.267)(297,58.3233)(298,57.3863)(299,56.4558)(300,55.5318)(301,54.6141)(302,53.7027)(303,52.7975)(304,51.8984)(305,51.0053)(306,50.1182)(307,49.2369)(308,48.3613)(309,47.4914)(310,46.6271)(311,45.7683)(312,44.9149)(313,44.0667)(314,43.2238)(315,42.3861)(316,41.5534)(317,40.7256)(318,39.9027)(319,39.0846)(320,38.2712)(321,37.4623)(322,36.658)(323,35.8581)(324,35.0626)(325,34.2713)(326,33.4841)(327,32.701)(328,31.9219)(329,31.1467)(330,30.3753)(331,29.6076)(332,28.8436)(333,28.083)(334,27.326)(335,26.5722)(336,25.8218)(337,25.0745)(338,24.3304)(339,23.5892)(340,22.851)(341,22.1155)(342,21.3829)(343,20.6528)(344,19.9254)(345,19.2004)(346,18.4778)(347,17.7575)(348,17.0395)(349,16.3235)(350,15.6096)(351,14.8977)(352,14.1876)(353,13.4793)(354,12.7727)(355,12.0676)(356,11.3641)(357,10.662)(358,9.96129)(359,9.2618)(360,8.56347)(361,7.86621)(362,7.16995)(363,6.47459)(364,5.78004)(365,5.08622)(366,4.39304)(367,3.70041)(368,3.00825)(369,2.31647)(370,1.62498) 
};

\end{axis}
\end{tikzpicture}

			\newframe
			\begin{tikzpicture}[scale=0.5]

\begin{axis}[%
scale only axis,
width=4.52083in,
height=3.56562in,
xmin=0, xmax=400,
ymin=0, ymax=400,
xlabel={Slab Length [cm]},
ylabel={Power [W]},
title={$\text{tstep }= 36$},
axis on top]
\addplot [
color=blue,
solid
]
coordinates{
 (1,13.9911)(2,20.3593)(3,26.7197)(4,33.0699)(5,39.4076)(6,45.7302)(7,52.0354)(8,58.3208)(9,64.584)(10,70.8228)(11,77.0348)(12,83.2176)(13,89.3691)(14,95.487)(15,101.569)(16,107.613)(17,113.617)(18,119.578)(19,125.496)(20,131.366)(21,137.189)(22,142.96)(23,148.68)(24,154.345)(25,159.955)(26,165.506)(27,170.998)(28,176.428)(29,181.795)(30,187.098)(31,192.335)(32,197.504)(33,202.604)(34,207.633)(35,212.591)(36,217.475)(37,222.285)(38,227.019)(39,231.677)(40,236.256)(41,240.757)(42,245.178)(43,249.519)(44,253.778)(45,257.954)(46,262.048)(47,266.057)(48,269.982)(49,273.822)(50,277.577)(51,281.245)(52,284.827)(53,288.323)(54,291.731)(55,295.052)(56,298.285)(57,301.43)(58,304.488)(59,307.457)(60,310.339)(61,313.133)(62,315.839)(63,318.457)(64,320.988)(65,323.431)(66,325.788)(67,328.057)(68,330.241)(69,332.338)(70,334.35)(71,336.276)(72,338.118)(73,339.876)(74,341.55)(75,343.141)(76,344.65)(77,346.078)(78,347.424)(79,348.69)(80,349.877)(81,350.985)(82,352.015)(83,352.967)(84,353.844)(85,354.645)(86,355.372)(87,356.025)(88,356.606)(89,357.114)(90,357.553)(91,357.921)(92,358.22)(93,358.452)(94,358.617)(95,358.716)(96,358.751)(97,358.722)(98,358.63)(99,358.477)(100,358.264)(101,357.991)(102,357.659)(103,357.271)(104,356.826)(105,356.327)(106,355.774)(107,355.167)(108,354.509)(109,353.801)(110,353.043)(111,352.236)(112,351.382)(113,350.482)(114,349.537)(115,348.547)(116,347.515)(117,346.441)(118,345.325)(119,344.17)(120,342.976)(121,341.744)(122,340.476)(123,339.172)(124,337.833)(125,336.461)(126,335.056)(127,333.619)(128,332.152)(129,330.654)(130,329.128)(131,327.575)(132,325.994)(133,324.387)(134,322.756)(135,321.1)(136,319.421)(137,317.719)(138,315.997)(139,314.253)(140,312.49)(141,310.707)(142,308.907)(143,307.089)(144,305.254)(145,303.404)(146,301.538)(147,299.659)(148,297.765)(149,295.859)(150,293.941)(151,292.012)(152,290.071)(153,288.121)(154,286.161)(155,284.192)(156,282.216)(157,280.232)(158,278.24)(159,276.243)(160,274.24)(161,272.232)(162,270.219)(163,268.202)(164,266.181)(165,264.158)(166,262.132)(167,260.104)(168,258.075)(169,256.044)(170,254.013)(171,251.982)(172,249.951)(173,247.92)(174,245.891)(175,243.864)(176,241.838)(177,239.815)(178,237.794)(179,235.776)(180,233.762)(181,231.752)(182,229.745)(183,227.743)(184,225.746)(185,223.753)(186,221.766)(187,219.785)(188,217.809)(189,215.839)(190,213.876)(191,211.919)(192,209.969)(193,208.027)(194,206.091)(195,204.163)(196,202.242)(197,200.33)(198,198.425)(199,196.529)(200,194.641)(201,192.762)(202,190.892)(203,189.03)(204,187.178)(205,185.334)(206,183.5)(207,181.676)(208,179.861)(209,178.055)(210,176.26)(211,174.474)(212,172.698)(213,170.933)(214,169.177)(215,167.432)(216,165.697)(217,163.972)(218,162.258)(219,160.554)(220,158.861)(221,157.178)(222,155.506)(223,153.845)(224,152.194)(225,150.554)(226,148.925)(227,147.307)(228,145.699)(229,144.102)(230,142.516)(231,140.941)(232,139.377)(233,137.823)(234,136.281)(235,134.749)(236,133.228)(237,131.718)(238,130.218)(239,128.73)(240,127.252)(241,125.785)(242,124.328)(243,122.883)(244,121.447)(245,120.023)(246,118.609)(247,117.206)(248,115.813)(249,114.43)(250,113.058)(251,111.697)(252,110.345)(253,109.004)(254,107.673)(255,106.352)(256,105.042)(257,103.741)(258,102.451)(259,101.17)(260,99.899)(261,98.6379)(262,97.3866)(263,96.1449)(264,94.9129)(265,93.6903)(266,92.4773)(267,91.2736)(268,90.0792)(269,88.8941)(270,87.7182)(271,86.5513)(272,85.3934)(273,84.2445)(274,83.1045)(275,81.9733)(276,80.8507)(277,79.7368)(278,78.6314)(279,77.5345)(280,76.4459)(281,75.3657)(282,74.2937)(283,73.2298)(284,72.1739)(285,71.126)(286,70.086)(287,69.0538)(288,68.0293)(289,67.0123)(290,66.003)(291,65.001)(292,64.0064)(293,63.0191)(294,62.039)(295,61.0659)(296,60.0999)(297,59.1407)(298,58.1884)(299,57.2428)(300,56.3038)(301,55.3714)(302,54.4454)(303,53.5259)(304,52.6125)(305,51.7054)(306,50.8044)(307,49.9094)(308,49.0203)(309,48.137)(310,47.2594)(311,46.3875)(312,45.5212)(313,44.6603)(314,43.8047)(315,42.9545)(316,42.1094)(317,41.2694)(318,40.4344)(319,39.6043)(320,38.779)(321,37.9585)(322,37.1426)(323,36.3312)(324,35.5243)(325,34.7217)(326,33.9234)(327,33.1293)(328,32.3393)(329,31.5532)(330,30.7711)(331,29.9928)(332,29.2182)(333,28.4472)(334,27.6798)(335,26.9158)(336,26.1551)(337,25.3978)(338,24.6436)(339,23.8925)(340,23.1443)(341,22.3991)(342,21.6567)(343,20.917)(344,20.18)(345,19.4455)(346,18.7134)(347,17.9837)(348,17.2563)(349,16.531)(350,15.8079)(351,15.0867)(352,14.3675)(353,13.65)(354,12.9343)(355,12.2203)(356,11.5078)(357,10.7967)(358,10.0871)(359,9.37866)(360,8.67146)(361,7.96537)(362,7.26029)(363,6.55613)(364,5.85281)(365,5.15023)(366,4.44831)(367,3.74696)(368,3.04609)(369,2.3456)(370,1.64541) 
};

\addplot [
color=red,
solid
]
coordinates{
 (1,5.45044)(2,8.07675)(3,10.9155)(4,14.0415)(5,17.5368)(6,21.4936)(7,26.016)(8,31.2233)(9,37.2528)(10,43.2644)(11,49.2557)(12,55.2244)(13,61.1684)(14,67.0854)(15,72.9733)(16,78.8297)(17,84.6528)(18,90.4402)(19,96.19)(20,101.9)(21,107.569)(22,113.193)(23,118.772)(24,124.304)(25,129.786)(26,135.217)(27,140.596)(28,145.919)(29,151.187)(30,156.396)(31,161.546)(32,166.634)(33,171.66)(34,176.622)(35,181.519)(36,186.349)(37,191.111)(38,195.803)(39,200.426)(40,204.976)(41,209.454)(42,213.858)(43,218.188)(44,222.442)(45,226.619)(46,230.719)(47,234.741)(48,238.685)(49,242.549)(50,246.333)(51,250.036)(52,253.658)(53,257.199)(54,260.658)(55,264.035)(56,267.329)(57,270.54)(58,273.669)(59,276.714)(60,279.676)(61,282.554)(62,285.35)(63,288.062)(64,290.691)(65,293.236)(66,295.699)(67,298.08)(68,300.377)(69,302.593)(70,304.726)(71,306.779)(72,308.75)(73,310.64)(74,312.451)(75,314.181)(76,315.833)(77,317.406)(78,318.9)(79,320.318)(80,321.659)(81,322.924)(82,324.113)(83,325.228)(84,326.269)(85,327.237)(86,328.133)(87,328.957)(88,329.711)(89,330.395)(90,331.01)(91,331.557)(92,332.037)(93,332.451)(94,332.8)(95,333.085)(96,333.306)(97,333.465)(98,333.562)(99,333.599)(100,333.577)(101,333.496)(102,333.358)(103,333.164)(104,332.914)(105,332.61)(106,332.253)(107,331.844)(108,331.383)(109,330.872)(110,330.312)(111,329.704)(112,329.049)(113,328.347)(114,327.601)(115,326.81)(116,325.977)(117,325.101)(118,324.185)(119,323.229)(120,322.233)(121,321.2)(122,320.129)(123,319.023)(124,317.882)(125,316.706)(126,315.497)(127,314.256)(128,312.984)(129,311.682)(130,310.35)(131,308.99)(132,307.602)(133,306.187)(134,304.747)(135,303.281)(136,301.792)(137,300.279)(138,298.744)(139,297.187)(140,295.61)(141,294.012)(142,292.396)(143,290.761)(144,289.108)(145,287.439)(146,285.753)(147,284.052)(148,282.336)(149,280.607)(150,278.864)(151,277.108)(152,275.341)(153,273.562)(154,271.772)(155,269.973)(156,268.164)(157,266.347)(158,264.521)(159,262.688)(160,260.848)(161,259.001)(162,257.148)(163,255.29)(164,253.427)(165,251.56)(166,249.689)(167,247.815)(168,245.938)(169,244.058)(170,242.177)(171,240.294)(172,238.41)(173,236.525)(174,234.64)(175,232.756)(176,230.872)(177,228.988)(178,227.107)(179,225.226)(180,223.348)(181,221.473)(182,219.6)(183,217.73)(184,215.863)(185,214)(186,212.141)(187,210.287)(188,208.436)(189,206.591)(190,204.75)(191,202.915)(192,201.086)(193,199.262)(194,197.444)(195,195.632)(196,193.827)(197,192.028)(198,190.237)(199,188.452)(200,186.674)(201,184.904)(202,183.141)(203,181.386)(204,179.639)(205,177.899)(206,176.168)(207,174.445)(208,172.731)(209,171.025)(210,169.327)(211,167.639)(212,165.959)(213,164.288)(214,162.626)(215,160.973)(216,159.329)(217,157.695)(218,156.07)(219,154.454)(220,152.848)(221,151.251)(222,149.664)(223,148.087)(224,146.519)(225,144.961)(226,143.413)(227,141.874)(228,140.345)(229,138.827)(230,137.317)(231,135.818)(232,134.329)(233,132.85)(234,131.38)(235,129.921)(236,128.471)(237,127.031)(238,125.601)(239,124.181)(240,122.771)(241,121.371)(242,119.981)(243,118.6)(244,117.229)(245,115.868)(246,114.517)(247,113.176)(248,111.844)(249,110.522)(250,109.21)(251,107.907)(252,106.613)(253,105.33)(254,104.055)(255,102.79)(256,101.535)(257,100.289)(258,99.0518)(259,97.8241)(260,96.6056)(261,95.3962)(262,94.1959)(263,93.0046)(264,91.8222)(265,90.6487)(266,89.4841)(267,88.3282)(268,87.181)(269,86.0425)(270,84.9125)(271,83.7911)(272,82.678)(273,81.5734)(274,80.477)(275,79.3889)(276,78.3089)(277,77.237)(278,76.1731)(279,75.1172)(280,74.0691)(281,73.0288)(282,71.9962)(283,70.9712)(284,69.9538)(285,68.9439)(286,67.9414)(287,66.9462)(288,65.9583)(289,64.9775)(290,64.0038)(291,63.0371)(292,62.0773)(293,61.1244)(294,60.1782)(295,59.2388)(296,58.3059)(297,57.3795)(298,56.4595)(299,55.5459)(300,54.6386)(301,53.7374)(302,52.8424)(303,51.9533)(304,51.0702)(305,50.1929)(306,49.3214)(307,48.4556)(308,47.5953)(309,46.7406)(310,45.8912)(311,45.0472)(312,44.2085)(313,43.3749)(314,42.5463)(315,41.7228)(316,40.9042)(317,40.0904)(318,39.2813)(319,38.4769)(320,37.677)(321,36.8816)(322,36.0906)(323,35.3039)(324,34.5215)(325,33.7431)(326,32.9688)(327,32.1985)(328,31.432)(329,30.6693)(330,29.9103)(331,29.1549)(332,28.4031)(333,27.6547)(334,26.9096)(335,26.1678)(336,25.4292)(337,24.6937)(338,23.9613)(339,23.2317)(340,22.505)(341,21.781)(342,21.0597)(343,20.341)(344,19.6248)(345,18.911)(346,18.1996)(347,17.4903)(348,16.7833)(349,16.0783)(350,15.3753)(351,14.6741)(352,13.9748)(353,13.2773)(354,12.5814)(355,11.887)(356,11.1941)(357,10.5026)(358,9.8124)(359,9.12342)(360,8.43557)(361,7.74877)(362,7.06294)(363,6.37798)(364,5.69382)(365,5.01037)(366,4.32754)(367,3.64525)(368,2.96341)(369,2.28195)(370,1.60076) 
};

\end{axis}
\end{tikzpicture}

			\newframe
			\begin{tikzpicture}[scale=0.5]

\begin{axis}[%
scale only axis,
width=4.52083in,
height=3.56562in,
xmin=0, xmax=400,
ymin=0, ymax=400,
xlabel={Slab Length [cm]},
ylabel={Power [W]},
title={$\text{tstep }= 42$},
axis on top]
\addplot [
color=blue,
solid
]
coordinates{
 (1,13.9911)(2,20.3593)(3,26.7197)(4,33.0699)(5,39.4076)(6,45.7302)(7,52.0354)(8,58.3208)(9,64.584)(10,70.8228)(11,77.0348)(12,83.2176)(13,89.3691)(14,95.487)(15,101.569)(16,107.613)(17,113.617)(18,119.578)(19,125.496)(20,131.366)(21,137.189)(22,142.96)(23,148.68)(24,154.345)(25,159.955)(26,165.506)(27,170.998)(28,176.428)(29,181.795)(30,187.098)(31,192.335)(32,197.504)(33,202.604)(34,207.633)(35,212.591)(36,217.475)(37,222.285)(38,227.019)(39,231.677)(40,236.256)(41,240.757)(42,245.178)(43,249.519)(44,253.778)(45,257.954)(46,262.048)(47,266.057)(48,269.982)(49,273.822)(50,277.577)(51,281.245)(52,284.827)(53,288.323)(54,291.731)(55,295.052)(56,298.285)(57,301.43)(58,304.488)(59,307.457)(60,310.339)(61,313.133)(62,315.839)(63,318.457)(64,320.988)(65,323.431)(66,325.788)(67,328.057)(68,330.241)(69,332.338)(70,334.35)(71,336.276)(72,338.118)(73,339.876)(74,341.55)(75,343.141)(76,344.65)(77,346.078)(78,347.424)(79,348.69)(80,349.877)(81,350.985)(82,352.015)(83,352.967)(84,353.844)(85,354.645)(86,355.372)(87,356.025)(88,356.606)(89,357.114)(90,357.553)(91,357.921)(92,358.22)(93,358.452)(94,358.617)(95,358.716)(96,358.751)(97,358.722)(98,358.63)(99,358.477)(100,358.264)(101,357.991)(102,357.659)(103,357.271)(104,356.826)(105,356.327)(106,355.774)(107,355.167)(108,354.509)(109,353.801)(110,353.043)(111,352.236)(112,351.382)(113,350.482)(114,349.537)(115,348.547)(116,347.515)(117,346.441)(118,345.325)(119,344.17)(120,342.976)(121,341.744)(122,340.476)(123,339.172)(124,337.833)(125,336.461)(126,335.056)(127,333.619)(128,332.152)(129,330.654)(130,329.128)(131,327.575)(132,325.994)(133,324.387)(134,322.756)(135,321.1)(136,319.421)(137,317.719)(138,315.997)(139,314.253)(140,312.49)(141,310.707)(142,308.907)(143,307.089)(144,305.254)(145,303.404)(146,301.538)(147,299.659)(148,297.765)(149,295.859)(150,293.941)(151,292.012)(152,290.071)(153,288.121)(154,286.161)(155,284.192)(156,282.216)(157,280.232)(158,278.24)(159,276.243)(160,274.24)(161,272.232)(162,270.219)(163,268.202)(164,266.181)(165,264.158)(166,262.132)(167,260.104)(168,258.075)(169,256.044)(170,254.013)(171,251.982)(172,249.951)(173,247.92)(174,245.891)(175,243.864)(176,241.838)(177,239.815)(178,237.794)(179,235.776)(180,233.762)(181,231.752)(182,229.745)(183,227.743)(184,225.746)(185,223.753)(186,221.766)(187,219.785)(188,217.809)(189,215.839)(190,213.876)(191,211.919)(192,209.969)(193,208.027)(194,206.091)(195,204.163)(196,202.242)(197,200.33)(198,198.425)(199,196.529)(200,194.641)(201,192.762)(202,190.892)(203,189.03)(204,187.178)(205,185.334)(206,183.5)(207,181.676)(208,179.861)(209,178.055)(210,176.26)(211,174.474)(212,172.698)(213,170.933)(214,169.177)(215,167.432)(216,165.697)(217,163.972)(218,162.258)(219,160.554)(220,158.861)(221,157.178)(222,155.506)(223,153.845)(224,152.194)(225,150.554)(226,148.925)(227,147.307)(228,145.699)(229,144.102)(230,142.516)(231,140.941)(232,139.377)(233,137.823)(234,136.281)(235,134.749)(236,133.228)(237,131.718)(238,130.218)(239,128.73)(240,127.252)(241,125.785)(242,124.328)(243,122.883)(244,121.447)(245,120.023)(246,118.609)(247,117.206)(248,115.813)(249,114.43)(250,113.058)(251,111.697)(252,110.345)(253,109.004)(254,107.673)(255,106.352)(256,105.042)(257,103.741)(258,102.451)(259,101.17)(260,99.899)(261,98.6379)(262,97.3866)(263,96.1449)(264,94.9129)(265,93.6903)(266,92.4773)(267,91.2736)(268,90.0792)(269,88.8941)(270,87.7182)(271,86.5513)(272,85.3934)(273,84.2445)(274,83.1045)(275,81.9733)(276,80.8507)(277,79.7368)(278,78.6314)(279,77.5345)(280,76.4459)(281,75.3657)(282,74.2937)(283,73.2298)(284,72.1739)(285,71.126)(286,70.086)(287,69.0538)(288,68.0293)(289,67.0123)(290,66.003)(291,65.001)(292,64.0064)(293,63.0191)(294,62.039)(295,61.0659)(296,60.0999)(297,59.1407)(298,58.1884)(299,57.2428)(300,56.3038)(301,55.3714)(302,54.4454)(303,53.5259)(304,52.6125)(305,51.7054)(306,50.8044)(307,49.9094)(308,49.0203)(309,48.137)(310,47.2594)(311,46.3875)(312,45.5212)(313,44.6603)(314,43.8047)(315,42.9545)(316,42.1094)(317,41.2694)(318,40.4344)(319,39.6043)(320,38.779)(321,37.9585)(322,37.1426)(323,36.3312)(324,35.5243)(325,34.7217)(326,33.9234)(327,33.1293)(328,32.3393)(329,31.5532)(330,30.7711)(331,29.9928)(332,29.2182)(333,28.4472)(334,27.6798)(335,26.9158)(336,26.1551)(337,25.3978)(338,24.6436)(339,23.8925)(340,23.1443)(341,22.3991)(342,21.6567)(343,20.917)(344,20.18)(345,19.4455)(346,18.7134)(347,17.9837)(348,17.2563)(349,16.531)(350,15.8079)(351,15.0867)(352,14.3675)(353,13.65)(354,12.9343)(355,12.2203)(356,11.5078)(357,10.7967)(358,10.0871)(359,9.37866)(360,8.67146)(361,7.96537)(362,7.26029)(363,6.55613)(364,5.85281)(365,5.15023)(366,4.44831)(367,3.74696)(368,3.04609)(369,2.3456)(370,1.64541) 
};

\addplot [
color=red,
solid
]
coordinates{
 (1,3.86178)(2,5.72245)(3,7.73341)(4,9.94747)(5,12.4228)(6,15.2244)(7,18.426)(8,22.1118)(9,26.3788)(10,31.3394)(11,37.1244)(12,42.8906)(13,48.6358)(14,54.3578)(15,60.0546)(16,65.7239)(17,71.3638)(18,76.972)(19,82.5466)(20,88.0856)(21,93.587)(22,99.0489)(23,104.469)(24,109.846)(25,115.178)(26,120.463)(27,125.699)(28,130.885)(29,136.019)(30,141.099)(31,146.123)(32,151.091)(33,156.001)(34,160.85)(35,165.639)(36,170.365)(37,175.027)(38,179.624)(39,184.155)(40,188.619)(41,193.014)(42,197.34)(43,201.594)(44,205.778)(45,209.889)(46,213.927)(47,217.891)(48,221.78)(49,225.594)(50,229.332)(51,232.993)(52,236.577)(53,240.083)(54,243.511)(55,246.86)(56,250.131)(57,253.323)(58,256.435)(59,259.468)(60,262.421)(61,265.294)(62,268.088)(63,270.801)(64,273.435)(65,275.989)(66,278.463)(67,280.858)(68,283.173)(69,285.409)(70,287.567)(71,289.646)(72,291.647)(73,293.57)(74,295.415)(75,297.184)(76,298.876)(77,300.493)(78,302.034)(79,303.5)(80,304.892)(81,306.21)(82,307.455)(83,308.628)(84,309.729)(85,310.76)(86,311.72)(87,312.611)(88,313.434)(89,314.188)(90,314.876)(91,315.497)(92,316.054)(93,316.545)(94,316.974)(95,317.339)(96,317.643)(97,317.886)(98,318.069)(99,318.193)(100,318.259)(101,318.268)(102,318.221)(103,318.119)(104,317.963)(105,317.753)(106,317.491)(107,317.178)(108,316.815)(109,316.402)(110,315.941)(111,315.433)(112,314.878)(113,314.278)(114,313.634)(115,312.946)(116,312.216)(117,311.444)(118,310.632)(119,309.78)(120,308.89)(121,307.962)(122,306.998)(123,305.998)(124,304.963)(125,303.894)(126,302.793)(127,301.659)(128,300.494)(129,299.299)(130,298.075)(131,296.822)(132,295.542)(133,294.235)(134,292.902)(135,291.544)(136,290.162)(137,288.757)(138,287.329)(139,285.879)(140,284.409)(141,282.918)(142,281.408)(143,279.879)(144,278.332)(145,276.768)(146,275.188)(147,273.591)(148,271.98)(149,270.355)(150,268.716)(151,267.063)(152,265.399)(153,263.723)(154,262.035)(155,260.337)(156,258.63)(157,256.913)(158,255.187)(159,253.453)(160,251.712)(161,249.964)(162,248.209)(163,246.449)(164,244.682)(165,242.911)(166,241.136)(167,239.357)(168,237.574)(169,235.788)(170,234)(171,232.209)(172,230.417)(173,228.623)(174,226.829)(175,225.034)(176,223.239)(177,221.444)(178,219.65)(179,217.857)(180,216.066)(181,214.276)(182,212.488)(183,210.703)(184,208.92)(185,207.14)(186,205.363)(187,203.59)(188,201.821)(189,200.055)(190,198.294)(191,196.538)(192,194.787)(193,193.04)(194,191.299)(195,189.563)(196,187.833)(197,186.109)(198,184.391)(199,182.68)(200,180.975)(201,179.276)(202,177.584)(203,175.9)(204,174.222)(205,172.552)(206,170.889)(207,169.234)(208,167.586)(209,165.947)(210,164.315)(211,162.691)(212,161.076)(213,159.468)(214,157.87)(215,156.279)(216,154.697)(217,153.124)(218,151.559)(219,150.003)(220,148.456)(221,146.918)(222,145.389)(223,143.869)(224,142.358)(225,140.855)(226,139.363)(227,137.879)(228,136.404)(229,134.939)(230,133.483)(231,132.036)(232,130.599)(233,129.171)(234,127.752)(235,126.342)(236,124.942)(237,123.552)(238,122.17)(239,120.798)(240,119.435)(241,118.082)(242,116.738)(243,115.403)(244,114.078)(245,112.761)(246,111.454)(247,110.157)(248,108.868)(249,107.589)(250,106.319)(251,105.057)(252,103.805)(253,102.562)(254,101.328)(255,100.103)(256,98.8871)(257,97.6798)(258,96.4814)(259,95.2917)(260,94.1108)(261,92.9385)(262,91.7749)(263,90.6199)(264,89.4734)(265,88.3354)(266,87.2057)(267,86.0845)(268,84.9715)(269,83.8668)(270,82.7703)(271,81.6818)(272,80.6015)(273,79.5291)(274,78.4647)(275,77.4081)(276,76.3593)(277,75.3182)(278,74.2848)(279,73.259)(280,72.2407)(281,71.2298)(282,70.2263)(283,69.2301)(284,68.2412)(285,67.2594)(286,66.2847)(287,65.317)(288,64.3563)(289,63.4024)(290,62.4553)(291,61.5149)(292,60.5812)(293,59.654)(294,58.7333)(295,57.819)(296,56.911)(297,56.0092)(298,55.1136)(299,54.2242)(300,53.3407)(301,52.4631)(302,51.5914)(303,50.7255)(304,49.8653)(305,49.0106)(306,48.1615)(307,47.3179)(308,46.4796)(309,45.6466)(310,44.8188)(311,43.9961)(312,43.1785)(313,42.3658)(314,41.558)(315,40.755)(316,39.9567)(317,39.163)(318,38.3739)(319,37.5893)(320,36.809)(321,36.0331)(322,35.2613)(323,34.4937)(324,33.7302)(325,32.9706)(326,32.2149)(327,31.463)(328,30.7149)(329,29.9704)(330,29.2294)(331,28.492)(332,27.7579)(333,27.0271)(334,26.2996)(335,25.5752)(336,24.8539)(337,24.1356)(338,23.4201)(339,22.7075)(340,21.9976)(341,21.2904)(342,20.5857)(343,19.8836)(344,19.1838)(345,18.4863)(346,17.7911)(347,17.0981)(348,16.4071)(349,15.7182)(350,15.0311)(351,14.3459)(352,13.6624)(353,12.9806)(354,12.3003)(355,11.6216)(356,10.9443)(357,10.2683)(358,9.5936)(359,8.92006)(360,8.2476)(361,7.57616)(362,6.90565)(363,6.23599)(364,5.56709)(365,4.89887)(366,4.23125)(367,3.56415)(368,2.8975)(369,2.23119)(370,1.56516) 
};

\end{axis}
\end{tikzpicture}

			\newframe
			\begin{tikzpicture}[scale=0.5]

\begin{axis}[%
scale only axis,
width=4.52083in,
height=3.56562in,
xmin=0, xmax=400,
ymin=0, ymax=400,
xlabel={Slab Length [cm]},
ylabel={Power [W]},
title={$\text{tstep }= 48$},
axis on top]
\addplot [
color=blue,
solid
]
coordinates{
 (1,13.9911)(2,20.3593)(3,26.7197)(4,33.0699)(5,39.4076)(6,45.7302)(7,52.0354)(8,58.3208)(9,64.584)(10,70.8228)(11,77.0348)(12,83.2176)(13,89.3691)(14,95.487)(15,101.569)(16,107.613)(17,113.617)(18,119.578)(19,125.496)(20,131.366)(21,137.189)(22,142.96)(23,148.68)(24,154.345)(25,159.955)(26,165.506)(27,170.998)(28,176.428)(29,181.795)(30,187.098)(31,192.335)(32,197.504)(33,202.604)(34,207.633)(35,212.591)(36,217.475)(37,222.285)(38,227.019)(39,231.677)(40,236.256)(41,240.757)(42,245.178)(43,249.519)(44,253.778)(45,257.954)(46,262.048)(47,266.057)(48,269.982)(49,273.822)(50,277.577)(51,281.245)(52,284.827)(53,288.323)(54,291.731)(55,295.052)(56,298.285)(57,301.43)(58,304.488)(59,307.457)(60,310.339)(61,313.133)(62,315.839)(63,318.457)(64,320.988)(65,323.431)(66,325.788)(67,328.057)(68,330.241)(69,332.338)(70,334.35)(71,336.276)(72,338.118)(73,339.876)(74,341.55)(75,343.141)(76,344.65)(77,346.078)(78,347.424)(79,348.69)(80,349.877)(81,350.985)(82,352.015)(83,352.967)(84,353.844)(85,354.645)(86,355.372)(87,356.025)(88,356.606)(89,357.114)(90,357.553)(91,357.921)(92,358.22)(93,358.452)(94,358.617)(95,358.716)(96,358.751)(97,358.722)(98,358.63)(99,358.477)(100,358.264)(101,357.991)(102,357.659)(103,357.271)(104,356.826)(105,356.327)(106,355.774)(107,355.167)(108,354.509)(109,353.801)(110,353.043)(111,352.236)(112,351.382)(113,350.482)(114,349.537)(115,348.547)(116,347.515)(117,346.441)(118,345.325)(119,344.17)(120,342.976)(121,341.744)(122,340.476)(123,339.172)(124,337.833)(125,336.461)(126,335.056)(127,333.619)(128,332.152)(129,330.654)(130,329.128)(131,327.575)(132,325.994)(133,324.387)(134,322.756)(135,321.1)(136,319.421)(137,317.719)(138,315.997)(139,314.253)(140,312.49)(141,310.707)(142,308.907)(143,307.089)(144,305.254)(145,303.404)(146,301.538)(147,299.659)(148,297.765)(149,295.859)(150,293.941)(151,292.012)(152,290.071)(153,288.121)(154,286.161)(155,284.192)(156,282.216)(157,280.232)(158,278.24)(159,276.243)(160,274.24)(161,272.232)(162,270.219)(163,268.202)(164,266.181)(165,264.158)(166,262.132)(167,260.104)(168,258.075)(169,256.044)(170,254.013)(171,251.982)(172,249.951)(173,247.92)(174,245.891)(175,243.864)(176,241.838)(177,239.815)(178,237.794)(179,235.776)(180,233.762)(181,231.752)(182,229.745)(183,227.743)(184,225.746)(185,223.753)(186,221.766)(187,219.785)(188,217.809)(189,215.839)(190,213.876)(191,211.919)(192,209.969)(193,208.027)(194,206.091)(195,204.163)(196,202.242)(197,200.33)(198,198.425)(199,196.529)(200,194.641)(201,192.762)(202,190.892)(203,189.03)(204,187.178)(205,185.334)(206,183.5)(207,181.676)(208,179.861)(209,178.055)(210,176.26)(211,174.474)(212,172.698)(213,170.933)(214,169.177)(215,167.432)(216,165.697)(217,163.972)(218,162.258)(219,160.554)(220,158.861)(221,157.178)(222,155.506)(223,153.845)(224,152.194)(225,150.554)(226,148.925)(227,147.307)(228,145.699)(229,144.102)(230,142.516)(231,140.941)(232,139.377)(233,137.823)(234,136.281)(235,134.749)(236,133.228)(237,131.718)(238,130.218)(239,128.73)(240,127.252)(241,125.785)(242,124.328)(243,122.883)(244,121.447)(245,120.023)(246,118.609)(247,117.206)(248,115.813)(249,114.43)(250,113.058)(251,111.697)(252,110.345)(253,109.004)(254,107.673)(255,106.352)(256,105.042)(257,103.741)(258,102.451)(259,101.17)(260,99.899)(261,98.6379)(262,97.3866)(263,96.1449)(264,94.9129)(265,93.6903)(266,92.4773)(267,91.2736)(268,90.0792)(269,88.8941)(270,87.7182)(271,86.5513)(272,85.3934)(273,84.2445)(274,83.1045)(275,81.9733)(276,80.8507)(277,79.7368)(278,78.6314)(279,77.5345)(280,76.4459)(281,75.3657)(282,74.2937)(283,73.2298)(284,72.1739)(285,71.126)(286,70.086)(287,69.0538)(288,68.0293)(289,67.0123)(290,66.003)(291,65.001)(292,64.0064)(293,63.0191)(294,62.039)(295,61.0659)(296,60.0999)(297,59.1407)(298,58.1884)(299,57.2428)(300,56.3038)(301,55.3714)(302,54.4454)(303,53.5259)(304,52.6125)(305,51.7054)(306,50.8044)(307,49.9094)(308,49.0203)(309,48.137)(310,47.2594)(311,46.3875)(312,45.5212)(313,44.6603)(314,43.8047)(315,42.9545)(316,42.1094)(317,41.2694)(318,40.4344)(319,39.6043)(320,38.779)(321,37.9585)(322,37.1426)(323,36.3312)(324,35.5243)(325,34.7217)(326,33.9234)(327,33.1293)(328,32.3393)(329,31.5532)(330,30.7711)(331,29.9928)(332,29.2182)(333,28.4472)(334,27.6798)(335,26.9158)(336,26.1551)(337,25.3978)(338,24.6436)(339,23.8925)(340,23.1443)(341,22.3991)(342,21.6567)(343,20.917)(344,20.18)(345,19.4455)(346,18.7134)(347,17.9837)(348,17.2563)(349,16.531)(350,15.8079)(351,15.0867)(352,14.3675)(353,13.65)(354,12.9343)(355,12.2203)(356,11.5078)(357,10.7967)(358,10.0871)(359,9.37866)(360,8.67146)(361,7.96537)(362,7.26029)(363,6.55613)(364,5.85281)(365,5.15023)(366,4.44831)(367,3.74696)(368,3.04609)(369,2.3456)(370,1.64541) 
};

\addplot [
color=red,
solid
]
coordinates{
 (1,2.67873)(2,3.96928)(3,5.36392)(4,6.89922)(5,8.61543)(6,10.5576)(7,12.7766)(8,15.3308)(9,18.2873)(10,21.7237)(11,25.7305)(12,30.4131)(13,35.8949)(14,41.3578)(15,46.7997)(16,52.2187)(17,57.6125)(18,62.9793)(19,68.3171)(20,73.6238)(21,78.8976)(22,84.1365)(23,89.3387)(24,94.5024)(25,99.6257)(26,104.707)(27,109.744)(28,114.736)(29,119.681)(30,124.577)(31,129.423)(32,134.217)(33,138.957)(34,143.643)(35,148.273)(36,152.845)(37,157.358)(38,161.812)(39,166.204)(40,170.534)(41,174.8)(42,179.002)(43,183.138)(44,187.208)(45,191.21)(46,195.145)(47,199.01)(48,202.805)(49,206.53)(50,210.183)(51,213.765)(52,217.274)(53,220.71)(54,224.073)(55,227.362)(56,230.577)(57,233.718)(58,236.783)(59,239.774)(60,242.689)(61,245.529)(62,248.293)(63,250.982)(64,253.595)(65,256.132)(66,258.594)(67,260.981)(68,263.292)(69,265.528)(70,267.689)(71,269.775)(72,271.787)(73,273.725)(74,275.59)(75,277.38)(76,279.098)(77,280.744)(78,282.317)(79,283.819)(80,285.25)(81,286.61)(82,287.901)(83,289.122)(84,290.275)(85,291.36)(86,292.377)(87,293.328)(88,294.213)(89,295.033)(90,295.788)(91,296.48)(92,297.109)(93,297.676)(94,298.182)(95,298.627)(96,299.013)(97,299.34)(98,299.609)(99,299.821)(100,299.977)(101,300.078)(102,300.124)(103,300.117)(104,300.058)(105,299.947)(106,299.785)(107,299.574)(108,299.313)(109,299.005)(110,298.65)(111,298.248)(112,297.802)(113,297.311)(114,296.777)(115,296.201)(116,295.583)(117,294.925)(118,294.227)(119,293.491)(120,292.717)(121,291.905)(122,291.058)(123,290.176)(124,289.26)(125,288.31)(126,287.328)(127,286.315)(128,285.27)(129,284.196)(130,283.093)(131,281.962)(132,280.804)(133,279.619)(134,278.408)(135,277.172)(136,275.913)(137,274.63)(138,273.325)(139,271.998)(140,270.65)(141,269.281)(142,267.893)(143,266.487)(144,265.062)(145,263.62)(146,262.162)(147,260.687)(148,259.198)(149,257.693)(150,256.175)(151,254.643)(152,253.099)(153,251.542)(154,249.974)(155,248.395)(156,246.806)(157,245.207)(158,243.599)(159,241.983)(160,240.358)(161,238.726)(162,237.087)(163,235.441)(164,233.789)(165,232.132)(166,230.47)(167,228.804)(168,227.133)(169,225.458)(170,223.781)(171,222.1)(172,220.418)(173,218.733)(174,217.047)(175,215.359)(176,213.671)(177,211.983)(178,210.294)(179,208.606)(180,206.918)(181,205.231)(182,203.546)(183,201.862)(184,200.18)(185,198.5)(186,196.823)(187,195.148)(188,193.477)(189,191.809)(190,190.144)(191,188.483)(192,186.827)(193,185.174)(194,183.526)(195,181.883)(196,180.245)(197,178.612)(198,176.984)(199,175.361)(200,173.745)(201,172.134)(202,170.529)(203,168.931)(204,167.339)(205,165.753)(206,164.174)(207,162.602)(208,161.037)(209,159.479)(210,157.928)(211,156.384)(212,154.848)(213,153.319)(214,151.798)(215,150.284)(216,148.779)(217,147.281)(218,145.791)(219,144.309)(220,142.835)(221,141.369)(222,139.912)(223,138.463)(224,137.022)(225,135.59)(226,134.166)(227,132.75)(228,131.343)(229,129.944)(230,128.555)(231,127.173)(232,125.8)(233,124.436)(234,123.081)(235,121.734)(236,120.396)(237,119.067)(238,117.746)(239,116.434)(240,115.131)(241,113.836)(242,112.55)(243,111.273)(244,110.004)(245,108.744)(246,107.493)(247,106.25)(248,105.016)(249,103.791)(250,102.574)(251,101.365)(252,100.165)(253,98.9736)(254,97.7906)(255,96.6159)(256,95.4496)(257,94.2916)(258,93.1419)(259,92.0004)(260,90.8672)(261,89.7421)(262,88.6251)(263,87.5162)(264,86.4153)(265,85.3223)(266,84.2373)(267,83.1602)(268,82.0909)(269,81.0293)(270,79.9754)(271,78.9292)(272,77.8906)(273,76.8595)(274,75.8358)(275,74.8196)(276,73.8108)(277,72.8092)(278,71.8148)(279,70.8277)(280,69.8476)(281,68.8745)(282,67.9084)(283,66.9493)(284,65.9969)(285,65.0514)(286,64.1125)(287,63.1802)(288,62.2546)(289,61.3354)(290,60.4226)(291,59.5162)(292,58.616)(293,57.7221)(294,56.8343)(295,55.9526)(296,55.0768)(297,54.207)(298,53.343)(299,52.4848)(300,51.6323)(301,50.7854)(302,49.944)(303,49.1081)(304,48.2776)(305,47.4525)(306,46.6325)(307,45.8178)(308,45.0081)(309,44.2034)(310,43.4037)(311,42.6089)(312,41.8188)(313,41.0335)(314,40.2527)(315,39.4765)(316,38.7048)(317,37.9375)(318,37.1746)(319,36.4158)(320,35.6613)(321,34.9108)(322,34.1643)(323,33.4218)(324,32.6831)(325,31.9482)(326,31.217)(327,30.4894)(328,29.7653)(329,29.0448)(330,28.3276)(331,27.6137)(332,26.903)(333,26.1955)(334,25.4911)(335,24.7896)(336,24.0911)(337,23.3954)(338,22.7025)(339,22.0122)(340,21.3246)(341,20.6395)(342,19.9568)(343,19.2765)(344,18.5985)(345,17.9226)(346,17.249)(347,16.5774)(348,15.9077)(349,15.24)(350,14.5741)(351,13.9099)(352,13.2474)(353,12.5864)(354,11.927)(355,11.269)(356,10.6124)(357,9.95704)(358,9.30288)(359,8.64983)(360,7.99783)(361,7.34678)(362,6.69662)(363,6.04727)(364,5.39865)(365,4.75067)(366,4.10327)(367,3.45637)(368,2.80988)(369,2.16373)(370,1.51784) 
};

\end{axis}
\end{tikzpicture}

			\newframe
			\begin{tikzpicture}[scale=0.5]

\begin{axis}[%
scale only axis,
width=4.52083in,
height=3.56562in,
xmin=0, xmax=400,
ymin=0, ymax=400,
xlabel={Slab Length [cm]},
ylabel={Power [W]},
title={$\text{tstep }= 54$},
axis on top]
\addplot [
color=blue,
solid
]
coordinates{
 (1,13.9911)(2,20.3593)(3,26.7197)(4,33.0699)(5,39.4076)(6,45.7302)(7,52.0354)(8,58.3208)(9,64.584)(10,70.8228)(11,77.0348)(12,83.2176)(13,89.3691)(14,95.487)(15,101.569)(16,107.613)(17,113.617)(18,119.578)(19,125.496)(20,131.366)(21,137.189)(22,142.96)(23,148.68)(24,154.345)(25,159.955)(26,165.506)(27,170.998)(28,176.428)(29,181.795)(30,187.098)(31,192.335)(32,197.504)(33,202.604)(34,207.633)(35,212.591)(36,217.475)(37,222.285)(38,227.019)(39,231.677)(40,236.256)(41,240.757)(42,245.178)(43,249.519)(44,253.778)(45,257.954)(46,262.048)(47,266.057)(48,269.982)(49,273.822)(50,277.577)(51,281.245)(52,284.827)(53,288.323)(54,291.731)(55,295.052)(56,298.285)(57,301.43)(58,304.488)(59,307.457)(60,310.339)(61,313.133)(62,315.839)(63,318.457)(64,320.988)(65,323.431)(66,325.788)(67,328.057)(68,330.241)(69,332.338)(70,334.35)(71,336.276)(72,338.118)(73,339.876)(74,341.55)(75,343.141)(76,344.65)(77,346.078)(78,347.424)(79,348.69)(80,349.877)(81,350.985)(82,352.015)(83,352.967)(84,353.844)(85,354.645)(86,355.372)(87,356.025)(88,356.606)(89,357.114)(90,357.553)(91,357.921)(92,358.22)(93,358.452)(94,358.617)(95,358.716)(96,358.751)(97,358.722)(98,358.63)(99,358.477)(100,358.264)(101,357.991)(102,357.659)(103,357.271)(104,356.826)(105,356.327)(106,355.774)(107,355.167)(108,354.509)(109,353.801)(110,353.043)(111,352.236)(112,351.382)(113,350.482)(114,349.537)(115,348.547)(116,347.515)(117,346.441)(118,345.325)(119,344.17)(120,342.976)(121,341.744)(122,340.476)(123,339.172)(124,337.833)(125,336.461)(126,335.056)(127,333.619)(128,332.152)(129,330.654)(130,329.128)(131,327.575)(132,325.994)(133,324.387)(134,322.756)(135,321.1)(136,319.421)(137,317.719)(138,315.997)(139,314.253)(140,312.49)(141,310.707)(142,308.907)(143,307.089)(144,305.254)(145,303.404)(146,301.538)(147,299.659)(148,297.765)(149,295.859)(150,293.941)(151,292.012)(152,290.071)(153,288.121)(154,286.161)(155,284.192)(156,282.216)(157,280.232)(158,278.24)(159,276.243)(160,274.24)(161,272.232)(162,270.219)(163,268.202)(164,266.181)(165,264.158)(166,262.132)(167,260.104)(168,258.075)(169,256.044)(170,254.013)(171,251.982)(172,249.951)(173,247.92)(174,245.891)(175,243.864)(176,241.838)(177,239.815)(178,237.794)(179,235.776)(180,233.762)(181,231.752)(182,229.745)(183,227.743)(184,225.746)(185,223.753)(186,221.766)(187,219.785)(188,217.809)(189,215.839)(190,213.876)(191,211.919)(192,209.969)(193,208.027)(194,206.091)(195,204.163)(196,202.242)(197,200.33)(198,198.425)(199,196.529)(200,194.641)(201,192.762)(202,190.892)(203,189.03)(204,187.178)(205,185.334)(206,183.5)(207,181.676)(208,179.861)(209,178.055)(210,176.26)(211,174.474)(212,172.698)(213,170.933)(214,169.177)(215,167.432)(216,165.697)(217,163.972)(218,162.258)(219,160.554)(220,158.861)(221,157.178)(222,155.506)(223,153.845)(224,152.194)(225,150.554)(226,148.925)(227,147.307)(228,145.699)(229,144.102)(230,142.516)(231,140.941)(232,139.377)(233,137.823)(234,136.281)(235,134.749)(236,133.228)(237,131.718)(238,130.218)(239,128.73)(240,127.252)(241,125.785)(242,124.328)(243,122.883)(244,121.447)(245,120.023)(246,118.609)(247,117.206)(248,115.813)(249,114.43)(250,113.058)(251,111.697)(252,110.345)(253,109.004)(254,107.673)(255,106.352)(256,105.042)(257,103.741)(258,102.451)(259,101.17)(260,99.899)(261,98.6379)(262,97.3866)(263,96.1449)(264,94.9129)(265,93.6903)(266,92.4773)(267,91.2736)(268,90.0792)(269,88.8941)(270,87.7182)(271,86.5513)(272,85.3934)(273,84.2445)(274,83.1045)(275,81.9733)(276,80.8507)(277,79.7368)(278,78.6314)(279,77.5345)(280,76.4459)(281,75.3657)(282,74.2937)(283,73.2298)(284,72.1739)(285,71.126)(286,70.086)(287,69.0538)(288,68.0293)(289,67.0123)(290,66.003)(291,65.001)(292,64.0064)(293,63.0191)(294,62.039)(295,61.0659)(296,60.0999)(297,59.1407)(298,58.1884)(299,57.2428)(300,56.3038)(301,55.3714)(302,54.4454)(303,53.5259)(304,52.6125)(305,51.7054)(306,50.8044)(307,49.9094)(308,49.0203)(309,48.137)(310,47.2594)(311,46.3875)(312,45.5212)(313,44.6603)(314,43.8047)(315,42.9545)(316,42.1094)(317,41.2694)(318,40.4344)(319,39.6043)(320,38.779)(321,37.9585)(322,37.1426)(323,36.3312)(324,35.5243)(325,34.7217)(326,33.9234)(327,33.1293)(328,32.3393)(329,31.5532)(330,30.7711)(331,29.9928)(332,29.2182)(333,28.4472)(334,27.6798)(335,26.9158)(336,26.1551)(337,25.3978)(338,24.6436)(339,23.8925)(340,23.1443)(341,22.3991)(342,21.6567)(343,20.917)(344,20.18)(345,19.4455)(346,18.7134)(347,17.9837)(348,17.2563)(349,16.531)(350,15.8079)(351,15.0867)(352,14.3675)(353,13.65)(354,12.9343)(355,12.2203)(356,11.5078)(357,10.7967)(358,10.0871)(359,9.37866)(360,8.67146)(361,7.96537)(362,7.26029)(363,6.55613)(364,5.85281)(365,5.15023)(366,4.44831)(367,3.74696)(368,3.04609)(369,2.3456)(370,1.64541) 
};

\addplot [
color=red,
solid
]
coordinates{
 (1,1.828)(2,2.70862)(3,3.66016)(4,4.70754)(5,5.87817)(6,7.20272)(7,8.71588)(8,10.4573)(9,12.4727)(10,14.8148)(11,17.5453)(12,20.7357)(13,24.47)(14,28.8465)(15,33.9802)(16,39.0954)(17,44.1903)(18,49.2628)(19,54.3111)(20,59.3332)(21,64.3273)(22,69.2916)(23,74.2242)(24,79.1234)(25,83.9874)(26,88.8146)(27,93.6032)(28,98.3517)(29,103.058)(30,107.722)(31,112.34)(32,116.912)(33,121.436)(34,125.911)(35,130.336)(36,134.708)(37,139.027)(38,143.292)(39,147.502)(40,151.654)(41,155.749)(42,159.785)(43,163.76)(44,167.675)(45,171.528)(46,175.319)(47,179.046)(48,182.708)(49,186.306)(50,189.838)(51,193.303)(52,196.702)(53,200.033)(54,203.296)(55,206.49)(56,209.616)(57,212.672)(58,215.659)(59,218.576)(60,221.423)(61,224.2)(62,226.906)(63,229.541)(64,232.106)(65,234.6)(66,237.023)(67,239.376)(68,241.658)(69,243.869)(70,246.011)(71,248.082)(72,250.083)(73,252.014)(74,253.876)(75,255.669)(76,257.393)(77,259.049)(78,260.637)(79,262.157)(80,263.611)(81,264.997)(82,266.318)(83,267.573)(84,268.763)(85,269.888)(86,270.95)(87,271.948)(88,272.884)(89,273.758)(90,274.571)(91,275.324)(92,276.016)(93,276.65)(94,277.225)(95,277.742)(96,278.203)(97,278.607)(98,278.957)(99,279.252)(100,279.493)(101,279.681)(102,279.817)(103,279.903)(104,279.937)(105,279.923)(106,279.859)(107,279.748)(108,279.59)(109,279.386)(110,279.136)(111,278.842)(112,278.505)(113,278.125)(114,277.703)(115,277.241)(116,276.738)(117,276.196)(118,275.616)(119,274.998)(120,274.344)(121,273.653)(122,272.928)(123,272.169)(124,271.377)(125,270.552)(126,269.696)(127,268.808)(128,267.891)(129,266.945)(130,265.97)(131,264.968)(132,263.939)(133,262.884)(134,261.804)(135,260.699)(136,259.57)(137,258.419)(138,257.245)(139,256.05)(140,254.834)(141,253.598)(142,252.343)(143,251.068)(144,249.776)(145,248.467)(146,247.141)(147,245.798)(148,244.441)(149,243.069)(150,241.682)(151,240.282)(152,238.869)(153,237.444)(154,236.007)(155,234.559)(156,233.1)(157,231.631)(158,230.153)(159,228.666)(160,227.17)(161,225.666)(162,224.155)(163,222.637)(164,221.113)(165,219.582)(166,218.046)(167,216.504)(168,214.959)(169,213.408)(170,211.854)(171,210.297)(172,208.737)(173,207.174)(174,205.609)(175,204.042)(176,202.473)(177,200.904)(178,199.334)(179,197.763)(180,196.192)(181,194.622)(182,193.052)(183,191.482)(184,189.914)(185,188.348)(186,186.783)(187,185.22)(188,183.66)(189,182.102)(190,180.546)(191,178.994)(192,177.445)(193,175.899)(194,174.358)(195,172.82)(196,171.286)(197,169.756)(198,168.231)(199,166.711)(200,165.195)(201,163.685)(202,162.18)(203,160.68)(204,159.186)(205,157.697)(206,156.214)(207,154.738)(208,153.267)(209,151.803)(210,150.344)(211,148.893)(212,147.448)(213,146.009)(214,144.578)(215,143.153)(216,141.735)(217,140.324)(218,138.921)(219,137.524)(220,136.135)(221,134.754)(222,133.379)(223,132.012)(224,130.653)(225,129.301)(226,127.957)(227,126.621)(228,125.292)(229,123.972)(230,122.659)(231,121.353)(232,120.056)(233,118.766)(234,117.485)(235,116.211)(236,114.946)(237,113.688)(238,112.438)(239,111.196)(240,109.962)(241,108.737)(242,107.519)(243,106.309)(244,105.107)(245,103.913)(246,102.727)(247,101.549)(248,100.379)(249,99.2166)(250,98.0623)(251,96.9158)(252,95.7772)(253,94.6464)(254,93.5234)(255,92.4082)(256,91.3007)(257,90.2009)(258,89.1088)(259,88.0243)(260,86.9474)(261,85.8781)(262,84.8163)(263,83.762)(264,82.7152)(265,81.6757)(266,80.6436)(267,79.6188)(268,78.6013)(269,77.591)(270,76.5878)(271,75.5918)(272,74.6028)(273,73.6208)(274,72.6458)(275,71.6777)(276,70.7164)(277,69.762)(278,68.8143)(279,67.8732)(280,66.9388)(281,66.0109)(282,65.0896)(283,64.1747)(284,63.2661)(285,62.3639)(286,61.4679)(287,60.5782)(288,59.6945)(289,58.817)(290,57.9454)(291,57.0798)(292,56.22)(293,55.3661)(294,54.5179)(295,53.6753)(296,52.8384)(297,52.007)(298,51.1811)(299,50.3606)(300,49.5454)(301,48.7355)(302,47.9307)(303,47.1311)(304,46.3366)(305,45.547)(306,44.7624)(307,43.9826)(308,43.2076)(309,42.4372)(310,41.6715)(311,40.9104)(312,40.1538)(313,39.4016)(314,38.6537)(315,37.9101)(316,37.1707)(317,36.4354)(318,35.7042)(319,34.977)(320,34.2537)(321,33.5342)(322,32.8185)(323,32.1065)(324,31.3981)(325,30.6933)(326,29.9919)(327,29.294)(328,28.5994)(329,27.908)(330,27.2198)(331,26.5347)(332,25.8527)(333,25.1736)(334,24.4974)(335,23.824)(336,23.1534)(337,22.4854)(338,21.8201)(339,21.1572)(340,20.4968)(341,19.8388)(342,19.1831)(343,18.5296)(344,17.8783)(345,17.2291)(346,16.5818)(347,15.9365)(348,15.2931)(349,14.6514)(350,14.0115)(351,13.3732)(352,12.7364)(353,12.1012)(354,11.4674)(355,10.8349)(356,10.2037)(357,9.57371)(358,8.94484)(359,8.31702)(360,7.69018)(361,7.06425)(362,6.43915)(363,5.81481)(364,5.19116)(365,4.56811)(366,3.94561)(367,3.32358)(368,2.70194)(369,2.08061)(370,1.45954) 
};

\end{axis}
\end{tikzpicture}

			\newframe
			\begin{tikzpicture}[scale=0.5]

\begin{axis}[%
scale only axis,
width=4.52083in,
height=3.56562in,
xmin=0, xmax=400,
ymin=0, ymax=400,
xlabel={Slab Length [cm]},
ylabel={Power [W]},
title={$\text{tstep }= 60$},
axis on top]
\addplot [
color=blue,
solid
]
coordinates{
 (1,13.9911)(2,20.3593)(3,26.7197)(4,33.0699)(5,39.4076)(6,45.7302)(7,52.0354)(8,58.3208)(9,64.584)(10,70.8228)(11,77.0348)(12,83.2176)(13,89.3691)(14,95.487)(15,101.569)(16,107.613)(17,113.617)(18,119.578)(19,125.496)(20,131.366)(21,137.189)(22,142.96)(23,148.68)(24,154.345)(25,159.955)(26,165.506)(27,170.998)(28,176.428)(29,181.795)(30,187.098)(31,192.335)(32,197.504)(33,202.604)(34,207.633)(35,212.591)(36,217.475)(37,222.285)(38,227.019)(39,231.677)(40,236.256)(41,240.757)(42,245.178)(43,249.519)(44,253.778)(45,257.954)(46,262.048)(47,266.057)(48,269.982)(49,273.822)(50,277.577)(51,281.245)(52,284.827)(53,288.323)(54,291.731)(55,295.052)(56,298.285)(57,301.43)(58,304.488)(59,307.457)(60,310.339)(61,313.133)(62,315.839)(63,318.457)(64,320.988)(65,323.431)(66,325.788)(67,328.057)(68,330.241)(69,332.338)(70,334.35)(71,336.276)(72,338.118)(73,339.876)(74,341.55)(75,343.141)(76,344.65)(77,346.078)(78,347.424)(79,348.69)(80,349.877)(81,350.985)(82,352.015)(83,352.967)(84,353.844)(85,354.645)(86,355.372)(87,356.025)(88,356.606)(89,357.114)(90,357.553)(91,357.921)(92,358.22)(93,358.452)(94,358.617)(95,358.716)(96,358.751)(97,358.722)(98,358.63)(99,358.477)(100,358.264)(101,357.991)(102,357.659)(103,357.271)(104,356.826)(105,356.327)(106,355.774)(107,355.167)(108,354.509)(109,353.801)(110,353.043)(111,352.236)(112,351.382)(113,350.482)(114,349.537)(115,348.547)(116,347.515)(117,346.441)(118,345.325)(119,344.17)(120,342.976)(121,341.744)(122,340.476)(123,339.172)(124,337.833)(125,336.461)(126,335.056)(127,333.619)(128,332.152)(129,330.654)(130,329.128)(131,327.575)(132,325.994)(133,324.387)(134,322.756)(135,321.1)(136,319.421)(137,317.719)(138,315.997)(139,314.253)(140,312.49)(141,310.707)(142,308.907)(143,307.089)(144,305.254)(145,303.404)(146,301.538)(147,299.659)(148,297.765)(149,295.859)(150,293.941)(151,292.012)(152,290.071)(153,288.121)(154,286.161)(155,284.192)(156,282.216)(157,280.232)(158,278.24)(159,276.243)(160,274.24)(161,272.232)(162,270.219)(163,268.202)(164,266.181)(165,264.158)(166,262.132)(167,260.104)(168,258.075)(169,256.044)(170,254.013)(171,251.982)(172,249.951)(173,247.92)(174,245.891)(175,243.864)(176,241.838)(177,239.815)(178,237.794)(179,235.776)(180,233.762)(181,231.752)(182,229.745)(183,227.743)(184,225.746)(185,223.753)(186,221.766)(187,219.785)(188,217.809)(189,215.839)(190,213.876)(191,211.919)(192,209.969)(193,208.027)(194,206.091)(195,204.163)(196,202.242)(197,200.33)(198,198.425)(199,196.529)(200,194.641)(201,192.762)(202,190.892)(203,189.03)(204,187.178)(205,185.334)(206,183.5)(207,181.676)(208,179.861)(209,178.055)(210,176.26)(211,174.474)(212,172.698)(213,170.933)(214,169.177)(215,167.432)(216,165.697)(217,163.972)(218,162.258)(219,160.554)(220,158.861)(221,157.178)(222,155.506)(223,153.845)(224,152.194)(225,150.554)(226,148.925)(227,147.307)(228,145.699)(229,144.102)(230,142.516)(231,140.941)(232,139.377)(233,137.823)(234,136.281)(235,134.749)(236,133.228)(237,131.718)(238,130.218)(239,128.73)(240,127.252)(241,125.785)(242,124.328)(243,122.883)(244,121.447)(245,120.023)(246,118.609)(247,117.206)(248,115.813)(249,114.43)(250,113.058)(251,111.697)(252,110.345)(253,109.004)(254,107.673)(255,106.352)(256,105.042)(257,103.741)(258,102.451)(259,101.17)(260,99.899)(261,98.6379)(262,97.3866)(263,96.1449)(264,94.9129)(265,93.6903)(266,92.4773)(267,91.2736)(268,90.0792)(269,88.8941)(270,87.7182)(271,86.5513)(272,85.3934)(273,84.2445)(274,83.1045)(275,81.9733)(276,80.8507)(277,79.7368)(278,78.6314)(279,77.5345)(280,76.4459)(281,75.3657)(282,74.2937)(283,73.2298)(284,72.1739)(285,71.126)(286,70.086)(287,69.0538)(288,68.0293)(289,67.0123)(290,66.003)(291,65.001)(292,64.0064)(293,63.0191)(294,62.039)(295,61.0659)(296,60.0999)(297,59.1407)(298,58.1884)(299,57.2428)(300,56.3038)(301,55.3714)(302,54.4454)(303,53.5259)(304,52.6125)(305,51.7054)(306,50.8044)(307,49.9094)(308,49.0203)(309,48.137)(310,47.2594)(311,46.3875)(312,45.5212)(313,44.6603)(314,43.8047)(315,42.9545)(316,42.1094)(317,41.2694)(318,40.4344)(319,39.6043)(320,38.779)(321,37.9585)(322,37.1426)(323,36.3312)(324,35.5243)(325,34.7217)(326,33.9234)(327,33.1293)(328,32.3393)(329,31.5532)(330,30.7711)(331,29.9928)(332,29.2182)(333,28.4472)(334,27.6798)(335,26.9158)(336,26.1551)(337,25.3978)(338,24.6436)(339,23.8925)(340,23.1443)(341,22.3991)(342,21.6567)(343,20.917)(344,20.18)(345,19.4455)(346,18.7134)(347,17.9837)(348,17.2563)(349,16.531)(350,15.8079)(351,15.0867)(352,14.3675)(353,13.65)(354,12.9343)(355,12.2203)(356,11.5078)(357,10.7967)(358,10.0871)(359,9.37866)(360,8.67146)(361,7.96537)(362,7.26029)(363,6.55613)(364,5.85281)(365,5.15023)(366,4.44831)(367,3.74696)(368,3.04609)(369,2.3456)(370,1.64541) 
};

\addplot [
color=red,
solid
]
coordinates{
 (1,0.893036)(2,1.32304)(3,1.7874)(4,2.29814)(5,2.86854)(6,3.51339)(7,4.24946)(8,5.09588)(9,6.07467)(10,7.21131)(11,8.53541)(12,10.0815)(13,11.8899)(14,14.008)(15,16.4911)(16,19.4042)(17,22.8237)(18,26.8394)(19,31.5568)(20,36.2545)(21,40.9308)(22,45.5838)(23,50.2119)(24,54.8132)(25,59.3862)(26,63.9292)(27,68.4405)(28,72.9186)(29,77.3619)(30,81.7688)(31,86.1379)(32,90.4677)(33,94.7568)(34,99.0039)(35,103.207)(36,107.366)(37,111.479)(38,115.545)(39,119.562)(40,123.529)(41,127.446)(42,131.311)(43,135.123)(44,138.882)(45,142.586)(46,146.234)(47,149.826)(48,153.36)(49,156.836)(50,160.254)(51,163.612)(52,166.91)(53,170.147)(54,173.323)(55,176.438)(56,179.49)(57,182.479)(58,185.405)(59,188.268)(60,191.067)(61,193.803)(62,196.474)(63,199.08)(64,201.622)(65,204.099)(66,206.512)(67,208.859)(68,211.142)(69,213.36)(70,215.514)(71,217.603)(72,219.627)(73,221.587)(74,223.482)(75,225.314)(76,227.083)(77,228.787)(78,230.429)(79,232.008)(80,233.525)(81,234.979)(82,236.372)(83,237.704)(84,238.976)(85,240.187)(86,241.338)(87,242.43)(88,243.464)(89,244.439)(90,245.357)(91,246.219)(92,247.023)(93,247.773)(94,248.467)(95,249.107)(96,249.693)(97,250.227)(98,250.708)(99,251.137)(100,251.516)(101,251.844)(102,252.123)(103,252.354)(104,252.536)(105,252.671)(106,252.76)(107,252.804)(108,252.802)(109,252.757)(110,252.668)(111,252.536)(112,252.363)(113,252.149)(114,251.894)(115,251.601)(116,251.268)(117,250.898)(118,250.491)(119,250.047)(120,249.569)(121,249.055)(122,248.507)(123,247.927)(124,247.314)(125,246.669)(126,245.994)(127,245.288)(128,244.553)(129,243.789)(130,242.998)(131,242.179)(132,241.335)(133,240.464)(134,239.568)(135,238.649)(136,237.705)(137,236.739)(138,235.751)(139,234.741)(140,233.71)(141,232.659)(142,231.589)(143,230.5)(144,229.392)(145,228.267)(146,227.125)(147,225.967)(148,224.793)(149,223.604)(150,222.4)(151,221.182)(152,219.951)(153,218.707)(154,217.451)(155,216.183)(156,214.903)(157,213.613)(158,212.313)(159,211.003)(160,209.684)(161,208.356)(162,207.02)(163,205.676)(164,204.325)(165,202.967)(166,201.602)(167,200.232)(168,198.856)(169,197.475)(170,196.089)(171,194.698)(172,193.304)(173,191.906)(174,190.505)(175,189.101)(176,187.695)(177,186.286)(178,184.876)(179,183.464)(180,182.051)(181,180.637)(182,179.223)(183,177.808)(184,176.394)(185,174.979)(186,173.566)(187,172.153)(188,170.741)(189,169.331)(190,167.922)(191,166.515)(192,165.11)(193,163.708)(194,162.308)(195,160.911)(196,159.517)(197,158.126)(198,156.738)(199,155.354)(200,153.974)(201,152.597)(202,151.225)(203,149.856)(204,148.492)(205,147.133)(206,145.778)(207,144.428)(208,143.083)(209,141.743)(210,140.409)(211,139.079)(212,137.755)(213,136.437)(214,135.124)(215,133.817)(216,132.516)(217,131.221)(218,129.932)(219,128.649)(220,127.372)(221,126.101)(222,124.837)(223,123.579)(224,122.328)(225,121.083)(226,119.844)(227,118.613)(228,117.387)(229,116.169)(230,114.958)(231,113.753)(232,112.555)(233,111.364)(234,110.18)(235,109.002)(236,107.832)(237,106.669)(238,105.512)(239,104.363)(240,103.221)(241,102.085)(242,100.957)(243,99.8358)(244,98.7216)(245,97.6144)(246,96.5143)(247,95.4212)(248,94.3351)(249,93.256)(250,92.1839)(251,91.1188)(252,90.0607)(253,89.0096)(254,87.9654)(255,86.9281)(256,85.8977)(257,84.8742)(258,83.8576)(259,82.8478)(260,81.8448)(261,80.8485)(262,79.859)(263,78.8762)(264,77.9001)(265,76.9306)(266,75.9677)(267,75.0114)(268,74.0616)(269,73.1184)(270,72.1815)(271,71.2511)(272,70.327)(273,69.4092)(274,68.4978)(275,67.5925)(276,66.6934)(277,65.8005)(278,64.9137)(279,64.0329)(280,63.158)(281,62.2892)(282,61.4262)(283,60.569)(284,59.7176)(285,58.8719)(286,58.0319)(287,57.1976)(288,56.3688)(289,55.5455)(290,54.7276)(291,53.9151)(292,53.108)(293,52.3062)(294,51.5095)(295,50.718)(296,49.9317)(297,49.1503)(298,48.374)(299,47.6025)(300,46.8359)(301,46.0742)(302,45.3171)(303,44.5647)(304,43.8169)(305,43.0737)(306,42.335)(307,41.6007)(308,40.8707)(309,40.145)(310,39.4236)(311,38.7063)(312,37.9932)(313,37.284)(314,36.5789)(315,35.8776)(316,35.1802)(317,34.4866)(318,33.7966)(319,33.1104)(320,32.4277)(321,31.7485)(322,31.0727)(323,30.4004)(324,29.7314)(325,29.0656)(326,28.403)(327,27.7435)(328,27.0871)(329,26.4336)(330,25.7831)(331,25.1354)(332,24.4905)(333,23.8483)(334,23.2088)(335,22.5719)(336,21.9374)(337,21.3055)(338,20.6759)(339,20.0486)(340,19.4236)(341,18.8007)(342,18.18)(343,17.5613)(344,16.9446)(345,16.3298)(346,15.7168)(347,15.1056)(348,14.4962)(349,13.8883)(350,13.2821)(351,12.6773)(352,12.074)(353,11.4721)(354,10.8715)(355,10.2721)(356,9.67387)(357,9.07676)(358,8.48068)(359,7.88557)(360,7.29136)(361,6.69798)(362,6.10537)(363,5.51346)(364,4.92218)(365,4.33146)(366,3.74124)(367,3.15144)(368,2.56201)(369,1.97287)(370,1.38396) 
};

\end{axis}
\end{tikzpicture}

			\newframe
			\begin{tikzpicture}[scale=0.5]

\begin{axis}[%
scale only axis,
width=4.52083in,
height=3.56562in,
xmin=0, xmax=400,
ymin=0, ymax=400,
xlabel={Slab Length [cm]},
ylabel={Power [W]},
title={$\text{tstep }= 66$},
axis on top]
\addplot [
color=blue,
solid
]
coordinates{
 (1,13.9911)(2,20.3593)(3,26.7197)(4,33.0699)(5,39.4076)(6,45.7302)(7,52.0354)(8,58.3208)(9,64.584)(10,70.8228)(11,77.0348)(12,83.2176)(13,89.3691)(14,95.487)(15,101.569)(16,107.613)(17,113.617)(18,119.578)(19,125.496)(20,131.366)(21,137.189)(22,142.96)(23,148.68)(24,154.345)(25,159.955)(26,165.506)(27,170.998)(28,176.428)(29,181.795)(30,187.098)(31,192.335)(32,197.504)(33,202.604)(34,207.633)(35,212.591)(36,217.475)(37,222.285)(38,227.019)(39,231.677)(40,236.256)(41,240.757)(42,245.178)(43,249.519)(44,253.778)(45,257.954)(46,262.048)(47,266.057)(48,269.982)(49,273.822)(50,277.577)(51,281.245)(52,284.827)(53,288.323)(54,291.731)(55,295.052)(56,298.285)(57,301.43)(58,304.488)(59,307.457)(60,310.339)(61,313.133)(62,315.839)(63,318.457)(64,320.988)(65,323.431)(66,325.788)(67,328.057)(68,330.241)(69,332.338)(70,334.35)(71,336.276)(72,338.118)(73,339.876)(74,341.55)(75,343.141)(76,344.65)(77,346.078)(78,347.424)(79,348.69)(80,349.877)(81,350.985)(82,352.015)(83,352.967)(84,353.844)(85,354.645)(86,355.372)(87,356.025)(88,356.606)(89,357.114)(90,357.553)(91,357.921)(92,358.22)(93,358.452)(94,358.617)(95,358.716)(96,358.751)(97,358.722)(98,358.63)(99,358.477)(100,358.264)(101,357.991)(102,357.659)(103,357.271)(104,356.826)(105,356.327)(106,355.774)(107,355.167)(108,354.509)(109,353.801)(110,353.043)(111,352.236)(112,351.382)(113,350.482)(114,349.537)(115,348.547)(116,347.515)(117,346.441)(118,345.325)(119,344.17)(120,342.976)(121,341.744)(122,340.476)(123,339.172)(124,337.833)(125,336.461)(126,335.056)(127,333.619)(128,332.152)(129,330.654)(130,329.128)(131,327.575)(132,325.994)(133,324.387)(134,322.756)(135,321.1)(136,319.421)(137,317.719)(138,315.997)(139,314.253)(140,312.49)(141,310.707)(142,308.907)(143,307.089)(144,305.254)(145,303.404)(146,301.538)(147,299.659)(148,297.765)(149,295.859)(150,293.941)(151,292.012)(152,290.071)(153,288.121)(154,286.161)(155,284.192)(156,282.216)(157,280.232)(158,278.24)(159,276.243)(160,274.24)(161,272.232)(162,270.219)(163,268.202)(164,266.181)(165,264.158)(166,262.132)(167,260.104)(168,258.075)(169,256.044)(170,254.013)(171,251.982)(172,249.951)(173,247.92)(174,245.891)(175,243.864)(176,241.838)(177,239.815)(178,237.794)(179,235.776)(180,233.762)(181,231.752)(182,229.745)(183,227.743)(184,225.746)(185,223.753)(186,221.766)(187,219.785)(188,217.809)(189,215.839)(190,213.876)(191,211.919)(192,209.969)(193,208.027)(194,206.091)(195,204.163)(196,202.242)(197,200.33)(198,198.425)(199,196.529)(200,194.641)(201,192.762)(202,190.892)(203,189.03)(204,187.178)(205,185.334)(206,183.5)(207,181.676)(208,179.861)(209,178.055)(210,176.26)(211,174.474)(212,172.698)(213,170.933)(214,169.177)(215,167.432)(216,165.697)(217,163.972)(218,162.258)(219,160.554)(220,158.861)(221,157.178)(222,155.506)(223,153.845)(224,152.194)(225,150.554)(226,148.925)(227,147.307)(228,145.699)(229,144.102)(230,142.516)(231,140.941)(232,139.377)(233,137.823)(234,136.281)(235,134.749)(236,133.228)(237,131.718)(238,130.218)(239,128.73)(240,127.252)(241,125.785)(242,124.328)(243,122.883)(244,121.447)(245,120.023)(246,118.609)(247,117.206)(248,115.813)(249,114.43)(250,113.058)(251,111.697)(252,110.345)(253,109.004)(254,107.673)(255,106.352)(256,105.042)(257,103.741)(258,102.451)(259,101.17)(260,99.899)(261,98.6379)(262,97.3866)(263,96.1449)(264,94.9129)(265,93.6903)(266,92.4773)(267,91.2736)(268,90.0792)(269,88.8941)(270,87.7182)(271,86.5513)(272,85.3934)(273,84.2445)(274,83.1045)(275,81.9733)(276,80.8507)(277,79.7368)(278,78.6314)(279,77.5345)(280,76.4459)(281,75.3657)(282,74.2937)(283,73.2298)(284,72.1739)(285,71.126)(286,70.086)(287,69.0538)(288,68.0293)(289,67.0123)(290,66.003)(291,65.001)(292,64.0064)(293,63.0191)(294,62.039)(295,61.0659)(296,60.0999)(297,59.1407)(298,58.1884)(299,57.2428)(300,56.3038)(301,55.3714)(302,54.4454)(303,53.5259)(304,52.6125)(305,51.7054)(306,50.8044)(307,49.9094)(308,49.0203)(309,48.137)(310,47.2594)(311,46.3875)(312,45.5212)(313,44.6603)(314,43.8047)(315,42.9545)(316,42.1094)(317,41.2694)(318,40.4344)(319,39.6043)(320,38.779)(321,37.9585)(322,37.1426)(323,36.3312)(324,35.5243)(325,34.7217)(326,33.9234)(327,33.1293)(328,32.3393)(329,31.5532)(330,30.7711)(331,29.9928)(332,29.2182)(333,28.4472)(334,27.6798)(335,26.9158)(336,26.1551)(337,25.3978)(338,24.6436)(339,23.8925)(340,23.1443)(341,22.3991)(342,21.6567)(343,20.917)(344,20.18)(345,19.4455)(346,18.7134)(347,17.9837)(348,17.2563)(349,16.531)(350,15.8079)(351,15.0867)(352,14.3675)(353,13.65)(354,12.9343)(355,12.2203)(356,11.5078)(357,10.7967)(358,10.0871)(359,9.37866)(360,8.67146)(361,7.96537)(362,7.26029)(363,6.55613)(364,5.85281)(365,5.15023)(366,4.44831)(367,3.74696)(368,3.04609)(369,2.3456)(370,1.64541) 
};

\addplot [
color=red,
solid
]
coordinates{
 (1,0.593712)(2,0.879567)(3,1.18821)(4,1.52764)(5,1.90665)(6,2.33507)(7,2.82399)(8,3.38611)(9,4.03602)(10,4.7906)(11,5.66947)(12,6.69549)(13,7.89538)(14,9.30043)(15,10.9473)(16,12.879)(17,15.146)(18,17.8077)(19,20.9338)(20,24.6064)(21,28.9218)(22,33.2189)(23,37.496)(24,41.7516)(25,45.9839)(26,50.1915)(27,54.3729)(28,58.5264)(29,62.6506)(30,66.744)(31,70.8053)(32,74.833)(33,78.8258)(34,82.7823)(35,86.7012)(36,90.5813)(37,94.4213)(38,98.2201)(39,101.977)(40,105.689)(41,109.358)(42,112.98)(43,116.556)(44,120.085)(45,123.565)(46,126.995)(47,130.375)(48,133.704)(49,136.981)(50,140.206)(51,143.378)(52,146.495)(53,149.558)(54,152.566)(55,155.518)(56,158.414)(57,161.254)(58,164.036)(59,166.762)(60,169.429)(61,172.038)(62,174.59)(63,177.082)(64,179.516)(65,181.891)(66,184.207)(67,186.464)(68,188.661)(69,190.8)(70,192.879)(71,194.9)(72,196.861)(73,198.763)(74,200.606)(75,202.391)(76,204.117)(77,205.785)(78,207.394)(79,208.946)(80,210.441)(81,211.878)(82,213.259)(83,214.583)(84,215.851)(85,217.063)(86,218.22)(87,219.323)(88,220.371)(89,221.365)(90,222.306)(91,223.193)(92,224.029)(93,224.813)(94,225.546)(95,226.228)(96,226.86)(97,227.443)(98,227.977)(99,228.462)(100,228.901)(101,229.292)(102,229.637)(103,229.937)(104,230.192)(105,230.402)(106,230.569)(107,230.693)(108,230.775)(109,230.816)(110,230.815)(111,230.775)(112,230.695)(113,230.577)(114,230.421)(115,230.228)(116,229.998)(117,229.732)(118,229.431)(119,229.096)(120,228.728)(121,228.326)(122,227.892)(123,227.427)(124,226.931)(125,226.404)(126,225.849)(127,225.264)(128,224.652)(129,224.012)(130,223.346)(131,222.653)(132,221.935)(133,221.193)(134,220.427)(135,219.637)(136,218.825)(137,217.99)(138,217.134)(139,216.258)(140,215.361)(141,214.444)(142,213.509)(143,212.556)(144,211.584)(145,210.596)(146,209.591)(147,208.569)(148,207.533)(149,206.481)(150,205.415)(151,204.336)(152,203.243)(153,202.137)(154,201.019)(155,199.89)(156,198.749)(157,197.597)(158,196.435)(159,195.263)(160,194.082)(161,192.892)(162,191.694)(163,190.487)(164,189.273)(165,188.052)(166,186.824)(167,185.59)(168,184.35)(169,183.104)(170,181.853)(171,180.598)(172,179.338)(173,178.074)(174,176.806)(175,175.535)(176,174.261)(177,172.984)(178,171.705)(179,170.424)(180,169.141)(181,167.857)(182,166.571)(183,165.285)(184,163.997)(185,162.71)(186,161.423)(187,160.135)(188,158.848)(189,157.562)(190,156.277)(191,154.993)(192,153.71)(193,152.429)(194,151.149)(195,149.872)(196,148.597)(197,147.324)(198,146.054)(199,144.786)(200,143.521)(201,142.26)(202,141.001)(203,139.747)(204,138.495)(205,137.248)(206,136.004)(207,134.764)(208,133.529)(209,132.297)(210,131.07)(211,129.848)(212,128.63)(213,127.416)(214,126.208)(215,125.005)(216,123.806)(217,122.613)(218,121.425)(219,120.242)(220,119.064)(221,117.892)(222,116.726)(223,115.564)(224,114.409)(225,113.259)(226,112.115)(227,110.977)(228,109.845)(229,108.719)(230,107.598)(231,106.484)(232,105.376)(233,104.273)(234,103.177)(235,102.087)(236,101.003)(237,99.9254)(238,98.8539)(239,97.7886)(240,96.7295)(241,95.6767)(242,94.6301)(243,93.5899)(244,92.5559)(245,91.5282)(246,90.5068)(247,89.4916)(248,88.4828)(249,87.4803)(250,86.484)(251,85.494)(252,84.5103)(253,83.5328)(254,82.5616)(255,81.5966)(256,80.6378)(257,79.6852)(258,78.7388)(259,77.7985)(260,76.8644)(261,75.9364)(262,75.0144)(263,74.0985)(264,73.1887)(265,72.2848)(266,71.387)(267,70.495)(268,69.609)(269,68.7289)(270,67.8546)(271,66.9861)(272,66.1233)(273,65.2663)(274,64.415)(275,63.5694)(276,62.7293)(277,61.8949)(278,61.0659)(279,60.2425)(280,59.4245)(281,58.6119)(282,57.8046)(283,57.0027)(284,56.206)(285,55.4145)(286,54.6282)(287,53.847)(288,53.0709)(289,52.2998)(290,51.5337)(291,50.7725)(292,50.0161)(293,49.2646)(294,48.5178)(295,47.7758)(296,47.0384)(297,46.3056)(298,45.5773)(299,44.8535)(300,44.1342)(301,43.4193)(302,42.7087)(303,42.0024)(304,41.3002)(305,40.6023)(306,39.9084)(307,39.2186)(308,38.5328)(309,37.8509)(310,37.1729)(311,36.4987)(312,35.8283)(313,35.1615)(314,34.4984)(315,33.8389)(316,33.1829)(317,32.5304)(318,31.8812)(319,31.2355)(320,30.593)(321,29.9537)(322,29.3176)(323,28.6846)(324,28.0546)(325,27.4276)(326,26.8035)(327,26.1823)(328,25.5639)(329,24.9483)(330,24.3353)(331,23.725)(332,23.1172)(333,22.5118)(334,21.909)(335,21.3085)(336,20.7103)(337,20.1144)(338,19.5206)(339,18.929)(340,18.3395)(341,17.7519)(342,17.1663)(343,16.5826)(344,16.0007)(345,15.4206)(346,14.8421)(347,14.2653)(348,13.6901)(349,13.1163)(350,12.5441)(351,11.9732)(352,11.4036)(353,10.8353)(354,10.2682)(355,9.70226)(356,9.13737)(357,8.5735)(358,8.01059)(359,7.44857)(360,6.88737)(361,6.32694)(362,5.76722)(363,5.20814)(364,4.64964)(365,4.09166)(366,3.53414)(367,2.97701)(368,2.42021)(369,1.86369)(370,1.30737) 
};

\end{axis}
\end{tikzpicture}

			\newframe
			\begin{tikzpicture}[scale=0.5]

\begin{axis}[%
scale only axis,
width=4.52083in,
height=3.56562in,
xmin=0, xmax=400,
ymin=0, ymax=400,
xlabel={Slab Length [cm]},
ylabel={Power [W]},
title={$\text{tstep }= 72$},
axis on top]
\addplot [
color=blue,
solid
]
coordinates{
 (1,13.9911)(2,20.3593)(3,26.7197)(4,33.0699)(5,39.4076)(6,45.7302)(7,52.0354)(8,58.3208)(9,64.584)(10,70.8228)(11,77.0348)(12,83.2176)(13,89.3691)(14,95.487)(15,101.569)(16,107.613)(17,113.617)(18,119.578)(19,125.496)(20,131.366)(21,137.189)(22,142.96)(23,148.68)(24,154.345)(25,159.955)(26,165.506)(27,170.998)(28,176.428)(29,181.795)(30,187.098)(31,192.335)(32,197.504)(33,202.604)(34,207.633)(35,212.591)(36,217.475)(37,222.285)(38,227.019)(39,231.677)(40,236.256)(41,240.757)(42,245.178)(43,249.519)(44,253.778)(45,257.954)(46,262.048)(47,266.057)(48,269.982)(49,273.822)(50,277.577)(51,281.245)(52,284.827)(53,288.323)(54,291.731)(55,295.052)(56,298.285)(57,301.43)(58,304.488)(59,307.457)(60,310.339)(61,313.133)(62,315.839)(63,318.457)(64,320.988)(65,323.431)(66,325.788)(67,328.057)(68,330.241)(69,332.338)(70,334.35)(71,336.276)(72,338.118)(73,339.876)(74,341.55)(75,343.141)(76,344.65)(77,346.078)(78,347.424)(79,348.69)(80,349.877)(81,350.985)(82,352.015)(83,352.967)(84,353.844)(85,354.645)(86,355.372)(87,356.025)(88,356.606)(89,357.114)(90,357.553)(91,357.921)(92,358.22)(93,358.452)(94,358.617)(95,358.716)(96,358.751)(97,358.722)(98,358.63)(99,358.477)(100,358.264)(101,357.991)(102,357.659)(103,357.271)(104,356.826)(105,356.327)(106,355.774)(107,355.167)(108,354.509)(109,353.801)(110,353.043)(111,352.236)(112,351.382)(113,350.482)(114,349.537)(115,348.547)(116,347.515)(117,346.441)(118,345.325)(119,344.17)(120,342.976)(121,341.744)(122,340.476)(123,339.172)(124,337.833)(125,336.461)(126,335.056)(127,333.619)(128,332.152)(129,330.654)(130,329.128)(131,327.575)(132,325.994)(133,324.387)(134,322.756)(135,321.1)(136,319.421)(137,317.719)(138,315.997)(139,314.253)(140,312.49)(141,310.707)(142,308.907)(143,307.089)(144,305.254)(145,303.404)(146,301.538)(147,299.659)(148,297.765)(149,295.859)(150,293.941)(151,292.012)(152,290.071)(153,288.121)(154,286.161)(155,284.192)(156,282.216)(157,280.232)(158,278.24)(159,276.243)(160,274.24)(161,272.232)(162,270.219)(163,268.202)(164,266.181)(165,264.158)(166,262.132)(167,260.104)(168,258.075)(169,256.044)(170,254.013)(171,251.982)(172,249.951)(173,247.92)(174,245.891)(175,243.864)(176,241.838)(177,239.815)(178,237.794)(179,235.776)(180,233.762)(181,231.752)(182,229.745)(183,227.743)(184,225.746)(185,223.753)(186,221.766)(187,219.785)(188,217.809)(189,215.839)(190,213.876)(191,211.919)(192,209.969)(193,208.027)(194,206.091)(195,204.163)(196,202.242)(197,200.33)(198,198.425)(199,196.529)(200,194.641)(201,192.762)(202,190.892)(203,189.03)(204,187.178)(205,185.334)(206,183.5)(207,181.676)(208,179.861)(209,178.055)(210,176.26)(211,174.474)(212,172.698)(213,170.933)(214,169.177)(215,167.432)(216,165.697)(217,163.972)(218,162.258)(219,160.554)(220,158.861)(221,157.178)(222,155.506)(223,153.845)(224,152.194)(225,150.554)(226,148.925)(227,147.307)(228,145.699)(229,144.102)(230,142.516)(231,140.941)(232,139.377)(233,137.823)(234,136.281)(235,134.749)(236,133.228)(237,131.718)(238,130.218)(239,128.73)(240,127.252)(241,125.785)(242,124.328)(243,122.883)(244,121.447)(245,120.023)(246,118.609)(247,117.206)(248,115.813)(249,114.43)(250,113.058)(251,111.697)(252,110.345)(253,109.004)(254,107.673)(255,106.352)(256,105.042)(257,103.741)(258,102.451)(259,101.17)(260,99.899)(261,98.6379)(262,97.3866)(263,96.1449)(264,94.9129)(265,93.6903)(266,92.4773)(267,91.2736)(268,90.0792)(269,88.8941)(270,87.7182)(271,86.5513)(272,85.3934)(273,84.2445)(274,83.1045)(275,81.9733)(276,80.8507)(277,79.7368)(278,78.6314)(279,77.5345)(280,76.4459)(281,75.3657)(282,74.2937)(283,73.2298)(284,72.1739)(285,71.126)(286,70.086)(287,69.0538)(288,68.0293)(289,67.0123)(290,66.003)(291,65.001)(292,64.0064)(293,63.0191)(294,62.039)(295,61.0659)(296,60.0999)(297,59.1407)(298,58.1884)(299,57.2428)(300,56.3038)(301,55.3714)(302,54.4454)(303,53.5259)(304,52.6125)(305,51.7054)(306,50.8044)(307,49.9094)(308,49.0203)(309,48.137)(310,47.2594)(311,46.3875)(312,45.5212)(313,44.6603)(314,43.8047)(315,42.9545)(316,42.1094)(317,41.2694)(318,40.4344)(319,39.6043)(320,38.779)(321,37.9585)(322,37.1426)(323,36.3312)(324,35.5243)(325,34.7217)(326,33.9234)(327,33.1293)(328,32.3393)(329,31.5532)(330,30.7711)(331,29.9928)(332,29.2182)(333,28.4472)(334,27.6798)(335,26.9158)(336,26.1551)(337,25.3978)(338,24.6436)(339,23.8925)(340,23.1443)(341,22.3991)(342,21.6567)(343,20.917)(344,20.18)(345,19.4455)(346,18.7134)(347,17.9837)(348,17.2563)(349,16.531)(350,15.8079)(351,15.0867)(352,14.3675)(353,13.65)(354,12.9343)(355,12.2203)(356,11.5078)(357,10.7967)(358,10.0871)(359,9.37866)(360,8.67146)(361,7.96537)(362,7.26029)(363,6.55613)(364,5.85281)(365,5.15023)(366,4.44831)(367,3.74696)(368,3.04609)(369,2.3456)(370,1.64541) 
};

\addplot [
color=red,
solid
]
coordinates{
 (1,0.391279)(2,0.57965)(3,0.783015)(4,1.00663)(5,1.25628)(6,1.53842)(7,1.86036)(8,2.23042)(9,2.6582)(10,3.15478)(11,3.73305)(12,4.40802)(13,5.19723)(14,6.12121)(15,7.204)(16,8.47381)(17,9.96377)(18,11.7128)(19,13.7665)(20,16.1787)(21,19.0126)(22,22.3424)(23,26.2555)(24,30.1517)(25,34.0295)(26,37.8875)(27,41.7243)(28,45.5383)(29,49.3282)(30,53.0926)(31,56.8301)(32,60.5395)(33,64.2195)(34,67.8687)(35,71.4859)(36,75.07)(37,78.6198)(38,82.1341)(39,85.6119)(40,89.052)(41,92.4534)(42,95.8151)(43,99.1362)(44,102.416)(45,105.653)(46,108.846)(47,111.996)(48,115.1)(49,118.159)(50,121.171)(51,124.136)(52,127.054)(53,129.922)(54,132.742)(55,135.513)(56,138.233)(57,140.903)(58,143.522)(59,146.089)(60,148.605)(61,151.069)(62,153.481)(63,155.839)(64,158.145)(65,160.398)(66,162.598)(67,164.745)(68,166.838)(69,168.877)(70,170.863)(71,172.796)(72,174.674)(73,176.5)(74,178.272)(75,179.99)(76,181.656)(77,183.268)(78,184.827)(79,186.334)(80,187.788)(81,189.19)(82,190.541)(83,191.839)(84,193.086)(85,194.283)(86,195.428)(87,196.524)(88,197.569)(89,198.565)(90,199.512)(91,200.411)(92,201.261)(93,202.064)(94,202.819)(95,203.528)(96,204.191)(97,204.808)(98,205.38)(99,205.907)(100,206.391)(101,206.831)(102,207.228)(103,207.583)(104,207.897)(105,208.169)(106,208.401)(107,208.593)(108,208.745)(109,208.86)(110,208.936)(111,208.974)(112,208.977)(113,208.943)(114,208.873)(115,208.769)(116,208.631)(117,208.459)(118,208.254)(119,208.017)(120,207.748)(121,207.449)(122,207.119)(123,206.759)(124,206.371)(125,205.954)(126,205.509)(127,205.037)(128,204.539)(129,204.015)(130,203.465)(131,202.891)(132,202.293)(133,201.671)(134,201.027)(135,200.36)(136,199.672)(137,198.963)(138,198.233)(139,197.483)(140,196.714)(141,195.926)(142,195.12)(143,194.297)(144,193.456)(145,192.599)(146,191.725)(147,190.836)(148,189.932)(149,189.014)(150,188.082)(151,187.136)(152,186.177)(153,185.206)(154,184.222)(155,183.228)(156,182.222)(157,181.205)(158,180.178)(159,179.141)(160,178.095)(161,177.04)(162,175.977)(163,174.906)(164,173.826)(165,172.74)(166,171.647)(167,170.547)(168,169.441)(169,168.329)(170,167.212)(171,166.089)(172,164.962)(173,163.831)(174,162.695)(175,161.556)(176,160.413)(177,159.267)(178,158.119)(179,156.968)(180,155.814)(181,154.659)(182,153.501)(183,152.343)(184,151.183)(185,150.023)(186,148.861)(187,147.7)(188,146.538)(189,145.376)(190,144.214)(191,143.053)(192,141.893)(193,140.733)(194,139.575)(195,138.418)(196,137.262)(197,136.109)(198,134.957)(199,133.807)(200,132.659)(201,131.513)(202,130.37)(203,129.23)(204,128.092)(205,126.958)(206,125.826)(207,124.698)(208,123.573)(209,122.452)(210,121.334)(211,120.22)(212,119.11)(213,118.003)(214,116.901)(215,115.803)(216,114.709)(217,113.619)(218,112.534)(219,111.453)(220,110.377)(221,109.306)(222,108.239)(223,107.177)(224,106.119)(225,105.067)(226,104.02)(227,102.977)(228,101.94)(229,100.908)(230,99.8809)(231,98.8591)(232,97.8426)(233,96.8314)(234,95.8255)(235,94.8249)(236,93.8298)(237,92.84)(238,91.8557)(239,90.8769)(240,89.9035)(241,88.9357)(242,87.9734)(243,87.0166)(244,86.0653)(245,85.1196)(246,84.1795)(247,83.2449)(248,82.3159)(249,81.3924)(250,80.4745)(251,79.5622)(252,78.6554)(253,77.7542)(254,76.8586)(255,75.9685)(256,75.0839)(257,74.2048)(258,73.3313)(259,72.4632)(260,71.6006)(261,70.7434)(262,69.8917)(263,69.0454)(264,68.2045)(265,67.3689)(266,66.5387)(267,65.7138)(268,64.8942)(269,64.0799)(270,63.2708)(271,62.4669)(272,61.6682)(273,60.8746)(274,60.0861)(275,59.3028)(276,58.5244)(277,57.7511)(278,56.9827)(279,56.2193)(280,55.4608)(281,54.7071)(282,53.9583)(283,53.2142)(284,52.4749)(285,51.7403)(286,51.0103)(287,50.2849)(288,49.5642)(289,48.8479)(290,48.1361)(291,47.4288)(292,46.7259)(293,46.0273)(294,45.333)(295,44.643)(296,43.9572)(297,43.2755)(298,42.598)(299,41.9245)(300,41.2551)(301,40.5896)(302,39.928)(303,39.2703)(304,38.6165)(305,37.9663)(306,37.32)(307,36.6772)(308,36.0381)(309,35.4026)(310,34.7706)(311,34.142)(312,33.5168)(313,32.895)(314,32.2765)(315,31.6613)(316,31.0492)(317,30.4403)(318,29.8345)(319,29.2317)(320,28.6319)(321,28.035)(322,27.441)(323,26.8498)(324,26.2614)(325,25.6757)(326,25.0927)(327,24.5123)(328,23.9344)(329,23.359)(330,22.786)(331,22.2154)(332,21.6472)(333,21.0812)(334,20.5174)(335,19.9558)(336,19.3963)(337,18.8389)(338,18.2834)(339,17.7299)(340,17.1782)(341,16.6284)(342,16.0804)(343,15.534)(344,14.9894)(345,14.4463)(346,13.9048)(347,13.3647)(348,12.8261)(349,12.2889)(350,11.753)(351,11.2184)(352,10.6849)(353,10.1526)(354,9.62145)(355,9.0913)(356,8.56213)(357,8.03389)(358,7.50652)(359,6.97995)(360,6.45415)(361,5.92904)(362,5.40458)(363,4.8807)(364,4.35735)(365,3.83448)(366,3.31202)(367,2.78992)(368,2.26813)(369,1.74658)(370,1.22522) 
};

\end{axis}
\end{tikzpicture}

			\newframe
			\begin{tikzpicture}[scale=0.5]

\begin{axis}[%
scale only axis,
width=4.52083in,
height=3.56562in,
xmin=0, xmax=400,
ymin=0, ymax=400,
xlabel={Slab Length [cm]},
ylabel={Power [W]},
title={$\text{tstep }= 78$},
axis on top]
\addplot [
color=blue,
solid
]
coordinates{
 (1,13.9911)(2,20.3593)(3,26.7197)(4,33.0699)(5,39.4076)(6,45.7302)(7,52.0354)(8,58.3208)(9,64.584)(10,70.8228)(11,77.0348)(12,83.2176)(13,89.3691)(14,95.487)(15,101.569)(16,107.613)(17,113.617)(18,119.578)(19,125.496)(20,131.366)(21,137.189)(22,142.96)(23,148.68)(24,154.345)(25,159.955)(26,165.506)(27,170.998)(28,176.428)(29,181.795)(30,187.098)(31,192.335)(32,197.504)(33,202.604)(34,207.633)(35,212.591)(36,217.475)(37,222.285)(38,227.019)(39,231.677)(40,236.256)(41,240.757)(42,245.178)(43,249.519)(44,253.778)(45,257.954)(46,262.048)(47,266.057)(48,269.982)(49,273.822)(50,277.577)(51,281.245)(52,284.827)(53,288.323)(54,291.731)(55,295.052)(56,298.285)(57,301.43)(58,304.488)(59,307.457)(60,310.339)(61,313.133)(62,315.839)(63,318.457)(64,320.988)(65,323.431)(66,325.788)(67,328.057)(68,330.241)(69,332.338)(70,334.35)(71,336.276)(72,338.118)(73,339.876)(74,341.55)(75,343.141)(76,344.65)(77,346.078)(78,347.424)(79,348.69)(80,349.877)(81,350.985)(82,352.015)(83,352.967)(84,353.844)(85,354.645)(86,355.372)(87,356.025)(88,356.606)(89,357.114)(90,357.553)(91,357.921)(92,358.22)(93,358.452)(94,358.617)(95,358.716)(96,358.751)(97,358.722)(98,358.63)(99,358.477)(100,358.264)(101,357.991)(102,357.659)(103,357.271)(104,356.826)(105,356.327)(106,355.774)(107,355.167)(108,354.509)(109,353.801)(110,353.043)(111,352.236)(112,351.382)(113,350.482)(114,349.537)(115,348.547)(116,347.515)(117,346.441)(118,345.325)(119,344.17)(120,342.976)(121,341.744)(122,340.476)(123,339.172)(124,337.833)(125,336.461)(126,335.056)(127,333.619)(128,332.152)(129,330.654)(130,329.128)(131,327.575)(132,325.994)(133,324.387)(134,322.756)(135,321.1)(136,319.421)(137,317.719)(138,315.997)(139,314.253)(140,312.49)(141,310.707)(142,308.907)(143,307.089)(144,305.254)(145,303.404)(146,301.538)(147,299.659)(148,297.765)(149,295.859)(150,293.941)(151,292.012)(152,290.071)(153,288.121)(154,286.161)(155,284.192)(156,282.216)(157,280.232)(158,278.24)(159,276.243)(160,274.24)(161,272.232)(162,270.219)(163,268.202)(164,266.181)(165,264.158)(166,262.132)(167,260.104)(168,258.075)(169,256.044)(170,254.013)(171,251.982)(172,249.951)(173,247.92)(174,245.891)(175,243.864)(176,241.838)(177,239.815)(178,237.794)(179,235.776)(180,233.762)(181,231.752)(182,229.745)(183,227.743)(184,225.746)(185,223.753)(186,221.766)(187,219.785)(188,217.809)(189,215.839)(190,213.876)(191,211.919)(192,209.969)(193,208.027)(194,206.091)(195,204.163)(196,202.242)(197,200.33)(198,198.425)(199,196.529)(200,194.641)(201,192.762)(202,190.892)(203,189.03)(204,187.178)(205,185.334)(206,183.5)(207,181.676)(208,179.861)(209,178.055)(210,176.26)(211,174.474)(212,172.698)(213,170.933)(214,169.177)(215,167.432)(216,165.697)(217,163.972)(218,162.258)(219,160.554)(220,158.861)(221,157.178)(222,155.506)(223,153.845)(224,152.194)(225,150.554)(226,148.925)(227,147.307)(228,145.699)(229,144.102)(230,142.516)(231,140.941)(232,139.377)(233,137.823)(234,136.281)(235,134.749)(236,133.228)(237,131.718)(238,130.218)(239,128.73)(240,127.252)(241,125.785)(242,124.328)(243,122.883)(244,121.447)(245,120.023)(246,118.609)(247,117.206)(248,115.813)(249,114.43)(250,113.058)(251,111.697)(252,110.345)(253,109.004)(254,107.673)(255,106.352)(256,105.042)(257,103.741)(258,102.451)(259,101.17)(260,99.899)(261,98.6379)(262,97.3866)(263,96.1449)(264,94.9129)(265,93.6903)(266,92.4773)(267,91.2736)(268,90.0792)(269,88.8941)(270,87.7182)(271,86.5513)(272,85.3934)(273,84.2445)(274,83.1045)(275,81.9733)(276,80.8507)(277,79.7368)(278,78.6314)(279,77.5345)(280,76.4459)(281,75.3657)(282,74.2937)(283,73.2298)(284,72.1739)(285,71.126)(286,70.086)(287,69.0538)(288,68.0293)(289,67.0123)(290,66.003)(291,65.001)(292,64.0064)(293,63.0191)(294,62.039)(295,61.0659)(296,60.0999)(297,59.1407)(298,58.1884)(299,57.2428)(300,56.3038)(301,55.3714)(302,54.4454)(303,53.5259)(304,52.6125)(305,51.7054)(306,50.8044)(307,49.9094)(308,49.0203)(309,48.137)(310,47.2594)(311,46.3875)(312,45.5212)(313,44.6603)(314,43.8047)(315,42.9545)(316,42.1094)(317,41.2694)(318,40.4344)(319,39.6043)(320,38.779)(321,37.9585)(322,37.1426)(323,36.3312)(324,35.5243)(325,34.7217)(326,33.9234)(327,33.1293)(328,32.3393)(329,31.5532)(330,30.7711)(331,29.9928)(332,29.2182)(333,28.4472)(334,27.6798)(335,26.9158)(336,26.1551)(337,25.3978)(338,24.6436)(339,23.8925)(340,23.1443)(341,22.3991)(342,21.6567)(343,20.917)(344,20.18)(345,19.4455)(346,18.7134)(347,17.9837)(348,17.2563)(349,16.531)(350,15.8079)(351,15.0867)(352,14.3675)(353,13.65)(354,12.9343)(355,12.2203)(356,11.5078)(357,10.7967)(358,10.0871)(359,9.37866)(360,8.67146)(361,7.96537)(362,7.26029)(363,6.55613)(364,5.85281)(365,5.15023)(366,4.44831)(367,3.74696)(368,3.04609)(369,2.3456)(370,1.64541) 
};

\addplot [
color=red,
solid
]
coordinates{
 (1,0.255677)(2,0.378754)(3,0.511611)(4,0.657677)(5,0.82072)(6,1.00495)(7,1.21513)(8,1.45669)(9,1.73587)(10,2.05989)(11,2.43716)(12,2.87742)(13,3.39212)(14,3.99459)(15,4.70049)(16,5.52816)(17,6.49915)(18,7.63873)(19,8.97661)(20,10.5477)(21,12.3931)(22,14.5609)(23,17.108)(24,20.101)(25,23.6184)(26,27.1205)(27,30.6059)(28,34.0732)(29,37.5212)(30,40.9486)(31,44.354)(32,47.7364)(33,51.0944)(34,54.4268)(35,57.7326)(36,61.0106)(37,64.2596)(38,67.4786)(39,70.6666)(40,73.8224)(41,76.9453)(42,80.0341)(43,83.088)(44,86.106)(45,89.0873)(46,92.0311)(47,94.9365)(48,97.8028)(49,100.629)(50,103.415)(51,106.16)(52,108.863)(53,111.523)(54,114.14)(55,116.714)(56,119.244)(57,121.729)(58,124.169)(59,126.564)(60,128.912)(61,131.215)(62,133.471)(63,135.681)(64,137.843)(65,139.958)(66,142.026)(67,144.046)(68,146.018)(69,147.942)(70,149.819)(71,151.647)(72,153.428)(73,155.16)(74,156.844)(75,158.481)(76,160.069)(77,161.61)(78,163.103)(79,164.548)(80,165.946)(81,167.297)(82,168.601)(83,169.858)(84,171.069)(85,172.233)(86,173.352)(87,174.425)(88,175.452)(89,176.435)(90,177.373)(91,178.266)(92,179.116)(93,179.922)(94,180.686)(95,181.406)(96,182.085)(97,182.722)(98,183.317)(99,183.872)(100,184.387)(101,184.861)(102,185.296)(103,185.693)(104,186.051)(105,186.372)(106,186.655)(107,186.901)(108,187.112)(109,187.286)(110,187.426)(111,187.531)(112,187.603)(113,187.64)(114,187.646)(115,187.618)(116,187.56)(117,187.47)(118,187.349)(119,187.199)(120,187.019)(121,186.81)(122,186.573)(123,186.309)(124,186.017)(125,185.699)(126,185.355)(127,184.986)(128,184.591)(129,184.173)(130,183.73)(131,183.265)(132,182.777)(133,182.267)(134,181.735)(135,181.183)(136,180.61)(137,180.017)(138,179.405)(139,178.774)(140,178.124)(141,177.457)(142,176.773)(143,176.071)(144,175.354)(145,174.62)(146,173.871)(147,173.108)(148,172.329)(149,171.538)(150,170.732)(151,169.914)(152,169.083)(153,168.239)(154,167.385)(155,166.519)(156,165.642)(157,164.754)(158,163.857)(159,162.95)(160,162.034)(161,161.109)(162,160.176)(163,159.235)(164,158.285)(165,157.329)(166,156.366)(167,155.396)(168,154.42)(169,153.437)(170,152.45)(171,151.457)(172,150.459)(173,149.456)(174,148.449)(175,147.438)(176,146.423)(177,145.405)(178,144.384)(179,143.36)(180,142.333)(181,141.303)(182,140.272)(183,139.239)(184,138.204)(185,137.168)(186,136.13)(187,135.092)(188,134.053)(189,133.013)(190,131.973)(191,130.933)(192,129.894)(193,128.854)(194,127.815)(195,126.777)(196,125.739)(197,124.703)(198,123.668)(199,122.634)(200,121.602)(201,120.571)(202,119.543)(203,118.516)(204,117.491)(205,116.469)(206,115.449)(207,114.431)(208,113.417)(209,112.405)(210,111.395)(211,110.389)(212,109.386)(213,108.386)(214,107.39)(215,106.397)(216,105.407)(217,104.421)(218,103.438)(219,102.46)(220,101.485)(221,100.514)(222,99.5465)(223,98.5834)(224,97.6245)(225,96.6697)(226,95.7191)(227,94.7728)(228,93.8309)(229,92.8933)(230,91.9601)(231,91.0314)(232,90.1072)(233,89.1876)(234,88.2725)(235,87.3621)(236,86.4563)(237,85.5552)(238,84.6588)(239,83.7671)(240,82.8802)(241,81.9981)(242,81.1208)(243,80.2483)(244,79.3806)(245,78.5178)(246,77.6598)(247,76.8067)(248,75.9585)(249,75.1151)(250,74.2766)(251,73.443)(252,72.6143)(253,71.7904)(254,70.9714)(255,70.1573)(256,69.3481)(257,68.5437)(258,67.7441)(259,66.9494)(260,66.1595)(261,65.3745)(262,64.5942)(263,63.8187)(264,63.048)(265,62.282)(266,61.5208)(267,60.7642)(268,60.0124)(269,59.2652)(270,58.5227)(271,57.7848)(272,57.0515)(273,56.3227)(274,55.5985)(275,54.8788)(276,54.1636)(277,53.4528)(278,52.7465)(279,52.0446)(280,51.347)(281,50.6537)(282,49.9648)(283,49.2801)(284,48.5996)(285,47.9234)(286,47.2512)(287,46.5833)(288,45.9193)(289,45.2595)(290,44.6036)(291,43.9517)(292,43.3038)(293,42.6597)(294,42.0195)(295,41.3831)(296,40.7504)(297,40.1215)(298,39.4963)(299,38.8747)(300,38.2567)(301,37.6422)(302,37.0313)(303,36.4239)(304,35.8198)(305,35.2192)(306,34.6219)(307,34.0278)(308,33.4371)(309,32.8495)(310,32.2651)(311,31.6837)(312,31.1055)(313,30.5302)(314,29.958)(315,29.3886)(316,28.8222)(317,28.2585)(318,27.6976)(319,27.1395)(320,26.584)(321,26.0312)(322,25.481)(323,24.9333)(324,24.3881)(325,23.8453)(326,23.305)(327,22.7669)(328,22.2312)(329,21.6977)(330,21.1664)(331,20.6373)(332,20.1102)(333,19.5852)(334,19.0622)(335,18.5412)(336,18.022)(337,17.5047)(338,16.9892)(339,16.4754)(340,15.9633)(341,15.4529)(342,14.9441)(343,14.4368)(344,13.931)(345,13.4266)(346,12.9237)(347,12.4221)(348,11.9218)(349,11.4227)(350,10.9248)(351,10.4281)(352,9.93246)(353,9.43785)(354,8.94423)(355,8.45155)(356,7.95976)(357,7.4688)(358,6.97862)(359,6.48918)(360,6.00043)(361,5.5123)(362,5.02476)(363,4.53774)(364,4.0512)(365,3.56509)(366,3.07936)(367,2.59395)(368,2.10882)(369,1.62391)(370,1.13917) 
};

\end{axis}
\end{tikzpicture}

			\newframe
			\begin{tikzpicture}[scale=0.5]

\begin{axis}[%
scale only axis,
width=4.52083in,
height=3.56562in,
xmin=0, xmax=400,
ymin=0, ymax=400,
xlabel={Slab Length [cm]},
ylabel={Power [W]},
title={$\text{tstep }= 84$},
axis on top]
\addplot [
color=blue,
solid
]
coordinates{
 (1,13.9911)(2,20.3593)(3,26.7197)(4,33.0699)(5,39.4076)(6,45.7302)(7,52.0354)(8,58.3208)(9,64.584)(10,70.8228)(11,77.0348)(12,83.2176)(13,89.3691)(14,95.487)(15,101.569)(16,107.613)(17,113.617)(18,119.578)(19,125.496)(20,131.366)(21,137.189)(22,142.96)(23,148.68)(24,154.345)(25,159.955)(26,165.506)(27,170.998)(28,176.428)(29,181.795)(30,187.098)(31,192.335)(32,197.504)(33,202.604)(34,207.633)(35,212.591)(36,217.475)(37,222.285)(38,227.019)(39,231.677)(40,236.256)(41,240.757)(42,245.178)(43,249.519)(44,253.778)(45,257.954)(46,262.048)(47,266.057)(48,269.982)(49,273.822)(50,277.577)(51,281.245)(52,284.827)(53,288.323)(54,291.731)(55,295.052)(56,298.285)(57,301.43)(58,304.488)(59,307.457)(60,310.339)(61,313.133)(62,315.839)(63,318.457)(64,320.988)(65,323.431)(66,325.788)(67,328.057)(68,330.241)(69,332.338)(70,334.35)(71,336.276)(72,338.118)(73,339.876)(74,341.55)(75,343.141)(76,344.65)(77,346.078)(78,347.424)(79,348.69)(80,349.877)(81,350.985)(82,352.015)(83,352.967)(84,353.844)(85,354.645)(86,355.372)(87,356.025)(88,356.606)(89,357.114)(90,357.553)(91,357.921)(92,358.22)(93,358.452)(94,358.617)(95,358.716)(96,358.751)(97,358.722)(98,358.63)(99,358.477)(100,358.264)(101,357.991)(102,357.659)(103,357.271)(104,356.826)(105,356.327)(106,355.774)(107,355.167)(108,354.509)(109,353.801)(110,353.043)(111,352.236)(112,351.382)(113,350.482)(114,349.537)(115,348.547)(116,347.515)(117,346.441)(118,345.325)(119,344.17)(120,342.976)(121,341.744)(122,340.476)(123,339.172)(124,337.833)(125,336.461)(126,335.056)(127,333.619)(128,332.152)(129,330.654)(130,329.128)(131,327.575)(132,325.994)(133,324.387)(134,322.756)(135,321.1)(136,319.421)(137,317.719)(138,315.997)(139,314.253)(140,312.49)(141,310.707)(142,308.907)(143,307.089)(144,305.254)(145,303.404)(146,301.538)(147,299.659)(148,297.765)(149,295.859)(150,293.941)(151,292.012)(152,290.071)(153,288.121)(154,286.161)(155,284.192)(156,282.216)(157,280.232)(158,278.24)(159,276.243)(160,274.24)(161,272.232)(162,270.219)(163,268.202)(164,266.181)(165,264.158)(166,262.132)(167,260.104)(168,258.075)(169,256.044)(170,254.013)(171,251.982)(172,249.951)(173,247.92)(174,245.891)(175,243.864)(176,241.838)(177,239.815)(178,237.794)(179,235.776)(180,233.762)(181,231.752)(182,229.745)(183,227.743)(184,225.746)(185,223.753)(186,221.766)(187,219.785)(188,217.809)(189,215.839)(190,213.876)(191,211.919)(192,209.969)(193,208.027)(194,206.091)(195,204.163)(196,202.242)(197,200.33)(198,198.425)(199,196.529)(200,194.641)(201,192.762)(202,190.892)(203,189.03)(204,187.178)(205,185.334)(206,183.5)(207,181.676)(208,179.861)(209,178.055)(210,176.26)(211,174.474)(212,172.698)(213,170.933)(214,169.177)(215,167.432)(216,165.697)(217,163.972)(218,162.258)(219,160.554)(220,158.861)(221,157.178)(222,155.506)(223,153.845)(224,152.194)(225,150.554)(226,148.925)(227,147.307)(228,145.699)(229,144.102)(230,142.516)(231,140.941)(232,139.377)(233,137.823)(234,136.281)(235,134.749)(236,133.228)(237,131.718)(238,130.218)(239,128.73)(240,127.252)(241,125.785)(242,124.328)(243,122.883)(244,121.447)(245,120.023)(246,118.609)(247,117.206)(248,115.813)(249,114.43)(250,113.058)(251,111.697)(252,110.345)(253,109.004)(254,107.673)(255,106.352)(256,105.042)(257,103.741)(258,102.451)(259,101.17)(260,99.899)(261,98.6379)(262,97.3866)(263,96.1449)(264,94.9129)(265,93.6903)(266,92.4773)(267,91.2736)(268,90.0792)(269,88.8941)(270,87.7182)(271,86.5513)(272,85.3934)(273,84.2445)(274,83.1045)(275,81.9733)(276,80.8507)(277,79.7368)(278,78.6314)(279,77.5345)(280,76.4459)(281,75.3657)(282,74.2937)(283,73.2298)(284,72.1739)(285,71.126)(286,70.086)(287,69.0538)(288,68.0293)(289,67.0123)(290,66.003)(291,65.001)(292,64.0064)(293,63.0191)(294,62.039)(295,61.0659)(296,60.0999)(297,59.1407)(298,58.1884)(299,57.2428)(300,56.3038)(301,55.3714)(302,54.4454)(303,53.5259)(304,52.6125)(305,51.7054)(306,50.8044)(307,49.9094)(308,49.0203)(309,48.137)(310,47.2594)(311,46.3875)(312,45.5212)(313,44.6603)(314,43.8047)(315,42.9545)(316,42.1094)(317,41.2694)(318,40.4344)(319,39.6043)(320,38.779)(321,37.9585)(322,37.1426)(323,36.3312)(324,35.5243)(325,34.7217)(326,33.9234)(327,33.1293)(328,32.3393)(329,31.5532)(330,30.7711)(331,29.9928)(332,29.2182)(333,28.4472)(334,27.6798)(335,26.9158)(336,26.1551)(337,25.3978)(338,24.6436)(339,23.8925)(340,23.1443)(341,22.3991)(342,21.6567)(343,20.917)(344,20.18)(345,19.4455)(346,18.7134)(347,17.9837)(348,17.2563)(349,16.531)(350,15.8079)(351,15.0867)(352,14.3675)(353,13.65)(354,12.9343)(355,12.2203)(356,11.5078)(357,10.7967)(358,10.0871)(359,9.37866)(360,8.67146)(361,7.96537)(362,7.26029)(363,6.55613)(364,5.85281)(365,5.15023)(366,4.44831)(367,3.74696)(368,3.04609)(369,2.3456)(370,1.64541) 
};

\addplot [
color=red,
solid
]
coordinates{
 (1,0.165758)(2,0.245542)(3,0.331655)(4,0.426316)(5,0.531962)(6,0.651316)(7,0.787456)(8,0.943892)(9,1.12466)(10,1.33443)(11,1.57862)(12,1.86354)(13,2.19657)(14,2.58633)(15,3.04292)(16,3.57817)(17,4.206)(18,4.9427)(19,5.80744)(20,6.82272)(21,8.01501)(22,9.41539)(23,11.0604)(24,12.9931)(25,15.2639)(26,17.9324)(27,21.0684)(28,24.1905)(29,27.2976)(30,30.3885)(31,33.4619)(32,36.5168)(33,39.552)(34,42.5664)(35,45.559)(36,48.5286)(37,51.4743)(38,54.395)(39,57.2897)(40,60.1576)(41,62.9976)(42,65.8088)(43,68.5905)(44,71.3417)(45,74.0616)(46,76.7494)(47,79.4044)(48,82.0258)(49,84.6129)(50,87.165)(51,89.6815)(52,92.1618)(53,94.6052)(54,97.0113)(55,99.3795)(56,101.709)(57,104)(58,106.251)(59,108.463)(60,110.635)(61,112.766)(62,114.856)(63,116.905)(64,118.913)(65,120.879)(66,122.803)(67,124.685)(68,126.524)(69,128.322)(70,130.077)(71,131.789)(72,133.458)(73,135.085)(74,136.669)(75,138.21)(76,139.709)(77,141.165)(78,142.578)(79,143.949)(80,145.278)(81,146.564)(82,147.808)(83,149.011)(84,150.171)(85,151.29)(86,152.368)(87,153.404)(88,154.4)(89,155.356)(90,156.271)(91,157.146)(92,157.982)(93,158.778)(94,159.535)(95,160.254)(96,160.935)(97,161.578)(98,162.183)(99,162.751)(100,163.283)(101,163.779)(102,164.239)(103,164.663)(104,165.053)(105,165.408)(106,165.729)(107,166.017)(108,166.272)(109,166.494)(110,166.684)(111,166.843)(112,166.971)(113,167.068)(114,167.134)(115,167.172)(116,167.18)(117,167.159)(118,167.111)(119,167.034)(120,166.931)(121,166.801)(122,166.646)(123,166.464)(124,166.258)(125,166.027)(126,165.772)(127,165.493)(128,165.192)(129,164.867)(130,164.521)(131,164.154)(132,163.765)(133,163.356)(134,162.926)(135,162.477)(136,162.009)(137,161.523)(138,161.018)(139,160.495)(140,159.956)(141,159.399)(142,158.826)(143,158.238)(144,157.634)(145,157.015)(146,156.381)(147,155.734)(148,155.073)(149,154.398)(150,153.711)(151,153.011)(152,152.299)(153,151.576)(154,150.842)(155,150.096)(156,149.341)(157,148.575)(158,147.799)(159,147.015)(160,146.221)(161,145.419)(162,144.608)(163,143.79)(164,142.963)(165,142.13)(166,141.29)(167,140.443)(168,139.59)(169,138.731)(170,137.867)(171,136.997)(172,136.122)(173,135.242)(174,134.358)(175,133.469)(176,132.577)(177,131.681)(178,130.781)(179,129.878)(180,128.973)(181,128.064)(182,127.154)(183,126.241)(184,125.326)(185,124.409)(186,123.491)(187,122.571)(188,121.65)(189,120.728)(190,119.806)(191,118.883)(192,117.96)(193,117.036)(194,116.112)(195,115.189)(196,114.266)(197,113.343)(198,112.421)(199,111.5)(200,110.58)(201,109.661)(202,108.743)(203,107.827)(204,106.912)(205,105.999)(206,105.087)(207,104.178)(208,103.27)(209,102.365)(210,101.461)(211,100.561)(212,99.6621)(213,98.7662)(214,97.8729)(215,96.9823)(216,96.0945)(217,95.2096)(218,94.3277)(219,93.4489)(220,92.5732)(221,91.7007)(222,90.8314)(223,89.9655)(224,89.1031)(225,88.244)(226,87.3885)(227,86.5366)(228,85.6883)(229,84.8437)(230,84.0029)(231,83.1658)(232,82.3325)(233,81.5031)(234,80.6775)(235,79.8559)(236,79.0383)(237,78.2247)(238,77.4151)(239,76.6095)(240,75.808)(241,75.0107)(242,74.2174)(243,73.4283)(244,72.6433)(245,71.8626)(246,71.086)(247,70.3136)(248,69.5454)(249,68.7814)(250,68.0216)(251,67.2661)(252,66.5148)(253,65.7678)(254,65.025)(255,64.2864)(256,63.552)(257,62.8219)(258,62.096)(259,61.3744)(260,60.6569)(261,59.9436)(262,59.2346)(263,58.5297)(264,57.829)(265,57.1324)(266,56.44)(267,55.7517)(268,55.0675)(269,54.3874)(270,53.7114)(271,53.0395)(272,52.3716)(273,51.7077)(274,51.0478)(275,50.3918)(276,49.7398)(277,49.0918)(278,48.4476)(279,47.8073)(280,47.1708)(281,46.5382)(282,45.9094)(283,45.2843)(284,44.663)(285,44.0453)(286,43.4313)(287,42.821)(288,42.2143)(289,41.6112)(290,41.0116)(291,40.4155)(292,39.8229)(293,39.2337)(294,38.648)(295,38.0656)(296,37.4866)(297,36.9108)(298,36.3384)(299,35.7691)(300,35.2031)(301,34.6402)(302,34.0805)(303,33.5238)(304,32.9701)(305,32.4195)(306,31.8718)(307,31.3271)(308,30.7852)(309,30.2462)(310,29.71)(311,29.1766)(312,28.6459)(313,28.1178)(314,27.5924)(315,27.0697)(316,26.5494)(317,26.0317)(318,25.5165)(319,25.0037)(320,24.4933)(321,23.9852)(322,23.4794)(323,22.976)(324,22.4747)(325,21.9756)(326,21.4786)(327,20.9838)(328,20.4909)(329,20.0001)(330,19.5113)(331,19.0243)(332,18.5392)(333,18.056)(334,17.5745)(335,17.0948)(336,16.6168)(337,16.1404)(338,15.6656)(339,15.1924)(340,14.7207)(341,14.2505)(342,13.7817)(343,13.3143)(344,12.8482)(345,12.3834)(346,11.9199)(347,11.4576)(348,10.9964)(349,10.5363)(350,10.0773)(351,9.61934)(352,9.16233)(353,8.70626)(354,8.25106)(355,7.79671)(356,7.34315)(357,6.89034)(358,6.43823)(359,5.98677)(360,5.53593)(361,5.08565)(362,4.6359)(363,4.18661)(364,3.73776)(365,3.28929)(366,2.84115)(367,2.39331)(368,1.94571)(369,1.49831)(370,1.05107) 
};

\end{axis}
\end{tikzpicture}

			\newframe
			\begin{tikzpicture}[scale=0.5]

\begin{axis}[%
scale only axis,
width=4.52083in,
height=3.56562in,
xmin=0, xmax=400,
ymin=0, ymax=400,
xlabel={Slab Length [cm]},
ylabel={Power [W]},
title={$\text{tstep }= 90$},
axis on top]
\addplot [
color=blue,
solid
]
coordinates{
 (1,13.9911)(2,20.3593)(3,26.7197)(4,33.0699)(5,39.4076)(6,45.7302)(7,52.0354)(8,58.3208)(9,64.584)(10,70.8228)(11,77.0348)(12,83.2176)(13,89.3691)(14,95.487)(15,101.569)(16,107.613)(17,113.617)(18,119.578)(19,125.496)(20,131.366)(21,137.189)(22,142.96)(23,148.68)(24,154.345)(25,159.955)(26,165.506)(27,170.998)(28,176.428)(29,181.795)(30,187.098)(31,192.335)(32,197.504)(33,202.604)(34,207.633)(35,212.591)(36,217.475)(37,222.285)(38,227.019)(39,231.677)(40,236.256)(41,240.757)(42,245.178)(43,249.519)(44,253.778)(45,257.954)(46,262.048)(47,266.057)(48,269.982)(49,273.822)(50,277.577)(51,281.245)(52,284.827)(53,288.323)(54,291.731)(55,295.052)(56,298.285)(57,301.43)(58,304.488)(59,307.457)(60,310.339)(61,313.133)(62,315.839)(63,318.457)(64,320.988)(65,323.431)(66,325.788)(67,328.057)(68,330.241)(69,332.338)(70,334.35)(71,336.276)(72,338.118)(73,339.876)(74,341.55)(75,343.141)(76,344.65)(77,346.078)(78,347.424)(79,348.69)(80,349.877)(81,350.985)(82,352.015)(83,352.967)(84,353.844)(85,354.645)(86,355.372)(87,356.025)(88,356.606)(89,357.114)(90,357.553)(91,357.921)(92,358.22)(93,358.452)(94,358.617)(95,358.716)(96,358.751)(97,358.722)(98,358.63)(99,358.477)(100,358.264)(101,357.991)(102,357.659)(103,357.271)(104,356.826)(105,356.327)(106,355.774)(107,355.167)(108,354.509)(109,353.801)(110,353.043)(111,352.236)(112,351.382)(113,350.482)(114,349.537)(115,348.547)(116,347.515)(117,346.441)(118,345.325)(119,344.17)(120,342.976)(121,341.744)(122,340.476)(123,339.172)(124,337.833)(125,336.461)(126,335.056)(127,333.619)(128,332.152)(129,330.654)(130,329.128)(131,327.575)(132,325.994)(133,324.387)(134,322.756)(135,321.1)(136,319.421)(137,317.719)(138,315.997)(139,314.253)(140,312.49)(141,310.707)(142,308.907)(143,307.089)(144,305.254)(145,303.404)(146,301.538)(147,299.659)(148,297.765)(149,295.859)(150,293.941)(151,292.012)(152,290.071)(153,288.121)(154,286.161)(155,284.192)(156,282.216)(157,280.232)(158,278.24)(159,276.243)(160,274.24)(161,272.232)(162,270.219)(163,268.202)(164,266.181)(165,264.158)(166,262.132)(167,260.104)(168,258.075)(169,256.044)(170,254.013)(171,251.982)(172,249.951)(173,247.92)(174,245.891)(175,243.864)(176,241.838)(177,239.815)(178,237.794)(179,235.776)(180,233.762)(181,231.752)(182,229.745)(183,227.743)(184,225.746)(185,223.753)(186,221.766)(187,219.785)(188,217.809)(189,215.839)(190,213.876)(191,211.919)(192,209.969)(193,208.027)(194,206.091)(195,204.163)(196,202.242)(197,200.33)(198,198.425)(199,196.529)(200,194.641)(201,192.762)(202,190.892)(203,189.03)(204,187.178)(205,185.334)(206,183.5)(207,181.676)(208,179.861)(209,178.055)(210,176.26)(211,174.474)(212,172.698)(213,170.933)(214,169.177)(215,167.432)(216,165.697)(217,163.972)(218,162.258)(219,160.554)(220,158.861)(221,157.178)(222,155.506)(223,153.845)(224,152.194)(225,150.554)(226,148.925)(227,147.307)(228,145.699)(229,144.102)(230,142.516)(231,140.941)(232,139.377)(233,137.823)(234,136.281)(235,134.749)(236,133.228)(237,131.718)(238,130.218)(239,128.73)(240,127.252)(241,125.785)(242,124.328)(243,122.883)(244,121.447)(245,120.023)(246,118.609)(247,117.206)(248,115.813)(249,114.43)(250,113.058)(251,111.697)(252,110.345)(253,109.004)(254,107.673)(255,106.352)(256,105.042)(257,103.741)(258,102.451)(259,101.17)(260,99.899)(261,98.6379)(262,97.3866)(263,96.1449)(264,94.9129)(265,93.6903)(266,92.4773)(267,91.2736)(268,90.0792)(269,88.8941)(270,87.7182)(271,86.5513)(272,85.3934)(273,84.2445)(274,83.1045)(275,81.9733)(276,80.8507)(277,79.7368)(278,78.6314)(279,77.5345)(280,76.4459)(281,75.3657)(282,74.2937)(283,73.2298)(284,72.1739)(285,71.126)(286,70.086)(287,69.0538)(288,68.0293)(289,67.0123)(290,66.003)(291,65.001)(292,64.0064)(293,63.0191)(294,62.039)(295,61.0659)(296,60.0999)(297,59.1407)(298,58.1884)(299,57.2428)(300,56.3038)(301,55.3714)(302,54.4454)(303,53.5259)(304,52.6125)(305,51.7054)(306,50.8044)(307,49.9094)(308,49.0203)(309,48.137)(310,47.2594)(311,46.3875)(312,45.5212)(313,44.6603)(314,43.8047)(315,42.9545)(316,42.1094)(317,41.2694)(318,40.4344)(319,39.6043)(320,38.779)(321,37.9585)(322,37.1426)(323,36.3312)(324,35.5243)(325,34.7217)(326,33.9234)(327,33.1293)(328,32.3393)(329,31.5532)(330,30.7711)(331,29.9928)(332,29.2182)(333,28.4472)(334,27.6798)(335,26.9158)(336,26.1551)(337,25.3978)(338,24.6436)(339,23.8925)(340,23.1443)(341,22.3991)(342,21.6567)(343,20.917)(344,20.18)(345,19.4455)(346,18.7134)(347,17.9837)(348,17.2563)(349,16.531)(350,15.8079)(351,15.0867)(352,14.3675)(353,13.65)(354,12.9343)(355,12.2203)(356,11.5078)(357,10.7967)(358,10.0871)(359,9.37866)(360,8.67146)(361,7.96537)(362,7.26029)(363,6.55613)(364,5.85281)(365,5.15023)(366,4.44831)(367,3.74696)(368,3.04609)(369,2.3456)(370,1.64541) 
};

\addplot [
color=red,
solid
]
coordinates{
 (1,0.0787067)(2,0.116561)(3,0.157375)(4,0.202184)(5,0.252128)(6,0.308472)(7,0.372651)(8,0.446293)(9,0.531275)(10,0.629758)(11,0.744254)(12,0.877684)(13,1.03345)(14,1.21555)(15,1.42863)(16,1.67815)(17,1.97052)(18,2.31325)(19,2.71517)(20,3.18661)(21,3.73975)(22,4.38887)(23,5.15077)(24,6.04519)(25,7.09531)(26,8.32843)(27,9.77662)(28,11.4776)(29,13.4758)(30,15.8233)(31,18.5815)(32,21.3261)(33,24.056)(34,26.7702)(35,29.4677)(36,32.1474)(37,34.8085)(38,37.4499)(39,40.0708)(40,42.6702)(41,45.2472)(42,47.8011)(43,50.331)(44,52.836)(45,55.3154)(46,57.7685)(47,60.1944)(48,62.5925)(49,64.962)(50,67.3024)(51,69.613)(52,71.8932)(53,74.1424)(54,76.36)(55,78.5455)(56,80.6985)(57,82.8183)(58,84.9047)(59,86.9571)(60,88.9751)(61,90.9584)(62,92.9067)(63,94.8195)(64,96.6966)(65,98.5377)(66,100.343)(67,102.111)(68,103.843)(69,105.538)(70,107.195)(71,108.816)(72,110.4)(73,111.946)(74,113.454)(75,114.926)(76,116.359)(77,117.756)(78,119.115)(79,120.436)(80,121.72)(81,122.966)(82,124.176)(83,125.348)(84,126.484)(85,127.582)(86,128.644)(87,129.669)(88,130.658)(89,131.611)(90,132.528)(91,133.409)(92,134.255)(93,135.066)(94,135.842)(95,136.584)(96,137.291)(97,137.964)(98,138.603)(99,139.209)(100,139.782)(101,140.323)(102,140.831)(103,141.307)(104,141.752)(105,142.165)(106,142.548)(107,142.9)(108,143.222)(109,143.515)(110,143.779)(111,144.013)(112,144.22)(113,144.398)(114,144.549)(115,144.673)(116,144.77)(117,144.841)(118,144.886)(119,144.905)(120,144.9)(121,144.871)(122,144.817)(123,144.74)(124,144.639)(125,144.516)(126,144.371)(127,144.203)(128,144.015)(129,143.805)(130,143.575)(131,143.325)(132,143.055)(133,142.766)(134,142.458)(135,142.132)(136,141.787)(137,141.426)(138,141.047)(139,140.651)(140,140.24)(141,139.812)(142,139.369)(143,138.911)(144,138.438)(145,137.951)(146,137.451)(147,136.936)(148,136.409)(149,135.869)(150,135.316)(151,134.752)(152,134.176)(153,133.589)(154,132.99)(155,132.382)(156,131.763)(157,131.134)(158,130.496)(159,129.848)(160,129.192)(161,128.527)(162,127.854)(163,127.173)(164,126.485)(165,125.789)(166,125.086)(167,124.377)(168,123.661)(169,122.939)(170,122.211)(171,121.477)(172,120.738)(173,119.995)(174,119.246)(175,118.493)(176,117.735)(177,116.974)(178,116.208)(179,115.439)(180,114.667)(181,113.892)(182,113.113)(183,112.332)(184,111.549)(185,110.763)(186,109.975)(187,109.185)(188,108.394)(189,107.601)(190,106.807)(191,106.011)(192,105.215)(193,104.418)(194,103.62)(195,102.821)(196,102.023)(197,101.224)(198,100.425)(199,99.6264)(200,98.8279)(201,98.0299)(202,97.2324)(203,96.4356)(204,95.6396)(205,94.8444)(206,94.0504)(207,93.2574)(208,92.4658)(209,91.6755)(210,90.8867)(211,90.0994)(212,89.3139)(213,88.5301)(214,87.7482)(215,86.9682)(216,86.1903)(217,85.4145)(218,84.6409)(219,83.8696)(220,83.1006)(221,82.3341)(222,81.5701)(223,80.8086)(224,80.0498)(225,79.2936)(226,78.5402)(227,77.7896)(228,77.0418)(229,76.297)(230,75.5551)(231,74.8162)(232,74.0803)(233,73.3476)(234,72.6179)(235,71.8915)(236,71.1682)(237,70.4482)(238,69.7314)(239,69.018)(240,68.3078)(241,67.601)(242,66.8976)(243,66.1976)(244,65.501)(245,64.8078)(246,64.1181)(247,63.4319)(248,62.7491)(249,62.0699)(250,61.3941)(251,60.7219)(252,60.0532)(253,59.388)(254,58.7263)(255,58.0682)(256,57.4137)(257,56.7627)(258,56.1152)(259,55.4713)(260,54.8309)(261,54.1941)(262,53.5608)(263,52.931)(264,52.3048)(265,51.682)(266,51.0628)(267,50.4471)(268,49.8348)(269,49.226)(270,48.6207)(271,48.0188)(272,47.4204)(273,46.8254)(274,46.2338)(275,45.6455)(276,45.0607)(277,44.4792)(278,43.901)(279,43.3261)(280,42.7545)(281,42.1862)(282,41.6211)(283,41.0592)(284,40.5006)(285,39.9451)(286,39.3928)(287,38.8436)(288,38.2975)(289,37.7545)(290,37.2145)(291,36.6776)(292,36.1436)(293,35.6127)(294,35.0846)(295,34.5595)(296,34.0372)(297,33.5178)(298,33.0013)(299,32.4875)(300,31.9764)(301,31.4682)(302,30.9626)(303,30.4596)(304,29.9593)(305,29.4616)(306,28.9665)(307,28.4739)(308,27.9838)(309,27.4961)(310,27.0109)(311,26.5281)(312,26.0477)(313,25.5696)(314,25.0938)(315,24.6202)(316,24.1489)(317,23.6798)(318,23.2128)(319,22.7479)(320,22.2851)(321,21.8244)(322,21.3656)(323,20.9088)(324,20.454)(325,20.0011)(326,19.55)(327,19.1007)(328,18.6532)(329,18.2075)(330,17.7635)(331,17.3211)(332,16.8804)(333,16.4413)(334,16.0037)(335,15.5676)(336,15.1331)(337,14.6999)(338,14.2682)(339,13.8378)(340,13.4088)(341,12.981)(342,12.5545)(343,12.1292)(344,11.7051)(345,11.2821)(346,10.8601)(347,10.4393)(348,10.0194)(349,9.60052)(350,9.18257)(351,8.76551)(352,8.3493)(353,7.9339)(354,7.51928)(355,7.1054)(356,6.6922)(357,6.27966)(358,5.86774)(359,5.45639)(360,5.04557)(361,4.63525)(362,4.22539)(363,3.81594)(364,3.40687)(365,2.99813)(366,2.58968)(367,2.18149)(368,1.77352)(369,1.36572)(370,0.958056) 
};

\end{axis}
\end{tikzpicture}

			\newframe
			\begin{tikzpicture}[scale=0.5]
\begin{axis}[%
scale only axis,
width=4.52083in,
height=3.56562in,
xmin=0, xmax=400,
ymin=0, ymax=400,
xlabel={Slab Length [cm]},
ylabel={Power [W]},
title={$\text{tstep }= 96$},
axis on top]
\addplot [
color=blue,
solid
]
coordinates{
 (1,13.9911)(2,20.3593)(3,26.7197)(4,33.0699)(5,39.4076)(6,45.7302)(7,52.0354)(8,58.3208)(9,64.584)(10,70.8228)(11,77.0348)(12,83.2176)(13,89.3691)(14,95.487)(15,101.569)(16,107.613)(17,113.617)(18,119.578)(19,125.496)(20,131.366)(21,137.189)(22,142.96)(23,148.68)(24,154.345)(25,159.955)(26,165.506)(27,170.998)(28,176.428)(29,181.795)(30,187.098)(31,192.335)(32,197.504)(33,202.604)(34,207.633)(35,212.591)(36,217.475)(37,222.285)(38,227.019)(39,231.677)(40,236.256)(41,240.757)(42,245.178)(43,249.519)(44,253.778)(45,257.954)(46,262.048)(47,266.057)(48,269.982)(49,273.822)(50,277.577)(51,281.245)(52,284.827)(53,288.323)(54,291.731)(55,295.052)(56,298.285)(57,301.43)(58,304.488)(59,307.457)(60,310.339)(61,313.133)(62,315.839)(63,318.457)(64,320.988)(65,323.431)(66,325.788)(67,328.057)(68,330.241)(69,332.338)(70,334.35)(71,336.276)(72,338.118)(73,339.876)(74,341.55)(75,343.141)(76,344.65)(77,346.078)(78,347.424)(79,348.69)(80,349.877)(81,350.985)(82,352.015)(83,352.967)(84,353.844)(85,354.645)(86,355.372)(87,356.025)(88,356.606)(89,357.114)(90,357.553)(91,357.921)(92,358.22)(93,358.452)(94,358.617)(95,358.716)(96,358.751)(97,358.722)(98,358.63)(99,358.477)(100,358.264)(101,357.991)(102,357.659)(103,357.271)(104,356.826)(105,356.327)(106,355.774)(107,355.167)(108,354.509)(109,353.801)(110,353.043)(111,352.236)(112,351.382)(113,350.482)(114,349.537)(115,348.547)(116,347.515)(117,346.441)(118,345.325)(119,344.17)(120,342.976)(121,341.744)(122,340.476)(123,339.172)(124,337.833)(125,336.461)(126,335.056)(127,333.619)(128,332.152)(129,330.654)(130,329.128)(131,327.575)(132,325.994)(133,324.387)(134,322.756)(135,321.1)(136,319.421)(137,317.719)(138,315.997)(139,314.253)(140,312.49)(141,310.707)(142,308.907)(143,307.089)(144,305.254)(145,303.404)(146,301.538)(147,299.659)(148,297.765)(149,295.859)(150,293.941)(151,292.012)(152,290.071)(153,288.121)(154,286.161)(155,284.192)(156,282.216)(157,280.232)(158,278.24)(159,276.243)(160,274.24)(161,272.232)(162,270.219)(163,268.202)(164,266.181)(165,264.158)(166,262.132)(167,260.104)(168,258.075)(169,256.044)(170,254.013)(171,251.982)(172,249.951)(173,247.92)(174,245.891)(175,243.864)(176,241.838)(177,239.815)(178,237.794)(179,235.776)(180,233.762)(181,231.752)(182,229.745)(183,227.743)(184,225.746)(185,223.753)(186,221.766)(187,219.785)(188,217.809)(189,215.839)(190,213.876)(191,211.919)(192,209.969)(193,208.027)(194,206.091)(195,204.163)(196,202.242)(197,200.33)(198,198.425)(199,196.529)(200,194.641)(201,192.762)(202,190.892)(203,189.03)(204,187.178)(205,185.334)(206,183.5)(207,181.676)(208,179.861)(209,178.055)(210,176.26)(211,174.474)(212,172.698)(213,170.933)(214,169.177)(215,167.432)(216,165.697)(217,163.972)(218,162.258)(219,160.554)(220,158.861)(221,157.178)(222,155.506)(223,153.845)(224,152.194)(225,150.554)(226,148.925)(227,147.307)(228,145.699)(229,144.102)(230,142.516)(231,140.941)(232,139.377)(233,137.823)(234,136.281)(235,134.749)(236,133.228)(237,131.718)(238,130.218)(239,128.73)(240,127.252)(241,125.785)(242,124.328)(243,122.883)(244,121.447)(245,120.023)(246,118.609)(247,117.206)(248,115.813)(249,114.43)(250,113.058)(251,111.697)(252,110.345)(253,109.004)(254,107.673)(255,106.352)(256,105.042)(257,103.741)(258,102.451)(259,101.17)(260,99.899)(261,98.6379)(262,97.3866)(263,96.1449)(264,94.9129)(265,93.6903)(266,92.4773)(267,91.2736)(268,90.0792)(269,88.8941)(270,87.7182)(271,86.5513)(272,85.3934)(273,84.2445)(274,83.1045)(275,81.9733)(276,80.8507)(277,79.7368)(278,78.6314)(279,77.5345)(280,76.4459)(281,75.3657)(282,74.2937)(283,73.2298)(284,72.1739)(285,71.126)(286,70.086)(287,69.0538)(288,68.0293)(289,67.0123)(290,66.003)(291,65.001)(292,64.0064)(293,63.0191)(294,62.039)(295,61.0659)(296,60.0999)(297,59.1407)(298,58.1884)(299,57.2428)(300,56.3038)(301,55.3714)(302,54.4454)(303,53.5259)(304,52.6125)(305,51.7054)(306,50.8044)(307,49.9094)(308,49.0203)(309,48.137)(310,47.2594)(311,46.3875)(312,45.5212)(313,44.6603)(314,43.8047)(315,42.9545)(316,42.1094)(317,41.2694)(318,40.4344)(319,39.6043)(320,38.779)(321,37.9585)(322,37.1426)(323,36.3312)(324,35.5243)(325,34.7217)(326,33.9234)(327,33.1293)(328,32.3393)(329,31.5532)(330,30.7711)(331,29.9928)(332,29.2182)(333,28.4472)(334,27.6798)(335,26.9158)(336,26.1551)(337,25.3978)(338,24.6436)(339,23.8925)(340,23.1443)(341,22.3991)(342,21.6567)(343,20.917)(344,20.18)(345,19.4455)(346,18.7134)(347,17.9837)(348,17.2563)(349,16.531)(350,15.8079)(351,15.0867)(352,14.3675)(353,13.65)(354,12.9343)(355,12.2203)(356,11.5078)(357,10.7967)(358,10.0871)(359,9.37866)(360,8.67146)(361,7.96537)(362,7.26029)(363,6.55613)(364,5.85281)(365,5.15023)(366,4.44831)(367,3.74696)(368,3.04609)(369,2.3456)(370,1.64541) 
};

\addplot [
color=red,
solid
]
coordinates{
 (1,0.0503272)(2,0.0745291)(3,0.100619)(4,0.129258)(5,0.16117)(6,0.197165)(7,0.238155)(8,0.285179)(9,0.33943)(10,0.402287)(11,0.475346)(12,0.560468)(13,0.65982)(14,0.775935)(15,0.911778)(16,1.07082)(17,1.25713)(18,1.47549)(19,1.73149)(20,2.0317)(21,2.38387)(22,2.79704)(23,3.28188)(24,3.85092)(25,4.51886)(26,5.303)(27,6.22368)(28,7.30481)(29,8.57448)(30,10.0658)(31,11.8175)(32,13.8755)(33,16.2935)(34,18.6995)(35,21.0925)(36,23.4717)(37,25.8362)(38,28.1852)(39,30.5176)(40,32.8329)(41,35.13)(42,37.4083)(43,39.667)(44,41.9053)(45,44.1225)(46,46.3178)(47,48.4907)(48,50.6404)(49,52.7663)(50,54.8677)(51,56.9441)(52,58.995)(53,61.0196)(54,63.0176)(55,64.9884)(56,66.9315)(57,68.8465)(58,70.7329)(59,72.5904)(60,74.4184)(61,76.2167)(62,77.9848)(63,79.7226)(64,81.4296)(65,83.1056)(66,84.7503)(67,86.3635)(68,87.945)(69,89.4946)(70,91.012)(71,92.4972)(72,93.9501)(73,95.3704)(74,96.7581)(75,98.1132)(76,99.4355)(77,100.725)(78,101.982)(79,103.206)(80,104.397)(81,105.556)(82,106.682)(83,107.775)(84,108.835)(85,109.864)(86,110.86)(87,111.823)(88,112.755)(89,113.655)(90,114.523)(91,115.36)(92,116.165)(93,116.939)(94,117.682)(95,118.395)(96,119.077)(97,119.728)(98,120.35)(99,120.942)(100,121.505)(101,122.039)(102,122.544)(103,123.02)(104,123.468)(105,123.889)(106,124.282)(107,124.647)(108,124.986)(109,125.298)(110,125.584)(111,125.844)(112,126.078)(113,126.288)(114,126.472)(115,126.633)(116,126.769)(117,126.881)(118,126.971)(119,127.037)(120,127.081)(121,127.103)(122,127.102)(123,127.081)(124,127.039)(125,126.975)(126,126.892)(127,126.789)(128,126.666)(129,126.525)(130,126.364)(131,126.185)(132,125.989)(133,125.774)(134,125.543)(135,125.295)(136,125.03)(137,124.749)(138,124.453)(139,124.141)(140,123.814)(141,123.473)(142,123.117)(143,122.748)(144,122.365)(145,121.968)(146,121.559)(147,121.138)(148,120.704)(149,120.258)(150,119.801)(151,119.333)(152,118.854)(153,118.364)(154,117.864)(155,117.355)(156,116.835)(157,116.307)(158,115.769)(159,115.223)(160,114.668)(161,114.106)(162,113.535)(163,112.957)(164,112.372)(165,111.78)(166,111.181)(167,110.575)(168,109.964)(169,109.346)(170,108.723)(171,108.094)(172,107.46)(173,106.821)(174,106.178)(175,105.53)(176,104.877)(177,104.221)(178,103.56)(179,102.896)(180,102.229)(181,101.558)(182,100.885)(183,100.208)(184,99.529)(185,98.8474)(186,98.1635)(187,97.4775)(188,96.7896)(189,96.1)(190,95.4087)(191,94.7161)(192,94.0222)(193,93.3271)(194,92.6312)(195,91.9344)(196,91.2369)(197,90.5389)(198,89.8406)(199,89.1419)(200,88.4432)(201,87.7445)(202,87.0459)(203,86.3475)(204,85.6495)(205,84.952)(206,84.255)(207,83.5588)(208,82.8634)(209,82.1688)(210,81.4753)(211,80.7828)(212,80.0916)(213,79.4016)(214,78.7129)(215,78.0258)(216,77.3401)(217,76.656)(218,75.9737)(219,75.293)(220,74.6142)(221,73.9373)(222,73.2624)(223,72.5894)(224,71.9186)(225,71.2498)(226,70.5833)(227,69.919)(228,69.257)(229,68.5974)(230,67.9402)(231,67.2854)(232,66.6331)(233,65.9833)(234,65.3361)(235,64.6914)(236,64.0495)(237,63.4102)(238,62.7735)(239,62.1397)(240,61.5086)(241,60.8802)(242,60.2547)(243,59.6321)(244,59.0122)(245,58.3953)(246,57.7813)(247,57.1701)(248,56.5619)(249,55.9567)(250,55.3544)(251,54.7551)(252,54.1587)(253,53.5654)(254,52.975)(255,52.3876)(256,51.8033)(257,51.2219)(258,50.6436)(259,50.0683)(260,49.496)(261,48.9267)(262,48.3605)(263,47.7972)(264,47.237)(265,46.6797)(266,46.1255)(267,45.5742)(268,45.026)(269,44.4807)(270,43.9384)(271,43.399)(272,42.8626)(273,42.3292)(274,41.7986)(275,41.271)(276,40.7463)(277,40.2244)(278,39.7054)(279,39.1893)(280,38.676)(281,38.1656)(282,37.6579)(283,37.153)(284,36.6509)(285,36.1516)(286,35.6549)(287,35.161)(288,34.6697)(289,34.1812)(290,33.6952)(291,33.2119)(292,32.7312)(293,32.2531)(294,31.7775)(295,31.3044)(296,30.8338)(297,30.3657)(298,29.9001)(299,29.4369)(300,28.9761)(301,28.5177)(302,28.0616)(303,27.6078)(304,27.1563)(305,26.7071)(306,26.2601)(307,25.8154)(308,25.3728)(309,24.9323)(310,24.494)(311,24.0578)(312,23.6236)(313,23.1915)(314,22.7614)(315,22.3332)(316,21.907)(317,21.4827)(318,21.0603)(319,20.6397)(320,20.2209)(321,19.804)(322,19.3888)(323,18.9753)(324,18.5635)(325,18.1533)(326,17.7448)(327,17.3379)(328,16.9325)(329,16.5287)(330,16.1264)(331,15.7255)(332,15.326)(333,14.928)(334,14.5313)(335,14.1359)(336,13.7419)(337,13.3491)(338,12.9575)(339,12.5671)(340,12.1779)(341,11.7898)(342,11.4028)(343,11.0169)(344,10.632)(345,10.2481)(346,9.86512)(347,9.48308)(348,9.10192)(349,8.72162)(350,8.34213)(351,7.96343)(352,7.58548)(353,7.20824)(354,6.83168)(355,6.45577)(356,6.08046)(357,5.70573)(358,5.33154)(359,4.95786)(360,4.58464)(361,4.21186)(362,3.83948)(363,3.46746)(364,3.09577)(365,2.72438)(366,2.35324)(367,1.98233)(368,1.61161)(369,1.24105)(370,0.870599) 
};

\end{axis}
\end{tikzpicture}

			\newframe
			\begin{tikzpicture}[scale=0.5]

\begin{axis}[%
scale only axis,
width=4.52083in,
height=3.56562in,
xmin=0, xmax=400,
ymin=0, ymax=400,
xlabel={Slab Length [cm]},
ylabel={Power [W]},
title={$\text{tstep }= 102$},
axis on top]
\addplot [
color=blue,
solid
]
coordinates{
 (1,13.9911)(2,20.3593)(3,26.7197)(4,33.0699)(5,39.4076)(6,45.7302)(7,52.0354)(8,58.3208)(9,64.584)(10,70.8228)(11,77.0348)(12,83.2176)(13,89.3691)(14,95.487)(15,101.569)(16,107.613)(17,113.617)(18,119.578)(19,125.496)(20,131.366)(21,137.189)(22,142.96)(23,148.68)(24,154.345)(25,159.955)(26,165.506)(27,170.998)(28,176.428)(29,181.795)(30,187.098)(31,192.335)(32,197.504)(33,202.604)(34,207.633)(35,212.591)(36,217.475)(37,222.285)(38,227.019)(39,231.677)(40,236.256)(41,240.757)(42,245.178)(43,249.519)(44,253.778)(45,257.954)(46,262.048)(47,266.057)(48,269.982)(49,273.822)(50,277.577)(51,281.245)(52,284.827)(53,288.323)(54,291.731)(55,295.052)(56,298.285)(57,301.43)(58,304.488)(59,307.457)(60,310.339)(61,313.133)(62,315.839)(63,318.457)(64,320.988)(65,323.431)(66,325.788)(67,328.057)(68,330.241)(69,332.338)(70,334.35)(71,336.276)(72,338.118)(73,339.876)(74,341.55)(75,343.141)(76,344.65)(77,346.078)(78,347.424)(79,348.69)(80,349.877)(81,350.985)(82,352.015)(83,352.967)(84,353.844)(85,354.645)(86,355.372)(87,356.025)(88,356.606)(89,357.114)(90,357.553)(91,357.921)(92,358.22)(93,358.452)(94,358.617)(95,358.716)(96,358.751)(97,358.722)(98,358.63)(99,358.477)(100,358.264)(101,357.991)(102,357.659)(103,357.271)(104,356.826)(105,356.327)(106,355.774)(107,355.167)(108,354.509)(109,353.801)(110,353.043)(111,352.236)(112,351.382)(113,350.482)(114,349.537)(115,348.547)(116,347.515)(117,346.441)(118,345.325)(119,344.17)(120,342.976)(121,341.744)(122,340.476)(123,339.172)(124,337.833)(125,336.461)(126,335.056)(127,333.619)(128,332.152)(129,330.654)(130,329.128)(131,327.575)(132,325.994)(133,324.387)(134,322.756)(135,321.1)(136,319.421)(137,317.719)(138,315.997)(139,314.253)(140,312.49)(141,310.707)(142,308.907)(143,307.089)(144,305.254)(145,303.404)(146,301.538)(147,299.659)(148,297.765)(149,295.859)(150,293.941)(151,292.012)(152,290.071)(153,288.121)(154,286.161)(155,284.192)(156,282.216)(157,280.232)(158,278.24)(159,276.243)(160,274.24)(161,272.232)(162,270.219)(163,268.202)(164,266.181)(165,264.158)(166,262.132)(167,260.104)(168,258.075)(169,256.044)(170,254.013)(171,251.982)(172,249.951)(173,247.92)(174,245.891)(175,243.864)(176,241.838)(177,239.815)(178,237.794)(179,235.776)(180,233.762)(181,231.752)(182,229.745)(183,227.743)(184,225.746)(185,223.753)(186,221.766)(187,219.785)(188,217.809)(189,215.839)(190,213.876)(191,211.919)(192,209.969)(193,208.027)(194,206.091)(195,204.163)(196,202.242)(197,200.33)(198,198.425)(199,196.529)(200,194.641)(201,192.762)(202,190.892)(203,189.03)(204,187.178)(205,185.334)(206,183.5)(207,181.676)(208,179.861)(209,178.055)(210,176.26)(211,174.474)(212,172.698)(213,170.933)(214,169.177)(215,167.432)(216,165.697)(217,163.972)(218,162.258)(219,160.554)(220,158.861)(221,157.178)(222,155.506)(223,153.845)(224,152.194)(225,150.554)(226,148.925)(227,147.307)(228,145.699)(229,144.102)(230,142.516)(231,140.941)(232,139.377)(233,137.823)(234,136.281)(235,134.749)(236,133.228)(237,131.718)(238,130.218)(239,128.73)(240,127.252)(241,125.785)(242,124.328)(243,122.883)(244,121.447)(245,120.023)(246,118.609)(247,117.206)(248,115.813)(249,114.43)(250,113.058)(251,111.697)(252,110.345)(253,109.004)(254,107.673)(255,106.352)(256,105.042)(257,103.741)(258,102.451)(259,101.17)(260,99.899)(261,98.6379)(262,97.3866)(263,96.1449)(264,94.9129)(265,93.6903)(266,92.4773)(267,91.2736)(268,90.0792)(269,88.8941)(270,87.7182)(271,86.5513)(272,85.3934)(273,84.2445)(274,83.1045)(275,81.9733)(276,80.8507)(277,79.7368)(278,78.6314)(279,77.5345)(280,76.4459)(281,75.3657)(282,74.2937)(283,73.2298)(284,72.1739)(285,71.126)(286,70.086)(287,69.0538)(288,68.0293)(289,67.0123)(290,66.003)(291,65.001)(292,64.0064)(293,63.0191)(294,62.039)(295,61.0659)(296,60.0999)(297,59.1407)(298,58.1884)(299,57.2428)(300,56.3038)(301,55.3714)(302,54.4454)(303,53.5259)(304,52.6125)(305,51.7054)(306,50.8044)(307,49.9094)(308,49.0203)(309,48.137)(310,47.2594)(311,46.3875)(312,45.5212)(313,44.6603)(314,43.8047)(315,42.9545)(316,42.1094)(317,41.2694)(318,40.4344)(319,39.6043)(320,38.779)(321,37.9585)(322,37.1426)(323,36.3312)(324,35.5243)(325,34.7217)(326,33.9234)(327,33.1293)(328,32.3393)(329,31.5532)(330,30.7711)(331,29.9928)(332,29.2182)(333,28.4472)(334,27.6798)(335,26.9158)(336,26.1551)(337,25.3978)(338,24.6436)(339,23.8925)(340,23.1443)(341,22.3991)(342,21.6567)(343,20.917)(344,20.18)(345,19.4455)(346,18.7134)(347,17.9837)(348,17.2563)(349,16.531)(350,15.8079)(351,15.0867)(352,14.3675)(353,13.65)(354,12.9343)(355,12.2203)(356,11.5078)(357,10.7967)(358,10.0871)(359,9.37866)(360,8.67146)(361,7.96537)(362,7.26029)(363,6.55613)(364,5.85281)(365,5.15023)(366,4.44831)(367,3.74696)(368,3.04609)(369,2.3456)(370,1.64541) 
};

\addplot [
color=red,
solid
]
coordinates{
 (1,0.111156)(2,0.164713)(3,0.222593)(4,0.286318)(5,0.357557)(6,0.438179)(7,0.5303)(8,0.636336)(9,0.759068)(10,0.901717)(11,1.06802)(12,1.26235)(13,1.48981)(14,1.75635)(15,2.06898)(16,2.4359)(17,2.86675)(18,3.37284)(19,3.96745)(20,4.66621)(21,5.48747)(22,6.4528)(23,7.58757)(24,8.92159)(25,10.4899)(26,12.3338)(27,14.5017)(28,16.6625)(29,18.8155)(30,20.9599)(31,23.0949)(32,25.2198)(33,27.3336)(34,29.4357)(35,31.5251)(36,33.6012)(37,35.663)(38,37.71)(39,39.7413)(40,41.7563)(41,43.7542)(42,45.7344)(43,47.6961)(44,49.6388)(45,51.5617)(46,53.4644)(47,55.3462)(48,57.2066)(49,59.0449)(50,60.8607)(51,62.6535)(52,64.4227)(53,66.1679)(54,67.8886)(55,69.5845)(56,71.255)(57,72.8999)(58,74.5187)(59,76.1111)(60,77.6767)(61,79.2153)(62,80.7266)(63,82.2102)(64,83.666)(65,85.0938)(66,86.4932)(67,87.8641)(68,89.2064)(69,90.5199)(70,91.8044)(71,93.0598)(72,94.2861)(73,95.4831)(74,96.6508)(75,97.7891)(76,98.8981)(77,99.9776)(78,101.028)(79,102.049)(80,103.04)(81,104.002)(82,104.935)(83,105.838)(84,106.713)(85,107.559)(86,108.375)(87,109.163)(88,109.923)(89,110.654)(90,111.356)(91,112.031)(92,112.678)(93,113.297)(94,113.888)(95,114.452)(96,114.99)(97,115.5)(98,115.984)(99,116.441)(100,116.873)(101,117.278)(102,117.658)(103,118.013)(104,118.343)(105,118.648)(106,118.929)(107,119.186)(108,119.419)(109,119.629)(110,119.815)(111,119.979)(112,120.12)(113,120.239)(114,120.337)(115,120.412)(116,120.467)(117,120.501)(118,120.514)(119,120.508)(120,120.481)(121,120.436)(122,120.371)(123,120.287)(124,120.185)(125,120.065)(126,119.927)(127,119.772)(128,119.6)(129,119.411)(130,119.206)(131,118.985)(132,118.749)(133,118.497)(134,118.23)(135,117.948)(136,117.653)(137,117.343)(138,117.02)(139,116.683)(140,116.334)(141,115.972)(142,115.597)(143,115.211)(144,114.813)(145,114.403)(146,113.983)(147,113.552)(148,113.11)(149,112.658)(150,112.197)(151,111.725)(152,111.245)(153,110.756)(154,110.258)(155,109.751)(156,109.237)(157,108.714)(158,108.184)(159,107.647)(160,107.102)(161,106.551)(162,105.993)(163,105.429)(164,104.859)(165,104.283)(166,103.701)(167,103.114)(168,102.522)(169,101.925)(170,101.324)(171,100.718)(172,100.107)(173,99.4928)(174,98.8745)(175,98.2527)(176,97.6274)(177,96.9989)(178,96.3673)(179,95.7328)(180,95.0957)(181,94.4561)(182,93.8141)(183,93.17)(184,92.5239)(185,91.876)(186,91.2264)(187,90.5753)(188,89.9229)(189,89.2692)(190,88.6145)(191,87.9589)(192,87.3025)(193,86.6455)(194,85.9879)(195,85.33)(196,84.6718)(197,84.0135)(198,83.3552)(199,82.697)(200,82.039)(201,81.3813)(202,80.7241)(203,80.0674)(204,79.4114)(205,78.7561)(206,78.1016)(207,77.4481)(208,76.7956)(209,76.1442)(210,75.494)(211,74.845)(212,74.1974)(213,73.5513)(214,72.9066)(215,72.2636)(216,71.6221)(217,70.9824)(218,70.3445)(219,69.7084)(220,69.0742)(221,68.4419)(222,67.8117)(223,67.1835)(224,66.5575)(225,65.9336)(226,65.3119)(227,64.6925)(228,64.0754)(229,63.4607)(230,62.8483)(231,62.2384)(232,61.6309)(233,61.0259)(234,60.4234)(235,59.8235)(236,59.2262)(237,58.6314)(238,58.0393)(239,57.4499)(240,56.8632)(241,56.2792)(242,55.6978)(243,55.1193)(244,54.5435)(245,53.9704)(246,53.4002)(247,52.8328)(248,52.2682)(249,51.7064)(250,51.1474)(251,50.5913)(252,50.038)(253,49.4876)(254,48.9401)(255,48.3954)(256,47.8535)(257,47.3146)(258,46.7785)(259,46.2453)(260,45.7149)(261,45.1874)(262,44.6628)(263,44.141)(264,43.622)(265,43.1059)(266,42.5927)(267,42.0823)(268,41.5747)(269,41.0699)(270,40.5679)(271,40.0687)(272,39.5722)(273,39.0786)(274,38.5877)(275,38.0995)(276,37.6141)(277,37.1313)(278,36.6513)(279,36.1739)(280,35.6993)(281,35.2272)(282,34.7578)(283,34.291)(284,33.8268)(285,33.3652)(286,32.9061)(287,32.4495)(288,31.9955)(289,31.544)(290,31.0949)(291,30.6483)(292,30.2041)(293,29.7623)(294,29.3229)(295,28.8859)(296,28.4512)(297,28.0188)(298,27.5887)(299,27.1609)(300,26.7353)(301,26.3119)(302,25.8907)(303,25.4717)(304,25.0548)(305,24.64)(306,24.2273)(307,23.8166)(308,23.408)(309,23.0014)(310,22.5967)(311,22.194)(312,21.7933)(313,21.3944)(314,20.9973)(315,20.6022)(316,20.2088)(317,19.8172)(318,19.4273)(319,19.0392)(320,18.6527)(321,18.2679)(322,17.8847)(323,17.5032)(324,17.1232)(325,16.7447)(326,16.3678)(327,15.9923)(328,15.6183)(329,15.2457)(330,14.8745)(331,14.5047)(332,14.1361)(333,13.7689)(334,13.4029)(335,13.0382)(336,12.6747)(337,12.3123)(338,11.9511)(339,11.591)(340,11.2319)(341,10.8739)(342,10.517)(343,10.161)(344,9.80593)(345,9.4518)(346,9.09856)(347,8.74617)(348,8.3946)(349,8.04382)(350,7.69381)(351,7.34451)(352,6.99591)(353,6.64798)(354,6.30067)(355,5.95396)(356,5.60781)(357,5.2622)(358,4.91708)(359,4.57244)(360,4.22823)(361,3.88442)(362,3.54098)(363,3.19788)(364,2.85509)(365,2.51256)(366,2.17028)(367,1.82821)(368,1.48631)(369,1.14455)(370,0.802909) 
};

\end{axis}
\end{tikzpicture}

			\newframe
			\begin{tikzpicture}[scale=0.5]

\begin{axis}[%
scale only axis,
width=4.52083in,
height=3.56562in,
xmin=0, xmax=400,
ymin=0, ymax=400,
xlabel={Slab Length [cm]},
ylabel={Power [W]},
title={$\text{tstep }= 108$},
axis on top]
\addplot [
color=blue,
solid
]
coordinates{
 (1,13.9911)(2,20.3593)(3,26.7197)(4,33.0699)(5,39.4076)(6,45.7302)(7,52.0354)(8,58.3208)(9,64.584)(10,70.8228)(11,77.0348)(12,83.2176)(13,89.3691)(14,95.487)(15,101.569)(16,107.613)(17,113.617)(18,119.578)(19,125.496)(20,131.366)(21,137.189)(22,142.96)(23,148.68)(24,154.345)(25,159.955)(26,165.506)(27,170.998)(28,176.428)(29,181.795)(30,187.098)(31,192.335)(32,197.504)(33,202.604)(34,207.633)(35,212.591)(36,217.475)(37,222.285)(38,227.019)(39,231.677)(40,236.256)(41,240.757)(42,245.178)(43,249.519)(44,253.778)(45,257.954)(46,262.048)(47,266.057)(48,269.982)(49,273.822)(50,277.577)(51,281.245)(52,284.827)(53,288.323)(54,291.731)(55,295.052)(56,298.285)(57,301.43)(58,304.488)(59,307.457)(60,310.339)(61,313.133)(62,315.839)(63,318.457)(64,320.988)(65,323.431)(66,325.788)(67,328.057)(68,330.241)(69,332.338)(70,334.35)(71,336.276)(72,338.118)(73,339.876)(74,341.55)(75,343.141)(76,344.65)(77,346.078)(78,347.424)(79,348.69)(80,349.877)(81,350.985)(82,352.015)(83,352.967)(84,353.844)(85,354.645)(86,355.372)(87,356.025)(88,356.606)(89,357.114)(90,357.553)(91,357.921)(92,358.22)(93,358.452)(94,358.617)(95,358.716)(96,358.751)(97,358.722)(98,358.63)(99,358.477)(100,358.264)(101,357.991)(102,357.659)(103,357.271)(104,356.826)(105,356.327)(106,355.774)(107,355.167)(108,354.509)(109,353.801)(110,353.043)(111,352.236)(112,351.382)(113,350.482)(114,349.537)(115,348.547)(116,347.515)(117,346.441)(118,345.325)(119,344.17)(120,342.976)(121,341.744)(122,340.476)(123,339.172)(124,337.833)(125,336.461)(126,335.056)(127,333.619)(128,332.152)(129,330.654)(130,329.128)(131,327.575)(132,325.994)(133,324.387)(134,322.756)(135,321.1)(136,319.421)(137,317.719)(138,315.997)(139,314.253)(140,312.49)(141,310.707)(142,308.907)(143,307.089)(144,305.254)(145,303.404)(146,301.538)(147,299.659)(148,297.765)(149,295.859)(150,293.941)(151,292.012)(152,290.071)(153,288.121)(154,286.161)(155,284.192)(156,282.216)(157,280.232)(158,278.24)(159,276.243)(160,274.24)(161,272.232)(162,270.219)(163,268.202)(164,266.181)(165,264.158)(166,262.132)(167,260.104)(168,258.075)(169,256.044)(170,254.013)(171,251.982)(172,249.951)(173,247.92)(174,245.891)(175,243.864)(176,241.838)(177,239.815)(178,237.794)(179,235.776)(180,233.762)(181,231.752)(182,229.745)(183,227.743)(184,225.746)(185,223.753)(186,221.766)(187,219.785)(188,217.809)(189,215.839)(190,213.876)(191,211.919)(192,209.969)(193,208.027)(194,206.091)(195,204.163)(196,202.242)(197,200.33)(198,198.425)(199,196.529)(200,194.641)(201,192.762)(202,190.892)(203,189.03)(204,187.178)(205,185.334)(206,183.5)(207,181.676)(208,179.861)(209,178.055)(210,176.26)(211,174.474)(212,172.698)(213,170.933)(214,169.177)(215,167.432)(216,165.697)(217,163.972)(218,162.258)(219,160.554)(220,158.861)(221,157.178)(222,155.506)(223,153.845)(224,152.194)(225,150.554)(226,148.925)(227,147.307)(228,145.699)(229,144.102)(230,142.516)(231,140.941)(232,139.377)(233,137.823)(234,136.281)(235,134.749)(236,133.228)(237,131.718)(238,130.218)(239,128.73)(240,127.252)(241,125.785)(242,124.328)(243,122.883)(244,121.447)(245,120.023)(246,118.609)(247,117.206)(248,115.813)(249,114.43)(250,113.058)(251,111.697)(252,110.345)(253,109.004)(254,107.673)(255,106.352)(256,105.042)(257,103.741)(258,102.451)(259,101.17)(260,99.899)(261,98.6379)(262,97.3866)(263,96.1449)(264,94.9129)(265,93.6903)(266,92.4773)(267,91.2736)(268,90.0792)(269,88.8941)(270,87.7182)(271,86.5513)(272,85.3934)(273,84.2445)(274,83.1045)(275,81.9733)(276,80.8507)(277,79.7368)(278,78.6314)(279,77.5345)(280,76.4459)(281,75.3657)(282,74.2937)(283,73.2298)(284,72.1739)(285,71.126)(286,70.086)(287,69.0538)(288,68.0293)(289,67.0123)(290,66.003)(291,65.001)(292,64.0064)(293,63.0191)(294,62.039)(295,61.0659)(296,60.0999)(297,59.1407)(298,58.1884)(299,57.2428)(300,56.3038)(301,55.3714)(302,54.4454)(303,53.5259)(304,52.6125)(305,51.7054)(306,50.8044)(307,49.9094)(308,49.0203)(309,48.137)(310,47.2594)(311,46.3875)(312,45.5212)(313,44.6603)(314,43.8047)(315,42.9545)(316,42.1094)(317,41.2694)(318,40.4344)(319,39.6043)(320,38.779)(321,37.9585)(322,37.1426)(323,36.3312)(324,35.5243)(325,34.7217)(326,33.9234)(327,33.1293)(328,32.3393)(329,31.5532)(330,30.7711)(331,29.9928)(332,29.2182)(333,28.4472)(334,27.6798)(335,26.9158)(336,26.1551)(337,25.3978)(338,24.6436)(339,23.8925)(340,23.1443)(341,22.3991)(342,21.6567)(343,20.917)(344,20.18)(345,19.4455)(346,18.7134)(347,17.9837)(348,17.2563)(349,16.531)(350,15.8079)(351,15.0867)(352,14.3675)(353,13.65)(354,12.9343)(355,12.2203)(356,11.5078)(357,10.7967)(358,10.0871)(359,9.37866)(360,8.67146)(361,7.96537)(362,7.26029)(363,6.55613)(364,5.85281)(365,5.15023)(366,4.44831)(367,3.74696)(368,3.04609)(369,2.3456)(370,1.64541) 
};

\addplot [
color=red,
solid
]
coordinates{
 (1,0.737081)(2,1.09247)(3,1.47692)(4,1.90065)(5,2.37492)(6,2.91233)(7,3.52715)(8,4.23571)(9,5.05681)(10,6.01226)(11,7.12739)(12,8.43179)(13,9.96006)(14,11.7527)(15,13.8573)(16,15.9579)(17,18.0539)(18,20.1449)(19,22.2303)(20,24.3094)(21,26.3816)(22,28.4465)(23,30.5032)(24,32.5513)(25,34.5899)(26,36.6186)(27,38.6366)(28,40.6432)(29,42.6378)(30,44.6195)(31,46.5879)(32,48.5421)(33,50.4814)(34,52.4052)(35,54.3127)(36,56.2033)(37,58.0764)(38,59.9314)(39,61.7675)(40,63.5843)(41,65.381)(42,67.1573)(43,68.9124)(44,70.646)(45,72.3574)(46,74.0462)(47,75.712)(48,77.3542)(49,78.9724)(50,80.5662)(51,82.1353)(52,83.6792)(53,85.1976)(54,86.6901)(55,88.1565)(56,89.5964)(57,91.0095)(58,92.3956)(59,93.7544)(60,95.0858)(61,96.3894)(62,97.6652)(63,98.9129)(64,100.132)(65,101.324)(66,102.486)(67,103.62)(68,104.726)(69,105.803)(70,106.851)(71,107.87)(72,108.861)(73,109.822)(74,110.755)(75,111.659)(76,112.535)(77,113.382)(78,114.2)(79,114.989)(80,115.751)(81,116.484)(82,117.189)(83,117.866)(84,118.515)(85,119.136)(86,119.73)(87,120.296)(88,120.835)(89,121.348)(90,121.833)(91,122.293)(92,122.725)(93,123.132)(94,123.513)(95,123.869)(96,124.199)(97,124.504)(98,124.785)(99,125.041)(100,125.273)(101,125.481)(102,125.666)(103,125.827)(104,125.966)(105,126.082)(106,126.175)(107,126.247)(108,126.297)(109,126.326)(110,126.333)(111,126.321)(112,126.287)(113,126.234)(114,126.162)(115,126.07)(116,125.959)(117,125.829)(118,125.682)(119,125.516)(120,125.333)(121,125.132)(122,124.915)(123,124.682)(124,124.432)(125,124.166)(126,123.885)(127,123.589)(128,123.278)(129,122.952)(130,122.612)(131,122.259)(132,121.892)(133,121.512)(134,121.12)(135,120.714)(136,120.297)(137,119.868)(138,119.427)(139,118.976)(140,118.513)(141,118.04)(142,117.556)(143,117.063)(144,116.56)(145,116.048)(146,115.526)(147,114.996)(148,114.457)(149,113.911)(150,113.356)(151,112.793)(152,112.224)(153,111.647)(154,111.063)(155,110.473)(156,109.876)(157,109.274)(158,108.665)(159,108.051)(160,107.432)(161,106.807)(162,106.178)(163,105.544)(164,104.905)(165,104.263)(166,103.616)(167,102.966)(168,102.312)(169,101.655)(170,100.994)(171,100.331)(172,99.6648)(173,98.9961)(174,98.325)(175,97.6517)(176,96.9763)(177,96.2991)(178,95.6202)(179,94.9398)(180,94.2581)(181,93.5751)(182,92.8911)(183,92.2062)(184,91.5205)(185,90.8342)(186,90.1475)(187,89.4604)(188,88.7732)(189,88.0859)(190,87.3986)(191,86.7116)(192,86.0248)(193,85.3385)(194,84.6527)(195,83.9676)(196,83.2832)(197,82.5997)(198,81.9171)(199,81.2356)(200,80.5553)(201,79.8763)(202,79.1985)(203,78.5222)(204,77.8474)(205,77.1743)(206,76.5028)(207,75.833)(208,75.1651)(209,74.4991)(210,73.835)(211,73.173)(212,72.513)(213,71.8553)(214,71.1997)(215,70.5465)(216,69.8955)(217,69.247)(218,68.6009)(219,67.9572)(220,67.3161)(221,66.6776)(222,66.0417)(223,65.4085)(224,64.7779)(225,64.1501)(226,63.5251)(227,62.9028)(228,62.2834)(229,61.6669)(230,61.0532)(231,60.4425)(232,59.8347)(233,59.2298)(234,58.628)(235,58.0292)(236,57.4333)(237,56.8406)(238,56.2509)(239,55.6642)(240,55.0806)(241,54.5002)(242,53.9228)(243,53.3486)(244,52.7775)(245,52.2095)(246,51.6446)(247,51.0829)(248,50.5243)(249,49.9689)(250,49.4166)(251,48.8675)(252,48.3215)(253,47.7786)(254,47.2389)(255,46.7023)(256,46.1689)(257,45.6385)(258,45.1113)(259,44.5872)(260,44.0662)(261,43.5483)(262,43.0335)(263,42.5218)(264,42.0131)(265,41.5075)(266,41.0049)(267,40.5053)(268,40.0087)(269,39.5152)(270,39.0246)(271,38.5369)(272,38.0523)(273,37.5705)(274,37.0917)(275,36.6157)(276,36.1426)(277,35.6724)(278,35.205)(279,34.7404)(280,34.2787)(281,33.8197)(282,33.3634)(283,32.9099)(284,32.4591)(285,32.0109)(286,31.5655)(287,31.1226)(288,30.6824)(289,30.2448)(290,29.8097)(291,29.3772)(292,28.9472)(293,28.5197)(294,28.0946)(295,27.672)(296,27.2518)(297,26.834)(298,26.4185)(299,26.0054)(300,25.5945)(301,25.186)(302,24.7796)(303,24.3756)(304,23.9737)(305,23.5739)(306,23.1763)(307,22.7808)(308,22.3874)(309,21.996)(310,21.6066)(311,21.2193)(312,20.8339)(313,20.4504)(314,20.0688)(315,19.6891)(316,19.3112)(317,18.9351)(318,18.5608)(319,18.1883)(320,17.8174)(321,17.4483)(322,17.0808)(323,16.7149)(324,16.3507)(325,15.9879)(326,15.6268)(327,15.2671)(328,14.9089)(329,14.5521)(330,14.1967)(331,13.8427)(332,13.49)(333,13.1387)(334,12.7886)(335,12.4398)(336,12.0921)(337,11.7457)(338,11.4004)(339,11.0563)(340,10.7132)(341,10.3712)(342,10.0301)(343,9.69013)(344,9.35107)(345,9.01294)(346,8.67569)(347,8.33931)(348,8.00375)(349,7.66899)(350,7.335)(351,7.00173)(352,6.66916)(353,6.33726)(354,6.00599)(355,5.67532)(356,5.34522)(357,5.01565)(358,4.68658)(359,4.35799)(360,4.02983)(361,3.70208)(362,3.3747)(363,3.04766)(364,2.72093)(365,2.39447)(366,2.06825)(367,1.74224)(368,1.41641)(369,1.09072)(370,0.76514) 
};

\end{axis}
\end{tikzpicture}

			\newframe
			\begin{tikzpicture}[scale=0.5]

\begin{axis}[%
scale only axis,
width=4.52083in,
height=3.56562in,
xmin=0, xmax=400,
ymin=0, ymax=400,
xlabel={Slab Length [cm]},
ylabel={Power [W]},
title={$\text{tstep }= 114$},
axis on top]
\addplot [
color=blue,
solid
]
coordinates{
 (1,13.9911)(2,20.3593)(3,26.7197)(4,33.0699)(5,39.4076)(6,45.7302)(7,52.0354)(8,58.3208)(9,64.584)(10,70.8228)(11,77.0348)(12,83.2176)(13,89.3691)(14,95.487)(15,101.569)(16,107.613)(17,113.617)(18,119.578)(19,125.496)(20,131.366)(21,137.189)(22,142.96)(23,148.68)(24,154.345)(25,159.955)(26,165.506)(27,170.998)(28,176.428)(29,181.795)(30,187.098)(31,192.335)(32,197.504)(33,202.604)(34,207.633)(35,212.591)(36,217.475)(37,222.285)(38,227.019)(39,231.677)(40,236.256)(41,240.757)(42,245.178)(43,249.519)(44,253.778)(45,257.954)(46,262.048)(47,266.057)(48,269.982)(49,273.822)(50,277.577)(51,281.245)(52,284.827)(53,288.323)(54,291.731)(55,295.052)(56,298.285)(57,301.43)(58,304.488)(59,307.457)(60,310.339)(61,313.133)(62,315.839)(63,318.457)(64,320.988)(65,323.431)(66,325.788)(67,328.057)(68,330.241)(69,332.338)(70,334.35)(71,336.276)(72,338.118)(73,339.876)(74,341.55)(75,343.141)(76,344.65)(77,346.078)(78,347.424)(79,348.69)(80,349.877)(81,350.985)(82,352.015)(83,352.967)(84,353.844)(85,354.645)(86,355.372)(87,356.025)(88,356.606)(89,357.114)(90,357.553)(91,357.921)(92,358.22)(93,358.452)(94,358.617)(95,358.716)(96,358.751)(97,358.722)(98,358.63)(99,358.477)(100,358.264)(101,357.991)(102,357.659)(103,357.271)(104,356.826)(105,356.327)(106,355.774)(107,355.167)(108,354.509)(109,353.801)(110,353.043)(111,352.236)(112,351.382)(113,350.482)(114,349.537)(115,348.547)(116,347.515)(117,346.441)(118,345.325)(119,344.17)(120,342.976)(121,341.744)(122,340.476)(123,339.172)(124,337.833)(125,336.461)(126,335.056)(127,333.619)(128,332.152)(129,330.654)(130,329.128)(131,327.575)(132,325.994)(133,324.387)(134,322.756)(135,321.1)(136,319.421)(137,317.719)(138,315.997)(139,314.253)(140,312.49)(141,310.707)(142,308.907)(143,307.089)(144,305.254)(145,303.404)(146,301.538)(147,299.659)(148,297.765)(149,295.859)(150,293.941)(151,292.012)(152,290.071)(153,288.121)(154,286.161)(155,284.192)(156,282.216)(157,280.232)(158,278.24)(159,276.243)(160,274.24)(161,272.232)(162,270.219)(163,268.202)(164,266.181)(165,264.158)(166,262.132)(167,260.104)(168,258.075)(169,256.044)(170,254.013)(171,251.982)(172,249.951)(173,247.92)(174,245.891)(175,243.864)(176,241.838)(177,239.815)(178,237.794)(179,235.776)(180,233.762)(181,231.752)(182,229.745)(183,227.743)(184,225.746)(185,223.753)(186,221.766)(187,219.785)(188,217.809)(189,215.839)(190,213.876)(191,211.919)(192,209.969)(193,208.027)(194,206.091)(195,204.163)(196,202.242)(197,200.33)(198,198.425)(199,196.529)(200,194.641)(201,192.762)(202,190.892)(203,189.03)(204,187.178)(205,185.334)(206,183.5)(207,181.676)(208,179.861)(209,178.055)(210,176.26)(211,174.474)(212,172.698)(213,170.933)(214,169.177)(215,167.432)(216,165.697)(217,163.972)(218,162.258)(219,160.554)(220,158.861)(221,157.178)(222,155.506)(223,153.845)(224,152.194)(225,150.554)(226,148.925)(227,147.307)(228,145.699)(229,144.102)(230,142.516)(231,140.941)(232,139.377)(233,137.823)(234,136.281)(235,134.749)(236,133.228)(237,131.718)(238,130.218)(239,128.73)(240,127.252)(241,125.785)(242,124.328)(243,122.883)(244,121.447)(245,120.023)(246,118.609)(247,117.206)(248,115.813)(249,114.43)(250,113.058)(251,111.697)(252,110.345)(253,109.004)(254,107.673)(255,106.352)(256,105.042)(257,103.741)(258,102.451)(259,101.17)(260,99.899)(261,98.6379)(262,97.3866)(263,96.1449)(264,94.9129)(265,93.6903)(266,92.4773)(267,91.2736)(268,90.0792)(269,88.8941)(270,87.7182)(271,86.5513)(272,85.3934)(273,84.2445)(274,83.1045)(275,81.9733)(276,80.8507)(277,79.7368)(278,78.6314)(279,77.5345)(280,76.4459)(281,75.3657)(282,74.2937)(283,73.2298)(284,72.1739)(285,71.126)(286,70.086)(287,69.0538)(288,68.0293)(289,67.0123)(290,66.003)(291,65.001)(292,64.0064)(293,63.0191)(294,62.039)(295,61.0659)(296,60.0999)(297,59.1407)(298,58.1884)(299,57.2428)(300,56.3038)(301,55.3714)(302,54.4454)(303,53.5259)(304,52.6125)(305,51.7054)(306,50.8044)(307,49.9094)(308,49.0203)(309,48.137)(310,47.2594)(311,46.3875)(312,45.5212)(313,44.6603)(314,43.8047)(315,42.9545)(316,42.1094)(317,41.2694)(318,40.4344)(319,39.6043)(320,38.779)(321,37.9585)(322,37.1426)(323,36.3312)(324,35.5243)(325,34.7217)(326,33.9234)(327,33.1293)(328,32.3393)(329,31.5532)(330,30.7711)(331,29.9928)(332,29.2182)(333,28.4472)(334,27.6798)(335,26.9158)(336,26.1551)(337,25.3978)(338,24.6436)(339,23.8925)(340,23.1443)(341,22.3991)(342,21.6567)(343,20.917)(344,20.18)(345,19.4455)(346,18.7134)(347,17.9837)(348,17.2563)(349,16.531)(350,15.8079)(351,15.0867)(352,14.3675)(353,13.65)(354,12.9343)(355,12.2203)(356,11.5078)(357,10.7967)(358,10.0871)(359,9.37866)(360,8.67146)(361,7.96537)(362,7.26029)(363,6.55613)(364,5.85281)(365,5.15023)(366,4.44831)(367,3.74696)(368,3.04609)(369,2.3456)(370,1.64541) 
};

\addplot [
color=red,
solid
]
coordinates{
 (1,4.26694)(2,6.32439)(3,8.55025)(4,10.7743)(5,12.9958)(6,15.2145)(7,17.4296)(8,19.6405)(9,21.8468)(10,24.0477)(11,26.2427)(12,28.4312)(13,30.6126)(14,32.7861)(15,34.9513)(16,37.1075)(17,39.254)(18,41.3902)(19,43.5154)(20,45.6291)(21,47.7306)(22,49.8192)(23,51.8943)(24,53.9552)(25,56.0013)(26,58.032)(27,60.0465)(28,62.0443)(29,64.0247)(30,65.987)(31,67.9306)(32,69.8549)(33,71.7592)(34,73.643)(35,75.5056)(36,77.3464)(37,79.165)(38,80.9607)(39,82.733)(40,84.4814)(41,86.2055)(42,87.9047)(43,89.5786)(44,91.2267)(45,92.8487)(46,94.4441)(47,96.0126)(48,97.5538)(49,99.0674)(50,100.553)(51,102.01)(52,103.439)(53,104.839)(54,106.21)(55,107.552)(56,108.865)(57,110.147)(58,111.4)(59,112.623)(60,113.816)(61,114.979)(62,116.111)(63,117.213)(64,118.285)(65,119.326)(66,120.337)(67,121.317)(68,122.267)(69,123.186)(70,124.075)(71,124.934)(72,125.762)(73,126.56)(74,127.328)(75,128.066)(76,128.775)(77,129.453)(78,130.102)(79,130.722)(80,131.313)(81,131.874)(82,132.407)(83,132.911)(84,133.387)(85,133.834)(86,134.254)(87,134.646)(88,135.011)(89,135.349)(90,135.66)(91,135.944)(92,136.203)(93,136.435)(94,136.642)(95,136.823)(96,136.979)(97,137.111)(98,137.219)(99,137.302)(100,137.362)(101,137.399)(102,137.412)(103,137.403)(104,137.372)(105,137.319)(106,137.245)(107,137.149)(108,137.033)(109,136.896)(110,136.739)(111,136.563)(112,136.367)(113,136.152)(114,135.919)(115,135.667)(116,135.398)(117,135.111)(118,134.808)(119,134.487)(120,134.151)(121,133.798)(122,133.43)(123,133.046)(124,132.648)(125,132.235)(126,131.809)(127,131.368)(128,130.914)(129,130.447)(130,129.967)(131,129.475)(132,128.971)(133,128.455)(134,127.928)(135,127.39)(136,126.841)(137,126.282)(138,125.713)(139,125.134)(140,124.546)(141,123.949)(142,123.343)(143,122.728)(144,122.106)(145,121.475)(146,120.837)(147,120.192)(148,119.54)(149,118.881)(150,118.216)(151,117.544)(152,116.867)(153,116.184)(154,115.496)(155,114.802)(156,114.104)(157,113.401)(158,112.694)(159,111.982)(160,111.267)(161,110.548)(162,109.826)(163,109.1)(164,108.372)(165,107.641)(166,106.907)(167,106.17)(168,105.432)(169,104.691)(170,103.949)(171,103.205)(172,102.46)(173,101.713)(174,100.966)(175,100.217)(176,99.4679)(177,98.718)(178,97.9677)(179,97.2172)(180,96.4665)(181,95.7159)(182,94.9654)(183,94.2153)(184,93.4656)(185,92.7165)(186,91.968)(187,91.2205)(188,90.4738)(189,89.7283)(190,88.9839)(191,88.2408)(192,87.4991)(193,86.7589)(194,86.0204)(195,85.2835)(196,84.5485)(197,83.8153)(198,83.0841)(199,82.355)(200,81.6281)(201,80.9033)(202,80.1809)(203,79.4609)(204,78.7433)(205,78.0283)(206,77.3158)(207,76.606)(208,75.8989)(209,75.1946)(210,74.4932)(211,73.7946)(212,73.099)(213,72.4063)(214,71.7167)(215,71.0302)(216,70.3468)(217,69.6666)(218,68.9896)(219,68.3158)(220,67.6453)(221,66.9781)(222,66.3143)(223,65.6538)(224,64.9967)(225,64.3431)(226,63.6929)(227,63.0462)(228,62.4029)(229,61.7632)(230,61.127)(231,60.4944)(232,59.8653)(233,59.2398)(234,58.6179)(235,57.9995)(236,57.3848)(237,56.7737)(238,56.1662)(239,55.5623)(240,54.962)(241,54.3654)(242,53.7724)(243,53.183)(244,52.5973)(245,52.0151)(246,51.4366)(247,50.8617)(248,50.2904)(249,49.7228)(250,49.1587)(251,48.5982)(252,48.0413)(253,47.4879)(254,46.9382)(255,46.3919)(256,45.8492)(257,45.31)(258,44.7744)(259,44.2422)(260,43.7135)(261,43.1882)(262,42.6664)(263,42.1481)(264,41.6331)(265,41.1215)(266,40.6133)(267,40.1085)(268,39.6069)(269,39.1087)(270,38.6138)(271,38.1221)(272,37.6337)(273,37.1485)(274,36.6665)(275,36.1877)(276,35.712)(277,35.2394)(278,34.7699)(279,34.3035)(280,33.8402)(281,33.3799)(282,32.9225)(283,32.4682)(284,32.0167)(285,31.5682)(286,31.1226)(287,30.6798)(288,30.2398)(289,29.8026)(290,29.3683)(291,28.9366)(292,28.5076)(293,28.0814)(294,27.6578)(295,27.2368)(296,26.8183)(297,26.4025)(298,25.9892)(299,25.5783)(300,25.17)(301,24.764)(302,24.3605)(303,23.9593)(304,23.5605)(305,23.164)(306,22.7698)(307,22.3778)(308,21.988)(309,21.6004)(310,21.2149)(311,20.8316)(312,20.4503)(313,20.0711)(314,19.6939)(315,19.3187)(316,18.9454)(317,18.574)(318,18.2045)(319,17.8369)(320,17.4711)(321,17.107)(322,16.7448)(323,16.3842)(324,16.0253)(325,15.6681)(326,15.3124)(327,14.9584)(328,14.6059)(329,14.2549)(330,13.9054)(331,13.5573)(332,13.2107)(333,12.8654)(334,12.5215)(335,12.1789)(336,11.8375)(337,11.4974)(338,11.1585)(339,10.8208)(340,10.4842)(341,10.1488)(342,9.81437)(343,9.48101)(344,9.14866)(345,8.81727)(346,8.48681)(347,8.15726)(348,7.82858)(349,7.50073)(350,7.17368)(351,6.84739)(352,6.52184)(353,6.19698)(354,5.87278)(355,5.54922)(356,5.22624)(357,4.90383)(358,4.58194)(359,4.26055)(360,3.93961)(361,3.61909)(362,3.29897)(363,2.9792)(364,2.65975)(365,2.34059)(366,2.02168)(367,1.70299)(368,1.38448)(369,1.06612)(370,0.747883) 
};

\end{axis}
\end{tikzpicture}

			\newframe
			\begin{tikzpicture}[scale=0.5]

\begin{axis}[%
scale only axis,
width=4.52083in,
height=3.56562in,
xmin=0, xmax=400,
ymin=0, ymax=400,
xlabel={Slab Length [cm]},
ylabel={Power [W]},
title={$\text{tstep }= 120$},
axis on top]
\addplot [
color=blue,
solid
]
coordinates{
 (1,13.9911)(2,20.3593)(3,26.7197)(4,33.0699)(5,39.4076)(6,45.7302)(7,52.0354)(8,58.3208)(9,64.584)(10,70.8228)(11,77.0348)(12,83.2176)(13,89.3691)(14,95.487)(15,101.569)(16,107.613)(17,113.617)(18,119.578)(19,125.496)(20,131.366)(21,137.189)(22,142.96)(23,148.68)(24,154.345)(25,159.955)(26,165.506)(27,170.998)(28,176.428)(29,181.795)(30,187.098)(31,192.335)(32,197.504)(33,202.604)(34,207.633)(35,212.591)(36,217.475)(37,222.285)(38,227.019)(39,231.677)(40,236.256)(41,240.757)(42,245.178)(43,249.519)(44,253.778)(45,257.954)(46,262.048)(47,266.057)(48,269.982)(49,273.822)(50,277.577)(51,281.245)(52,284.827)(53,288.323)(54,291.731)(55,295.052)(56,298.285)(57,301.43)(58,304.488)(59,307.457)(60,310.339)(61,313.133)(62,315.839)(63,318.457)(64,320.988)(65,323.431)(66,325.788)(67,328.057)(68,330.241)(69,332.338)(70,334.35)(71,336.276)(72,338.118)(73,339.876)(74,341.55)(75,343.141)(76,344.65)(77,346.078)(78,347.424)(79,348.69)(80,349.877)(81,350.985)(82,352.015)(83,352.967)(84,353.844)(85,354.645)(86,355.372)(87,356.025)(88,356.606)(89,357.114)(90,357.553)(91,357.921)(92,358.22)(93,358.452)(94,358.617)(95,358.716)(96,358.751)(97,358.722)(98,358.63)(99,358.477)(100,358.264)(101,357.991)(102,357.659)(103,357.271)(104,356.826)(105,356.327)(106,355.774)(107,355.167)(108,354.509)(109,353.801)(110,353.043)(111,352.236)(112,351.382)(113,350.482)(114,349.537)(115,348.547)(116,347.515)(117,346.441)(118,345.325)(119,344.17)(120,342.976)(121,341.744)(122,340.476)(123,339.172)(124,337.833)(125,336.461)(126,335.056)(127,333.619)(128,332.152)(129,330.654)(130,329.128)(131,327.575)(132,325.994)(133,324.387)(134,322.756)(135,321.1)(136,319.421)(137,317.719)(138,315.997)(139,314.253)(140,312.49)(141,310.707)(142,308.907)(143,307.089)(144,305.254)(145,303.404)(146,301.538)(147,299.659)(148,297.765)(149,295.859)(150,293.941)(151,292.012)(152,290.071)(153,288.121)(154,286.161)(155,284.192)(156,282.216)(157,280.232)(158,278.24)(159,276.243)(160,274.24)(161,272.232)(162,270.219)(163,268.202)(164,266.181)(165,264.158)(166,262.132)(167,260.104)(168,258.075)(169,256.044)(170,254.013)(171,251.982)(172,249.951)(173,247.92)(174,245.891)(175,243.864)(176,241.838)(177,239.815)(178,237.794)(179,235.776)(180,233.762)(181,231.752)(182,229.745)(183,227.743)(184,225.746)(185,223.753)(186,221.766)(187,219.785)(188,217.809)(189,215.839)(190,213.876)(191,211.919)(192,209.969)(193,208.027)(194,206.091)(195,204.163)(196,202.242)(197,200.33)(198,198.425)(199,196.529)(200,194.641)(201,192.762)(202,190.892)(203,189.03)(204,187.178)(205,185.334)(206,183.5)(207,181.676)(208,179.861)(209,178.055)(210,176.26)(211,174.474)(212,172.698)(213,170.933)(214,169.177)(215,167.432)(216,165.697)(217,163.972)(218,162.258)(219,160.554)(220,158.861)(221,157.178)(222,155.506)(223,153.845)(224,152.194)(225,150.554)(226,148.925)(227,147.307)(228,145.699)(229,144.102)(230,142.516)(231,140.941)(232,139.377)(233,137.823)(234,136.281)(235,134.749)(236,133.228)(237,131.718)(238,130.218)(239,128.73)(240,127.252)(241,125.785)(242,124.328)(243,122.883)(244,121.447)(245,120.023)(246,118.609)(247,117.206)(248,115.813)(249,114.43)(250,113.058)(251,111.697)(252,110.345)(253,109.004)(254,107.673)(255,106.352)(256,105.042)(257,103.741)(258,102.451)(259,101.17)(260,99.899)(261,98.6379)(262,97.3866)(263,96.1449)(264,94.9129)(265,93.6903)(266,92.4773)(267,91.2736)(268,90.0792)(269,88.8941)(270,87.7182)(271,86.5513)(272,85.3934)(273,84.2445)(274,83.1045)(275,81.9733)(276,80.8507)(277,79.7368)(278,78.6314)(279,77.5345)(280,76.4459)(281,75.3657)(282,74.2937)(283,73.2298)(284,72.1739)(285,71.126)(286,70.086)(287,69.0538)(288,68.0293)(289,67.0123)(290,66.003)(291,65.001)(292,64.0064)(293,63.0191)(294,62.039)(295,61.0659)(296,60.0999)(297,59.1407)(298,58.1884)(299,57.2428)(300,56.3038)(301,55.3714)(302,54.4454)(303,53.5259)(304,52.6125)(305,51.7054)(306,50.8044)(307,49.9094)(308,49.0203)(309,48.137)(310,47.2594)(311,46.3875)(312,45.5212)(313,44.6603)(314,43.8047)(315,42.9545)(316,42.1094)(317,41.2694)(318,40.4344)(319,39.6043)(320,38.779)(321,37.9585)(322,37.1426)(323,36.3312)(324,35.5243)(325,34.7217)(326,33.9234)(327,33.1293)(328,32.3393)(329,31.5532)(330,30.7711)(331,29.9928)(332,29.2182)(333,28.4472)(334,27.6798)(335,26.9158)(336,26.1551)(337,25.3978)(338,24.6436)(339,23.8925)(340,23.1443)(341,22.3991)(342,21.6567)(343,20.917)(344,20.18)(345,19.4455)(346,18.7134)(347,17.9837)(348,17.2563)(349,16.531)(350,15.8079)(351,15.0867)(352,14.3675)(353,13.65)(354,12.9343)(355,12.2203)(356,11.5078)(357,10.7967)(358,10.0871)(359,9.37866)(360,8.67146)(361,7.96537)(362,7.26029)(363,6.55613)(364,5.85281)(365,5.15023)(366,4.44831)(367,3.74696)(368,3.04609)(369,2.3456)(370,1.64541) 
};

\addplot [
color=red,
solid
]
coordinates{
 (1,5.1416)(2,7.48253)(3,9.82151)(4,12.1579)(5,14.491)(6,16.8201)(7,19.1446)(8,21.4637)(9,23.7767)(10,26.0828)(11,28.3814)(12,30.6718)(13,32.9532)(14,35.2249)(15,37.4862)(16,39.7364)(17,41.9747)(18,44.2005)(19,46.4131)(20,48.6117)(21,50.7956)(22,52.9642)(23,55.1168)(24,57.2526)(25,59.371)(26,61.4713)(27,63.5529)(28,65.6151)(29,67.6573)(30,69.6787)(31,71.6788)(32,73.657)(33,75.6126)(34,77.5451)(35,79.4539)(36,81.3383)(37,83.198)(38,85.0324)(39,86.8409)(40,88.6231)(41,90.3785)(42,92.1067)(43,93.8072)(44,95.4797)(45,97.1237)(46,98.739)(47,100.325)(48,101.882)(49,103.409)(50,104.905)(51,106.372)(52,107.808)(53,109.213)(54,110.587)(55,111.93)(56,113.242)(57,114.522)(58,115.771)(59,116.988)(60,118.173)(61,119.326)(62,120.447)(63,121.536)(64,122.594)(65,123.619)(66,124.612)(67,125.574)(68,126.503)(69,127.401)(70,128.267)(71,129.102)(72,129.904)(73,130.676)(74,131.416)(75,132.125)(76,132.803)(77,133.451)(78,134.068)(79,134.654)(80,135.211)(81,135.737)(82,136.234)(83,136.702)(84,137.14)(85,137.549)(86,137.93)(87,138.283)(88,138.607)(89,138.904)(90,139.173)(91,139.416)(92,139.631)(93,139.82)(94,139.984)(95,140.121)(96,140.233)(97,140.32)(98,140.382)(99,140.42)(100,140.435)(101,140.425)(102,140.393)(103,140.337)(104,140.26)(105,140.16)(106,140.039)(107,139.896)(108,139.733)(109,139.549)(110,139.345)(111,139.121)(112,138.878)(113,138.616)(114,138.336)(115,138.038)(116,137.722)(117,137.388)(118,137.038)(119,136.671)(120,136.288)(121,135.89)(122,135.475)(123,135.046)(124,134.603)(125,134.145)(126,133.673)(127,133.187)(128,132.689)(129,132.178)(130,131.654)(131,131.118)(132,130.571)(133,130.012)(134,129.443)(135,128.862)(136,128.272)(137,127.671)(138,127.061)(139,126.441)(140,125.813)(141,125.176)(142,124.53)(143,123.877)(144,123.215)(145,122.547)(146,121.871)(147,121.188)(148,120.499)(149,119.804)(150,119.103)(151,118.395)(152,117.683)(153,116.965)(154,116.243)(155,115.516)(156,114.784)(157,114.048)(158,113.309)(159,112.566)(160,111.819)(161,111.069)(162,110.316)(163,109.561)(164,108.802)(165,108.042)(166,107.279)(167,106.515)(168,105.749)(169,104.981)(170,104.212)(171,103.442)(172,102.671)(173,101.899)(174,101.126)(175,100.353)(176,99.5799)(177,98.8065)(178,98.0331)(179,97.2599)(180,96.4871)(181,95.7147)(182,94.9429)(183,94.1718)(184,93.4017)(185,92.6325)(186,91.8645)(187,91.0977)(188,90.3323)(189,89.5684)(190,88.8061)(191,88.0455)(192,87.2868)(193,86.5299)(194,85.775)(195,85.0222)(196,84.2717)(197,83.5234)(198,82.7775)(199,82.034)(200,81.293)(201,80.5547)(202,79.819)(203,79.0861)(204,78.356)(205,77.6288)(206,76.9046)(207,76.1833)(208,75.4651)(209,74.75)(210,74.0381)(211,73.3294)(212,72.624)(213,71.9219)(214,71.2231)(215,70.5278)(216,69.8358)(217,69.1474)(218,68.4625)(219,67.7811)(220,67.1032)(221,66.429)(222,65.7584)(223,65.0915)(224,64.4282)(225,63.7687)(226,63.1128)(227,62.4607)(228,61.8124)(229,61.1678)(230,60.527)(231,59.89)(232,59.2568)(233,58.6275)(234,58.0019)(235,57.3802)(236,56.7623)(237,56.1482)(238,55.538)(239,54.9316)(240,54.3291)(241,53.7304)(242,53.1355)(243,52.5445)(244,51.9573)(245,51.3739)(246,50.7943)(247,50.2186)(248,49.6466)(249,49.0784)(250,48.5141)(251,47.9534)(252,47.3966)(253,46.8434)(254,46.2941)(255,45.7484)(256,45.2064)(257,44.6681)(258,44.1334)(259,43.6025)(260,43.0751)(261,42.5513)(262,42.0312)(263,41.5146)(264,41.0015)(265,40.492)(266,39.986)(267,39.4835)(268,38.9844)(269,38.4888)(270,37.9965)(271,37.5077)(272,37.0222)(273,36.5401)(274,36.0613)(275,35.5857)(276,35.1134)(277,34.6444)(278,34.1785)(279,33.7158)(280,33.2563)(281,32.7999)(282,32.3466)(283,31.8963)(284,31.4491)(285,31.0048)(286,30.5636)(287,30.1252)(288,29.6898)(289,29.2573)(290,28.8276)(291,28.4008)(292,27.9767)(293,27.5554)(294,27.1368)(295,26.7209)(296,26.3076)(297,25.897)(298,25.489)(299,25.0835)(300,24.6806)(301,24.2802)(302,23.8822)(303,23.4867)(304,23.0935)(305,22.7027)(306,22.3143)(307,21.9281)(308,21.5442)(309,21.1626)(310,20.7831)(311,20.4058)(312,20.0307)(313,19.6576)(314,19.2866)(315,18.9176)(316,18.5506)(317,18.1855)(318,17.8224)(319,17.4612)(320,17.1018)(321,16.7442)(322,16.3884)(323,16.0344)(324,15.6821)(325,15.3315)(326,14.9825)(327,14.6351)(328,14.2894)(329,13.9451)(330,13.6024)(331,13.2611)(332,12.9213)(333,12.5829)(334,12.2458)(335,11.9101)(336,11.5757)(337,11.2425)(338,10.9106)(339,10.5799)(340,10.2503)(341,9.92188)(342,9.59453)(343,9.26824)(344,8.94297)(345,8.61869)(346,8.29536)(347,7.97294)(348,7.65141)(349,7.33073)(350,7.01086)(351,6.69177)(352,6.37342)(353,6.05578)(354,5.73881)(355,5.42249)(356,5.10677)(357,4.79161)(358,4.47699)(359,4.16288)(360,3.84922)(361,3.536)(362,3.22318)(363,2.91071)(364,2.59857)(365,2.28673)(366,1.97513)(367,1.66377)(368,1.35259)(369,1.04156)(370,0.730647) 
};

\end{axis}
\end{tikzpicture}

		\end{animateinline}	

	\end{center}

\end{frame}
%\end{comment}
%-----------------------------------------------------------------------------------------------------------
\begin{frame}
\begin{center}
	\begin{tikzpicture}

	% Draw fuel pin
	\tikzstyle{fuelcirc}    = [draw = black, shape = circle, fill = red,   inner sep = 3*0.4096cm]
	\tikzstyle{gascirc}     = [draw = black, shape = circle, fill = green, inner sep = 3*0.4178cm]
	\tikzstyle{cladcirc}    = [draw = black, shape = circle, fill = gray,  inner sep = 3*0.4750cm]
	\tikzstyle{coolantsqu}  = [draw = black, very thick, shape = rectangle, fill = blue, minimum height=6*1.26cm, minimum width=6*1.26cm]
	\begin{scope}
		\node[coolantsqu] (cool) at (0,0) {};
		\node[cladcirc]   (clad) at (0,0) {};
		\node[gascirc]    (gas)  at (0,0) {};
		\node[fuelcirc]   (fuel) at (0,0) {};
	\end{scope}
	
	% Draw labels
	\tikzstyle{pcolor} = [shape = rectangle, minimum height=0.1cm, minimum width=0.1cm];
	\tikzstyle{plabel} = [shape = rectangle, minimum height=0.5cm, minimum width=1cm];
	\node[pcolor,fill=blue]  (cool_color) at (-3*1.26cm+0.5cm,-3*1.26cm - 0.5cm) {};
	\node[plabel,right = 0.1cm of cool_color.east]  (cool_label) {Coolant};
	\node[pcolor,fill=gray,right = 0.15cm of cool_label.east]  (clad_color) {};
	\node[plabel,right = 0.1cm of clad_color.east]  (clad_label) {Clad};
	\node[pcolor,fill=green,right = 0.15cm of clad_label.east]  (gas_color) {};
	\node[plabel,right = 0.1cm of gas_color.east]  (gas_label) {Gas Gap};
	\node[pcolor,fill=red,right = 0.15cm of gas_label.east]  (fuel_color) {};
	\node[plabel,right = 0.1cm of fuel_color.east]  (fuel_label) {Fuel};

\end{tikzpicture}
\end{center}
\end{frame}
%-----------------------------------------------------------------------------------------------------------
\end{section}
%======================================================
\begin{section}{Conclusions}
\end{section}
%======================================================
%======================================================
\end{document}


