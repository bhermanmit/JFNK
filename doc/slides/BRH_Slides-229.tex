%%%%%%%%%%%%   LaTeX Preamble %%%%%%%%%%%%%%

\documentclass{beamer}

% list all packages
\usefonttheme[onlymath]{serif}
\usepackage{amsmath}
\usepackage{comment}
\usepackage{hyperref}
\usepackage{pgfpages}
\usepackage{amsmath}
\usepackage{latexsym}
\usepackage{color}
\usepackage{ifthen}
\usepackage{animate}
\usepackage{tikz,pgfplots}
\usepackage{mycommands}
\pgfplotsset{compat=1.3}
\usetikzlibrary{plotmarks,shapes,arrows,positioning,snakes}
\usepackage[latin1]{inputenc}
\usepackage{xcolor}
\usepackage{tikz}
\usetikzlibrary{decorations.pathmorphing,shapes.multipart}
\usepackage{graphicx}
\usepackage{scalefnt}
\usepackage{relsize}
\usepackage{algorithm}
\usepackage{algorithmic}
\usepackage{animate}
\usepackage[Gray]{SIunits}

% slide theme
\usetheme{Berlin}
\usecolortheme{mit}

% Set Logo
\pgfdeclareimage[height=0.5cm]{mit-logo}{mit-logo.pdf}
\logo{\vspace{-0.25cm}\pgfuseimage{mit-logo}\hspace*{0.025cm}}

% Show outline at beginning of each section
%\AtBeginSection[]
%{
%  \begin{frame}<beamer>
%    \frametitle{Outline}
%    \tableofcontents[currentsection]
%  \end{frame}
%}

% Include Custom environments
% \setbeamertemplate{blocks}[rounded]
\setbeamertemplate{blocks}[rounded][shadow=true]

% \setbeamertemplate{headline}[default]
\setbeamertemplate{navigation symbols}{}

% \beamerdefaultoverlayspecification{<+->}

% Set color for 'alert' text
\setbeamercolor{alerted text}{fg=blue}

%\setbeamercolor{section in toc}

% Modify some default font sizes
\setbeamerfont{itemize/enumerate body}{size=\normalfont}
\setbeamerfont{itemize/enumerate subbody}{size=\smaller, shape=\upshape}
\setbeamerfont{frametitle}{size=\large, series=\bfseries}


% \setbeamertemplate{bibliography entry title}{}
% \setbeamertemplate{bibliography entry location}{}
% \setbeamertemplate{bibliography entry note}{}
\setbeamertemplate{bibliography item}[text] 

% \setbeamertemplate{items}[ball]
% \setbeamertemplate{itemize subitem}[circle-symbol]
% \setbeamertemplate{background canvas}[vertical shading][bottom=mitgray!25,top=white]

% Use the shrink option to squeeze lots of text on a slide

% \frame[shrink]{

% …

% }

\colorlet{dark green}{green!50!black}

\newcommand{\packin}{\setlength\abovedisplayskip{2pt}\setlength\belowdisplayskip{2pt}}

\tikzstyle{refbox} = [shape = rectangle, fill = mitred, inner sep = 2pt, text=white, font=\footnotesize]

%\newcommand{\numberInBox}[2][0.9]%
%	{\scalebox{#1}{{\tikz \draw (0,0) node[refbox] {\makebox[\totalheight]{#2}};}}}

%\newcommand{\enumref}[2][0.9] {\numberInBox[#1]{\ref{#2}}}


% Small arrow pointing down and hooking right
\newcommand{\drarrow}{\scalebox{1.5}{\reflectbox{\rotatebox[c]{180}{$\boldsymbol{\smash[b]{\Rsh}}$}}}}

\newenvironment{prettydescript}[1]
	{\begin{list}{}%
		{\renewcommand\makelabel[1]{\itshape\bfseries\color{mitred} ##1:\hfill}%
		\settowidth\labelwidth{\makelabel{#1}}%
		\setlength\leftmargin{\labelwidth}%
		\addtolength\leftmargin{\labelsep}}}%
	{\end{list}}

\newenvironment{customdescript}[1]
	{\begin{list}{}%
		{\renewcommand\makelabel[1]{\bfseries\color{mitred} ##1\hfill}%
		\settowidth\labelwidth{\makelabel{#1}}%
		\setlength\leftmargin{\labelwidth}%
		\addtolength\leftmargin{\labelsep}}}%
	{\end{list}}

\makeatletter

\newenvironment{customlist}[2]{
  \ifnum\@itemdepth >2\relax\@toodeep\else
      \advance\@itemdepth\@ne%
      \beamer@computepref\@itemdepth%
      \usebeamerfont{itemize/enumerate \beameritemnestingprefix body}%
      \usebeamercolor[fg]{itemize/enumerate \beameritemnestingprefix body}%
      \usebeamertemplate{itemize/enumerate \beameritemnestingprefix body begin}%
      \begin{list}
        {
            \usebeamertemplate{itemize \beameritemnestingprefix item}
        }
        { \leftmargin=#1 \itemindent=#2
            \def\makelabel##1{%
              {%  
                  \hss\llap{{%
                    \usebeamerfont*{itemize \beameritemnestingprefix item}%
                        \usebeamercolor[fg]{itemize \beameritemnestingprefix item}##1}}%
              }%  
            }%  
        }
  \fi
}
{
  \end{list}
  \usebeamertemplate{itemize/enumerate \beameritemnestingprefix body end}%
}
\makeatother

\newenvironment<>{varblock}[2][\textwidth]{%
  \setlength{\textwidth}{#1}
  \begin{actionenv}#3%
    \def\insertblocktitle{#2}%
    \par%
    \usebeamertemplate{block begin}}
  {\par%
    \usebeamertemplate{block end}%
  \end{actionenv}}

%% Notational commands:
\newcommand{\params}{\ensuremath{\xi}}
\newcommand{\vparms}{\ensuremath{\gvect{\params}}}
\renewcommand{\thefootnote}{\ensuremath{\fnsymbol{footnote}}}
\setcounter{footnote}{2}
\renewcommand{\thempfootnote}{\ensuremath{\fnsymbol{mpfootnote}}}
\newcommand{\newsubsection}[1]{\subsection{#1}\setcounter{subsection}{0}}

% For Loop 
\newcommand{\forloop}[5][1]% 
{% 
\setcounter{#2}{#3}% 
\ifthenelse{#4}% 
  {% 
  #5% 
  \addtocounter{#2}{#1}% 
  \forloop[#1]{#2}{\value{#2}}{#4}{#5}% 
  }% 
% Else 
  {% 
  }% 
}%

% Title Page
\title[JFNK Methods for Coupled Nonlinear Systems]{Jacobian-Free Newton-Krylov (JFNK) Methods for Nonlinear Neutronics/Thermal-Hydraulic Equations}
\author[]{Bryan Herman}
\institute[\insertpagenumber]{}
\date{December 14, 2011} 

% -----------------------------------------------------------------------------
\begin{document}
% -----------------------------------------------------------------------------

% Inset title page
\frame{\titlepage}

% Outline slide
\begin{frame}{Outline}
  \tableofcontents{}
\end{frame}

%==============================================================================
\begin{section}{Introduction}

%------------------------------------------------------------------------------
\begin{comment}
\begin{frame}{Motivation and Objective}

  \begin{customlist}{2ex}{0pt}

    \item Eventually will be part of thesis work
    \vfill\item JFNK method not currently used in nuclear reactor 
		analysis
    \vfill\item Incorporates a lot of ideas from 2.29 class
    \vfill\item Implement JFNK framework for coupled physics
    \vfill\item Test implementation on a simple model 

  \end{customlist}

\end{frame}
\end{comment}
%------------------------------------------------------------------------------
\begin{frame}{Nuclear Reactor Plant}
  \scalebox{0.45}{% Pressurized Water Reactor
% Author: Gloria Faccanoni <http://www.science.unitn.it/~gloria/home.htm>
%
\begin{tikzpicture}[
        scale=0.7,
        annotline/.style = {stealth-},
        arrows1loop/.style={->,red},
        arrows2loop/.style={->,white},
        arrows3loop/.style={->,draw=gray},
    ]
\draw[draw=gray,double=gray!10,double distance=4pt]
    (12,12) to[out=135,in=45](0,12)--(0,0)--(22,0)--(22,12)--(12,12)--(12,0);
\node[text width=4cm, text centered,font=\small] at (6,13)
    {Containment\\structure};
% legend
\begin{scope}[yshift=-2cm]
    \filldraw[draw=red,fill=red!10] (1,0) rectangle ++(2,1);
    \node[text width=4cm, font=\small,right] at (3,0.5)
        {Pressurized water\\(primary loop)};
    \filldraw[draw=blue,bottom color=blue!40,top color=gray!30]
        (11,0) rectangle ++(2,1);
    \node[text width=4cm, font=\small,right] at (13,0.5)
        {Water and steam\\(secondary loop)};
    \filldraw[draw=blue,fill=blue!10] (21,0) rectangle ++(2,1);
    \node[text width=4cm, font=\small,right] at (23,0.5)
        {Water\\(cooling loop)};
\end{scope}
% 2nd loop --------------------------------------------------------------------
\begin{scope}[xshift=7.25cm,yshift=3cm]
    % vessel left
    \filldraw[draw=blue,bottom color=blue!40,top color=gray!30]
        (0,0) to[out=-20,in=200] (3.5,0) --
        (3.5,4.5) to[out=120,in=60] (0,4.5) -- (0,0);
    % vessel right
    \filldraw[draw=blue,bottom color=blue!40,top color=gray!30,xshift=7cm]
        (0,0) to[out=-20,in=200] (3.5,0) --
        (3.5,5) to[out=120,in=60] (0,5) -- (0,0);
    % circuits
    \draw[draw=blue,double=blue!40,double distance=4pt]
      (1.75,-0.3) -- ++(0,-1) -- ++(7,0) -- ++(0,1);
    \draw[draw=blue,double=gray!30,double distance=4pt]
        (1.75,5.38) -- ++(0,1) -- ++(4,0) -- ++(0,1) -- ++(3,0) -- ++(0,-1.5);
    % arrows
    \draw[arrows2loop] (3.5,-1.3) -- (3,-1.3);
    \draw[arrows2loop] (1.75,-0.9) -- (1.75,-0.4);
    \draw[arrows2loop] (4.5,6.38) -- (5,6.38);
    \draw[arrows2loop] (7,7.38) -- (7.5,7.38);
    \draw[arrows2loop] (8.75,6.4) -- (8.75,5.9);
    \draw[arrows2loop] (8.75,-0.4) -- (8.75,-0.9);
    %
    \foreach \x in {0.5,1,...,3}
        \draw[arrows2loop,xshift=7cm] (\x,3) -- (\x,2.5);
    % labels
    \draw[annotline] (2.5,-1.3) -- ++(3.5,1.3)
        node[text width=1cm,font=\small,above] {Liquid};
    \draw[annotline] (2.5,6.38) -- ++(3.5,-1.3)
        node[text width=1cm,font=\small,below] {Vapor};
    % pump
    \begin{scope}[xshift=160,yshift=-40]
        \filldraw[fill=blue!20,draw=blue] (0,0) circle (0.5cm);
        \node[below,font=\small] at (0,-0.5) {Pump};
        \filldraw[fill=blue!40,draw=blue,yshift=-0.5cm]
            (0,0) arc (240:180:0.4cm)  arc (200:280:0.4cm) ;
        \filldraw[fill=blue!40,draw=blue,yshift=+0.5cm,rotate=180]
            (0,0) arc (240:180:0.4cm)  arc (200:280:0.4cm) ;
        \filldraw[fill=blue!40,draw=blue,xshift=+0.5cm,rotate=90]
            (0,0) arc (240:180:0.4cm)  arc (200:280:0.4cm) ;
        \filldraw[fill=blue!40,draw=blue,xshift=-0.5cm,rotate=-90]
            (0,0) arc (240:180:0.4cm)  arc (200:280:0.4cm) ;
    \end{scope}
    % generator ...
    \draw[xshift=6.5cm,draw=gray,double=gray!10,double distance=4pt] 
        (3,4) -- ++(2,0);
    \filldraw[xshift=6.5cm,fill=orange!10,draw=orange] 
        (1.8,4) -- (3.0,3.3) -- (3.0,4.7) -- cycle;
    \filldraw[xshift=6.5cm,fill=orange!10,draw=orange] 
        (1.5,4) -- (2.5,3.4) -- (2.5,4.6) -- cycle;
    \filldraw[xshift=6.5cm,fill=orange!10,draw=orange] 
        (1.2,4) -- (2  ,3.5) -- (2  ,4.5) -- cycle;
    \filldraw[xshift=6.5cm,fill=orange!10,draw=orange] 
        (4.5,3.3) rectangle (7.3,4.7);
    %labels
    \node[text width=3cm, text centered,font=\small] at (1.75,4) 
        {Steam generator\\ (heat change)};
    \node[text width=2cm, text centered,font=\small] at (8.8,5) {Turbine};
    \node[text width=2cm, text centered,font=\small] at (12.4,4) {Generator};
    % transmission lines
    \node (aa) at (11.1,4.6) {};
    \node (bb) at (11.6,4.6) {};
    \node (cc) at (12.1,4.6) {};
    \node (dd) at (12.6,4.6) {};
    \node (ee) at (13.1,4.6) {};
    \node (ff) at (13.6,4.6) {};

\end{scope}
% 3 loop --------------------------------------------------------------------
\begin{scope}[xshift=23cm,yshift=1cm]
    % circuit
    \draw[draw=blue,double=blue!10,double distance=4pt]
      (1,2.5) -- ++(-8.5,0) -- ++(0,+1.5) -- ++(8.5,0);
    % arrows
    \draw[arrows3loop] (-5.5,2.5) -- (-6,2.5);
    \draw[arrows3loop] (-1.5,2.5) -- (-2,2.5);
    \draw[arrows3loop] (-6,4) -- (-5.5,4);
    \draw[arrows3loop] (-2,4) -- (-1.5,4);
    % tower
    \filldraw[draw=gray,fill=gray!20] (1,7) to[out=270,in=80]
                  (0,0) to[out=-20,in=200]
                  (6,0) to[out=100,in=270]
                  (5,7);
    \filldraw[draw=gray,fill=gray!40] (1,7) to[out=30,in=150]
                  (5,7) to[out=200,in=-20]
                  (1,7);
    % labels
    \node[text width=3cm, text centered,font=\small] at (3,3.5)
        {Cooling\\tower};
    \node[text width=2cm, text centered,font=\small] at (-3.5,1.5)
        {Cooling\\water};
    \node[text width=2cm, text centered,font=\small] at (-6,3.25)
        {Condenser};
    % pump
    \begin{scope}[xshift=-10,yshift=115]
        \filldraw[fill=purple!20,draw=purple] (0,0) circle (0.5cm);
        \node[below,font=\small] at (0,-0.5) {Pump};
        \filldraw[fill=purple!40,draw=purple,yshift=-0.5cm]
            (0,0) arc (240:180:0.4cm)  arc (200:280:0.4cm) ;
        \filldraw[fill=purple!40,draw=purple,yshift=+0.5cm,rotate=180]
            (0,0) arc (240:180:0.4cm)  arc (200:280:0.4cm) ;
        \filldraw[fill=purple!40,draw=purple,xshift=+0.5cm,rotate=90]
            (0,0) arc (240:180:0.4cm)  arc (200:280:0.4cm) ;
        \filldraw[fill=purple!40,draw=purple,xshift=-0.5cm,rotate=-90]
            (0,0) arc (240:180:0.4cm)  arc (200:280:0.4cm) ;
    \end{scope}
\end{scope}
%1 loop --------------------------------------------------------------------
\begin{scope}[xshift=2cm,yshift=4cm]
% Reactor vessel
\filldraw[draw=red,fill=red!10] (0,-0.5) to[out=-20,in=200]
              (3.5,-0.5) --
              (3.5,4.5) to[out=160,in=20]
              (0,4.5) --
              (0,-0.5);
% circuit
\draw[draw=red,double=red!10,double distance=4pt]
  (0.1,1) --  ++(-1,0) -- ++(0,-3) -- ++(5,0) -- ++(0,1.5) --
  ++(3,0) -- ++(0,2) -- ++(-3.7,0);
% Pressurizer
\draw[draw=red,double=red!10,double distance=4pt] (4.2,1.6) -- ++(0,0.8);
\filldraw[draw=green,bottom color=red!40,top color=green!20]
              (4,2.4) to[out=-20,in=200]
              (4.5,2.4) --
              (4.5,3.6) to[out=160,in=20]
              (3.9,3.6) --
              (3.9,2.4);
% arrows
\draw[arrows1loop] (-0.7,1) -- (-0.2,1);
\draw[arrows1loop] (-0.9,-0.5) -- (-0.9,0);
\draw[arrows1loop] (0.7,-2) -- (0.2,-2);
\draw[arrows1loop] (4.5,1.5) -- (5,1.5);
\draw[arrows1loop] (7.1,0.5) -- (7.1,0);
\draw[arrows1loop] (5.5,-0.5) -- (5,-0.5);

% pump
\begin{scope}[xshift=75,yshift=-55,fill=red!20,draw=red]
    \filldraw (0,0) circle (0.5cm);
    \node[below,font=\small] at (0,-0.5) {Pump};
    \filldraw[yshift=-0.5cm] (0,0) arc (240:180:0.4cm)  arc (200:280:0.4cm) ;
    \filldraw[yshift=+0.5cm,rotate=180]
        (0,0) arc (240:180:0.4cm)  arc (200:280:0.4cm) ;
    \filldraw[xshift=+0.5cm,rotate=90]
        (0,0) arc (240:180:0.4cm)  arc (200:280:0.4cm) ;
    \filldraw[xshift=-0.5cm,rotate=-90]
        (0,0) arc (240:180:0.4cm)  arc (200:280:0.4cm) ;
\end{scope}
% reactor core
\filldraw[fill=red!30,draw=red] (0.7,0) rectangle (2.8,2);

% control rods
\foreach \x in {1.0,1.5,2.0,2.5}
  \draw[draw=gray,double=gray!50,double distance=0.5pt] (\x,0.3) -- (\x,3.7);

%labels
\draw[annotline] (0.6,0.5) -- ++(-3.3,-1.5)
    node[text width=1cm,font=\small,left] {Reactor core};
\node[text width=2cm, text centered,font=\small] at (1.75,5.4) {Reactor vessel};
\draw[annotline] (0.9,2.8) -- ++(-3.3,1.5)
    node[text width=2cm, text centered,font=\small,left=-8pt] {Control\\rods};
\draw[annotline] (4.2,3.7) -- ++(0.5,1.5)
    node[text width=2cm, text centered,font=\small,above] {Pressurizer};
\draw[annotline] (3.9,1.5) -- ++(1.3,-0.6)
    node[text width=2.4cm, text centered,below=-2pt,font=\small]
        {Water coolant (\unit{330}{\degreecelsius})};
\draw[annotline] (-0.1,-2) -- ++(-0.3,-0.6)
    node[text width=2.4cm, text centered,below=-2pt,font=\small]
        {Water coolant (\unit{280}{\degreecelsius})};
\end{scope}
% clouds ----------------------------------
\begin{scope}[xshift=26cm,yshift=10cm, fill=blue!10, draw=blue,
    decoration={bumps,segment length=0.5cm}]
    \filldraw[yshift=-1.5cm,rotate=-25,decorate]
        (0,0) -- ++(-0.4,1.25)-- ++(-0.1,0.75)-- ++(0.2,0.5)-- ++(0.3,0.5)--
        ++(0.3,-0.5)-- ++(0.2,-0.5)-- ++(-0.1,-0.75)-- ++(-0.4,-1.25);
    \filldraw[xshift=0.5cm,yshift=-2cm,rotate=-30,decorate]
        (0,0) -- ++(-0.4,1.25)-- ++(-0.1,0.75)-- ++(0.2,0.5)-- ++(0.3,0.5)--
        ++(0.3,-0.5)-- ++(0.2,-0.5)-- ++(-0.1,-0.75)-- ++(-0.4,-1.25);
    \filldraw[xshift=-1.05cm,yshift=-2.15cm,rotate=-20,decorate]
        (0,0) -- ++(-0.4,1.25)-- ++(-0.1,0.75)-- ++(0.2,0.5)-- ++(0.3,0.5)--
        ++(0.3,-0.5)-- ++(0.2,-0.5)-- ++(-0.1,-0.75)-- ++(-0.4,-1.25);
    %labels
    \node[text width=1cm, text centered,font=\small] at (0.2,1.5) {Water vapor};
\end{scope}

% palo della luce
\begin{scope}[xscale=0.2,xshift=113cm,yshift=19cm,line width=1pt,brown]
    \draw (0,0) -- (-6,-6)
          (0,0) -- ( 6,-6)
          (-1,-1) -- ( 1,-1)
          (-1,-1) -- ( 2,-2)
          ( 1,-1) -- (-2,-2)
          (-2,-2) -- ( 2,-2)
          (-2,-2) -- ( 3,-3)
          ( 2,-2) -- (-3,-3)
          ( 3,-3) -- (-3,-3)
          (-3,-3) -- ( 4,-4)
          ( 3,-3) -- (-4,-4)
          ( 4,-4) -- (-4,-4)
          (-4,-4) -- ( 5,-5)
          ( 4,-4) -- (-5,-5)
          ( 5,-5) -- (-5,-5)
          (-6,-6) -- ( 0,-5.2)
          ( 6,-6) -- ( 0,-5.2);
    \draw (-1.5,-1.5) -- (-4,-1.5) -- (-1,-1)
          ( 1.5,-1.5) -- ( 4,-1.5) -- ( 1,-1);
    \path (-4,-1.4) node (a) {}
          ( 4,-1.4) node (b) {};
    \draw[line width=1pt,brown] (-3.5,-3.5) -- (-7.5,-3.5) -- (-3,-3)
                                ( 3.5,-3.5) -- ( 7.5,-3.5) -- ( 3,-3);
    \path (-7.5,-3.4) node (c) {}
          ( 7.5,-3.4) node (d) {}
          (-5.5,-3.4) node (e) {}
          ( 5.5,-3.4) node (f) {};
\end{scope}
% transmission lines
\draw[dashed,gray] (c) -- (aa)
                   (a) -- (bb)
                   (e) -- (cc)
                   (b) -- (dd)
                   (f) -- (ee)
                   (d) -- (ff);
\end{tikzpicture}}
  \vfill
\end{frame}
%------------------------------------------------------------------------------
\begin{frame}{Nuclear Feedback}
  \begin{center}
    \scalebox{0.4}{\begin{tikzpicture}[scale=0.65]
	
	% Draw nucleus
	\tikzstyle{nucleon}=[shape = circle, shading=ball, minimum size=0.25cm];
	\node[nucleon, ball color = green] (neut) at (0.1,0.3) {};
	\node[nucleon, ball color = red] (prot) at (0.0,0.0) {};
	\node[nucleon, ball color = green] (neut2) at (0.3,0.2) {};
        \node[nucleon, ball color = red] (prot2) at (-0.2,0.1) {};
	\node[nucleon, ball color = green] (neut3) at (-0.1,0.3) {};
        \node[nucleon, ball color = red] (prot3) at (0.2,-0.15) {};
	\node[nucleon, ball color = green] (neut4) at (-0.05,-0.12) {};
        \node[nucleon, ball color = red] (prot4) at (0.17,0.21) {};
	\draw (0.17,0.21) node{$+$};

	% Draw electrons
	\draw[rotate = 80] (0,0) ellipse (1.5 and 0.75)[color=blue];
	\shade[ball color=yellow] (0,0.75)[rotate=80] circle (.1);
        \draw[rotate = 260] (0,0) ellipse (1.2 and 1.4)[color=blue];
        \shade[ball color=yellow] (0,1.4)[rotate=260] circle (.1);
        \draw[rotate = 30] (0,0) ellipse (4 and 2)[color=blue];
        \shade[ball color=yellow] (0,2)[rotate=30] circle (.1);
        \draw[rotate = 60] (0,0) ellipse (5 and 1)[color=blue];
        \shade[ball color=yellow] (0,1)[rotate=60] circle (.1);
        \draw[rotate = 150] (0,0) ellipse (5.5 and 1.5)[color=blue];
        \shade[ball color=yellow] (0,1.5)[rotate=150] circle (.1);
        \draw[rotate = 80] (0,0) ellipse (2.8 and 2.25)[color=blue];
        \shade[ball color=yellow] (0,2.25)[rotate=80] circle (.1);

	% Draw incident neutron
	\node[nucleon, ball color = green] (neut) at (-5.0,-5.0) {};
	\draw[snake=coil,segment aspect=0,very thick,segment length=20pt,line after snake=1mm,->,draw=orange] (-4.5cm,-4.5cm) -- (-0.4,-0.4);

        % Draw text
	\node[shape = rectangle,color=black,minimum width=8,minimum height=5,draw=black,above=5cm of prot.north] (N)   {Neutronics};
	\node[shape = rectangle,color=black,minimum width=8,minimum height=5,draw=black,below=5cm of prot.south] (T)   {Thermal Hydraulics};
	\node[shape = rectangle,color=black,minimum width=8,minimum height=5,draw=black,right=5cm of prot.east]  (NT)  {Neut/TH};
	\node[shape = rectangle,color=black,minimum width=8,minimum height=5,draw=black,left=5cm of prot.west]   (TN)  {TH/Neut};

	% Draw arrows
	\tikzstyle{connector} = [->,>=stealth, thick];
	\draw[connector] (N)  to [out=0,in=90]    (NT);
	\draw[connector] (NT) to [out=270,in=0]   (T);
	\draw[connector] (T)  to [out=180,in=270] (TN);
	\draw[connector] (TN) to [out=90,in=180]  (N);

\end{tikzpicture}
}
    \begin{itemize}
      \begin{scriptsize}
	\item Fuel Temperature Feedback -  $T_{f} \uparrow$, U-238 
	      Capture $\uparrow$, Fission Rate $\downarrow$, 
	      Power $\downarrow$
	\item \alert{Coolant Density Feedback - $\rho \downarrow$,
	      $E_{n} \uparrow$, Fission Rate $\downarrow$, 
	      Power $\downarrow$}
      \end{scriptsize}
    \end{itemize}
  \end{center}
\end{frame}
%------------------------------------------------------------------------------
\begin{frame}{Common Approach to Coupling: Operator Splitting}
\begin{customlist}{2ex}{0pt}
  \item Solve physics independently and iterate between them
  \begin{center}
    \scalebox{0.6}{\begin{tikzpicture}

	% Define background color
	\colorlet{bgcolor}{mitgray!20};

	% Define Dimensions
	\def\blockwidth{2.75cm};
	\def\blockheight{1.5cm};
	\def\elemheight{1cm};
	\def\elemwidth{2cm};
	\def\elemvsep{0.5cm};
	\def\elemhsep{0.75cm};
	
	% Define Outer Frame
	\tikzstyle{frame} = [rounded corners, fill=bgcolor, draw=black, double, very thick];
	\draw[frame] (0.0,0.0) rectangle (4*\blockwidth+0.1cm,2*\blockheight+0.5cm);

	% Define Elements in Frame
	\begin{scope}
		\tikzstyle{element} = [rectangle, rounded corners, draw = black, very thick, minimum height=\elemheight, minimum width=\elemwidth];	
		\node[element] (time 1t) at (0.5*\elemwidth+0.5*\elemhsep,0.5*\elemheight+0.5*\elemvsep) {T/H};
		\node[element,above = \elemvsep of time 1t.north] (time 1n) {Neutronics};
		\node[element,right = \elemhsep of time 1t.east] (time 2t) {T/H};
		\node[element,above = \elemvsep of time 2t.north] (time 2n) {Neutronics};
		\node[element,right = \elemhsep of time 2t.east] (time 3t) {T/H};
		\node[element,above = \elemvsep of time 3t.north] (time 3n) {Neutronics};
		\node[element,right = \elemhsep of time 3t.east] (time 4t) {T/H};
		\node[element,above = \elemvsep of time 4t.north] (time 4n) {Neutronics};
	\end{scope}
		
	% Draw connectors between elements
	\tikzstyle{connector} = [->, >=stealth, thick, shorten >=2pt];
	\draw[connector] (time 1n) to [out=270,in=90] (time 1t);
	\draw[connector] (time 1t) to [out=0,in=180]  (time 2n);
	\draw[connector] (time 2n) to [out=270,in=90] (time 2t);
	\draw[connector] (time 2t) to [out=0,in=180]  (time 3n);
	\draw[connector] (time 3n) to [out=270,in=90] (time 3t);
	\draw[connector,dashed] (time 3t) to [out=0,in=180]  (time 4n);
	\draw[connector] (time 4n) to [out=270,in=90] (time 4t);
	
	% Put labels in
	\tikzstyle{labels}    = [minimum height = 1.0cm, text centered, anchor = south];
	\node[labels,above = 0.4cm of time 1n.center] {$t=0$};
	\node[labels,above = 0.4cm of time 2n.center] {$t=\Delta t$};
	\node[labels,above = 0.4cm of time 3n.center] {$t=2\Delta t$};
	\node[labels,above = 0.4cm of time 4n.center] {$t=n\Delta t$};

\end{tikzpicture}}
  \end{center}
  \item Fully coupled approach solves the nonlinear physics together
  \begin{center}
    \scalebox{0.6}{\begin{tikzpicture}

	% Define background color
	\colorlet{bgcolor}{mitred!30};

	% Define Dimensions
	\def\blockwidth{2.75cm};
	\def\blockheight{1.5cm};
	\def\elemheight{2.5cm};
	\def\elemwidth{2cm};
	\def\elemvsep{0.5cm};
	\def\elemhsep{0.75cm};
	
	% Define Outer Frame
	\tikzstyle{frame} = [rounded corners, fill=bgcolor, draw=black, double, very thick];
	\draw[frame] (0.0,0.0) rectangle (4*\blockwidth+0.1cm,2*\blockheight+0.5cm);

	% Define Elements in Frame
	\begin{scope}
		\tikzstyle{element} = [rectangle, rounded corners, draw = black, very thick, minimum height=\elemheight, minimum width=\elemwidth];	
		\node[element] (time 1) at (0.5*\elemwidth+0.5*\elemhsep,0.5*\elemheight+0.5*\elemvsep) {};
		\node[element,right = \elemhsep of time 1.east] (time 2) {};
		\node[element,right = \elemhsep of time 2.east] (time 3) {};
		\node[element,right = \elemhsep of time 3.east] (time 4) {};
	\end{scope}

	% put text in elements
	\tikzstyle{textelem} = [rectangle, minimum height=1.0cm, minimum width=\elemwidth];
	\node[textelem] (TH1) at (0.5*\elemwidth+0.5*\elemhsep,0.5*1.0cm+0.5*\elemvsep) {and T/H};
	\node[textelem,right = \elemhsep of TH1.east] (TH2) {and T/H};
	\node[textelem,right = \elemhsep of TH2.east] (TH3) {and T/H};
	\node[textelem,right = \elemhsep of TH3.east] (TH4) {and T/H};
	\node[textelem,above = \elemvsep of TH1.north] (N1) {Neutronics};
	\node[textelem,above = \elemvsep of TH2.north] (N2) {Neutronics};
	\node[textelem,above = \elemvsep of TH3.north] (N3) {Neutronics};
	\node[textelem,above = \elemvsep of TH4.north] (N4) {Neutronics};

		
	% Draw connectors between elements
	\tikzstyle{connector} = [->, >=stealth, thick, shorten >=2pt];
	\draw[connector] (time 1) to [out=0,in=180] (time 2);
	\draw[connector] (time 2) to [out=0,in=180] (time 3);
	\draw[connector,dashed] (time 3) to [out=0,in=180]  (time 4);
	
	% Put labels in
	\tikzstyle{labels}    = [minimum height = 1.0cm, text centered, anchor = south];
	\node[labels,above = 1.1cm of time 1.center] {$t=0$};
	\node[labels,above = 1.1cm of time 2.center] {$t=\Delta t$};
	\node[labels,above = 1.1cm of time 3.center] {$t=2\Delta t$};
	\node[labels,above = 1.1cm of time 4.center] {$t=n\Delta t$};

\end{tikzpicture}}
  \end{center}
\end{customlist}
\end{frame}
%------------------------------------------------------------------------------
\begin{frame}{1-D Slab Reactor Geometry}
\begin{columns}
  \begin{column}{0.5\textwidth}
  \begin{center}
    \scalebox{0.5}{\begin{tikzpicture}

	% Draw fuel pin
	\tikzstyle{fuelcirc}    = [draw = black, shape = circle, fill = red,   inner sep = 3*0.4096cm]
	\tikzstyle{gascirc}     = [draw = black, shape = circle, fill = green, inner sep = 3*0.4178cm]
	\tikzstyle{cladcirc}    = [draw = black, shape = circle, fill = gray,  inner sep = 3*0.4750cm]
	\tikzstyle{coolantsqu}  = [draw = black, very thick, shape = rectangle, fill = blue, minimum height=6*1.26cm, minimum width=6*1.26cm]
	\begin{scope}
		\node[coolantsqu] (cool) at (0,0) {};
		\node[cladcirc]   (clad) at (0,0) {};
		\node[gascirc]    (gas)  at (0,0) {};
		\node[fuelcirc]   (fuel) at (0,0) {};
	\end{scope}
	
	% Draw labels
	\tikzstyle{pcolor} = [shape = rectangle, minimum height=0.1cm, minimum width=0.1cm];
	\tikzstyle{plabel} = [shape = rectangle, minimum height=0.5cm, minimum width=1cm];
	\node[pcolor,fill=blue]  (cool_color) at (-3*1.26cm+0.5cm,-3*1.26cm - 0.5cm) {};
	\node[plabel,right = 0.1cm of cool_color.east]  (cool_label) {Coolant};
	\node[pcolor,fill=gray,right = 0.15cm of cool_label.east]  (clad_color) {};
	\node[plabel,right = 0.1cm of clad_color.east]  (clad_label) {Clad};
	\node[pcolor,fill=green,right = 0.15cm of clad_label.east]  (gas_color) {};
	\node[plabel,right = 0.1cm of gas_color.east]  (gas_label) {Gas Gap};
	\node[pcolor,fill=red,right = 0.15cm of gas_label.east]  (fuel_color) {};
	\node[plabel,right = 0.1cm of fuel_color.east]  (fuel_label) {Fuel};

\end{tikzpicture}}
    \\ Top-view
    \\ Fuel Rod Unit-Cell
  \end{center}
  \end{column}
  \begin{column}{0.5\textwidth}
  \begin{center}
    \scalebox{0.45}{\begin{tikzpicture}

	% Draw slab reactor
	\draw[draw = black, very thick,fill = red!50] (0.0,0.0) rectangle (12.0cm,1cm);

	% Label and Draw Width
	\node[rectangle, minimum height=1cm, minimum width=5cm, fill=none] (title) at (6.0cm,0.5cm) {1-D Reactor, 370cm};
	\draw[->, >=stealth, very thick, shorten >=2pt] (title) to [out=180,in=0] (0.0,0.5cm);
	\draw[->, >=stealth, very thick, shorten >=2pt] (title) to [out=0,in=180] (12.0cm,0.5cm);	

	% Draw coolant wavy arrows
	\draw[snake=coil,segment aspect=0,very thick,segment length=20pt,line after snake=1mm,->,draw=blue] (0.0cm,1.5cm) -- (12.0cm,1.5cm);
	\draw[snake=coil,segment aspect=0,very thick,segment length=20pt,line after snake=1mm,->,draw=blue] (0.0cm,2.0cm) -- (12.0cm,2.0cm);
	\draw[snake=coil,segment aspect=0,very thick,segment length=20pt,line after snake=1mm,->,draw=blue] (0.0cm,2.5cm) -- (12.0cm,2.5cm);

	% Label as Coolant
	\node[rectangle, minimum height=0.5cm, minimum width=1cm, fill=white] (cool) at (6.0cm,2.0cm) {Coolant};
	
\end{tikzpicture}}
      \\ Side-view (vertical)
      \\ 1-D Model of Reactor
  \end{center}
  \end{column}
\end{columns}
\end{frame}
%------------------------------------------------------------------------------
\end{section}
%==============================================================================
\begin{section}{Governing Equations}
%------------------------------------------------------------------------------
\begin{frame}{Neutronics}
\relsize{-1}
\begin{customlist}{2ex}{0pt}

    \item Basic Neutron Conservation:
    \[
      \nonumber Change + Leakage + Interactions = Scattering + Fission
    \]

    \item Neutron Diffusion Equation (1-D Energy Integrated):
    \begin{multline}
    \nonumber
\underbrace{\frac{1}{v}\frac{\partial\phi}{\partial t}}_{\mathrm{time-dependent}}-\underbrace{D\left(x,t\right)\frac{\partial^{2} \phi}{\partial^{2} x}}_{\mathrm{diffusion}}+\underbrace{\Sigma_{a}\left(x,t\right)\phi\left(x,t\right)}_{\mathrm{absorption}}= \\ \underbrace{\frac{1-\beta}{k_{eff}}\nu\Sigma_{f}\left(x,t\right)\phi\left(x,t\right)}_{\mathrm{fission}} + \underbrace{\lambda_{d}c\left(x,t\right)}_{\mathrm{decay}}
    \end{multline}

    \vspace{-0.04cm}\item Precursor Concentration Equation:
    \[
    \nonumber {\underbrace{\frac{\partial c}{\partial t}}_{\mathrm{time-dependent}}=\underbrace{\frac{\beta}{k_{eff}}\nu\Sigma_{f}\left(x,t\right)\phi\left(x,t\right)}_{\mathrm{fission}}-\underbrace{\lambda_{d}c\left(x,t\right)}_{\mathrm{decay}}   }
    \]

  \alert{Note:} Precursors are unstable isotopes created from fission that decay through neutron emission

\end{customlist}
\end{frame}
%------------------------------------------------------------------------------
\begin{frame}{Discretization of Neutronics Equations}
\relsize{-2}
\begin{block}{Assumptions:}
\begin{enumerate}
  \item One-dimensional finite volume spatial discretization, uniform $\Delta x$
  \item Central difference scheme for diffusion term
  \item No incoming current of neutrons at boundaries
  \item Implicit Euler time discretization
\end{enumerate}
\end{block}
\begin{customlist}{2ex}{0pt}
  \item Discretized neutronics equation for interior cell
  \begin{multline}
    \nonumber
    \frac{1}{v}\frac{d\bar{\phi}_{i}}{dt}-\frac{2}{\Delta x^{2}}\frac{D_{i}D_{i-1}}{D_{i}+D_{i-1}}\bar{\phi}_{i-1}+\left(\frac{2}{\Delta x^{2}}\frac{D_{i+1}D_{i}}{D_{i+1}+D_{i}}+\frac{2}{\Delta x^{2}}\frac{D_{i}D_{i-1}}{D_{i}+D_{i-1}}+\Sigma_{a,i}\right)\bar{\phi}_{i}-\\
    \frac{2}{\Delta x^{2}}\frac{D_{i+1}D_{i}}{D_{i+1}+D_{i}}\bar{\phi}_{i+1}=\frac{1-\beta}{k_{eff}}\nu\Sigma_{f,i}\bar{\phi}_{i}+\lambda_{d}\bar{c}_{i}
  \end{multline}
  \item Matrix-form of neutronics equations
  \[
    \bar{\mathbf{\Phi}}^{n+1}-\bar{\mathbf{\Phi}}^{n}+v\Delta t\left(\mathbb{M}\bar{\mathbf{\Phi}}^{n+1}-\left(1-\beta\right)\lambda\mathbb{F}\bar{\mathbf{\Phi}}^{n+1}-\lambda_{d}\mathbf{\bar{c}}^{n+1}\right)=0
  \]
  \item Matrix-form of precursors
  \[
    \mathbf{\bar{c}}^{n+1}-\mathbf{\bar{c}}^{n}+\Delta t\left(\lambda_{d}\mathbf{\bar{c}}^{n+1}-\beta\lambda\mathbb{F}\bar{\mathbf{\Phi}}^{n+1}\right)=0
  \]
\end{customlist}
\end{frame}
%------------------------------------------------------------------------------
\begin{frame}{Thermal Hydraulics}
\relsize{-1}
\begin{customlist}{2ex}{0pt}
  \item Energy Equation - single phase fluid and inviscid fluid
  \[
   \frac{\partial\left(\rho h\right)}{\partial t}+\nabla\cdot\left(\rho h\mathbf{u}\right)=-\nabla\cdot\mathbf{q}^{\prime\prime}+q^{\prime\prime\prime}
  \]
  \vfill\item Assuming fissions are a volumetic heat source in 1-D
  \[
   \rho A\frac{\partial h}{\partial t}+\dot{m}\frac{\partial h}{\partial x}=q^{\prime}
  \]
  \vfill\item For an incompressible fluid, $dh=c_{p}dT$
  \[
   \rho Ac_{p}\frac{\partial T}{\partial t}+\dot{m}c_{p}\frac{\partial T}{\partial x}=q^{\prime}
  \]
\end{customlist}
\end{frame}
%------------------------------------------------------------------------------
\begin{frame}{Discretization of Energy Equation}
\relsize{-1}
\begin{block}{Assumptions:}
\begin{enumerate}
  \item One-dimensional finite volume spatial discretization, uniform $\Delta x$
  \item Upwind difference scheme for flux
  \item Specify inlet conditions and mass flow rate
  \item Implicit Euler time discretization
\end{enumerate}
\end{block}
\begin{customlist}{2ex}{0pt}
  \item Spatial discretization
  \[
   \frac{\rho A\Delta x}{\dot{m}}\frac{d\bar{T}_{i}}{dt}+\bar{T}_{i}-\bar{T}_{i-1}=\frac{1}{2 \dot{m} c_{p}}\left(Q_{i-1}+Q_{i}\right)
  \]
  \item Matrix-form with time discretization
  \[
   \mathbf{\bar{T}}^{n+1}-\mathbf{\bar{T}}^{n}+\frac{\dot{m} \Delta t}{\mathcal{P}^{n+1}A\Delta x}\left(\mathbb{S}\mathbf{\bar{T}}^{n+1}-\mathbb{R}\mathbf{Q}^{n+1}\right)=0
  \]
  \alert{Note:} $\mathcal{P}$ is a vector of cell-averaged coolant densities
\end{customlist}
\end{frame}
%------------------------------------------------------------------------------
\begin{frame}{Physics Coupling}
\relsize{-2}
\begin{block}{Neutronics-Thermal Hydraulics}
  \begin{itemize}
    \item Neutrons cause fission
    \item Large portion of fission energy deposited in coolant
    \item This is represented by
    \[
      \mathbf{Q}=\tilde{c}\mathbb{E}\bar{\mathbf{\Phi}}\Delta x
    \]
    \\where $\mathbb{E}=\mathrm{diag}\left\{\kappa\Sigma_{f}\right\}$ characterizes energy per fission and \\  $\tilde{c}$ is the flux-to-power normalization constant
  \end{itemize}
\end{block}
\begin{alertblock}{Thermal Hydraulics-Neutronics}
  \begin{itemize}
    \item Diffusion theory parameters depend on coolant density
    \item This dependence is determined with a transport theory code
    \item $D$, $\Sigma_{a}$, $\nu\Sigma_{f}$, $\kappa\Sigma_{f}$ are all affected by coolant density variations
    \item Data is correlated with a linear regression of the form:
    \[
     \mathbf{\Sigma} = \mathbf{\Sigma}^{ref} + \frac{\partial\Sigma}{\partial\rho}\left(\mathcal{P} - \rho^{ref}\right)
    \]
  \end{itemize}
\end{alertblock}
\end{frame}
%------------------------------------------------------------------------------
\begin{frame}{The Steady State Eigenvalue Problem}
\begin{customlist}{2ex}{0pt}
  \item The steady state equations must be solved first
  \vfill\item Reducing the neutronics equation to steady state form:
  \[
    \mathbb{M}\bar{\mathbf{\Phi}}=\lambda\mathbb{F}\bar{\mathbf{\Phi}},\qquad \lambda = \frac{1}{k_{e\!f\!f}}
  \]
  \vfill\item Eigenvalue, $\lambda$, and eigenvector, $\bar{\mathbf{\Phi}}$, must be determined
  \vfill\item Flux-to-power normalization constant determined from reactor power:
  \[
    Q_{R}=\tilde{c}\int_{0}^{L}dx\kappa\Sigma_{f}\left(x\right)\phi\left(x\right)=\tilde{c}\sum_{i}\kappa\Sigma_{f,i}\bar{\phi}_{i}\Delta x=\tilde{c}\kappa\mathbf{\Sigma}_{f}^{\mathrm{T}}\bar{\mathbf{\Phi}}\Delta x
  \]
  \item $\lambda$ and $\tilde{c}$ are specified as constants for time-dependent calculations 
  \vspace{0.5cm}
\end{customlist}
\end{frame}
%------------------------------------------------------------------------------
\end{section}
%==============================================================================
\begin{section}{Solvers}
%------------------------------------------------------------------------------
\begin{frame}{Newton's Method}
\relsize{-2}
\begin{block}{Procedure:}
\begin{algorithmic}[1]
  \color{blue}
  \STATE Goal:$\mathbf{F}\left(\mathbf{x}\right) = \mathbf{0}$
  \color{black}
  \STATE Guess $\mathbf{x}$
  \FOR{$n=1,2,3,...$}
    \STATE $\mathbf{r} = \mathbf{F}\left(\mathbf{x}\right)$
    \IF{$\left\Vert \mathbf{r} \right\Vert < ntol$} \STATE DONE! \ENDIF
    \STATE $\mathbf{dx} = -\mathbb{J}^{-1}\left(\mathbf{x}\right)\mathbf{F}\left(\mathbf{x}\right)$
    \STATE $\mathbf{x} = \mathbf{x} + \mathbf{dx}$
  \ENDFOR
\end{algorithmic}
\end{block}
Three major tasks:
\begin{enumerate}
  \item Evaluate residual vector in external function
  \item Evaluate Jacobian in external function (Do we have to?)
  \item Calculate $\mathbf{dx}$ - Direct or Iterative Solvers?
\end{enumerate}
\end{frame}
%------------------------------------------------------------------------------
\begin{frame}{Krylov Subspace Methods}
\relsize{-1}
\begin{customlist}{2ex}{0pt}
  \item A class of iterative methods for sparse systems
  \item Solves $\mathbb{A}\mathbf{x}=\mathbf{b}$ by projecting a $m$ dimensional problem into a lower dimensional Krylov subspace
  \[
   \mathcal{K}_{n}\left(\mathbb{A},\mathbf{v}\right)=\mathrm{span}\left\{ \mathbf{v},\mathbb{A}\mathbf{v},\mathbb{A}^{2}\mathbf{v},...,\mathbb{A}^{n-1}\mathbf{v}\right\} 
  \]
  \item Here $\mathbb{A}$ is nonhermitian and we use the GMRES method
  \item GMRES uses the Arnoldi method to reduce the system to Hessenberg form
  \[
   \mathbb{A}\mathbb{Q}=\mathbb{Q}\mathbb{H}
  \]
  \[
   \mathbb{H}=\left[\begin{array}{cccc}
    h_{11} &  & \cdots & h_{1n}\\
    h_{21} & h_{22}\\
    & \ddots & \ddots & \vdots\\
    &  & h_{n,n-1} & h_{n,n}
   \end{array}\right]
  \]
\end{customlist}
\end{frame}
%------------------------------------------------------------------------------
\begin{frame}{Generalized Minimal RESidual Method}
\relsize{-2}
\begin{customlist}{2ex}{0pt}
  \item Goal: $\mathbf{x}_{*}=\mathbb{A}^{-1}\mathbf{b}$
  \item A step $n$, $\mathbf{x}_{*}$ is approximated by $\mathbf{x}_{n}\in\mathcal{K}_{n}$ that minimizes the norm of the residual $\mathbf{r}_{n} = \mathbf{b} - \mathbb{A}\mathbf{x}_{n}$
  \begin{block}{Procedure:}
  \begin{algorithmic}[1]
    \STATE $q_{1} = b/\left\Vert b \right\Vert$
    \FOR{$n=1,2,3,...$}
      \STATE Perform step $n$ of Arnoldi (Creates Hessenberg matrix)
      \STATE Find $y$ to minimize $\left\Vert \widetilde{\mathbb{H}}_{n}\mathbf{y}-\left\Vert \mathbf{b}\right\Vert \mathbf{e}_{1}\right\Vert$
      \IF{$\left\Vert \mathbf{r} \right\Vert < ltol$} \STATE DONE! \ENDIF
    \ENDFOR
    \STATE $\mathbf{x}=\mathbb{Q}_{n}\mathbf{y}$
  \end{algorithmic}
  \end{block}
  \item Saad et al.\footnote{\relsize{-3}Youcef Saad and Martin H. Schultz. GMRES: A generalized minimal residual algorithm for solving nonsymmetric linear systems. Society for Industrial and Applied Mathematics, 7:856-859, 1986.} defines a novel method to compute $\widetilde{\mathbb{H}}_{n}$ from $\widetilde{\mathbb{H}}_{n-1}$ from Givens rotations
\end{customlist}
\end{frame}
%------------------------------------------------------------------------------
\begin{frame}{Inexact Newton's Method}
\relsize{-1}
\begin{customlist}{2ex}{0pt}
  \item Newton-Krylov is an inexact Newton method since the linear step is not determined \emph{exactly}
  \vfill\item In Newton-Krylov framework, two tolerances were defined:
  \begin{enumerate}
    \item Nonlinear tolerance for Newton iteration
    \item Linear tolerance for GMRES iteration
  \end{enumerate}
  \vfill\item Why have tight linear convergence when nonlinear residual is large?
  \vfill\item Instead, a relative residual tolerance, $\eta$, is used
  \[
   \left\Vert \mathbb{J}\left(\mathbf{x}^{n}\right)\mathbf{dx}_{m}^{n}+\mathbf{F}\left(\mathbf{x}^{n}\right)\right\Vert <\eta\left\Vert \mathbf{F}\left(\mathbf{x}^{n}\right)\right\Vert
  \]
  \vfill\item At initial Newton iterations, GMRES will not be converged very tightly
  \vfill\item For the last couple of Newton iterations, convergence may be too tight \\ $\therefore$ limit how small linear tolerance can get
\end{customlist}
\end{frame}
%------------------------------------------------------------------------------
\begin{frame}{Jacobian-Free Approximation}
\relsize{-2}
\begin{customlist}{2ex}{0pt}
  \item Recall a Krylov subspace: $\mathcal{K}_{n}\left(\mathbb{A},\mathbf{v}\right)=\mathrm{span}\left\{ \mathbf{v},\mathbb{A}\mathbf{v},\mathbb{A}^{2}\mathbf{v},...,\mathbb{A}^{n-1}\mathbf{v}\right\}$
  \vfill\item Why create $\mathbb{A}$ when it is only used to multiply a vector?
  \vfill\item Option 1: Perform Jacobian-vector product analytically
  \[
   \mathbb{J}\mathbf{y}=\left[\begin{array}{cc}
    \mathbb{M}-\lambda\mathbb{F} & -\mathbb{F}\bar{\mathbf{\Phi}}\\
    -\bar{\mathbf{\Phi}}^{\top} & 0
    \end{array}\right]\left[\begin{array}{c}
    y_{\phi}\\
    y_{\lambda}
    \end{array}\right]=\left[\begin{array}{c}
    \left(\mathbb{M}-\lambda\mathbb{F}\right)y_{\phi}-\mathbb{F}\bar{\mathbf{\Phi}}y_{\lambda}\\
    -\bar{\mathbf{\Phi}}^{\top}y_{\phi}
    \end{array}\right]
  \]
  \vfill\item Option 2: Approximate Jacobian-vector product with finite difference
  \[
   \mathbb{J}\mathbf{y}\approx\frac{\mathbf{F}\left(\mathbf{x}+\epsilon\mathbf{y}\right)-\mathbf{F}\left(\mathbf{x}\right)}{\epsilon}
  \]
  \vfill\item Advantages: Saves memory and possibly computational time to form Jacobian
  \vfill\item $\epsilon$ is the perturbation parameter and is somewhat arbitrary - Mousseau\footnote{\relsize{-2}V.A. Mousseau. Implicitly balanced solution of the two-phase flow equations couple to nonlinear heat
    conduction. Journal of Computational Physics, 200:104-132, 2004.} recommends:
  \[
   \epsilon=\frac{\sum_{i=1}^{N}bx_{i}}{N\left\Vert \mathbf{y}\right\Vert _{2}}\qquad b=1\times10^{-8}
  \]
\end{customlist}
\end{frame}
%------------------------------------------------------------------------------
\begin{frame}{Preconditioning}
\begin{customlist}{2ex}{0pt}
  \item Want to limit number of GMRES iterations
  \vfill\item Before a calculation, a Jacobian matrix is formed analytically and a zero-fill Incomplete LU (ILU) is performed:
  \[
   \mathbb{R}=\mathbb{L}\mathbb{U}-\mathbb{A}
  \]
  \vfill\item In ILU, residual matrix $\mathbb{R}$ is constrained to certain conditions
  \vfill\item Zero-fill implies that the number and location of nonzeros is preserved
  \vfill\item Left preconditioning is used in this project:
  \[
   \mathbb{U}^{-1}\mathbb{L}^{-1}\mathbb{A}\mathbf{x}=\mathbb{U}^{-1}\mathbb{L}^{-1}\mathbf{b}
  \]
\end{customlist}
\end{frame}
%------------------------------------------------------------------------------
\end{section}
%==============================================================================
\begin{section}{Results}
%------------------------------------------------------------------------------
\begin{frame}{Steady State - Neutronics}
\relsize{-1}
\begin{customlist}{2ex}{0pt}
  \item Residual Equations
  \[
    \mathbf{F}=\left[\begin{array}{c}
    \mathbb{M}\bar{\mathbf{\Phi}}-\lambda\mathbb{F}\bar{\mathbf{\Phi}}\\
    -\frac{1}{2}\bar{\mathbf{\Phi}}^{\top}\bar{\mathbf{\Phi}}+\frac{1}{2}
    \end{array}\right]
  \]
  \item Resulting neutron flux distribution:
\end{customlist}
  \begin{center}
  \begin{columns}
    \begin{column}{0.5\textwidth}
      \begin{animateinline}[poster = first, controls]{2}
	\scalebox{0.5}{\begin{tikzpicture}[scale=0.8]

\begin{axis}[%
scale only axis,
width=4.52083in,
height=3.56562in,
xmin=0, xmax=600,
ymin=-0.008, ymax=0.008,
xlabel={Slab Length [cm]},
ylabel={Unnormalized Flux [-]},
axis on top,
legend entries={Power Iteration,Analytic JFNK,FD JFNK},
legend style={nodes=right},
legend pos= south east]
\addplot [
color=blue,
solid,
line width=2.0pt
]
coordinates{
 (1,3.07627e-005)(2,4.43097e-005)(3,5.78554e-005)(4,7.13996e-005)(5,8.49418e-005)(6,9.84818e-005)(7,0.000112019)(8,0.000125553)(9,0.000139084)(10,0.000152611)(11,0.000166134)(12,0.000179653)(13,0.000193166)(14,0.000206674)(15,0.000220177)(16,0.000233674)(17,0.000247164)(18,0.000260648)(19,0.000274125)(20,0.000287594)(21,0.000301056)(22,0.000314509)(23,0.000327954)(24,0.00034139)(25,0.000354816)(26,0.000368233)(27,0.00038164)(28,0.000395037)(29,0.000408423)(30,0.000421798)(31,0.000435161)(32,0.000448513)(33,0.000461853)(34,0.00047518)(35,0.000488494)(36,0.000501795)(37,0.000515082)(38,0.000528356)(39,0.000541615)(40,0.000554859)(41,0.000568088)(42,0.000581302)(43,0.000594501)(44,0.000607683)(45,0.000620848)(46,0.000633997)(47,0.000647129)(48,0.000660243)(49,0.000673339)(50,0.000686417)(51,0.000699477)(52,0.000712517)(53,0.000725538)(54,0.00073854)(55,0.000751521)(56,0.000764482)(57,0.000777423)(58,0.000790342)(59,0.00080324)(60,0.000816116)(61,0.00082897)(62,0.000841802)(63,0.00085461)(64,0.000867396)(65,0.000880158)(66,0.000892896)(67,0.00090561)(68,0.0009183)(69,0.000930965)(70,0.000943604)(71,0.000956218)(72,0.000968806)(73,0.000981368)(74,0.000993903)(75,0.00100641)(76,0.00101889)(77,0.00103135)(78,0.00104377)(79,0.00105617)(80,0.00106854)(81,0.00108088)(82,0.00109319)(83,0.00110547)(84,0.00111772)(85,0.00112994)(86,0.00114213)(87,0.00115429)(88,0.00116641)(89,0.00117851)(90,0.00119057)(91,0.00120261)(92,0.0012146)(93,0.00122657)(94,0.0012385)(95,0.0012504)(96,0.00126227)(97,0.0012741)(98,0.0012859)(99,0.00129766)(100,0.00130939)(101,0.00132108)(102,0.00133273)(103,0.00134435)(104,0.00135594)(105,0.00136748)(106,0.00137899)(107,0.00139046)(108,0.0014019)(109,0.00141329)(110,0.00142465)(111,0.00143597)(112,0.00144725)(113,0.00145849)(114,0.00146969)(115,0.00148086)(116,0.00149198)(117,0.00150306)(118,0.0015141)(119,0.0015251)(120,0.00153606)(121,0.00154697)(122,0.00155785)(123,0.00156868)(124,0.00157947)(125,0.00159022)(126,0.00160092)(127,0.00161158)(128,0.0016222)(129,0.00163277)(130,0.0016433)(131,0.00165379)(132,0.00166423)(133,0.00167462)(134,0.00168497)(135,0.00169527)(136,0.00170553)(137,0.00171574)(138,0.00172591)(139,0.00173602)(140,0.0017461)(141,0.00175612)(142,0.0017661)(143,0.00177602)(144,0.0017859)(145,0.00179574)(146,0.00180552)(147,0.00181525)(148,0.00182494)(149,0.00183457)(150,0.00184416)(151,0.0018537)(152,0.00186318)(153,0.00187262)(154,0.001882)(155,0.00189133)(156,0.00190062)(157,0.00190985)(158,0.00191903)(159,0.00192815)(160,0.00193723)(161,0.00194625)(162,0.00195522)(163,0.00196414)(164,0.001973)(165,0.00198181)(166,0.00199057)(167,0.00199927)(168,0.00200792)(169,0.00201651)(170,0.00202505)(171,0.00203353)(172,0.00204196)(173,0.00205034)(174,0.00205865)(175,0.00206692)(176,0.00207512)(177,0.00208327)(178,0.00209137)(179,0.0020994)(180,0.00210738)(181,0.00211531)(182,0.00212317)(183,0.00213098)(184,0.00213873)(185,0.00214642)(186,0.00215406)(187,0.00216163)(188,0.00216915)(189,0.00217661)(190,0.00218401)(191,0.00219135)(192,0.00219863)(193,0.00220585)(194,0.00221301)(195,0.00222012)(196,0.00222716)(197,0.00223414)(198,0.00224106)(199,0.00224792)(200,0.00225472)(201,0.00226146)(202,0.00226813)(203,0.00227475)(204,0.0022813)(205,0.0022878)(206,0.00229423)(207,0.00230059)(208,0.0023069)(209,0.00231314)(210,0.00231932)(211,0.00232544)(212,0.00233149)(213,0.00233748)(214,0.00234341)(215,0.00234928)(216,0.00235508)(217,0.00236081)(218,0.00236649)(219,0.0023721)(220,0.00237764)(221,0.00238312)(222,0.00238854)(223,0.00239389)(224,0.00239917)(225,0.0024044)(226,0.00240955)(227,0.00241464)(228,0.00241967)(229,0.00242463)(230,0.00242952)(231,0.00243435)(232,0.00243911)(233,0.00244381)(234,0.00244844)(235,0.002453)(236,0.0024575)(237,0.00246193)(238,0.00246629)(239,0.00247059)(240,0.00247482)(241,0.00247898)(242,0.00248308)(243,0.00248711)(244,0.00249107)(245,0.00249496)(246,0.00249879)(247,0.00250254)(248,0.00250623)(249,0.00250986)(250,0.00251341)(251,0.0025169)(252,0.00252032)(253,0.00252367)(254,0.00252695)(255,0.00253016)(256,0.00253331)(257,0.00253638)(258,0.00253939)(259,0.00254233)(260,0.0025452)(261,0.002548)(262,0.00255073)(263,0.00255339)(264,0.00255598)(265,0.00255851)(266,0.00256096)(267,0.00256335)(268,0.00256566)(269,0.00256791)(270,0.00257009)(271,0.00257219)(272,0.00257423)(273,0.0025762)(274,0.0025781)(275,0.00257993)(276,0.00258168)(277,0.00258337)(278,0.00258499)(279,0.00258654)(280,0.00258802)(281,0.00258942)(282,0.00259076)(283,0.00259203)(284,0.00259323)(285,0.00259435)(286,0.00259541)(287,0.0025964)(288,0.00259731)(289,0.00259816)(290,0.00259893)(291,0.00259964)(292,0.00260027)(293,0.00260084)(294,0.00260133)(295,0.00260175)(296,0.00260211)(297,0.00260239)(298,0.0026026)(299,0.00260274)(300,0.00260281)(301,0.00260281)(302,0.00260274)(303,0.0026026)(304,0.00260239)(305,0.00260211)(306,0.00260175)(307,0.00260133)(308,0.00260084)(309,0.00260027)(310,0.00259964)(311,0.00259893)(312,0.00259816)(313,0.00259731)(314,0.0025964)(315,0.00259541)(316,0.00259435)(317,0.00259323)(318,0.00259203)(319,0.00259076)(320,0.00258942)(321,0.00258802)(322,0.00258654)(323,0.00258499)(324,0.00258337)(325,0.00258168)(326,0.00257993)(327,0.0025781)(328,0.0025762)(329,0.00257423)(330,0.00257219)(331,0.00257009)(332,0.00256791)(333,0.00256566)(334,0.00256335)(335,0.00256096)(336,0.00255851)(337,0.00255598)(338,0.00255339)(339,0.00255073)(340,0.002548)(341,0.0025452)(342,0.00254233)(343,0.00253939)(344,0.00253638)(345,0.00253331)(346,0.00253016)(347,0.00252695)(348,0.00252367)(349,0.00252032)(350,0.0025169)(351,0.00251341)(352,0.00250986)(353,0.00250623)(354,0.00250254)(355,0.00249879)(356,0.00249496)(357,0.00249107)(358,0.00248711)(359,0.00248308)(360,0.00247898)(361,0.00247482)(362,0.00247059)(363,0.00246629)(364,0.00246193)(365,0.0024575)(366,0.002453)(367,0.00244844)(368,0.00244381)(369,0.00243911)(370,0.00243435)(371,0.00242952)(372,0.00242463)(373,0.00241967)(374,0.00241464)(375,0.00240955)(376,0.0024044)(377,0.00239917)(378,0.00239389)(379,0.00238854)(380,0.00238312)(381,0.00237764)(382,0.0023721)(383,0.00236649)(384,0.00236081)(385,0.00235508)(386,0.00234928)(387,0.00234341)(388,0.00233748)(389,0.00233149)(390,0.00232544)(391,0.00231932)(392,0.00231314)(393,0.0023069)(394,0.00230059)(395,0.00229423)(396,0.0022878)(397,0.0022813)(398,0.00227475)(399,0.00226813)(400,0.00226146)(401,0.00225472)(402,0.00224792)(403,0.00224106)(404,0.00223414)(405,0.00222716)(406,0.00222012)(407,0.00221301)(408,0.00220585)(409,0.00219863)(410,0.00219135)(411,0.00218401)(412,0.00217661)(413,0.00216915)(414,0.00216163)(415,0.00215406)(416,0.00214642)(417,0.00213873)(418,0.00213098)(419,0.00212317)(420,0.00211531)(421,0.00210738)(422,0.0020994)(423,0.00209137)(424,0.00208327)(425,0.00207512)(426,0.00206692)(427,0.00205865)(428,0.00205034)(429,0.00204196)(430,0.00203353)(431,0.00202505)(432,0.00201651)(433,0.00200792)(434,0.00199927)(435,0.00199057)(436,0.00198181)(437,0.001973)(438,0.00196414)(439,0.00195522)(440,0.00194625)(441,0.00193723)(442,0.00192815)(443,0.00191903)(444,0.00190985)(445,0.00190062)(446,0.00189133)(447,0.001882)(448,0.00187262)(449,0.00186318)(450,0.0018537)(451,0.00184416)(452,0.00183457)(453,0.00182494)(454,0.00181525)(455,0.00180552)(456,0.00179574)(457,0.0017859)(458,0.00177602)(459,0.0017661)(460,0.00175612)(461,0.0017461)(462,0.00173602)(463,0.00172591)(464,0.00171574)(465,0.00170553)(466,0.00169527)(467,0.00168497)(468,0.00167462)(469,0.00166423)(470,0.00165379)(471,0.0016433)(472,0.00163277)(473,0.0016222)(474,0.00161158)(475,0.00160092)(476,0.00159022)(477,0.00157947)(478,0.00156868)(479,0.00155785)(480,0.00154697)(481,0.00153606)(482,0.0015251)(483,0.0015141)(484,0.00150306)(485,0.00149198)(486,0.00148086)(487,0.00146969)(488,0.00145849)(489,0.00144725)(490,0.00143597)(491,0.00142465)(492,0.00141329)(493,0.0014019)(494,0.00139046)(495,0.00137899)(496,0.00136748)(497,0.00135594)(498,0.00134435)(499,0.00133273)(500,0.00132108)(501,0.00130939)(502,0.00129766)(503,0.0012859)(504,0.0012741)(505,0.00126227)(506,0.0012504)(507,0.0012385)(508,0.00122657)(509,0.0012146)(510,0.00120261)(511,0.00119057)(512,0.00117851)(513,0.00116641)(514,0.00115429)(515,0.00114213)(516,0.00112994)(517,0.00111772)(518,0.00110547)(519,0.00109319)(520,0.00108088)(521,0.00106854)(522,0.00105617)(523,0.00104377)(524,0.00103135)(525,0.00101889)(526,0.00100641)(527,0.000993903)(528,0.000981368)(529,0.000968806)(530,0.000956218)(531,0.000943604)(532,0.000930965)(533,0.0009183)(534,0.00090561)(535,0.000892896)(536,0.000880158)(537,0.000867396)(538,0.00085461)(539,0.000841802)(540,0.00082897)(541,0.000816116)(542,0.00080324)(543,0.000790342)(544,0.000777423)(545,0.000764482)(546,0.000751521)(547,0.00073854)(548,0.000725538)(549,0.000712517)(550,0.000699477)(551,0.000686417)(552,0.000673339)(553,0.000660243)(554,0.000647129)(555,0.000633997)(556,0.000620848)(557,0.000607683)(558,0.000594501)(559,0.000581302)(560,0.000568088)(561,0.000554859)(562,0.000541615)(563,0.000528356)(564,0.000515082)(565,0.000501795)(566,0.000488494)(567,0.00047518)(568,0.000461853)(569,0.000448513)(570,0.000435161)(571,0.000421798)(572,0.000408423)(573,0.000395037)(574,0.00038164)(575,0.000368233)(576,0.000354816)(577,0.00034139)(578,0.000327954)(579,0.000314509)(580,0.000301056)(581,0.000287594)(582,0.000274125)(583,0.000260648)(584,0.000247164)(585,0.000233674)(586,0.000220177)(587,0.000206674)(588,0.000193166)(589,0.000179653)(590,0.000166134)(591,0.000152611)(592,0.000139084)(593,0.000125553)(594,0.000112019)(595,9.84818e-005)(596,8.49418e-005)(597,7.13996e-005)(598,5.78554e-005)(599,4.43097e-005)(600,3.07627e-005) 
};

\addplot [
color=red,
solid,
line width=2.0pt
]
coordinates{
 (1,0.000276929)(2,0.00039882)(3,0.000520613)(4,0.00064228)(5,0.00076379)(6,0.000885113)(7,0.00100622)(8,0.00112708)(9,0.00124767)(10,0.00136796)(11,0.0014879)(12,0.00160749)(13,0.00172669)(14,0.00184546)(15,0.00196379)(16,0.00208163)(17,0.00219897)(18,0.00231577)(19,0.00243201)(20,0.00254765)(21,0.00266267)(22,0.00277705)(23,0.00289074)(24,0.00300373)(25,0.00311599)(26,0.00322749)(27,0.0033382)(28,0.0034481)(29,0.00355716)(30,0.00366535)(31,0.00377264)(32,0.00387902)(33,0.00398445)(34,0.00408891)(35,0.00419237)(36,0.00429481)(37,0.0043962)(38,0.00449652)(39,0.00459574)(40,0.00469384)(41,0.0047908)(42,0.00488659)(43,0.00498119)(44,0.00507457)(45,0.00516672)(46,0.00525761)(47,0.00534721)(48,0.00543551)(49,0.00552248)(50,0.00560811)(51,0.00569237)(52,0.00577525)(53,0.00585671)(54,0.00593675)(55,0.00601533)(56,0.00609246)(57,0.00616809)(58,0.00624222)(59,0.00631483)(60,0.0063859)(61,0.00645542)(62,0.00652335)(63,0.0065897)(64,0.00665444)(65,0.00671756)(66,0.00677904)(67,0.00683887)(68,0.00689703)(69,0.0069535)(70,0.00700829)(71,0.00706136)(72,0.00711271)(73,0.00716233)(74,0.0072102)(75,0.00725631)(76,0.00730065)(77,0.00734321)(78,0.00738399)(79,0.00742296)(80,0.00746012)(81,0.00749546)(82,0.00752897)(83,0.00756065)(84,0.00759049)(85,0.00761847)(86,0.0076446)(87,0.00766886)(88,0.00769125)(89,0.00771176)(90,0.0077304)(91,0.00774715)(92,0.00776201)(93,0.00777498)(94,0.00778605)(95,0.00779523)(96,0.0078025)(97,0.00780787)(98,0.00781134)(99,0.0078129)(100,0.00781256)(101,0.00781031)(102,0.00780615)(103,0.0078001)(104,0.00779214)(105,0.00778228)(106,0.00777052)(107,0.00775687)(108,0.00774133)(109,0.0077239)(110,0.00770459)(111,0.0076834)(112,0.00766033)(113,0.0076354)(114,0.0076086)(115,0.00757995)(116,0.00754945)(117,0.00751711)(118,0.00748294)(119,0.00744694)(120,0.00740912)(121,0.0073695)(122,0.00732809)(123,0.00728488)(124,0.0072399)(125,0.00719315)(126,0.00714465)(127,0.00709441)(128,0.00704244)(129,0.00698875)(130,0.00693335)(131,0.00687627)(132,0.00681751)(133,0.00675708)(134,0.00669501)(135,0.0066313)(136,0.00656598)(137,0.00649906)(138,0.00643055)(139,0.00636048)(140,0.00628885)(141,0.00621569)(142,0.00614102)(143,0.00606484)(144,0.00598719)(145,0.00590808)(146,0.00582752)(147,0.00574555)(148,0.00566218)(149,0.00557742)(150,0.0054913)(151,0.00540385)(152,0.00531508)(153,0.00522501)(154,0.00513367)(155,0.00504107)(156,0.00494725)(157,0.00485222)(158,0.004756)(159,0.00465863)(160,0.00456012)(161,0.0044605)(162,0.00435979)(163,0.00425802)(164,0.00415521)(165,0.00405139)(166,0.00394657)(167,0.0038408)(168,0.00373409)(169,0.00362647)(170,0.00351796)(171,0.0034086)(172,0.00329841)(173,0.00318741)(174,0.00307563)(175,0.0029631)(176,0.00284986)(177,0.00273591)(178,0.0026213)(179,0.00250605)(180,0.00239019)(181,0.00227375)(182,0.00215675)(183,0.00203923)(184,0.00192121)(185,0.00180272)(186,0.00168379)(187,0.00156445)(188,0.00144472)(189,0.00132465)(190,0.00120425)(191,0.00108356)(192,0.00096261)(193,0.00084142)(194,0.000720026)(195,0.000598456)(196,0.00047674)(197,0.000354907)(198,0.000232989)(199,0.000111013)(200,-1.09896e-005)(201,-0.00013299)(202,-0.000254957)(203,-0.000376863)(204,-0.000498676)(205,-0.000620368)(206,-0.000741909)(207,-0.000863268)(208,-0.000984417)(209,-0.00110533)(210,-0.00122597)(211,-0.00134631)(212,-0.00146632)(213,-0.00158597)(214,-0.00170524)(215,-0.00182409)(216,-0.0019425)(217,-0.00206044)(218,-0.00217787)(219,-0.00229477)(220,-0.00241111)(221,-0.00252686)(222,-0.002642)(223,-0.00275649)(224,-0.00287031)(225,-0.00298343)(226,-0.00309582)(227,-0.00320746)(228,-0.00331832)(229,-0.00342836)(230,-0.00353757)(231,-0.00364592)(232,-0.00375338)(233,-0.00385992)(234,-0.00396553)(235,-0.00407016)(236,-0.0041738)(237,-0.00427643)(238,-0.00437801)(239,-0.00447853)(240,-0.00457795)(241,-0.00467625)(242,-0.00477342)(243,-0.00486942)(244,-0.00496424)(245,-0.00505784)(246,-0.00515021)(247,-0.00524133)(248,-0.00533116)(249,-0.0054197)(250,-0.00550691)(251,-0.00559279)(252,-0.0056773)(253,-0.00576042)(254,-0.00584214)(255,-0.00592243)(256,-0.00600128)(257,-0.00607867)(258,-0.00615458)(259,-0.00622898)(260,-0.00630187)(261,-0.00637321)(262,-0.00644301)(263,-0.00651123)(264,-0.00657787)(265,-0.0066429)(266,-0.00670631)(267,-0.00676809)(268,-0.00682821)(269,-0.00688667)(270,-0.00694346)(271,-0.00699855)(272,-0.00705193)(273,-0.00710359)(274,-0.00715352)(275,-0.00720171)(276,-0.00724814)(277,-0.0072928)(278,-0.00733568)(279,-0.00737678)(280,-0.00741607)(281,-0.00745356)(282,-0.00748923)(283,-0.00752308)(284,-0.00755509)(285,-0.00758525)(286,-0.00761357)(287,-0.00764003)(288,-0.00766463)(289,-0.00768736)(290,-0.00770821)(291,-0.00772719)(292,-0.00774428)(293,-0.00775948)(294,-0.00777279)(295,-0.00778421)(296,-0.00779373)(297,-0.00780134)(298,-0.00780706)(299,-0.00781087)(300,-0.00781277)(301,-0.00781277)(302,-0.00781087)(303,-0.00780706)(304,-0.00780134)(305,-0.00779373)(306,-0.00778421)(307,-0.0077728)(308,-0.00775949)(309,-0.00774429)(310,-0.0077272)(311,-0.00770822)(312,-0.00768737)(313,-0.00766464)(314,-0.00764004)(315,-0.00761358)(316,-0.00758527)(317,-0.0075551)(318,-0.00752309)(319,-0.00748925)(320,-0.00745358)(321,-0.00741609)(322,-0.0073768)(323,-0.0073357)(324,-0.00729282)(325,-0.00724816)(326,-0.00720173)(327,-0.00715354)(328,-0.00710361)(329,-0.00705195)(330,-0.00699857)(331,-0.00694348)(332,-0.0068867)(333,-0.00682824)(334,-0.00676812)(335,-0.00670634)(336,-0.00664293)(337,-0.0065779)(338,-0.00651126)(339,-0.00644304)(340,-0.00637325)(341,-0.0063019)(342,-0.00622901)(343,-0.00615461)(344,-0.0060787)(345,-0.00600132)(346,-0.00592247)(347,-0.00584217)(348,-0.00576045)(349,-0.00567733)(350,-0.00559282)(351,-0.00550695)(352,-0.00541973)(353,-0.0053312)(354,-0.00524136)(355,-0.00515025)(356,-0.00505788)(357,-0.00496427)(358,-0.00486946)(359,-0.00477346)(360,-0.00467629)(361,-0.00457799)(362,-0.00447856)(363,-0.00437805)(364,-0.00427647)(365,-0.00417384)(366,-0.0040702)(367,-0.00396556)(368,-0.00385996)(369,-0.00375342)(370,-0.00364596)(371,-0.00353761)(372,-0.0034284)(373,-0.00331835)(374,-0.0032075)(375,-0.00309586)(376,-0.00298347)(377,-0.00287035)(378,-0.00275653)(379,-0.00264204)(380,-0.0025269)(381,-0.00241115)(382,-0.00229481)(383,-0.00217791)(384,-0.00206047)(385,-0.00194254)(386,-0.00182413)(387,-0.00170528)(388,-0.00158601)(389,-0.00146636)(390,-0.00134634)(391,-0.001226)(392,-0.00110536)(393,-0.000984455)(394,-0.000863306)(395,-0.000741946)(396,-0.000620405)(397,-0.000498713)(398,-0.0003769)(399,-0.000254994)(400,-0.000133026)(401,-1.10262e-005)(402,0.000110977)(403,0.000232952)(404,0.000354871)(405,0.000476704)(406,0.00059842)(407,0.00071999)(408,0.000841385)(409,0.000962575)(410,0.00108353)(411,0.00120422)(412,0.00132462)(413,0.00144469)(414,0.00156441)(415,0.00168375)(416,0.00180268)(417,0.00192117)(418,0.00203919)(419,0.00215672)(420,0.00227372)(421,0.00239016)(422,0.00250602)(423,0.00262127)(424,0.00273588)(425,0.00284983)(426,0.00296308)(427,0.0030756)(428,0.00318738)(429,0.00329838)(430,0.00340857)(431,0.00351794)(432,0.00362644)(433,0.00373406)(434,0.00384077)(435,0.00394655)(436,0.00405136)(437,0.00415518)(438,0.00425799)(439,0.00435977)(440,0.00446048)(441,0.0045601)(442,0.00465861)(443,0.00475598)(444,0.00485219)(445,0.00494722)(446,0.00504105)(447,0.00513364)(448,0.00522499)(449,0.00531506)(450,0.00540383)(451,0.00549128)(452,0.0055774)(453,0.00566216)(454,0.00574553)(455,0.00582751)(456,0.00590806)(457,0.00598717)(458,0.00606483)(459,0.006141)(460,0.00621568)(461,0.00628884)(462,0.00636046)(463,0.00643054)(464,0.00649905)(465,0.00656597)(466,0.0066313)(467,0.006695)(468,0.00675707)(469,0.0068175)(470,0.00687626)(471,0.00693335)(472,0.00698874)(473,0.00704243)(474,0.00709441)(475,0.00714465)(476,0.00719315)(477,0.0072399)(478,0.00728488)(479,0.00732809)(480,0.0073695)(481,0.00740913)(482,0.00744694)(483,0.00748294)(484,0.00751711)(485,0.00754945)(486,0.00757995)(487,0.00760861)(488,0.0076354)(489,0.00766034)(490,0.0076834)(491,0.0077046)(492,0.00772391)(493,0.00774134)(494,0.00775689)(495,0.00777054)(496,0.00778229)(497,0.00779215)(498,0.00780011)(499,0.00780617)(500,0.00781032)(501,0.00781257)(502,0.00781292)(503,0.00781135)(504,0.00780789)(505,0.00780252)(506,0.00779525)(507,0.00778607)(508,0.007775)(509,0.00776203)(510,0.00774717)(511,0.00773042)(512,0.00771178)(513,0.00769127)(514,0.00766888)(515,0.00764462)(516,0.00761849)(517,0.00759051)(518,0.00756067)(519,0.007529)(520,0.00749548)(521,0.00746014)(522,0.00742298)(523,0.00738401)(524,0.00734324)(525,0.00730067)(526,0.00725633)(527,0.00721022)(528,0.00716235)(529,0.00711273)(530,0.00706138)(531,0.00700831)(532,0.00695353)(533,0.00689705)(534,0.00683889)(535,0.00677906)(536,0.00671758)(537,0.00665446)(538,0.00658972)(539,0.00652338)(540,0.00645544)(541,0.00638592)(542,0.00631485)(543,0.00624224)(544,0.00616811)(545,0.00609248)(546,0.00601536)(547,0.00593677)(548,0.00585673)(549,0.00577527)(550,0.00569239)(551,0.00560813)(552,0.0055225)(553,0.00543553)(554,0.00534723)(555,0.00525762)(556,0.00516674)(557,0.00507459)(558,0.00498121)(559,0.00488661)(560,0.00479082)(561,0.00469386)(562,0.00459576)(563,0.00449654)(564,0.00439622)(565,0.00429482)(566,0.00419238)(567,0.00408892)(568,0.00398446)(569,0.00387903)(570,0.00377266)(571,0.00366536)(572,0.00355717)(573,0.00344811)(574,0.00333822)(575,0.0032275)(576,0.003116)(577,0.00300374)(578,0.00289075)(579,0.00277706)(580,0.00266268)(581,0.00254766)(582,0.00243202)(583,0.00231578)(584,0.00219898)(585,0.00208164)(586,0.00196379)(587,0.00184547)(588,0.00172669)(589,0.0016075)(590,0.00148791)(591,0.00136796)(592,0.00124768)(593,0.00112709)(594,0.00100623)(595,0.000885118)(596,0.000763793)(597,0.000642283)(598,0.000520616)(599,0.000398821)(600,0.00027693) 
};

\addplot [
color=green,
dashed,
line width=2.0pt
]
coordinates{
 (1,0.000276929)(2,0.00039882)(3,0.000520613)(4,0.00064228)(5,0.00076379)(6,0.000885113)(7,0.00100622)(8,0.00112708)(9,0.00124767)(10,0.00136796)(11,0.0014879)(12,0.00160749)(13,0.00172669)(14,0.00184546)(15,0.00196379)(16,0.00208163)(17,0.00219897)(18,0.00231577)(19,0.00243201)(20,0.00254765)(21,0.00266267)(22,0.00277705)(23,0.00289074)(24,0.00300373)(25,0.00311599)(26,0.00322749)(27,0.0033382)(28,0.0034481)(29,0.00355716)(30,0.00366535)(31,0.00377264)(32,0.00387902)(33,0.00398445)(34,0.00408891)(35,0.00419237)(36,0.00429481)(37,0.0043962)(38,0.00449652)(39,0.00459574)(40,0.00469384)(41,0.0047908)(42,0.00488659)(43,0.00498119)(44,0.00507457)(45,0.00516672)(46,0.00525761)(47,0.00534721)(48,0.00543551)(49,0.00552248)(50,0.00560811)(51,0.00569237)(52,0.00577525)(53,0.00585671)(54,0.00593675)(55,0.00601533)(56,0.00609246)(57,0.00616809)(58,0.00624222)(59,0.00631483)(60,0.0063859)(61,0.00645542)(62,0.00652335)(63,0.0065897)(64,0.00665444)(65,0.00671756)(66,0.00677904)(67,0.00683887)(68,0.00689703)(69,0.0069535)(70,0.00700829)(71,0.00706136)(72,0.00711271)(73,0.00716233)(74,0.0072102)(75,0.00725631)(76,0.00730065)(77,0.00734321)(78,0.00738399)(79,0.00742296)(80,0.00746012)(81,0.00749546)(82,0.00752897)(83,0.00756065)(84,0.00759049)(85,0.00761847)(86,0.0076446)(87,0.00766886)(88,0.00769125)(89,0.00771176)(90,0.0077304)(91,0.00774715)(92,0.00776201)(93,0.00777498)(94,0.00778605)(95,0.00779523)(96,0.0078025)(97,0.00780787)(98,0.00781134)(99,0.0078129)(100,0.00781256)(101,0.00781031)(102,0.00780615)(103,0.0078001)(104,0.00779214)(105,0.00778228)(106,0.00777052)(107,0.00775687)(108,0.00774133)(109,0.0077239)(110,0.00770459)(111,0.0076834)(112,0.00766033)(113,0.0076354)(114,0.0076086)(115,0.00757995)(116,0.00754945)(117,0.00751711)(118,0.00748294)(119,0.00744694)(120,0.00740912)(121,0.0073695)(122,0.00732809)(123,0.00728488)(124,0.0072399)(125,0.00719315)(126,0.00714465)(127,0.00709441)(128,0.00704244)(129,0.00698875)(130,0.00693335)(131,0.00687627)(132,0.00681751)(133,0.00675708)(134,0.00669501)(135,0.0066313)(136,0.00656598)(137,0.00649906)(138,0.00643055)(139,0.00636048)(140,0.00628885)(141,0.00621569)(142,0.00614102)(143,0.00606484)(144,0.00598719)(145,0.00590808)(146,0.00582752)(147,0.00574555)(148,0.00566218)(149,0.00557742)(150,0.0054913)(151,0.00540385)(152,0.00531508)(153,0.00522501)(154,0.00513367)(155,0.00504107)(156,0.00494725)(157,0.00485222)(158,0.004756)(159,0.00465863)(160,0.00456012)(161,0.0044605)(162,0.00435979)(163,0.00425802)(164,0.00415521)(165,0.00405139)(166,0.00394657)(167,0.0038408)(168,0.00373409)(169,0.00362647)(170,0.00351796)(171,0.0034086)(172,0.00329841)(173,0.00318741)(174,0.00307563)(175,0.0029631)(176,0.00284986)(177,0.00273591)(178,0.0026213)(179,0.00250605)(180,0.00239019)(181,0.00227375)(182,0.00215675)(183,0.00203923)(184,0.00192121)(185,0.00180272)(186,0.00168379)(187,0.00156445)(188,0.00144472)(189,0.00132465)(190,0.00120425)(191,0.00108356)(192,0.00096261)(193,0.00084142)(194,0.000720026)(195,0.000598456)(196,0.00047674)(197,0.000354907)(198,0.000232989)(199,0.000111013)(200,-1.09896e-005)(201,-0.00013299)(202,-0.000254957)(203,-0.000376863)(204,-0.000498676)(205,-0.000620368)(206,-0.000741909)(207,-0.000863268)(208,-0.000984417)(209,-0.00110533)(210,-0.00122597)(211,-0.00134631)(212,-0.00146632)(213,-0.00158597)(214,-0.00170524)(215,-0.00182409)(216,-0.0019425)(217,-0.00206044)(218,-0.00217787)(219,-0.00229477)(220,-0.00241111)(221,-0.00252686)(222,-0.002642)(223,-0.00275649)(224,-0.00287031)(225,-0.00298343)(226,-0.00309582)(227,-0.00320746)(228,-0.00331832)(229,-0.00342836)(230,-0.00353757)(231,-0.00364592)(232,-0.00375338)(233,-0.00385992)(234,-0.00396553)(235,-0.00407016)(236,-0.0041738)(237,-0.00427643)(238,-0.00437801)(239,-0.00447853)(240,-0.00457795)(241,-0.00467625)(242,-0.00477342)(243,-0.00486942)(244,-0.00496424)(245,-0.00505784)(246,-0.00515021)(247,-0.00524133)(248,-0.00533116)(249,-0.0054197)(250,-0.00550691)(251,-0.00559279)(252,-0.0056773)(253,-0.00576042)(254,-0.00584214)(255,-0.00592243)(256,-0.00600128)(257,-0.00607867)(258,-0.00615458)(259,-0.00622898)(260,-0.00630187)(261,-0.00637321)(262,-0.00644301)(263,-0.00651123)(264,-0.00657787)(265,-0.0066429)(266,-0.00670631)(267,-0.00676809)(268,-0.00682821)(269,-0.00688667)(270,-0.00694346)(271,-0.00699855)(272,-0.00705193)(273,-0.00710359)(274,-0.00715352)(275,-0.00720171)(276,-0.00724814)(277,-0.0072928)(278,-0.00733568)(279,-0.00737678)(280,-0.00741607)(281,-0.00745356)(282,-0.00748923)(283,-0.00752308)(284,-0.00755509)(285,-0.00758525)(286,-0.00761357)(287,-0.00764003)(288,-0.00766463)(289,-0.00768736)(290,-0.00770821)(291,-0.00772719)(292,-0.00774428)(293,-0.00775948)(294,-0.00777279)(295,-0.00778421)(296,-0.00779373)(297,-0.00780134)(298,-0.00780706)(299,-0.00781087)(300,-0.00781277)(301,-0.00781277)(302,-0.00781087)(303,-0.00780706)(304,-0.00780134)(305,-0.00779373)(306,-0.00778421)(307,-0.0077728)(308,-0.00775949)(309,-0.00774429)(310,-0.0077272)(311,-0.00770822)(312,-0.00768737)(313,-0.00766464)(314,-0.00764004)(315,-0.00761358)(316,-0.00758527)(317,-0.0075551)(318,-0.00752309)(319,-0.00748925)(320,-0.00745358)(321,-0.00741609)(322,-0.0073768)(323,-0.0073357)(324,-0.00729282)(325,-0.00724816)(326,-0.00720173)(327,-0.00715354)(328,-0.00710361)(329,-0.00705195)(330,-0.00699857)(331,-0.00694348)(332,-0.0068867)(333,-0.00682824)(334,-0.00676812)(335,-0.00670634)(336,-0.00664293)(337,-0.0065779)(338,-0.00651126)(339,-0.00644304)(340,-0.00637325)(341,-0.0063019)(342,-0.00622901)(343,-0.00615461)(344,-0.0060787)(345,-0.00600132)(346,-0.00592247)(347,-0.00584217)(348,-0.00576045)(349,-0.00567733)(350,-0.00559282)(351,-0.00550695)(352,-0.00541973)(353,-0.0053312)(354,-0.00524136)(355,-0.00515025)(356,-0.00505788)(357,-0.00496427)(358,-0.00486946)(359,-0.00477346)(360,-0.00467629)(361,-0.00457799)(362,-0.00447856)(363,-0.00437805)(364,-0.00427647)(365,-0.00417384)(366,-0.0040702)(367,-0.00396556)(368,-0.00385996)(369,-0.00375342)(370,-0.00364596)(371,-0.00353761)(372,-0.0034284)(373,-0.00331835)(374,-0.0032075)(375,-0.00309586)(376,-0.00298347)(377,-0.00287035)(378,-0.00275653)(379,-0.00264204)(380,-0.0025269)(381,-0.00241115)(382,-0.00229481)(383,-0.00217791)(384,-0.00206047)(385,-0.00194254)(386,-0.00182413)(387,-0.00170528)(388,-0.00158601)(389,-0.00146636)(390,-0.00134634)(391,-0.001226)(392,-0.00110536)(393,-0.000984455)(394,-0.000863306)(395,-0.000741946)(396,-0.000620405)(397,-0.000498713)(398,-0.0003769)(399,-0.000254994)(400,-0.000133026)(401,-1.10262e-005)(402,0.000110977)(403,0.000232952)(404,0.000354871)(405,0.000476704)(406,0.00059842)(407,0.00071999)(408,0.000841385)(409,0.000962575)(410,0.00108353)(411,0.00120422)(412,0.00132462)(413,0.00144469)(414,0.00156441)(415,0.00168375)(416,0.00180268)(417,0.00192117)(418,0.00203919)(419,0.00215672)(420,0.00227372)(421,0.00239016)(422,0.00250602)(423,0.00262127)(424,0.00273588)(425,0.00284983)(426,0.00296308)(427,0.0030756)(428,0.00318738)(429,0.00329838)(430,0.00340857)(431,0.00351794)(432,0.00362644)(433,0.00373406)(434,0.00384077)(435,0.00394655)(436,0.00405136)(437,0.00415518)(438,0.00425799)(439,0.00435977)(440,0.00446048)(441,0.0045601)(442,0.00465861)(443,0.00475598)(444,0.00485219)(445,0.00494722)(446,0.00504105)(447,0.00513364)(448,0.00522499)(449,0.00531506)(450,0.00540383)(451,0.00549128)(452,0.0055774)(453,0.00566216)(454,0.00574553)(455,0.00582751)(456,0.00590806)(457,0.00598717)(458,0.00606483)(459,0.006141)(460,0.00621568)(461,0.00628884)(462,0.00636046)(463,0.00643054)(464,0.00649905)(465,0.00656597)(466,0.0066313)(467,0.006695)(468,0.00675707)(469,0.0068175)(470,0.00687626)(471,0.00693335)(472,0.00698874)(473,0.00704243)(474,0.00709441)(475,0.00714465)(476,0.00719315)(477,0.0072399)(478,0.00728488)(479,0.00732809)(480,0.0073695)(481,0.00740913)(482,0.00744694)(483,0.00748294)(484,0.00751711)(485,0.00754945)(486,0.00757995)(487,0.00760861)(488,0.0076354)(489,0.00766034)(490,0.0076834)(491,0.0077046)(492,0.00772391)(493,0.00774134)(494,0.00775689)(495,0.00777054)(496,0.00778229)(497,0.00779215)(498,0.00780011)(499,0.00780617)(500,0.00781032)(501,0.00781257)(502,0.00781292)(503,0.00781135)(504,0.00780789)(505,0.00780252)(506,0.00779525)(507,0.00778607)(508,0.007775)(509,0.00776203)(510,0.00774717)(511,0.00773042)(512,0.00771178)(513,0.00769127)(514,0.00766888)(515,0.00764462)(516,0.00761849)(517,0.00759051)(518,0.00756067)(519,0.007529)(520,0.00749548)(521,0.00746014)(522,0.00742298)(523,0.00738401)(524,0.00734324)(525,0.00730067)(526,0.00725633)(527,0.00721022)(528,0.00716235)(529,0.00711273)(530,0.00706138)(531,0.00700831)(532,0.00695353)(533,0.00689705)(534,0.00683889)(535,0.00677906)(536,0.00671758)(537,0.00665446)(538,0.00658972)(539,0.00652338)(540,0.00645544)(541,0.00638592)(542,0.00631485)(543,0.00624224)(544,0.00616811)(545,0.00609248)(546,0.00601536)(547,0.00593677)(548,0.00585673)(549,0.00577527)(550,0.00569239)(551,0.00560813)(552,0.0055225)(553,0.00543553)(554,0.00534723)(555,0.00525762)(556,0.00516674)(557,0.00507459)(558,0.00498121)(559,0.00488661)(560,0.00479082)(561,0.00469386)(562,0.00459576)(563,0.00449654)(564,0.00439622)(565,0.00429482)(566,0.00419238)(567,0.00408892)(568,0.00398446)(569,0.00387903)(570,0.00377266)(571,0.00366536)(572,0.00355717)(573,0.00344811)(574,0.00333822)(575,0.0032275)(576,0.003116)(577,0.00300374)(578,0.00289075)(579,0.00277706)(580,0.00266268)(581,0.00254766)(582,0.00243202)(583,0.00231578)(584,0.00219898)(585,0.00208164)(586,0.00196379)(587,0.00184547)(588,0.00172669)(589,0.0016075)(590,0.00148791)(591,0.00136796)(592,0.00124768)(593,0.00112709)(594,0.00100623)(595,0.000885118)(596,0.000763793)(597,0.000642283)(598,0.000520616)(599,0.000398821)(600,0.00027693) 
};

\end{axis}

\end{tikzpicture}
}
	\newframe \scalebox{0.5}{\begin{tikzpicture}[scale=0.8]

\begin{axis}[%
scale only axis,
width=4.52083in,
height=3.56562in,
xmin=0, xmax=600,
ymin=0, ymax=0.003,
xlabel={Slab Length [cm]},
ylabel={Unnormalized Flux [-]},
axis on top,
legend entries={Power Iteration,Analytic JFNK,FD JFNK},
legend style={nodes=right},
legend pos= north east]
\addplot [
color=blue,
solid,
line width=2.0pt
]
coordinates{
 (1,3.07627e-005)(2,4.43097e-005)(3,5.78554e-005)(4,7.13996e-005)(5,8.49418e-005)(6,9.84818e-005)(7,0.000112019)(8,0.000125553)(9,0.000139084)(10,0.000152611)(11,0.000166134)(12,0.000179653)(13,0.000193166)(14,0.000206674)(15,0.000220177)(16,0.000233674)(17,0.000247164)(18,0.000260648)(19,0.000274125)(20,0.000287594)(21,0.000301056)(22,0.000314509)(23,0.000327954)(24,0.00034139)(25,0.000354816)(26,0.000368233)(27,0.00038164)(28,0.000395037)(29,0.000408423)(30,0.000421798)(31,0.000435161)(32,0.000448513)(33,0.000461853)(34,0.00047518)(35,0.000488494)(36,0.000501795)(37,0.000515082)(38,0.000528356)(39,0.000541615)(40,0.000554859)(41,0.000568088)(42,0.000581302)(43,0.000594501)(44,0.000607683)(45,0.000620848)(46,0.000633997)(47,0.000647129)(48,0.000660243)(49,0.000673339)(50,0.000686417)(51,0.000699477)(52,0.000712517)(53,0.000725538)(54,0.00073854)(55,0.000751521)(56,0.000764482)(57,0.000777423)(58,0.000790342)(59,0.00080324)(60,0.000816116)(61,0.00082897)(62,0.000841802)(63,0.00085461)(64,0.000867396)(65,0.000880158)(66,0.000892896)(67,0.00090561)(68,0.0009183)(69,0.000930965)(70,0.000943604)(71,0.000956218)(72,0.000968806)(73,0.000981368)(74,0.000993903)(75,0.00100641)(76,0.00101889)(77,0.00103135)(78,0.00104377)(79,0.00105617)(80,0.00106854)(81,0.00108088)(82,0.00109319)(83,0.00110547)(84,0.00111772)(85,0.00112994)(86,0.00114213)(87,0.00115429)(88,0.00116641)(89,0.00117851)(90,0.00119057)(91,0.00120261)(92,0.0012146)(93,0.00122657)(94,0.0012385)(95,0.0012504)(96,0.00126227)(97,0.0012741)(98,0.0012859)(99,0.00129766)(100,0.00130939)(101,0.00132108)(102,0.00133273)(103,0.00134435)(104,0.00135594)(105,0.00136748)(106,0.00137899)(107,0.00139046)(108,0.0014019)(109,0.00141329)(110,0.00142465)(111,0.00143597)(112,0.00144725)(113,0.00145849)(114,0.00146969)(115,0.00148086)(116,0.00149198)(117,0.00150306)(118,0.0015141)(119,0.0015251)(120,0.00153606)(121,0.00154697)(122,0.00155785)(123,0.00156868)(124,0.00157947)(125,0.00159022)(126,0.00160092)(127,0.00161158)(128,0.0016222)(129,0.00163277)(130,0.0016433)(131,0.00165379)(132,0.00166423)(133,0.00167462)(134,0.00168497)(135,0.00169527)(136,0.00170553)(137,0.00171574)(138,0.00172591)(139,0.00173602)(140,0.0017461)(141,0.00175612)(142,0.0017661)(143,0.00177602)(144,0.0017859)(145,0.00179574)(146,0.00180552)(147,0.00181525)(148,0.00182494)(149,0.00183457)(150,0.00184416)(151,0.0018537)(152,0.00186318)(153,0.00187262)(154,0.001882)(155,0.00189133)(156,0.00190062)(157,0.00190985)(158,0.00191903)(159,0.00192815)(160,0.00193723)(161,0.00194625)(162,0.00195522)(163,0.00196414)(164,0.001973)(165,0.00198181)(166,0.00199057)(167,0.00199927)(168,0.00200792)(169,0.00201651)(170,0.00202505)(171,0.00203353)(172,0.00204196)(173,0.00205034)(174,0.00205865)(175,0.00206692)(176,0.00207512)(177,0.00208327)(178,0.00209137)(179,0.0020994)(180,0.00210738)(181,0.00211531)(182,0.00212317)(183,0.00213098)(184,0.00213873)(185,0.00214642)(186,0.00215406)(187,0.00216163)(188,0.00216915)(189,0.00217661)(190,0.00218401)(191,0.00219135)(192,0.00219863)(193,0.00220585)(194,0.00221301)(195,0.00222012)(196,0.00222716)(197,0.00223414)(198,0.00224106)(199,0.00224792)(200,0.00225472)(201,0.00226146)(202,0.00226813)(203,0.00227475)(204,0.0022813)(205,0.0022878)(206,0.00229423)(207,0.00230059)(208,0.0023069)(209,0.00231314)(210,0.00231932)(211,0.00232544)(212,0.00233149)(213,0.00233748)(214,0.00234341)(215,0.00234928)(216,0.00235508)(217,0.00236081)(218,0.00236649)(219,0.0023721)(220,0.00237764)(221,0.00238312)(222,0.00238854)(223,0.00239389)(224,0.00239917)(225,0.0024044)(226,0.00240955)(227,0.00241464)(228,0.00241967)(229,0.00242463)(230,0.00242952)(231,0.00243435)(232,0.00243911)(233,0.00244381)(234,0.00244844)(235,0.002453)(236,0.0024575)(237,0.00246193)(238,0.00246629)(239,0.00247059)(240,0.00247482)(241,0.00247898)(242,0.00248308)(243,0.00248711)(244,0.00249107)(245,0.00249496)(246,0.00249879)(247,0.00250254)(248,0.00250623)(249,0.00250986)(250,0.00251341)(251,0.0025169)(252,0.00252032)(253,0.00252367)(254,0.00252695)(255,0.00253016)(256,0.00253331)(257,0.00253638)(258,0.00253939)(259,0.00254233)(260,0.0025452)(261,0.002548)(262,0.00255073)(263,0.00255339)(264,0.00255598)(265,0.00255851)(266,0.00256096)(267,0.00256335)(268,0.00256566)(269,0.00256791)(270,0.00257009)(271,0.00257219)(272,0.00257423)(273,0.0025762)(274,0.0025781)(275,0.00257993)(276,0.00258168)(277,0.00258337)(278,0.00258499)(279,0.00258654)(280,0.00258802)(281,0.00258942)(282,0.00259076)(283,0.00259203)(284,0.00259323)(285,0.00259435)(286,0.00259541)(287,0.0025964)(288,0.00259731)(289,0.00259816)(290,0.00259893)(291,0.00259964)(292,0.00260027)(293,0.00260084)(294,0.00260133)(295,0.00260175)(296,0.00260211)(297,0.00260239)(298,0.0026026)(299,0.00260274)(300,0.00260281)(301,0.00260281)(302,0.00260274)(303,0.0026026)(304,0.00260239)(305,0.00260211)(306,0.00260175)(307,0.00260133)(308,0.00260084)(309,0.00260027)(310,0.00259964)(311,0.00259893)(312,0.00259816)(313,0.00259731)(314,0.0025964)(315,0.00259541)(316,0.00259435)(317,0.00259323)(318,0.00259203)(319,0.00259076)(320,0.00258942)(321,0.00258802)(322,0.00258654)(323,0.00258499)(324,0.00258337)(325,0.00258168)(326,0.00257993)(327,0.0025781)(328,0.0025762)(329,0.00257423)(330,0.00257219)(331,0.00257009)(332,0.00256791)(333,0.00256566)(334,0.00256335)(335,0.00256096)(336,0.00255851)(337,0.00255598)(338,0.00255339)(339,0.00255073)(340,0.002548)(341,0.0025452)(342,0.00254233)(343,0.00253939)(344,0.00253638)(345,0.00253331)(346,0.00253016)(347,0.00252695)(348,0.00252367)(349,0.00252032)(350,0.0025169)(351,0.00251341)(352,0.00250986)(353,0.00250623)(354,0.00250254)(355,0.00249879)(356,0.00249496)(357,0.00249107)(358,0.00248711)(359,0.00248308)(360,0.00247898)(361,0.00247482)(362,0.00247059)(363,0.00246629)(364,0.00246193)(365,0.0024575)(366,0.002453)(367,0.00244844)(368,0.00244381)(369,0.00243911)(370,0.00243435)(371,0.00242952)(372,0.00242463)(373,0.00241967)(374,0.00241464)(375,0.00240955)(376,0.0024044)(377,0.00239917)(378,0.00239389)(379,0.00238854)(380,0.00238312)(381,0.00237764)(382,0.0023721)(383,0.00236649)(384,0.00236081)(385,0.00235508)(386,0.00234928)(387,0.00234341)(388,0.00233748)(389,0.00233149)(390,0.00232544)(391,0.00231932)(392,0.00231314)(393,0.0023069)(394,0.00230059)(395,0.00229423)(396,0.0022878)(397,0.0022813)(398,0.00227475)(399,0.00226813)(400,0.00226146)(401,0.00225472)(402,0.00224792)(403,0.00224106)(404,0.00223414)(405,0.00222716)(406,0.00222012)(407,0.00221301)(408,0.00220585)(409,0.00219863)(410,0.00219135)(411,0.00218401)(412,0.00217661)(413,0.00216915)(414,0.00216163)(415,0.00215406)(416,0.00214642)(417,0.00213873)(418,0.00213098)(419,0.00212317)(420,0.00211531)(421,0.00210738)(422,0.0020994)(423,0.00209137)(424,0.00208327)(425,0.00207512)(426,0.00206692)(427,0.00205865)(428,0.00205034)(429,0.00204196)(430,0.00203353)(431,0.00202505)(432,0.00201651)(433,0.00200792)(434,0.00199927)(435,0.00199057)(436,0.00198181)(437,0.001973)(438,0.00196414)(439,0.00195522)(440,0.00194625)(441,0.00193723)(442,0.00192815)(443,0.00191903)(444,0.00190985)(445,0.00190062)(446,0.00189133)(447,0.001882)(448,0.00187262)(449,0.00186318)(450,0.0018537)(451,0.00184416)(452,0.00183457)(453,0.00182494)(454,0.00181525)(455,0.00180552)(456,0.00179574)(457,0.0017859)(458,0.00177602)(459,0.0017661)(460,0.00175612)(461,0.0017461)(462,0.00173602)(463,0.00172591)(464,0.00171574)(465,0.00170553)(466,0.00169527)(467,0.00168497)(468,0.00167462)(469,0.00166423)(470,0.00165379)(471,0.0016433)(472,0.00163277)(473,0.0016222)(474,0.00161158)(475,0.00160092)(476,0.00159022)(477,0.00157947)(478,0.00156868)(479,0.00155785)(480,0.00154697)(481,0.00153606)(482,0.0015251)(483,0.0015141)(484,0.00150306)(485,0.00149198)(486,0.00148086)(487,0.00146969)(488,0.00145849)(489,0.00144725)(490,0.00143597)(491,0.00142465)(492,0.00141329)(493,0.0014019)(494,0.00139046)(495,0.00137899)(496,0.00136748)(497,0.00135594)(498,0.00134435)(499,0.00133273)(500,0.00132108)(501,0.00130939)(502,0.00129766)(503,0.0012859)(504,0.0012741)(505,0.00126227)(506,0.0012504)(507,0.0012385)(508,0.00122657)(509,0.0012146)(510,0.00120261)(511,0.00119057)(512,0.00117851)(513,0.00116641)(514,0.00115429)(515,0.00114213)(516,0.00112994)(517,0.00111772)(518,0.00110547)(519,0.00109319)(520,0.00108088)(521,0.00106854)(522,0.00105617)(523,0.00104377)(524,0.00103135)(525,0.00101889)(526,0.00100641)(527,0.000993903)(528,0.000981368)(529,0.000968806)(530,0.000956218)(531,0.000943604)(532,0.000930965)(533,0.0009183)(534,0.00090561)(535,0.000892896)(536,0.000880158)(537,0.000867396)(538,0.00085461)(539,0.000841802)(540,0.00082897)(541,0.000816116)(542,0.00080324)(543,0.000790342)(544,0.000777423)(545,0.000764482)(546,0.000751521)(547,0.00073854)(548,0.000725538)(549,0.000712517)(550,0.000699477)(551,0.000686417)(552,0.000673339)(553,0.000660243)(554,0.000647129)(555,0.000633997)(556,0.000620848)(557,0.000607683)(558,0.000594501)(559,0.000581302)(560,0.000568088)(561,0.000554859)(562,0.000541615)(563,0.000528356)(564,0.000515082)(565,0.000501795)(566,0.000488494)(567,0.00047518)(568,0.000461853)(569,0.000448513)(570,0.000435161)(571,0.000421798)(572,0.000408423)(573,0.000395037)(574,0.00038164)(575,0.000368233)(576,0.000354816)(577,0.00034139)(578,0.000327954)(579,0.000314509)(580,0.000301056)(581,0.000287594)(582,0.000274125)(583,0.000260648)(584,0.000247164)(585,0.000233674)(586,0.000220177)(587,0.000206674)(588,0.000193166)(589,0.000179653)(590,0.000166134)(591,0.000152611)(592,0.000139084)(593,0.000125553)(594,0.000112019)(595,9.84818e-005)(596,8.49418e-005)(597,7.13996e-005)(598,5.78554e-005)(599,4.43097e-005)(600,3.07627e-005) 
};

\addplot [
color=red,
solid,
line width=2.0pt
]
coordinates{
 (1,3.07628e-005)(2,4.43098e-005)(3,5.78555e-005)(4,7.13998e-005)(5,8.4942e-005)(6,9.8482e-005)(7,0.000112019)(8,0.000125554)(9,0.000139084)(10,0.000152612)(11,0.000166134)(12,0.000179653)(13,0.000193167)(14,0.000206675)(15,0.000220178)(16,0.000233674)(17,0.000247165)(18,0.000260649)(19,0.000274125)(20,0.000287595)(21,0.000301056)(22,0.000314509)(23,0.000327954)(24,0.00034139)(25,0.000354817)(26,0.000368234)(27,0.000381641)(28,0.000395038)(29,0.000408424)(30,0.000421799)(31,0.000435162)(32,0.000448514)(33,0.000461854)(34,0.000475181)(35,0.000488495)(36,0.000501796)(37,0.000515083)(38,0.000528357)(39,0.000541616)(40,0.00055486)(41,0.00056809)(42,0.000581304)(43,0.000594502)(44,0.000607684)(45,0.00062085)(46,0.000633999)(47,0.00064713)(48,0.000660244)(49,0.000673341)(50,0.000686419)(51,0.000699478)(52,0.000712519)(53,0.00072554)(54,0.000738541)(55,0.000751523)(56,0.000764484)(57,0.000777424)(58,0.000790344)(59,0.000803242)(60,0.000816118)(61,0.000828972)(62,0.000841803)(63,0.000854612)(64,0.000867398)(65,0.00088016)(66,0.000892898)(67,0.000905612)(68,0.000918302)(69,0.000930967)(70,0.000943606)(71,0.00095622)(72,0.000968808)(73,0.00098137)(74,0.000993905)(75,0.00100641)(76,0.00101889)(77,0.00103135)(78,0.00104377)(79,0.00105617)(80,0.00106854)(81,0.00108088)(82,0.00109319)(83,0.00110547)(84,0.00111772)(85,0.00112994)(86,0.00114213)(87,0.00115429)(88,0.00116642)(89,0.00117851)(90,0.00119058)(91,0.00120261)(92,0.00121461)(93,0.00122657)(94,0.00123851)(95,0.00125041)(96,0.00126227)(97,0.0012741)(98,0.0012859)(99,0.00129766)(100,0.00130939)(101,0.00132108)(102,0.00133274)(103,0.00134435)(104,0.00135594)(105,0.00136748)(106,0.00137899)(107,0.00139047)(108,0.0014019)(109,0.0014133)(110,0.00142465)(111,0.00143597)(112,0.00144725)(113,0.0014585)(114,0.0014697)(115,0.00148086)(116,0.00149198)(117,0.00150306)(118,0.0015141)(119,0.0015251)(120,0.00153606)(121,0.00154698)(122,0.00155785)(123,0.00156868)(124,0.00157947)(125,0.00159022)(126,0.00160093)(127,0.00161159)(128,0.0016222)(129,0.00163278)(130,0.00164331)(131,0.00165379)(132,0.00166423)(133,0.00167462)(134,0.00168497)(135,0.00169528)(136,0.00170553)(137,0.00171575)(138,0.00172591)(139,0.00173603)(140,0.0017461)(141,0.00175612)(142,0.0017661)(143,0.00177603)(144,0.00178591)(145,0.00179574)(146,0.00180552)(147,0.00181526)(148,0.00182494)(149,0.00183458)(150,0.00184416)(151,0.0018537)(152,0.00186319)(153,0.00187262)(154,0.00188201)(155,0.00189134)(156,0.00190062)(157,0.00190985)(158,0.00191903)(159,0.00192816)(160,0.00193723)(161,0.00194625)(162,0.00195522)(163,0.00196414)(164,0.001973)(165,0.00198181)(166,0.00199057)(167,0.00199927)(168,0.00200792)(169,0.00201651)(170,0.00202505)(171,0.00203354)(172,0.00204197)(173,0.00205034)(174,0.00205866)(175,0.00206692)(176,0.00207513)(177,0.00208328)(178,0.00209137)(179,0.00209941)(180,0.00210739)(181,0.00211531)(182,0.00212318)(183,0.00213098)(184,0.00213874)(185,0.00214643)(186,0.00215406)(187,0.00216164)(188,0.00216916)(189,0.00217661)(190,0.00218401)(191,0.00219135)(192,0.00219864)(193,0.00220586)(194,0.00221302)(195,0.00222012)(196,0.00222716)(197,0.00223414)(198,0.00224106)(199,0.00224792)(200,0.00225472)(201,0.00226146)(202,0.00226814)(203,0.00227475)(204,0.00228131)(205,0.0022878)(206,0.00229423)(207,0.0023006)(208,0.0023069)(209,0.00231315)(210,0.00231933)(211,0.00232544)(212,0.0023315)(213,0.00233749)(214,0.00234342)(215,0.00234928)(216,0.00235508)(217,0.00236082)(218,0.00236649)(219,0.0023721)(220,0.00237764)(221,0.00238312)(222,0.00238854)(223,0.00239389)(224,0.00239918)(225,0.0024044)(226,0.00240955)(227,0.00241464)(228,0.00241967)(229,0.00242463)(230,0.00242952)(231,0.00243435)(232,0.00243911)(233,0.00244381)(234,0.00244844)(235,0.002453)(236,0.0024575)(237,0.00246193)(238,0.00246629)(239,0.00247059)(240,0.00247482)(241,0.00247898)(242,0.00248308)(243,0.00248711)(244,0.00249107)(245,0.00249496)(246,0.00249879)(247,0.00250255)(248,0.00250624)(249,0.00250986)(250,0.00251341)(251,0.0025169)(252,0.00252032)(253,0.00252367)(254,0.00252695)(255,0.00253016)(256,0.00253331)(257,0.00253638)(258,0.00253939)(259,0.00254233)(260,0.0025452)(261,0.002548)(262,0.00255073)(263,0.00255339)(264,0.00255599)(265,0.00255851)(266,0.00256096)(267,0.00256335)(268,0.00256567)(269,0.00256791)(270,0.00257009)(271,0.0025722)(272,0.00257423)(273,0.0025762)(274,0.0025781)(275,0.00257993)(276,0.00258168)(277,0.00258337)(278,0.00258499)(279,0.00258654)(280,0.00258802)(281,0.00258942)(282,0.00259076)(283,0.00259203)(284,0.00259323)(285,0.00259435)(286,0.00259541)(287,0.0025964)(288,0.00259731)(289,0.00259816)(290,0.00259893)(291,0.00259964)(292,0.00260027)(293,0.00260084)(294,0.00260133)(295,0.00260175)(296,0.00260211)(297,0.00260239)(298,0.0026026)(299,0.00260274)(300,0.00260281)(301,0.00260281)(302,0.00260274)(303,0.0026026)(304,0.00260239)(305,0.00260211)(306,0.00260175)(307,0.00260133)(308,0.00260084)(309,0.00260027)(310,0.00259964)(311,0.00259893)(312,0.00259816)(313,0.00259731)(314,0.0025964)(315,0.00259541)(316,0.00259435)(317,0.00259323)(318,0.00259203)(319,0.00259076)(320,0.00258942)(321,0.00258801)(322,0.00258654)(323,0.00258499)(324,0.00258337)(325,0.00258168)(326,0.00257992)(327,0.0025781)(328,0.0025762)(329,0.00257423)(330,0.00257219)(331,0.00257009)(332,0.00256791)(333,0.00256566)(334,0.00256335)(335,0.00256096)(336,0.00255851)(337,0.00255598)(338,0.00255339)(339,0.00255073)(340,0.002548)(341,0.00254519)(342,0.00254233)(343,0.00253939)(344,0.00253638)(345,0.0025333)(346,0.00253016)(347,0.00252695)(348,0.00252366)(349,0.00252031)(350,0.0025169)(351,0.00251341)(352,0.00250985)(353,0.00250623)(354,0.00250254)(355,0.00249878)(356,0.00249496)(357,0.00249106)(358,0.0024871)(359,0.00248307)(360,0.00247898)(361,0.00247482)(362,0.00247059)(363,0.00246629)(364,0.00246192)(365,0.00245749)(366,0.002453)(367,0.00244843)(368,0.0024438)(369,0.00243911)(370,0.00243435)(371,0.00242952)(372,0.00242462)(373,0.00241966)(374,0.00241464)(375,0.00240955)(376,0.00240439)(377,0.00239917)(378,0.00239389)(379,0.00238853)(380,0.00238312)(381,0.00237764)(382,0.00237209)(383,0.00236648)(384,0.00236081)(385,0.00235507)(386,0.00234927)(387,0.00234341)(388,0.00233748)(389,0.00233149)(390,0.00232544)(391,0.00231932)(392,0.00231314)(393,0.0023069)(394,0.00230059)(395,0.00229422)(396,0.00228779)(397,0.0022813)(398,0.00227475)(399,0.00226813)(400,0.00226145)(401,0.00225472)(402,0.00224792)(403,0.00224106)(404,0.00223414)(405,0.00222715)(406,0.00222011)(407,0.00221301)(408,0.00220585)(409,0.00219863)(410,0.00219135)(411,0.00218401)(412,0.00217661)(413,0.00216915)(414,0.00216163)(415,0.00215405)(416,0.00214642)(417,0.00213873)(418,0.00213098)(419,0.00212317)(420,0.0021153)(421,0.00210738)(422,0.0020994)(423,0.00209136)(424,0.00208327)(425,0.00207512)(426,0.00206691)(427,0.00205865)(428,0.00205033)(429,0.00204196)(430,0.00203353)(431,0.00202504)(432,0.00201651)(433,0.00200791)(434,0.00199926)(435,0.00199056)(436,0.00198181)(437,0.001973)(438,0.00196413)(439,0.00195522)(440,0.00194625)(441,0.00193722)(442,0.00192815)(443,0.00191902)(444,0.00190984)(445,0.00190061)(446,0.00189133)(447,0.001882)(448,0.00187261)(449,0.00186318)(450,0.00185369)(451,0.00184416)(452,0.00183457)(453,0.00182493)(454,0.00181525)(455,0.00180552)(456,0.00179573)(457,0.0017859)(458,0.00177602)(459,0.00176609)(460,0.00175612)(461,0.00174609)(462,0.00173602)(463,0.0017259)(464,0.00171574)(465,0.00170553)(466,0.00169527)(467,0.00168497)(468,0.00167462)(469,0.00166422)(470,0.00165378)(471,0.0016433)(472,0.00163277)(473,0.0016222)(474,0.00161158)(475,0.00160092)(476,0.00159021)(477,0.00157947)(478,0.00156868)(479,0.00155785)(480,0.00154697)(481,0.00153605)(482,0.0015251)(483,0.0015141)(484,0.00150306)(485,0.00149197)(486,0.00148085)(487,0.00146969)(488,0.00145849)(489,0.00144725)(490,0.00143597)(491,0.00142465)(492,0.00141329)(493,0.00140189)(494,0.00139046)(495,0.00137899)(496,0.00136748)(497,0.00135593)(498,0.00134435)(499,0.00133273)(500,0.00132107)(501,0.00130938)(502,0.00129766)(503,0.00128589)(504,0.0012741)(505,0.00126227)(506,0.0012504)(507,0.0012385)(508,0.00122657)(509,0.0012146)(510,0.0012026)(511,0.00119057)(512,0.00117851)(513,0.00116641)(514,0.00115428)(515,0.00114212)(516,0.00112993)(517,0.00111771)(518,0.00110546)(519,0.00109318)(520,0.00108087)(521,0.00106853)(522,0.00105616)(523,0.00104377)(524,0.00103134)(525,0.00101889)(526,0.00100641)(527,0.0009939)(528,0.000981365)(529,0.000968803)(530,0.000956215)(531,0.000943601)(532,0.000930962)(533,0.000918297)(534,0.000905608)(535,0.000892894)(536,0.000880156)(537,0.000867394)(538,0.000854608)(539,0.000841799)(540,0.000828968)(541,0.000816114)(542,0.000803238)(543,0.00079034)(544,0.00077742)(545,0.00076448)(546,0.000751519)(547,0.000738537)(548,0.000725536)(549,0.000712515)(550,0.000699475)(551,0.000686415)(552,0.000673337)(553,0.000660241)(554,0.000647127)(555,0.000633995)(556,0.000620847)(557,0.000607681)(558,0.000594499)(559,0.000581301)(560,0.000568087)(561,0.000554857)(562,0.000541613)(563,0.000528354)(564,0.000515081)(565,0.000501793)(566,0.000488492)(567,0.000475178)(568,0.000461851)(569,0.000448512)(570,0.00043516)(571,0.000421797)(572,0.000408422)(573,0.000395036)(574,0.000381639)(575,0.000368232)(576,0.000354815)(577,0.000341388)(578,0.000327953)(579,0.000314508)(580,0.000301055)(581,0.000287593)(582,0.000274124)(583,0.000260647)(584,0.000247164)(585,0.000233673)(586,0.000220177)(587,0.000206674)(588,0.000193166)(589,0.000179652)(590,0.000166134)(591,0.000152611)(592,0.000139084)(593,0.000125553)(594,0.000112019)(595,9.84815e-005)(596,8.49416e-005)(597,7.13994e-005)(598,5.78552e-005)(599,4.43095e-005)(600,3.07626e-005) 
};

\addplot [
color=green,
dashed,
line width=2.0pt
]
coordinates{
 (1,3.07628e-005)(2,4.43098e-005)(3,5.78555e-005)(4,7.13998e-005)(5,8.4942e-005)(6,9.8482e-005)(7,0.000112019)(8,0.000125554)(9,0.000139084)(10,0.000152612)(11,0.000166134)(12,0.000179653)(13,0.000193167)(14,0.000206675)(15,0.000220178)(16,0.000233674)(17,0.000247165)(18,0.000260649)(19,0.000274125)(20,0.000287595)(21,0.000301056)(22,0.000314509)(23,0.000327954)(24,0.00034139)(25,0.000354817)(26,0.000368234)(27,0.000381641)(28,0.000395038)(29,0.000408424)(30,0.000421799)(31,0.000435162)(32,0.000448514)(33,0.000461854)(34,0.000475181)(35,0.000488495)(36,0.000501796)(37,0.000515083)(38,0.000528357)(39,0.000541616)(40,0.00055486)(41,0.00056809)(42,0.000581304)(43,0.000594502)(44,0.000607684)(45,0.00062085)(46,0.000633999)(47,0.00064713)(48,0.000660244)(49,0.000673341)(50,0.000686419)(51,0.000699478)(52,0.000712519)(53,0.00072554)(54,0.000738541)(55,0.000751523)(56,0.000764484)(57,0.000777424)(58,0.000790344)(59,0.000803242)(60,0.000816118)(61,0.000828972)(62,0.000841803)(63,0.000854612)(64,0.000867398)(65,0.00088016)(66,0.000892898)(67,0.000905612)(68,0.000918302)(69,0.000930967)(70,0.000943606)(71,0.00095622)(72,0.000968808)(73,0.00098137)(74,0.000993905)(75,0.00100641)(76,0.00101889)(77,0.00103135)(78,0.00104377)(79,0.00105617)(80,0.00106854)(81,0.00108088)(82,0.00109319)(83,0.00110547)(84,0.00111772)(85,0.00112994)(86,0.00114213)(87,0.00115429)(88,0.00116642)(89,0.00117851)(90,0.00119058)(91,0.00120261)(92,0.00121461)(93,0.00122657)(94,0.00123851)(95,0.00125041)(96,0.00126227)(97,0.0012741)(98,0.0012859)(99,0.00129766)(100,0.00130939)(101,0.00132108)(102,0.00133274)(103,0.00134435)(104,0.00135594)(105,0.00136748)(106,0.00137899)(107,0.00139047)(108,0.0014019)(109,0.0014133)(110,0.00142465)(111,0.00143597)(112,0.00144725)(113,0.0014585)(114,0.0014697)(115,0.00148086)(116,0.00149198)(117,0.00150306)(118,0.0015141)(119,0.0015251)(120,0.00153606)(121,0.00154698)(122,0.00155785)(123,0.00156868)(124,0.00157947)(125,0.00159022)(126,0.00160093)(127,0.00161159)(128,0.0016222)(129,0.00163278)(130,0.00164331)(131,0.00165379)(132,0.00166423)(133,0.00167462)(134,0.00168497)(135,0.00169528)(136,0.00170553)(137,0.00171575)(138,0.00172591)(139,0.00173603)(140,0.0017461)(141,0.00175612)(142,0.0017661)(143,0.00177603)(144,0.00178591)(145,0.00179574)(146,0.00180552)(147,0.00181526)(148,0.00182494)(149,0.00183458)(150,0.00184416)(151,0.0018537)(152,0.00186319)(153,0.00187262)(154,0.00188201)(155,0.00189134)(156,0.00190062)(157,0.00190985)(158,0.00191903)(159,0.00192816)(160,0.00193723)(161,0.00194625)(162,0.00195522)(163,0.00196414)(164,0.001973)(165,0.00198181)(166,0.00199057)(167,0.00199927)(168,0.00200792)(169,0.00201651)(170,0.00202505)(171,0.00203354)(172,0.00204197)(173,0.00205034)(174,0.00205866)(175,0.00206692)(176,0.00207513)(177,0.00208328)(178,0.00209137)(179,0.00209941)(180,0.00210739)(181,0.00211531)(182,0.00212318)(183,0.00213098)(184,0.00213874)(185,0.00214643)(186,0.00215406)(187,0.00216164)(188,0.00216916)(189,0.00217661)(190,0.00218401)(191,0.00219135)(192,0.00219864)(193,0.00220586)(194,0.00221302)(195,0.00222012)(196,0.00222716)(197,0.00223414)(198,0.00224106)(199,0.00224792)(200,0.00225472)(201,0.00226146)(202,0.00226814)(203,0.00227475)(204,0.00228131)(205,0.0022878)(206,0.00229423)(207,0.0023006)(208,0.0023069)(209,0.00231315)(210,0.00231933)(211,0.00232544)(212,0.0023315)(213,0.00233749)(214,0.00234342)(215,0.00234928)(216,0.00235508)(217,0.00236082)(218,0.00236649)(219,0.0023721)(220,0.00237764)(221,0.00238312)(222,0.00238854)(223,0.00239389)(224,0.00239918)(225,0.0024044)(226,0.00240955)(227,0.00241464)(228,0.00241967)(229,0.00242463)(230,0.00242952)(231,0.00243435)(232,0.00243911)(233,0.00244381)(234,0.00244844)(235,0.002453)(236,0.0024575)(237,0.00246193)(238,0.00246629)(239,0.00247059)(240,0.00247482)(241,0.00247898)(242,0.00248308)(243,0.00248711)(244,0.00249107)(245,0.00249496)(246,0.00249879)(247,0.00250255)(248,0.00250624)(249,0.00250986)(250,0.00251341)(251,0.0025169)(252,0.00252032)(253,0.00252367)(254,0.00252695)(255,0.00253016)(256,0.00253331)(257,0.00253638)(258,0.00253939)(259,0.00254233)(260,0.0025452)(261,0.002548)(262,0.00255073)(263,0.00255339)(264,0.00255599)(265,0.00255851)(266,0.00256096)(267,0.00256335)(268,0.00256567)(269,0.00256791)(270,0.00257009)(271,0.0025722)(272,0.00257423)(273,0.0025762)(274,0.0025781)(275,0.00257993)(276,0.00258168)(277,0.00258337)(278,0.00258499)(279,0.00258654)(280,0.00258802)(281,0.00258942)(282,0.00259076)(283,0.00259203)(284,0.00259323)(285,0.00259435)(286,0.00259541)(287,0.0025964)(288,0.00259731)(289,0.00259816)(290,0.00259893)(291,0.00259964)(292,0.00260027)(293,0.00260084)(294,0.00260133)(295,0.00260175)(296,0.00260211)(297,0.00260239)(298,0.0026026)(299,0.00260274)(300,0.00260281)(301,0.00260281)(302,0.00260274)(303,0.0026026)(304,0.00260239)(305,0.00260211)(306,0.00260175)(307,0.00260133)(308,0.00260084)(309,0.00260027)(310,0.00259964)(311,0.00259893)(312,0.00259816)(313,0.00259731)(314,0.0025964)(315,0.00259541)(316,0.00259435)(317,0.00259323)(318,0.00259203)(319,0.00259076)(320,0.00258942)(321,0.00258801)(322,0.00258654)(323,0.00258499)(324,0.00258337)(325,0.00258168)(326,0.00257992)(327,0.0025781)(328,0.0025762)(329,0.00257423)(330,0.00257219)(331,0.00257009)(332,0.00256791)(333,0.00256566)(334,0.00256335)(335,0.00256096)(336,0.00255851)(337,0.00255598)(338,0.00255339)(339,0.00255073)(340,0.002548)(341,0.00254519)(342,0.00254233)(343,0.00253939)(344,0.00253638)(345,0.0025333)(346,0.00253016)(347,0.00252695)(348,0.00252366)(349,0.00252031)(350,0.0025169)(351,0.00251341)(352,0.00250985)(353,0.00250623)(354,0.00250254)(355,0.00249878)(356,0.00249496)(357,0.00249106)(358,0.0024871)(359,0.00248307)(360,0.00247898)(361,0.00247482)(362,0.00247059)(363,0.00246629)(364,0.00246192)(365,0.00245749)(366,0.002453)(367,0.00244843)(368,0.0024438)(369,0.00243911)(370,0.00243435)(371,0.00242952)(372,0.00242462)(373,0.00241966)(374,0.00241464)(375,0.00240955)(376,0.00240439)(377,0.00239917)(378,0.00239389)(379,0.00238853)(380,0.00238312)(381,0.00237764)(382,0.00237209)(383,0.00236648)(384,0.00236081)(385,0.00235507)(386,0.00234927)(387,0.00234341)(388,0.00233748)(389,0.00233149)(390,0.00232544)(391,0.00231932)(392,0.00231314)(393,0.0023069)(394,0.00230059)(395,0.00229422)(396,0.00228779)(397,0.0022813)(398,0.00227475)(399,0.00226813)(400,0.00226145)(401,0.00225472)(402,0.00224792)(403,0.00224106)(404,0.00223414)(405,0.00222715)(406,0.00222011)(407,0.00221301)(408,0.00220585)(409,0.00219863)(410,0.00219135)(411,0.00218401)(412,0.00217661)(413,0.00216915)(414,0.00216163)(415,0.00215405)(416,0.00214642)(417,0.00213873)(418,0.00213098)(419,0.00212317)(420,0.0021153)(421,0.00210738)(422,0.0020994)(423,0.00209136)(424,0.00208327)(425,0.00207512)(426,0.00206691)(427,0.00205865)(428,0.00205033)(429,0.00204196)(430,0.00203353)(431,0.00202504)(432,0.00201651)(433,0.00200791)(434,0.00199926)(435,0.00199056)(436,0.00198181)(437,0.001973)(438,0.00196413)(439,0.00195522)(440,0.00194625)(441,0.00193722)(442,0.00192815)(443,0.00191902)(444,0.00190984)(445,0.00190061)(446,0.00189133)(447,0.001882)(448,0.00187261)(449,0.00186318)(450,0.00185369)(451,0.00184416)(452,0.00183457)(453,0.00182493)(454,0.00181525)(455,0.00180552)(456,0.00179573)(457,0.0017859)(458,0.00177602)(459,0.00176609)(460,0.00175612)(461,0.00174609)(462,0.00173602)(463,0.0017259)(464,0.00171574)(465,0.00170553)(466,0.00169527)(467,0.00168497)(468,0.00167462)(469,0.00166422)(470,0.00165378)(471,0.0016433)(472,0.00163277)(473,0.0016222)(474,0.00161158)(475,0.00160092)(476,0.00159021)(477,0.00157947)(478,0.00156868)(479,0.00155785)(480,0.00154697)(481,0.00153605)(482,0.0015251)(483,0.0015141)(484,0.00150306)(485,0.00149197)(486,0.00148085)(487,0.00146969)(488,0.00145849)(489,0.00144725)(490,0.00143597)(491,0.00142465)(492,0.00141329)(493,0.00140189)(494,0.00139046)(495,0.00137899)(496,0.00136748)(497,0.00135593)(498,0.00134435)(499,0.00133273)(500,0.00132107)(501,0.00130938)(502,0.00129766)(503,0.00128589)(504,0.0012741)(505,0.00126227)(506,0.0012504)(507,0.0012385)(508,0.00122657)(509,0.0012146)(510,0.0012026)(511,0.00119057)(512,0.00117851)(513,0.00116641)(514,0.00115428)(515,0.00114212)(516,0.00112993)(517,0.00111771)(518,0.00110546)(519,0.00109318)(520,0.00108087)(521,0.00106853)(522,0.00105616)(523,0.00104377)(524,0.00103134)(525,0.00101889)(526,0.00100641)(527,0.0009939)(528,0.000981365)(529,0.000968803)(530,0.000956215)(531,0.000943601)(532,0.000930962)(533,0.000918297)(534,0.000905608)(535,0.000892894)(536,0.000880156)(537,0.000867394)(538,0.000854608)(539,0.000841799)(540,0.000828968)(541,0.000816114)(542,0.000803238)(543,0.00079034)(544,0.00077742)(545,0.00076448)(546,0.000751519)(547,0.000738537)(548,0.000725536)(549,0.000712515)(550,0.000699475)(551,0.000686415)(552,0.000673337)(553,0.000660241)(554,0.000647127)(555,0.000633995)(556,0.000620847)(557,0.000607681)(558,0.000594499)(559,0.000581301)(560,0.000568087)(561,0.000554857)(562,0.000541613)(563,0.000528354)(564,0.000515081)(565,0.000501793)(566,0.000488492)(567,0.000475178)(568,0.000461851)(569,0.000448512)(570,0.00043516)(571,0.000421797)(572,0.000408422)(573,0.000395036)(574,0.000381639)(575,0.000368232)(576,0.000354815)(577,0.000341388)(578,0.000327953)(579,0.000314508)(580,0.000301055)(581,0.000287593)(582,0.000274124)(583,0.000260647)(584,0.000247164)(585,0.000233673)(586,0.000220177)(587,0.000206674)(588,0.000193166)(589,0.000179652)(590,0.000166134)(591,0.000152611)(592,0.000139084)(593,0.000125553)(594,0.000112019)(595,9.84815e-005)(596,8.49416e-005)(597,7.13994e-005)(598,5.78552e-005)(599,4.43095e-005)(600,3.07626e-005) 
};

\end{axis}
\end{tikzpicture}
}
      \end{animateinline}
    \end{column}
    \begin{column}{0.5\textwidth}
      \begin{customlist}{2ex}{0pt}
	\item A Newton method cannot guarantee that the fundamental eigenmode is calculated
	\item Any mode satisfies the nonlinear set of equations
	\item Here the ratio of the first two eigenvalues is close to unity (dominance ratio)
	\item Use a few power iterations to get gross flux shape
      \end{customlist}
    \end{column}
  \end{columns}
  \end{center}
\end{frame}
%------------------------------------------------------------------------------
\begin{frame}{Steady State - Coupled Neutronics/Thermal Hydraulics [1/2]}
\relsize{-2}
  \begin{block}{Residual Equations}
    \begin{center}
    \begin{columns}
      \begin{column}{0.5\textwidth}
	\[
	    \mathbf{F}=\left[\begin{array}{c}
	    \mathbb{M}\mathbf{\Phi}-\lambda\mathbb{F}\mathbf{\Phi}\\
	    Q_{R}-\tilde{c}\kappa\mathbf{\Sigma}_{f}^{\mathrm{T}}\mathbf{\Phi}\Delta x.\\
	    \mathbf{Q}-\tilde{c}\mathbb{E}\mathbf{\Phi}\Delta x\\
	    \mathbb{S}\mathbf{T}-\mathbb{R}\mathbf{Q}\\
	    \mathcal{P}-\rho\left(\mathbf{T},p\right)\\
	    \mathbf{\Sigma}_{a}-\Sigma_{a}^{ref}-\frac{\partial\Sigma_{a}}{\partial\rho}\left[\mathcal{P}-\rho^{ref}\right]\\
	    \nu\mathbf{\Sigma}_{f}-\nu\Sigma_{f}^{ref}-\frac{\partial\nu\Sigma_{f}}{\partial\rho}\left[\mathcal{P}-\rho^{ref}\right]\\
	    \mathbf{D}-D^{ref}-\frac{\partial D}{\partial\rho}\left[\mathcal{P}-\rho^{ref}\right],\\
	    \kappa\mathbf{\Sigma}_{f}-\kappa\Sigma_{f}^{ref}-\frac{\partial\kappa\Sigma_{f}}{\partial\rho}\left[\mathcal{P}-\rho^{ref}\right]\\
	    -\frac{1}{2}\mathbf{\Phi}^{\mathrm{T}}\mathbf{\Phi}+\frac{1}{2}
	    \end{array}\right]
	\]
      \end{column}
      \begin{column}{0.4\textwidth}
	\begin{customlist}{2ex}{0pt}
	  \item Neutronics
	  \item Flux normalization
	  \item Energy deposition
	  \item Temperature distribution
	  \item Density distribution
	  \item Absorption xs
	  \item Fission xs
	  \item Diffusion 
	  \item Energy deposition xs
	  \item Eigenvalue
	\end{customlist}
      \end{column}
    \end{columns}
    \end{center}
  \end{block}
  \begin{customlist}{2ex}{0pt}
    \vfill\item X-Steam MATLAB tables were used to look up density as a function of temperature
    \vfill\item An analytical Jacobian-vector product can be formed for everything except $\frac{\partial\rho}{\partial T}$
    \vfill\item For this block a finite difference approximation must be used
  \end{customlist}
\end{frame}
%------------------------------------------------------------------------------
\begin{frame}{Steady State - Coupled Neutronics/Thermal Hydraulics [2/2]}
  \begin{block}{Results}
    \begin{center}
      \scalebox{0.4}{\begin{tikzpicture}[scale=0.9]

\begin{axis}[%
name=plot1,
scale only axis,
width=4.52083in,
height=3.56562in,
xmin=0, xmax=400,
ymin=0, ymax=400,
xlabel={Slab Length [cm]},
ylabel={Power [W]},
axis on top]
\addplot [
color=blue,
solid
]
coordinates{
 (1,13.9911)(2,20.3593)(3,26.7197)(4,33.0699)(5,39.4076)(6,45.7302)(7,52.0354)(8,58.3208)(9,64.584)(10,70.8228)(11,77.0348)(12,83.2176)(13,89.3691)(14,95.487)(15,101.569)(16,107.613)(17,113.617)(18,119.578)(19,125.496)(20,131.366)(21,137.189)(22,142.96)(23,148.68)(24,154.345)(25,159.955)(26,165.506)(27,170.998)(28,176.428)(29,181.795)(30,187.098)(31,192.335)(32,197.504)(33,202.604)(34,207.633)(35,212.591)(36,217.475)(37,222.285)(38,227.019)(39,231.677)(40,236.256)(41,240.757)(42,245.178)(43,249.519)(44,253.778)(45,257.954)(46,262.048)(47,266.057)(48,269.982)(49,273.822)(50,277.577)(51,281.245)(52,284.827)(53,288.323)(54,291.731)(55,295.052)(56,298.285)(57,301.43)(58,304.488)(59,307.457)(60,310.339)(61,313.133)(62,315.839)(63,318.457)(64,320.988)(65,323.431)(66,325.788)(67,328.057)(68,330.241)(69,332.338)(70,334.35)(71,336.276)(72,338.118)(73,339.876)(74,341.55)(75,343.141)(76,344.65)(77,346.078)(78,347.424)(79,348.69)(80,349.877)(81,350.985)(82,352.015)(83,352.967)(84,353.844)(85,354.645)(86,355.372)(87,356.025)(88,356.606)(89,357.114)(90,357.553)(91,357.921)(92,358.22)(93,358.452)(94,358.617)(95,358.716)(96,358.751)(97,358.722)(98,358.63)(99,358.477)(100,358.264)(101,357.991)(102,357.659)(103,357.271)(104,356.826)(105,356.327)(106,355.774)(107,355.167)(108,354.509)(109,353.801)(110,353.043)(111,352.236)(112,351.382)(113,350.482)(114,349.537)(115,348.547)(116,347.515)(117,346.441)(118,345.325)(119,344.17)(120,342.976)(121,341.744)(122,340.476)(123,339.172)(124,337.833)(125,336.461)(126,335.056)(127,333.619)(128,332.152)(129,330.654)(130,329.128)(131,327.575)(132,325.994)(133,324.387)(134,322.756)(135,321.1)(136,319.421)(137,317.719)(138,315.997)(139,314.253)(140,312.49)(141,310.707)(142,308.907)(143,307.089)(144,305.254)(145,303.404)(146,301.538)(147,299.659)(148,297.765)(149,295.859)(150,293.941)(151,292.012)(152,290.071)(153,288.121)(154,286.161)(155,284.192)(156,282.216)(157,280.232)(158,278.24)(159,276.243)(160,274.24)(161,272.232)(162,270.219)(163,268.202)(164,266.181)(165,264.158)(166,262.132)(167,260.104)(168,258.075)(169,256.044)(170,254.013)(171,251.982)(172,249.951)(173,247.92)(174,245.891)(175,243.864)(176,241.838)(177,239.815)(178,237.794)(179,235.776)(180,233.762)(181,231.752)(182,229.745)(183,227.743)(184,225.746)(185,223.753)(186,221.766)(187,219.785)(188,217.809)(189,215.839)(190,213.876)(191,211.919)(192,209.969)(193,208.027)(194,206.091)(195,204.163)(196,202.242)(197,200.33)(198,198.425)(199,196.529)(200,194.641)(201,192.762)(202,190.892)(203,189.03)(204,187.178)(205,185.334)(206,183.5)(207,181.676)(208,179.861)(209,178.055)(210,176.26)(211,174.474)(212,172.698)(213,170.933)(214,169.177)(215,167.432)(216,165.697)(217,163.972)(218,162.258)(219,160.554)(220,158.861)(221,157.178)(222,155.506)(223,153.845)(224,152.194)(225,150.554)(226,148.925)(227,147.307)(228,145.699)(229,144.102)(230,142.516)(231,140.941)(232,139.377)(233,137.823)(234,136.281)(235,134.749)(236,133.228)(237,131.718)(238,130.218)(239,128.73)(240,127.252)(241,125.785)(242,124.328)(243,122.883)(244,121.447)(245,120.023)(246,118.609)(247,117.206)(248,115.813)(249,114.43)(250,113.058)(251,111.697)(252,110.345)(253,109.004)(254,107.673)(255,106.352)(256,105.042)(257,103.741)(258,102.451)(259,101.17)(260,99.899)(261,98.6379)(262,97.3866)(263,96.1449)(264,94.9129)(265,93.6903)(266,92.4773)(267,91.2736)(268,90.0792)(269,88.8941)(270,87.7182)(271,86.5513)(272,85.3934)(273,84.2445)(274,83.1045)(275,81.9733)(276,80.8507)(277,79.7368)(278,78.6314)(279,77.5345)(280,76.4459)(281,75.3657)(282,74.2937)(283,73.2298)(284,72.1739)(285,71.126)(286,70.086)(287,69.0538)(288,68.0293)(289,67.0123)(290,66.003)(291,65.001)(292,64.0064)(293,63.0191)(294,62.039)(295,61.0659)(296,60.0999)(297,59.1407)(298,58.1884)(299,57.2428)(300,56.3038)(301,55.3714)(302,54.4454)(303,53.5259)(304,52.6125)(305,51.7054)(306,50.8044)(307,49.9094)(308,49.0203)(309,48.137)(310,47.2594)(311,46.3875)(312,45.5212)(313,44.6603)(314,43.8047)(315,42.9545)(316,42.1094)(317,41.2694)(318,40.4344)(319,39.6043)(320,38.779)(321,37.9585)(322,37.1426)(323,36.3312)(324,35.5243)(325,34.7217)(326,33.9234)(327,33.1293)(328,32.3393)(329,31.5532)(330,30.7711)(331,29.9928)(332,29.2182)(333,28.4472)(334,27.6798)(335,26.9158)(336,26.1551)(337,25.3978)(338,24.6436)(339,23.8925)(340,23.1443)(341,22.3991)(342,21.6567)(343,20.917)(344,20.18)(345,19.4455)(346,18.7134)(347,17.9837)(348,17.2563)(349,16.531)(350,15.8079)(351,15.0867)(352,14.3675)(353,13.65)(354,12.9343)(355,12.2203)(356,11.5078)(357,10.7967)(358,10.0871)(359,9.37866)(360,8.67146)(361,7.96537)(362,7.26029)(363,6.55613)(364,5.85281)(365,5.15023)(366,4.44831)(367,3.74696)(368,3.04609)(369,2.3456)(370,1.64541) 
};

\end{axis}

\begin{axis}[%
at=(plot1.right of east), anchor=left of west,
scale only axis,
width=4.52083in,
height=3.56562in,
xmin=0, xmax=400,
ymin=0.65, ymax=0.75,
xlabel={Slab Length [cm]},
ylabel={$\text{Density [g}/\text{cc]}$},
axis on top]
\addplot [
color=red,
solid
]
coordinates{
 (1,0.744285)(2,0.744264)(3,0.744236)(4,0.744199)(5,0.744155)(6,0.744103)(7,0.744044)(8,0.743977)(9,0.743902)(10,0.74382)(11,0.74373)(12,0.743633)(13,0.743528)(14,0.743415)(15,0.743296)(16,0.743168)(17,0.743034)(18,0.742892)(19,0.742743)(20,0.742587)(21,0.742424)(22,0.742253)(23,0.742076)(24,0.741892)(25,0.741701)(26,0.741503)(27,0.741299)(28,0.741087)(29,0.74087)(30,0.740645)(31,0.740415)(32,0.740178)(33,0.739934)(34,0.739685)(35,0.73943)(36,0.739168)(37,0.738901)(38,0.738628)(39,0.738349)(40,0.738065)(41,0.737775)(42,0.737479)(43,0.737178)(44,0.736872)(45,0.736561)(46,0.736245)(47,0.735924)(48,0.735598)(49,0.735268)(50,0.734933)(51,0.734593)(52,0.734249)(53,0.7339)(54,0.733548)(55,0.733191)(56,0.732831)(57,0.732466)(58,0.732098)(59,0.731726)(60,0.73135)(61,0.730971)(62,0.730589)(63,0.730203)(64,0.729814)(65,0.729423)(66,0.729028)(67,0.728631)(68,0.72823)(69,0.727828)(70,0.727422)(71,0.727015)(72,0.726605)(73,0.726193)(74,0.725778)(75,0.725362)(76,0.724944)(77,0.724524)(78,0.724103)(79,0.72368)(80,0.723255)(81,0.722829)(82,0.722402)(83,0.721973)(84,0.721543)(85,0.721113)(86,0.720681)(87,0.720249)(88,0.719815)(89,0.719382)(90,0.718947)(91,0.718512)(92,0.718077)(93,0.717641)(94,0.717205)(95,0.716769)(96,0.716333)(97,0.715897)(98,0.715461)(99,0.715025)(100,0.714589)(101,0.714154)(102,0.713719)(103,0.713284)(104,0.71285)(105,0.712417)(106,0.711984)(107,0.711552)(108,0.71112)(109,0.71069)(110,0.71026)(111,0.709831)(112,0.709404)(113,0.708977)(114,0.708551)(115,0.708127)(116,0.707704)(117,0.707282)(118,0.706862)(119,0.706442)(120,0.706025)(121,0.705608)(122,0.705194)(123,0.704781)(124,0.704369)(125,0.703959)(126,0.703551)(127,0.703145)(128,0.70274)(129,0.702337)(130,0.701936)(131,0.701537)(132,0.701139)(133,0.700744)(134,0.700351)(135,0.699959)(136,0.69957)(137,0.699183)(138,0.698797)(139,0.698414)(140,0.698033)(141,0.697654)(142,0.697278)(143,0.696903)(144,0.696531)(145,0.696161)(146,0.695793)(147,0.695428)(148,0.695065)(149,0.694704)(150,0.694345)(151,0.693989)(152,0.693635)(153,0.693284)(154,0.692935)(155,0.692588)(156,0.692244)(157,0.691902)(158,0.691562)(159,0.691225)(160,0.690891)(161,0.690558)(162,0.690229)(163,0.689901)(164,0.689577)(165,0.689254)(166,0.688934)(167,0.688617)(168,0.688302)(169,0.687989)(170,0.687679)(171,0.687372)(172,0.687067)(173,0.686764)(174,0.686464)(175,0.686166)(176,0.685871)(177,0.685578)(178,0.685288)(179,0.685)(180,0.684714)(181,0.684431)(182,0.684151)(183,0.683873)(184,0.683597)(185,0.683324)(186,0.683053)(187,0.682785)(188,0.682519)(189,0.682255)(190,0.681994)(191,0.681735)(192,0.681478)(193,0.681224)(194,0.680973)(195,0.680723)(196,0.680476)(197,0.680231)(198,0.679989)(199,0.679749)(200,0.679511)(201,0.679276)(202,0.679043)(203,0.678812)(204,0.678583)(205,0.678356)(206,0.678132)(207,0.67791)(208,0.67769)(209,0.677473)(210,0.677258)(211,0.677044)(212,0.676833)(213,0.676624)(214,0.676418)(215,0.676213)(216,0.676011)(217,0.67581)(218,0.675612)(219,0.675416)(220,0.675221)(221,0.675029)(222,0.674839)(223,0.674651)(224,0.674465)(225,0.674281)(226,0.674099)(227,0.673919)(228,0.673741)(229,0.673565)(230,0.67339)(231,0.673218)(232,0.673048)(233,0.672879)(234,0.672713)(235,0.672548)(236,0.672385)(237,0.672224)(238,0.672065)(239,0.671907)(240,0.671752)(241,0.671598)(242,0.671446)(243,0.671296)(244,0.671147)(245,0.671)(246,0.670855)(247,0.670712)(248,0.67057)(249,0.67043)(250,0.670292)(251,0.670155)(252,0.67002)(253,0.669887)(254,0.669755)(255,0.669625)(256,0.669497)(257,0.66937)(258,0.669244)(259,0.669121)(260,0.668998)(261,0.668878)(262,0.668759)(263,0.668641)(264,0.668525)(265,0.66841)(266,0.668297)(267,0.668185)(268,0.668075)(269,0.667966)(270,0.667859)(271,0.667753)(272,0.667648)(273,0.667545)(274,0.667444)(275,0.667343)(276,0.667244)(277,0.667147)(278,0.66705)(279,0.666955)(280,0.666862)(281,0.66677)(282,0.666679)(283,0.666589)(284,0.666501)(285,0.666413)(286,0.666328)(287,0.666243)(288,0.66616)(289,0.666078)(290,0.665997)(291,0.665917)(292,0.665839)(293,0.665761)(294,0.665685)(295,0.665611)(296,0.665537)(297,0.665464)(298,0.665393)(299,0.665323)(300,0.665254)(301,0.665186)(302,0.665119)(303,0.665054)(304,0.664989)(305,0.664926)(306,0.664863)(307,0.664802)(308,0.664742)(309,0.664683)(310,0.664625)(311,0.664568)(312,0.664512)(313,0.664457)(314,0.664404)(315,0.664351)(316,0.664299)(317,0.664248)(318,0.664199)(319,0.66415)(320,0.664103)(321,0.664056)(322,0.66401)(323,0.663966)(324,0.663922)(325,0.663879)(326,0.663837)(327,0.663797)(328,0.663757)(329,0.663718)(330,0.66368)(331,0.663643)(332,0.663607)(333,0.663572)(334,0.663538)(335,0.663505)(336,0.663473)(337,0.663441)(338,0.663411)(339,0.663381)(340,0.663353)(341,0.663325)(342,0.663298)(343,0.663272)(344,0.663247)(345,0.663223)(346,0.6632)(347,0.663178)(348,0.663156)(349,0.663136)(350,0.663116)(351,0.663097)(352,0.66308)(353,0.663062)(354,0.663046)(355,0.663031)(356,0.663017)(357,0.663003)(358,0.66299)(359,0.662979)(360,0.662968)(361,0.662957)(362,0.662948)(363,0.66294)(364,0.662932)(365,0.662926)(366,0.66292)(367,0.662915)(368,0.662911)(369,0.662907)(370,0.662905) 
};

\end{axis}

\end{tikzpicture}
}
    \end{center}
  \end{block}
  \begin{customlist}{2ex}{0pt}
    \vfill\item \small Timing results using MATLAB profiler:
    \begin{customlist}{2ex}{0pt}
      \vfill\item 7 seconds to converged overall
      \vfill\item 6+ out of 7 seconds in X-Steam!
      \vfill\item Solution: fit density range with polynomial - with linear fit $<$ 0.5 seconds
    \end{customlist}
  \end{customlist}
\end{frame}
%------------------------------------------------------------------------------
\begin{frame}{Transient - Coupled Neutronics/Thermal Hydraulics [1/2]}
\relsize{-1}
  \begin{block}{Residual Equations}
	\[
	    \mathbf{F}=\left[\begin{array}{c}
	    \mathbf{\Phi}^{n+1}-\mathbf{\Phi}^{n}+v\Delta t\left[\mathbb{M}\mathbf{\Phi}^{n+1}-\left(1-\beta\right)\lambda\mathbb{F}\mathbf{\Phi}^{n+1}-\lambda_{d}\mathbf{c}^{n+1}\right]\\
	    \mathbf{c}^{n+1}-\mathbf{c}^{n}+\Delta t\left(\lambda_{d}\mathbf{c}^{n+1}-\beta\lambda\mathbb{F}\mathbf{\Phi}^{n+1}\right)\\
	    \mathbf{Q}-\tilde{c}\mathbb{E}\mathbf{\Phi}\Delta x\\
	    \mathbf{T}^{n+1}-\mathbf{T}^{n}+\frac{\dot{m} \Delta t}{\mathcal{P}^{n+1}A\Delta x}\left(\mathbb{S}\mathbf{T}^{n+1}-\mathbb{R}\mathbf{Q}^{n+1}\right)\\
	    \mathcal{P}-\rho^{ref}-\frac{\partial\rho}{\partial T}\left(\mathbf{T}-T^{ref}\right)\\
	    \mathbf{\Sigma}_{a}-\Sigma_{a}^{ref}-\frac{\partial\Sigma_{a}}{\partial\rho}\left[\mathcal{P}-\rho^{ref}\right]\\
	    \nu\mathbf{\Sigma}_{f}-\nu\Sigma_{f}^{ref}-\frac{\partial\nu\Sigma_{f}}{\partial\rho}\left[\mathcal{P}-\rho^{ref}\right]\\
	    \mathbf{D}-D^{ref}-\frac{\partial D}{\partial\rho}\left[\mathcal{P}-\rho^{ref}\right],\\
	    \kappa\mathbf{\Sigma}_{f}-\kappa\Sigma_{f}^{ref}-\frac{\partial\kappa\Sigma_{f}}{\partial\rho}\left[\mathcal{P}-\rho^{ref}\right]
	    \end{array}\right]
	\]
  \end{block}
  \begin{customlist}{2ex}{0pt}
    \item Each time step took ~0.4 seconds 
    \item 5-8 Newton iterations per time step and 10-20 GMRES per Newton step
  \end{customlist}
\end{frame}
%------------------------------------------------------------------------------
\begin{frame}{Transient - Coupled Neutronics/Thermal Hydraulics [2/2]}
\def \dx{0.0135cm}
\newcounter{cr}
\setcounter{cr}{92}
\begin{animateinline}[poster = first, controls]{2}
  \begin{columns}
    \begin{column}{0.75\textwidth}
      \begin{center}
      \scalebox{0.5}{% This file was created by matlab2tikz v0.1.4.
% Copyright (c) 2008--2011, Nico Schlömer <nico.schloemer@gmail.com>
% All rights reserved.
% 
% The latest updates can be retrieved from
%   http://www.mathworks.com/matlabcentral/fileexchange/22022-matlab2tikz
% where you can also make suggestions and rate matlab2tikz.
% 
\begin{tikzpicture}

\begin{axis}[%
name=plot1,
scale only axis,
width=1.89157in,
height=1.4272in,
xmin=0, xmax=400,
ymin=0, ymax=400,
xlabel={Power [W]},
ylabel={Slab Length [cm]},
title={Power},
axis on top]
\addplot [
color=blue,
solid
]
coordinates{
 (13.9911,1)(20.3593,2)(26.7197,3)(33.0699,4)(39.4076,5)(45.7302,6)(52.0354,7)(58.3208,8)(64.584,9)(70.8228,10)(77.0348,11)(83.2176,12)(89.3691,13)(95.487,14)(101.569,15)(107.613,16)(113.617,17)(119.578,18)(125.496,19)(131.366,20)(137.189,21)(142.96,22)(148.68,23)(154.345,24)(159.955,25)(165.506,26)(170.998,27)(176.428,28)(181.795,29)(187.098,30)(192.335,31)(197.504,32)(202.604,33)(207.633,34)(212.591,35)(217.475,36)(222.285,37)(227.019,38)(231.677,39)(236.256,40)(240.757,41)(245.178,42)(249.519,43)(253.778,44)(257.954,45)(262.048,46)(266.057,47)(269.982,48)(273.822,49)(277.577,50)(281.245,51)(284.827,52)(288.323,53)(291.731,54)(295.052,55)(298.285,56)(301.43,57)(304.488,58)(307.457,59)(310.339,60)(313.133,61)(315.839,62)(318.457,63)(320.988,64)(323.431,65)(325.788,66)(328.057,67)(330.241,68)(332.338,69)(334.35,70)(336.276,71)(338.118,72)(339.876,73)(341.55,74)(343.141,75)(344.65,76)(346.078,77)(347.424,78)(348.69,79)(349.877,80)(350.985,81)(352.015,82)(352.967,83)(353.844,84)(354.645,85)(355.372,86)(356.025,87)(356.606,88)(357.114,89)(357.553,90)(357.921,91)(358.22,92)(358.452,93)(358.617,94)(358.716,95)(358.751,96)(358.722,97)(358.63,98)(358.477,99)(358.264,100)(357.991,101)(357.659,102)(357.271,103)(356.826,104)(356.327,105)(355.774,106)(355.167,107)(354.509,108)(353.801,109)(353.043,110)(352.236,111)(351.382,112)(350.482,113)(349.537,114)(348.547,115)(347.515,116)(346.441,117)(345.325,118)(344.17,119)(342.976,120)(341.744,121)(340.476,122)(339.172,123)(337.833,124)(336.461,125)(335.056,126)(333.619,127)(332.152,128)(330.654,129)(329.128,130)(327.575,131)(325.994,132)(324.387,133)(322.756,134)(321.1,135)(319.421,136)(317.719,137)(315.997,138)(314.253,139)(312.49,140)(310.707,141)(308.907,142)(307.089,143)(305.254,144)(303.404,145)(301.538,146)(299.659,147)(297.765,148)(295.859,149)(293.941,150)(292.012,151)(290.071,152)(288.121,153)(286.161,154)(284.192,155)(282.216,156)(280.232,157)(278.24,158)(276.243,159)(274.24,160)(272.232,161)(270.219,162)(268.202,163)(266.181,164)(264.158,165)(262.132,166)(260.104,167)(258.075,168)(256.044,169)(254.013,170)(251.982,171)(249.951,172)(247.92,173)(245.891,174)(243.864,175)(241.838,176)(239.815,177)(237.794,178)(235.776,179)(233.762,180)(231.752,181)(229.745,182)(227.743,183)(225.746,184)(223.753,185)(221.766,186)(219.785,187)(217.809,188)(215.839,189)(213.876,190)(211.919,191)(209.969,192)(208.027,193)(206.091,194)(204.163,195)(202.242,196)(200.33,197)(198.425,198)(196.529,199)(194.641,200)(192.762,201)(190.892,202)(189.03,203)(187.178,204)(185.334,205)(183.5,206)(181.676,207)(179.861,208)(178.055,209)(176.26,210)(174.474,211)(172.698,212)(170.933,213)(169.177,214)(167.432,215)(165.697,216)(163.972,217)(162.258,218)(160.554,219)(158.861,220)(157.178,221)(155.506,222)(153.845,223)(152.194,224)(150.554,225)(148.925,226)(147.307,227)(145.699,228)(144.102,229)(142.516,230)(140.941,231)(139.377,232)(137.823,233)(136.281,234)(134.749,235)(133.228,236)(131.718,237)(130.218,238)(128.73,239)(127.252,240)(125.785,241)(124.328,242)(122.883,243)(121.447,244)(120.023,245)(118.609,246)(117.206,247)(115.813,248)(114.43,249)(113.058,250)(111.697,251)(110.345,252)(109.004,253)(107.673,254)(106.352,255)(105.042,256)(103.741,257)(102.451,258)(101.17,259)(99.899,260)(98.6379,261)(97.3866,262)(96.1449,263)(94.9129,264)(93.6903,265)(92.4773,266)(91.2736,267)(90.0792,268)(88.8941,269)(87.7182,270)(86.5513,271)(85.3934,272)(84.2445,273)(83.1045,274)(81.9733,275)(80.8507,276)(79.7368,277)(78.6314,278)(77.5345,279)(76.4459,280)(75.3657,281)(74.2937,282)(73.2298,283)(72.1739,284)(71.126,285)(70.086,286)(69.0538,287)(68.0293,288)(67.0123,289)(66.003,290)(65.001,291)(64.0064,292)(63.0191,293)(62.039,294)(61.0659,295)(60.0999,296)(59.1407,297)(58.1884,298)(57.2428,299)(56.3038,300)(55.3714,301)(54.4454,302)(53.5259,303)(52.6125,304)(51.7054,305)(50.8044,306)(49.9094,307)(49.0203,308)(48.137,309)(47.2594,310)(46.3875,311)(45.5212,312)(44.6603,313)(43.8047,314)(42.9545,315)(42.1094,316)(41.2694,317)(40.4344,318)(39.6043,319)(38.779,320)(37.9585,321)(37.1426,322)(36.3312,323)(35.5243,324)(34.7217,325)(33.9234,326)(33.1293,327)(32.3393,328)(31.5532,329)(30.7711,330)(29.9928,331)(29.2182,332)(28.4472,333)(27.6798,334)(26.9158,335)(26.1551,336)(25.3978,337)(24.6436,338)(23.8925,339)(23.1443,340)(22.3991,341)(21.6567,342)(20.917,343)(20.18,344)(19.4455,345)(18.7134,346)(17.9837,347)(17.2563,348)(16.531,349)(15.8079,350)(15.0867,351)(14.3675,352)(13.65,353)(12.9343,354)(12.2203,355)(11.5078,356)(10.7967,357)(10.0871,358)(9.37866,359)(8.67146,360)(7.96537,361)(7.26029,362)(6.55613,363)(5.85281,364)(5.15023,365)(4.44831,366)(3.74696,367)(3.04609,368)(2.3456,369)(1.64541,370) 
};

\end{axis}

\begin{axis}[%
name=plot2,
at=(plot1.right of south east), anchor=left of south west,
scale only axis,
width=1.89157in,
height=1.4272in,
xmin=0.65, xmax=0.75,
ymin=0, ymax=400,
xlabel={$\text{Density [g}/\text{cc]}$},
ylabel={Slab Length [cm]},
title={Density},
axis on top]
\addplot [
color=blue,
solid
]
coordinates{
 (0.744285,1)(0.744264,2)(0.744236,3)(0.744199,4)(0.744155,5)(0.744103,6)(0.744044,7)(0.743977,8)(0.743902,9)(0.74382,10)(0.74373,11)(0.743633,12)(0.743528,13)(0.743415,14)(0.743296,15)(0.743168,16)(0.743034,17)(0.742892,18)(0.742743,19)(0.742587,20)(0.742424,21)(0.742253,22)(0.742076,23)(0.741892,24)(0.741701,25)(0.741503,26)(0.741299,27)(0.741087,28)(0.74087,29)(0.740645,30)(0.740415,31)(0.740178,32)(0.739934,33)(0.739685,34)(0.73943,35)(0.739168,36)(0.738901,37)(0.738628,38)(0.738349,39)(0.738065,40)(0.737775,41)(0.737479,42)(0.737178,43)(0.736872,44)(0.736561,45)(0.736245,46)(0.735924,47)(0.735598,48)(0.735268,49)(0.734933,50)(0.734593,51)(0.734249,52)(0.7339,53)(0.733548,54)(0.733191,55)(0.732831,56)(0.732466,57)(0.732098,58)(0.731726,59)(0.73135,60)(0.730971,61)(0.730589,62)(0.730203,63)(0.729814,64)(0.729423,65)(0.729028,66)(0.728631,67)(0.72823,68)(0.727828,69)(0.727422,70)(0.727015,71)(0.726605,72)(0.726193,73)(0.725778,74)(0.725362,75)(0.724944,76)(0.724524,77)(0.724103,78)(0.72368,79)(0.723255,80)(0.722829,81)(0.722402,82)(0.721973,83)(0.721543,84)(0.721113,85)(0.720681,86)(0.720249,87)(0.719815,88)(0.719382,89)(0.718947,90)(0.718512,91)(0.718077,92)(0.717641,93)(0.717205,94)(0.716769,95)(0.716333,96)(0.715897,97)(0.715461,98)(0.715025,99)(0.714589,100)(0.714154,101)(0.713719,102)(0.713284,103)(0.71285,104)(0.712417,105)(0.711984,106)(0.711552,107)(0.71112,108)(0.71069,109)(0.71026,110)(0.709831,111)(0.709404,112)(0.708977,113)(0.708551,114)(0.708127,115)(0.707704,116)(0.707282,117)(0.706862,118)(0.706442,119)(0.706025,120)(0.705608,121)(0.705194,122)(0.704781,123)(0.704369,124)(0.703959,125)(0.703551,126)(0.703145,127)(0.70274,128)(0.702337,129)(0.701936,130)(0.701537,131)(0.701139,132)(0.700744,133)(0.700351,134)(0.699959,135)(0.69957,136)(0.699183,137)(0.698797,138)(0.698414,139)(0.698033,140)(0.697654,141)(0.697278,142)(0.696903,143)(0.696531,144)(0.696161,145)(0.695793,146)(0.695428,147)(0.695065,148)(0.694704,149)(0.694345,150)(0.693989,151)(0.693635,152)(0.693284,153)(0.692935,154)(0.692588,155)(0.692244,156)(0.691902,157)(0.691562,158)(0.691225,159)(0.690891,160)(0.690558,161)(0.690229,162)(0.689901,163)(0.689577,164)(0.689254,165)(0.688934,166)(0.688617,167)(0.688302,168)(0.687989,169)(0.687679,170)(0.687372,171)(0.687067,172)(0.686764,173)(0.686464,174)(0.686166,175)(0.685871,176)(0.685578,177)(0.685288,178)(0.685,179)(0.684714,180)(0.684431,181)(0.684151,182)(0.683873,183)(0.683597,184)(0.683324,185)(0.683053,186)(0.682785,187)(0.682519,188)(0.682255,189)(0.681994,190)(0.681735,191)(0.681478,192)(0.681224,193)(0.680973,194)(0.680723,195)(0.680476,196)(0.680231,197)(0.679989,198)(0.679749,199)(0.679511,200)(0.679276,201)(0.679043,202)(0.678812,203)(0.678583,204)(0.678356,205)(0.678132,206)(0.67791,207)(0.67769,208)(0.677473,209)(0.677258,210)(0.677044,211)(0.676833,212)(0.676624,213)(0.676418,214)(0.676213,215)(0.676011,216)(0.67581,217)(0.675612,218)(0.675416,219)(0.675221,220)(0.675029,221)(0.674839,222)(0.674651,223)(0.674465,224)(0.674281,225)(0.674099,226)(0.673919,227)(0.673741,228)(0.673565,229)(0.67339,230)(0.673218,231)(0.673048,232)(0.672879,233)(0.672713,234)(0.672548,235)(0.672385,236)(0.672224,237)(0.672065,238)(0.671907,239)(0.671752,240)(0.671598,241)(0.671446,242)(0.671296,243)(0.671147,244)(0.671,245)(0.670855,246)(0.670712,247)(0.67057,248)(0.67043,249)(0.670292,250)(0.670155,251)(0.67002,252)(0.669887,253)(0.669755,254)(0.669625,255)(0.669497,256)(0.66937,257)(0.669244,258)(0.669121,259)(0.668998,260)(0.668878,261)(0.668759,262)(0.668641,263)(0.668525,264)(0.66841,265)(0.668297,266)(0.668185,267)(0.668075,268)(0.667966,269)(0.667859,270)(0.667753,271)(0.667648,272)(0.667545,273)(0.667444,274)(0.667343,275)(0.667244,276)(0.667147,277)(0.66705,278)(0.666955,279)(0.666862,280)(0.66677,281)(0.666679,282)(0.666589,283)(0.666501,284)(0.666413,285)(0.666328,286)(0.666243,287)(0.66616,288)(0.666078,289)(0.665997,290)(0.665917,291)(0.665839,292)(0.665761,293)(0.665685,294)(0.665611,295)(0.665537,296)(0.665464,297)(0.665393,298)(0.665323,299)(0.665254,300)(0.665186,301)(0.665119,302)(0.665054,303)(0.664989,304)(0.664926,305)(0.664863,306)(0.664802,307)(0.664742,308)(0.664683,309)(0.664625,310)(0.664568,311)(0.664512,312)(0.664457,313)(0.664404,314)(0.664351,315)(0.664299,316)(0.664248,317)(0.664199,318)(0.66415,319)(0.664103,320)(0.664056,321)(0.66401,322)(0.663966,323)(0.663922,324)(0.663879,325)(0.663837,326)(0.663797,327)(0.663757,328)(0.663718,329)(0.66368,330)(0.663643,331)(0.663607,332)(0.663572,333)(0.663538,334)(0.663505,335)(0.663473,336)(0.663441,337)(0.663411,338)(0.663381,339)(0.663353,340)(0.663325,341)(0.663298,342)(0.663272,343)(0.663247,344)(0.663223,345)(0.6632,346)(0.663178,347)(0.663156,348)(0.663136,349)(0.663116,350)(0.663097,351)(0.66308,352)(0.663062,353)(0.663046,354)(0.663031,355)(0.663017,356)(0.663003,357)(0.66299,358)(0.662979,359)(0.662968,360)(0.662957,361)(0.662948,362)(0.66294,363)(0.662932,364)(0.662926,365)(0.66292,366)(0.662915,367)(0.662911,368)(0.662907,369)(0.662905,370) 
};

\end{axis}

\begin{axis}[%
name=plot4,
at=(plot2.below south west), anchor=above north west,
scale only axis,
width=1.89157in,
height=1.4272in,
xmin=0, xmax=100,
ymin=304, ymax=316,
xlabel={Time [s]},
ylabel={Temperature [C]},
title={Average Temperature},
axis on top]
\addplot [
color=blue,
solid
]
coordinates{
 (0,315.093)(0.1,315.093)(0.196321,11.5933) 
};

\end{axis}

\begin{axis}[%
at=(plot4.left of south west), anchor=right of south east,
scale only axis,
width=1.89157in,
height=1.4272in,
xmin=0, xmax=100,
ymin=52000, ymax=70000,
xlabel={Time [s]},
ylabel={Power [W]},
title={Reactor Power},
axis on top]
\addplot [
color=blue,
solid
]
coordinates{
 (0,66945.4)(0.1,66945.4)(0.177675,14945.9) 
};

\end{axis}
\end{tikzpicture}
}
      \end{center}
    \end{column}
    \begin{column}{0.25\textwidth}
      \begin{center}
	\begin{tikzpicture}
	  \node[rectangle, rounded corners, draw = black, very thick, minimum height=5cm, minimum width=1cm,fill=red!50,anchor = south west]  (slab) at (0.0,0.0) {};
	  \draw[snake=coil,segment aspect=0,very thick,segment length=20pt,line after snake=1mm,->,draw=blue] (1.25cm,0.0cm) -- (1.25cm,5cm);
	  \draw[snake=coil,segment aspect=0,very thick,segment length=20pt,line after snake=1mm,->,draw=blue] (1.5cm,0.0cm) --  (1.5cm,5cm);
	  \draw[snake=coil,segment aspect=0,very thick,segment length=20pt,line after snake=1mm,->,draw=blue] (1.75cm,0.0cm) -- (1.75cm,5cm);
	\end{tikzpicture}
      \end{center}
    \end{column}
  \end{columns}
  \newcounter{ct}
  \forloop[10]{ct}{11}{\value{ct} < 992}
  {%
  \newframe
    \begin{columns}
      \begin{column}{0.75\textwidth}
	\begin{center}
	  \scalebox{0.5}{\input{./tikz/anim/rod_trans_\arabic{ct}.tikz}}
	\end{center}
      \end{column}
    \begin{column}{0.25\textwidth}
      \begin{center}
	\ifthenelse{\value{ct} > 50 \and \value{ct} < 585}
	{
	  \ifthenelse{\value{ct} < 56}
	  {
	    \begin{tikzpicture}
	      \node[rectangle, rounded corners, draw = black, very thick, minimum height=5cm, minimum width=1cm,fill=red!50,anchor= south west]  (slab) at (0.0,0.0) {};
	      \draw[snake=coil,segment aspect=0,very thick,segment length=20pt,line after snake=1mm,->,draw=blue] (1.25cm,0.0cm) -- (1.25cm,5cm);
	      \draw[snake=coil,segment aspect=0,very thick,segment length=20pt,line after snake=1mm,->,draw=blue] (1.5cm,0.0cm) --  (1.5cm,5cm);
	      \draw[snake=coil,segment aspect=0,very thick,segment length=20pt,line after snake=1mm,->,draw=blue] (1.75cm,0.0cm) -- (1.75cm,5cm);
	      \node[rectangle, rounded corners,draw = black, very thick, minimum height=2*\dx, minimum width=0.25cm,left=0cm of slab.south west, anchor=south west,fill=black] {};
	    \end{tikzpicture}   
	  }
	  {
	    \ifthenelse{\value{ct} < 97}
	    {
	      \begin{tikzpicture}
		\node[rectangle, rounded corners, draw = black, very thick, minimum height=5cm, minimum width=1cm,fill=red!50,anchor= south west]  (slab) at (0.0,0.0) {};
		\draw[snake=coil,segment aspect=0,very thick,segment length=20pt,line after snake=1mm,->,draw=blue] (1.25cm,0.0cm) -- (1.25cm,5cm);
		\draw[snake=coil,segment aspect=0,very thick,segment length=20pt,line after snake=1mm,->,draw=blue] (1.5cm,0.0cm) --  (1.5cm,5cm);
		\draw[snake=coil,segment aspect=0,very thick,segment length=20pt,line after snake=1mm,->,draw=blue] (1.75cm,0.0cm) -- (1.75cm,5cm);
		\def \crheight{2+floor((\value{ct}-50)/10)*20}
		\node[rectangle,rounded corners, draw = black, very thick, minimum height=\crheight*\dx, minimum width=0.25cm,left=0cm of slab.south west, anchor=south west,fill=black] {};
	      \end{tikzpicture}   
	    }
	    {
	      \ifthenelse{\value{ct} < 500}
	      {
		\begin{tikzpicture}
		  \node[rectangle, rounded corners, draw = black, very thick, minimum height=5cm, minimum width=1cm,fill=red!50,anchor=south west]  (slab) at (0.0,0.0) {};
		  \draw[snake=coil,segment aspect=0,very thick,segment length=20pt,line after snake=1mm,->,draw=blue] (1.25cm,0.0cm) -- (1.25cm,5cm);
		  \draw[snake=coil,segment aspect=0,very thick,segment length=20pt,line after snake=1mm,->,draw=blue] (1.5cm,0.0cm) --  (1.5cm,5cm);
		  \draw[snake=coil,segment aspect=0,very thick,segment length=20pt,line after snake=1mm,->,draw=blue] (1.75cm,0.0cm) -- (1.75cm,5cm);
		  \node[rectangle, rounded corners,draw = black, very thick, minimum height=92*\dx, minimum width=0.25cm,left=0cm of slab.south west, anchor=south west,fill=black] {};
		\end{tikzpicture}
	      }
	      {
		\begin{tikzpicture}
		  \node[rectangle, rounded corners, draw = black, very thick, minimum height=5cm, minimum width=1cm,fill=red!50,anchor=south west]  (slab) at (0.0,0.0) {};
		  \draw[snake=coil,segment aspect=0,very thick,segment length=20pt,line after snake=1mm,->,draw=blue] (1.25cm,0.0cm) -- (1.25cm,5cm);
		  \draw[snake=coil,segment aspect=0,very thick,segment length=20pt,line after snake=1mm,->,draw=blue] (1.5cm,0.0cm) --  (1.5cm,5cm);
		  \draw[snake=coil,segment aspect=0,very thick,segment length=20pt,line after snake=1mm,->,draw=blue] (1.75cm,0.0cm) -- (1.75cm,5cm);
		  \addtocounter{cr}{-10}
		  \node[rectangle, draw = black,rounded corners, very thick, minimum height=\value{cr}*\dx, minimum width=0.25cm,left=0cm of slab.south west, anchor=south west,fill=black] {};
		\end{tikzpicture}
	      }
	    }
	  }
	}
	{
	  \begin{tikzpicture}
	    \node[rectangle, rounded corners, draw = black, very thick, minimum height=5cm, minimum width=1cm,fill=red!50,anchor= south west]  (slab) at (0.0,0.0) {};
	    \draw[snake=coil,segment aspect=0,very thick,segment length=20pt,line after snake=1mm,->,draw=blue] (1.25cm,0.0cm) -- (1.25cm,5cm);
	    \draw[snake=coil,segment aspect=0,very thick,segment length=20pt,line after snake=1mm,->,draw=blue] (1.5cm,0.0cm) --  (1.5cm,5cm);
	    \draw[snake=coil,segment aspect=0,very thick,segment length=20pt,line after snake=1mm,->,draw=blue] (1.75cm,0.0cm) -- (1.75cm,5cm);
	  \end{tikzpicture}
	}
      \end{center}
    \end{column}
  \end{columns}
  }
\end{animateinline}
\end{frame}
%------------------------------------------------------------------------------
\end{section}
%==============================================================================
\begin{section}{Conclusions}
%------------------------------------------------------------------------------
\begin{frame}{Conclusions}
  \begin{customlist}{2ex}{0pt}
    \item JFNK methods are well suited for solving coupled systems
    \vfill\item Finite difference method for Jacobian-vector product works well
    \vfill\item Write your own steam tables or fit data with polynomial
    \vfill\item Watch out for scaling issues when coupling physics
  \end{customlist}
\end{frame}
%------------------------------------------------------------------------------
\begin{frame}{Future Work}
  \begin{customlist}{2ex}{0pt}
    \item Rewrite code in a compiled language (Fortran)
    \vfill\item Use PETSc Nonlinear solvers
    \vfill\item Model other important nuclear feedback mechanisms
    \vfill\item Compare JFNK to Operator-splitting methods
    \vfill\item This project can be downloaded from GitHub:
    \begin{center}
      \url{www.github.com/bhermanmit/JFNK} \\
      \tt{git clone git@github.com:bhermanmit/JFNK}
    \end{center}
  \end{customlist}
\end{frame}
%------------------------------------------------------------------------------
\end{section}
%==============================================================================
\end{document}
